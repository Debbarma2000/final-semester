\documentclass[oneside, 12pt]{scrbook}

%packages to import
\usepackage{scrhack}
\usepackage[utf8]{inputenc}
\usepackage{amsmath}
\usepackage{amssymb}
\usepackage{amsthm}
\usepackage{float}
\usepackage{minitoc}
\usepackage[width=15cm, height=24cm]{geometry}
\usepackage{hyperref}
\usepackage[usenames,svgnames,dvipsnames]{xcolor}
\usepackage{mathrsfs}
\usepackage{mathtools}
\usepackage{thmtools}
%\usepackage{kpfonts}
\usepackage[lf]{venturis} %% lf option gives lining figures as default; 
			  %% remove option to get oldstyle figures as default
\renewcommand*\familydefault{\sfdefault} %% Only if the base font of the document is to be sans serif
\usepackage{fontenc}
\usepackage{setspace}
\usepackage{cleveref}
\usepackage{backref}
\usepackage{graphicx}
\usepackage{tikz-cd}
\usepackage{quiver}
\usepackage{hyperref}
\usepackage{extpfeil}
\usepackage{stmaryrd}
%\usepackage{newtxtext,newtxmath}

%section symbol


%no indentation paragraph
\setlength{\parindent}{0pt}


% color those links
\hypersetup{
colorlinks=true,
urlcolor= BrickRed,
linkcolor= BrickRed,
citecolor= Cerulean
}
%colour page black and text white
%\pagecolor{gray}
%\color{white}

%\usepackage{multicol}
\usepackage[utf8]{inputenc}

%custom commands
\newtheorem{statement}{Statement}
\newcommand{\CC}{\mathbb C}
\newcommand{\FF}{\mathbb F}
\newcommand{\NN}{\mathbb N}
\newcommand{\QQ}{\mathbb Q}
\newcommand{\RR}{\mathbb R}
\newcommand{\ZZ}{\mathbb Z}
\newcommand{\p}{\mathfrak{p}}
\newcommand{\V}{\mathcal{V}}
\newcommand{\I}{\mathcal{I}}
\newcommand{\spec}{\mathrm{Spec}}
\newcommand{\Aa}{\mathbb{A}}
\newcommand{\PP}{\mathbb{P}}
\newcommand{\im}{\mathrm{im}}
\newcommand{\pr}{\mathfrak{p}}
\newcommand{\m}{\mathfrak{m}}
\newcommand{\nil}{\mathrm{Nil}}
\newcommand{\ord}{\mathrm{ord}}
\newcommand{\SL}{\mathrm{SL}}
\newcommand{\GL}{\mathrm{GL}}
\newcommand{\Hom}{\mathrm{Hom}}
\newcommand{\Char}{\mathrm{char}}
\newcommand{\Ext}{\mathrm{Ext}}
\newcommand{\Tor}{\mathrm{Tor}}
\newcommand{\iM}{\mathrm{Im}}
\newcommand{\coker}{\mathrm{Coker}}
\newcommand{\HH}{\mathbb{H}}
\newcommand{\bs}{\backslash}
\newcommand{\ds}{\displaystyle}
\newcommand{\Stab}{\mathrm{Stab}}
\newcommand{\id}{\mathrm{id}}
\newcommand{\Der}{\mathrm{Der}}
\newcommand{\Ht}{\mathrm{ht}}
\newcommand{\coht}{\mathrm{co.ht}}
\newcommand{\Hh}{\mathcal{H}}
\newcommand{\Mm}{\mathcal{M}}
\newcommand{\Ss}{\mathcal{S}}

%theoremstyles 
\usepackage[framemethod=TikZ]{mdframed}

%theorem
\mdfdefinestyle{mdthmbox}{
	linewidth= 3pt,
	rightline=false,
	leftline=true,
	topline=false,
	bottomline=false,
	linecolor= WildStrawberry,
	backgroundcolor= CarnationPink!25 
}

\declaretheoremstyle[
	spaceabove=6pt, spacebelow=6pt,
	headfont=\normalfont \bfseries,
	notefont=\mdseries, notebraces={(}{)},
	bodyfont=\normalfont,
	postheadspace = \newline,
	mdframed = {style= mdthmbox}
]{theobox}

\declaretheorem[style=theobox,name=Theorem,numberwithin=section]{theorem}
\declaretheorem[style=theobox,name=Conjecture,numberwithin=section]{conjecture}

%lemma, proposition, corollary
\mdfdefinestyle{mdlembox}{
	linewidth= 3pt,
	rightline=false,
	leftline=true,
	topline=false,
	bottomline=false,
	linecolor= RoyalBlue,
	backgroundcolor= SkyBlue!30
}

\declaretheoremstyle[
	postheadspace= \newline,
	spaceabove=6pt, spacebelow=6pt,
	headfont=\normalfont \bfseries,
	notefont=\mdseries, notebraces={(}{)},
	bodyfont=\normalfont,
	postheadspace = \newline,
	mdframed = {style= mdlembox}
]{lembox}
\declaretheorem[style=lembox,name=Lemma,sibling=theorem]{lemma}
\declaretheorem[style=lembox,name=Proposition,sibling=theorem]{proposition}
\declaretheorem[style=lembox,name=Corollary,sibling=theorem]{corollary}

% definition
\mdfdefinestyle{mddefbox}{
	linewidth= 3pt,
	rightline=false,
	leftline=true,
	topline=false,
	bottomline=false,
	linecolor= LimeGreen,
	backgroundcolor= GreenYellow!25 
}

\declaretheoremstyle[
	postheadspace= \newline,
	spaceabove=6pt, spacebelow=6pt,
	headfont=\normalfont \bfseries,
	notefont=\mdseries, notebraces={(}{)},
	bodyfont=\normalfont,
	postheadspace = \newline,
	mdframed = {style= mddefbox}
]{defbox}


\declaretheorem[style=defbox, name= Definition, sibling=theorem]{definition}

\mdfdefinestyle{mdrembox}{
	linewidth= 2pt,
	rightline=false,
	leftline=true,
	topline=false,
	bottomline=false,
	linecolor= black,
}

\declaretheoremstyle[
	postheadspace= \newline,
	spaceabove=6pt, spacebelow=6pt,
	headfont=\normalfont \bfseries,
	notefont=\mdseries, notebraces={(}{)},
	bodyfont=\normalfont,
	postheadspace = \newline,
	mdframed = {style= mdrembox}
]{rembox}

\declaretheorem[name=Remark,sibling=theorem,style=rembox]{remark}
\declaretheorem[name=Note,sibling=theorem,style=rembox]{note}


% example
\mdfdefinestyle{mdredbox}{
	skipabove=8pt,
	linewidth=3pt,
	rightline=false,
	leftline=true,
	topline=false,
	bottomline=false,
	linecolor=red,
	backgroundcolor=Salmon!5
}

\declaretheoremstyle[
	postheadspace= \newline,
	spaceabove=6pt, spacebelow=6pt,
	headfont=\normalfont \bfseries,
	notefont=\mdseries, notebraces={(}{)},
	bodyfont=\normalfont,
	postheadspace = \newline,
	mdframed = {style= mdredbox}
]{thmredbox}

\mdfdefinestyle{mdgreybox}{
	skipabove=8pt,
	linewidth=3pt,
	rightline=false,
	leftline=true,
	topline=false,
	bottomline=false,
	linecolor=Goldenrod,
	backgroundcolor=Grey!5
}

\declaretheoremstyle[
	postheadspace= \newline,
	spaceabove=6pt, spacebelow=6pt,
	headfont=\normalfont \bfseries,
	notefont=\mdseries, notebraces={(}{)},
	bodyfont=\normalfont,
	postheadspace = \newline,
	mdframed = {style= mdredbox}
]{thmredbox}

\theoremstyle{theorem}
\declaretheorem[name= Exercise, style= thmgreybox, sibling = theorem]{exercise}
\declaretheorem[name= Example, style= thmredbox, sibling = theorem]{example}

%citations
%\usepackage[
%backend=biber,
%style=alphabetic,
%sorting=nyt
%]{biblatex}
%\addbibresource{citations.bib}

\begin{document}

%titlepage

\begin{center}
\begin{minipage}{0.75\linewidth}
    \centering
%University logo
    \includegraphics[width=1\linewidth]{IISc_logo.png}
    \rule{0.4\linewidth}{0.15\linewidth}\par
    \vspace{2cm}
%Thesis title
    {\uppercase{\Large Semester notes\par  }}
    \vspace{2cm}
%Author's name
    {\Large Irish Debbarma\par}
    \vspace{2cm}
%Degree
    {\Large Department of Mathematics \\ Indian Institute of Science, Bangalore\par}
    \vspace{1cm}
%Date
    {\Large December 2022}
\pagenumbering{gobble}   
\end{minipage}
\end{center}
\clearpage

\frontmatter

\chapter{Overview}

These notes probably contain all the basics you will need in number theory. The Elliptic Curves book follows Silverman's Arithmetic of Elliptic Curves, Modular Forms follows Diamond Shurman's A first course in modular forms (it will contain a lot of theory of Riemann surfaces if I manage to do it ofc), Algebraic Geometry I covers range of topics, Basic Algebraic Geometry covers Fulton's Algebraic Curves. On top on that, I am also working to include relevant Commutative algebra, Algebraic Number Theory, Representation theory, Galois theory (finite and infinite). \\

I am trying to make it self contained and accessible to undergraduates, thus whenever you see interlude: it just means it was not covered in class but filled in by me later on to cover the gap of knowledge I had. \\

I will also try to mention external resources used so that original material is kept on record and properly referenced. \\

This section is just a reminder to myself on how to organise and stuff. So, feel free to ignore this part. 

\tableofcontents

\mainmatter

\part{Modular Forms}

\chapter{Lecture-1 (3rd January): Introduction }

\chapter{Lecture-2 (5th January, 2023): Modular Group and Binary quadratic forms}

\section{Modular Group and Upper Half Plane}

The modular group is the group of $2\times 2$ matrices with integer entries and determinant $1$, 
\begin{equation}
\SL_{2}(\ZZ) = \left\{ \begin{pmatrix}
a & b \\ c & d
\end{pmatrix} : a,b,c,d \in \ZZ , ad-bc=1\right\}
\end{equation}

\begin{proposition}[DS, exercise 1.1.1]
The modular group $\SL_{2}(\ZZ)$ is generated by $\ds{S = \begin{pmatrix}
1 & 1 \\ 0 & 1
\end{pmatrix} , T = \begin{pmatrix}
0 & -1 \\ 1 & 0
\end{pmatrix}}$
\end{proposition}

\begin{proof}
Let $\Gamma$ be the subgroup of $\SL_{2}(\ZZ)$ generated by $S,T$. Observe that 
\begin{equation}
S^n = \begin{pmatrix}
1 & n \\ 0 & 1
\end{pmatrix} \text{ for } n \in \ZZ
\end{equation}
Next, take an element $\gamma = \begin{pmatrix}
a & b \\ c & d
\end{pmatrix}$. Then, 
\begin{align*}
\gamma S^n &= \begin{pmatrix}
a & b \\ c & d
\end{pmatrix}\begin{pmatrix}
1 & n \\ 0 & 1
\end{pmatrix} \\
&= \begin{pmatrix}
a & an + b = :b' \\ c & nc +d =: d'
\end{pmatrix} \\
\gamma T &= \begin{pmatrix}
a & b \\ c & d
\end{pmatrix}\begin{pmatrix}
0 & -1 \\ 1 & 0
\end{pmatrix} \\
&= \begin{pmatrix}
b & -a \\ d & -c
\end{pmatrix}
\end{align*}
Suppose $c \neq 0$. If $|d| \geq |c|$, then we can employ Euclidean algorithm to get $d = cq + r$ with $0 \le r < |c|$. If we perform $\gamma S^{-q}$, then $d'$ in the following matrix has the value $r < |c|$. \\

Now, apply $T$ to switch $(c,r)$ to $(r, -c)$ and now we have $|c|>r$ and we can continue the process. Finally after multiplying by $S^{\alpha}$'s and $T$'s we get $\gamma \gamma_{1} \in \SL_{2}(\ZZ)$ with the last row $(0, *)$. \\

This matrix is integral with determinant $1$. Therefore it has the form $\begin{pmatrix}
\pm 1 & m \\ 0 & \pm 1
\end{pmatrix}, m \in \ZZ$. Therefore, the final form is either $S^m$ or $S^{-m}$. Hence, there exists a $\gamma_{1} \in \Gamma$ such that $\gamma \gamma_{1} \in \Gamma \; \forall \; \gamma \in \SL_{2}(\ZZ)$. We can thus conclude that $\gamma \in \SL_{2}(\ZZ)$. Thus, $\Gamma$ is the entire group.
\end{proof}

An element of the modular group can also be viewed as an automorphism of the Riemann surface $\widehat{\CC} =\CC \cup \{\infty\}$ through the fractional linear transformation 
\begin{equation}
\begin{pmatrix}
a & b \\ c & d
\end{pmatrix} = \gamma \cdot z = \frac{az+b}{cz+d} , z\in \widehat{\CC}
\end{equation}
If $c\neq 0$, then $-d/c$ maps to $\infty$ and $\infty$ maps to $a/c$. If $c=0$, then $\infty$ maps to $\infty$. \\

We also note that $I$ and $-I$ give the identity transformation. Moreover, each pair $\pm \gamma$ give the same transformation. The group of transformations of the modular group is determined by the transformations carried out by its generators $S,T$ as $$ z \mapsto z + 1, \;\;\; z \mapsto -1/z$$

\begin{proposition}[DS, Exercise 1.1.2]
\begin{enumerate}
\item $\iM(\gamma(z)) = \iM(z)/|cz+d|^2$ for all $\gamma \in \begin{pmatrix}
a & b \\ c & d
\end{pmatrix} \in \SL_{2}(\ZZ)$
\item $(\gamma \gamma')(z) = \gamma(\gamma'(z)) \; \forall \; \gamma , \gamma' \in \SL_{2}(\ZZ), z \in \Hh$
\item $d\gamma(z)/dz = 1/(cz+d)^2$ for $\gamma = \begin{pmatrix}
a & b \\ c & d
\end{pmatrix} \in \SL_{2}(\ZZ)$
\end{enumerate}
\end{proposition}

\begin{proof}
\begin{enumerate}
\item Observe that 
\begin{align*}
\gamma \cdot z &= \frac{az+b}{cz+d} \\
\iM(\gamma \cdot z) &= \frac{1}{|cz+d|^2} \iM(adz + bc\bar{z})\\
&= \frac{1}{|cz+d|^2} (ad(x+iy) + bc(x-iy))\\
&= \frac{\iM(z)}{|cz+d|^2}
\end{align*}
\item 
\begin{align*}
\gamma \gamma ' &= \begin{pmatrix}
a & b \\ c & d
\end{pmatrix}\begin{pmatrix}
a' & b' \\ c' & d'
\end{pmatrix} \\
\gamma' \cdot z &= \frac{a'z + b'}{c'z+d} \\
\gamma (\gamma ' (z)) &= \gamma \cdot \frac{a'z + b'}{c'z+d} \\
&= \frac{aa'z + ab' + bc'z + bd'}{ca'z + b'c + c'dz + dd'}
\end{align*}

\item \begin{align*}
\frac{d}{dz} \gamma(z) &= \frac{d}{dz} \left( \frac{az +b}{cz+d} \right)\\
&= \frac{a}{cz+d} - \frac{(az+b)c}{(cz+d)^2} \\
&= \frac{acz + ad - acz - bc}{(cz+d)^2} \\
&= \frac{1}{(cz+d)^2}
\end{align*}
\end{enumerate}
\end{proof}

\begin{definition}
Let $k$ be an integer. A meromorphic function $f: \Hh \rightarrow \CC$ is a weakly modular of weight $k$ if 
\begin{equation*}
f(\gamma(z)) = (cz+d)^k f(z) \; \text{ for }\; \gamma = \begin{pmatrix}
a & b \\ c & d
\end{pmatrix}\in \SL_{2}(\ZZ) \; \text{ and } z \in \Hh
\end{equation*}
\end{definition}

\begin{remark}
Since $\SL_{2}(\ZZ)$ is generated by $S,T$, we just need to check the invariance under the action of these two matrices, i.e., that $$f(z +1)=f(z) \; \text{ and } \; f(-1/z) = z^k f(z)$$ for all $z \in \Hh$
\end{remark}

\begin{lemma}
There are no non-zero weakly modular functions of odd weights.
\end{lemma}

\begin{proof}
Let $k$ be odd and $f$ be a weakly modular function of weight $k$. Therefore, $f(-I \cdot z)=(-1)^k f(z) \; \forall \; z\in \Hh$ and thus $f(z)=0 \; \forall \;z \in \Hh$.
\end{proof}

\begin{remark}
Suppose we want a function that is holomorphic on the upper half plane $\Hh$ and $\infty$. Note that $\SL_{2}(\ZZ)$ contains the translation matrix $S : z \mapsto z+1$, so that $f(z+1)=f(z)$ for every weakly modular function $f: \Hh \rightarrow \CC$. That is to say that $f$ is $\ZZ$-periodic. Let $D'$ be the open complex unit disc punctured at the origin. We know that $\ZZ$-periodic holomorphic map $z \mapsto \exp(2\pi i z) = q$ that takes $\Hh$ to $D'$. Thus, corresponding to $f$ we have a function $g: D' \rightarrow \CC$ such that $g(q) = f(\log (q)/ (2 \pi i))$. Then, $g$ is well-defined even though log is defined only upto $2 \pi i \ZZ$ and $f(z)=g(\exp(2 \pi iz))$. If $f$ is holomorphic on the upper half plane, then $g$ is holomorphic on the punctured disc as the log is defined holomorphically around each point and thus $g$ has a Laurent series expansion $g(q) = \sum_{n \in \ZZ} a_{n} q^n , q \in D'$. The relation $|q| = \exp(2 \pi \iM(z))$ shows that $q \rightarrow 0$ if $\iM(z) \rightarrow \infty$. So, thinking of $\infty$ as lying far in the imaginary direction, we define $f$ to be holomorphic at $\infty$ if $g$ extends to a holomorphic function at $q=0$, i.e., the Laurent series sums over $\NN$. This means that $f$ has the Fourier expansion $$f(z) = \sum_{n=0}^{\infty} a_{n}(f)q^n , q = \exp(2 \pi i z)$$

Since $q=0$ iff $\iM(z) \rightarrow \infty$, showing that to check if a weakly holomorphic function is holomorphic at $\infty$ we just need to check whether $\lim_{\iM(z) \rightarrow \infty} f(z)$ exists or even bounded works. 
\end{remark}

\begin{definition}
Let $k$ be an integer. A function $f: \Hh \rightarrow \CC$ is a modular form of weight $k$ if 
\begin{enumerate}
\item $f$ is holomorphic on $\Hh$
\item $f$ is weakly modular of weight $k$
\item $f$ is holomorphic at $\infty$ \\

If in addition, we have 
\item $a_{0}=0$ in the Fourier expansion of $f$, then we say $f$ is a cusp form of weight $k$
\end{enumerate}
We denote the space of modular forms of weight $k$ by $\Mm_{k}(\SL_{2}(\ZZ))$ and the space of cusp forms of weight $k$ by $\Ss_{k}(\SL_{2}(\ZZ))$.
\end{definition}

\begin{proposition}[DS, Exercise 1.1.3]
\begin{enumerate}
\item $\Mm_{k}(\SL_{2}(\ZZ))$ is a vector space over $\CC$
\item If $f$ is a modular form of weight $k$ and $g$ a modular form of weight $\ell$, then $fg$ is a modular form of weight $k \ell$
\item $\Ss_{k}(\SL_{2}(\ZZ))$ is a vector subspace of $\Mm_{k}(\SL_{2}(\ZZ))$ and further, $\Ss_{k}(\SL_{2}(\ZZ))$ is an ideal of $\Mm_{k}(\SL_{2}(\ZZ))$
\end{enumerate}
\end{proposition}

\begin{proof}
Really not in a mood. Someday
\end{proof}

\begin{remark}
The second property in previous proposition gives the space $\Mm_{k} (\SL_{2}(\ZZ))$ a graded structure, therefore 
\begin{equation*}
\Mm_{k} (\SL_{2}(\ZZ)) = \bigoplus_{k \in \ZZ}\Mm_{k} (\SL_{2}(\ZZ))
\end{equation*}
Similarly, using part $3$ of the previous theorem and the above observation, we also have 
\begin{equation*}
\Ss_{k} (\SL_{2}(\ZZ)) = \bigoplus_{k \in \ZZ}\Ss_{k} (\SL_{2}(\ZZ))
\end{equation*}
\end{remark}

\section{Fundamental domain}

\begin{definition}
Let $\Gamma$ be a group acting on $\Hh$. A fundamental domain of $\Gamma$ is a closed subset $\mathcal{D} \subseteq \Hh$ such that 
\begin{enumerate}
\item The set $\mathcal{D}$ is the closure of its interior.
\item Every point in $\Hh$ is $\Gamma$ equivalent to a point in $\mathcal{D}$.
\item If $z,w \in \mathcal{D}$ are $\Gamma$ equivalent, then they lie on the boundary of $\mathcal{D}$.
\end{enumerate}
\end{definition}

\begin{proposition}
The set $$\mathcal{F}_{1} =\{z \in \Hh : |z|>1 , |\mathfrak{R}z|<1/2\}$$ is a fundamental domain of the full modular group $\Gamma_{1}$/
\end{proposition}

\begin{proof}

\end{proof}


\section{Binary Quadratic Forms}

\section{Compactification of $\SL_{2}(\ZZ)$ and cusps}

We can compactify $\SL_{2}(\ZZ) \bs \Hh$ by adding a cusp at $\infty$. 
\begin{align*}
\Hh &\rightarrow D' \\
z &\mapsto q=\exp(2\pi i z)
\end{align*}
Adding a cusp at $\infty$ is the same as adding $q=0$. $\overline{\SL_{2}(\ZZ)\bs \Hh} = \SL_{2}(\ZZ)\bs \Hh \cup \{\infty\}$. \\

The open neighbourhoods of $\infty$ are $$U_{y} = \{z \in \Hh : \mathfrak{I}(z)>y\}$$ with $y>1$. In the disc $D'$, $U_{y}$ goes to $\{0 < q < \exp(-2\pi y)\}$

\begin{definition}
We will call $\PP^1(\QQ)= \QQ \cup \{\infty\}$ the projective rational line. The action of $\gamma \in \SL_{2}(\ZZ)$ is given by the usual fractional transformation $$\gamma \cdot z  = \frac{az+ b}{cz+d}$$
Here, $\gamma \cdot \infty = a/c$ and $\gamma \cdot z = \infty$ if $cz+d=0$
\end{definition}

\begin{exercise}
The two ways to compactify are the same.
\end{exercise}

\begin{proposition}
The action of $\SL_{2}(\ZZ)$ is transitive on $\PP^1(\QQ)$
\end{proposition}

\begin{proof}
Take an element $t\in \QQ$. Write $t = a/c$ in the reduced form, then there exists integers $b,d$ such that $ad-bc=1$. Therefore, we have $\gamma = \begin{pmatrix}
a & b \\ c & d 
\end{pmatrix} \in \SL_{2}(\ZZ)$. Clearly, $\gamma \cdot \infty = t$.
\end{proof}

\begin{remark}
Note that $\SL_{2}(\ZZ)_{\infty} = \{\begin{pmatrix}
a & b \\ c & d 
\end{pmatrix}: \frac{a}{c}=\infty\} = \{\pm\begin{pmatrix}
1 & b \\ 0 & 1 
\end{pmatrix}: b \in \ZZ \}$ \\
Therefore, there is a bijection 
\begin{equation*}
\SL_{2}(\ZZ)/\SL_{2}(\ZZ)_{\infty} \xlongrightarrow{\sim} \PP^1(\QQ)
\end{equation*}
\end{remark}

We want to study stabilisers of each point of $\overline{\mathcal{F}}_{1}$. Let $z \in \overline{\mathcal{F}}_{1}$ and $\SL_{2}(\ZZ) \ni \gamma =\begin{pmatrix}
a & b \\ c & d
\end{pmatrix}  $\\

We want to examine 
\begin{align*}
\gamma \cdot z &= z \\
\frac{az+b}{cz+d} &= z \\ 
cz^2 + (d-a)z - b &=0
\end{align*}
Since $c\neq 0 , z \not \in \QQ$, therefore the discriminant must be negative or $(a+d)^2 - 4 < 0 \Rightarrow |a+d|<2$

\begin{exercise}
$\gamma$ satisfies $X^2 + 1 = 0$ or $X^2 \pm X +1 =0$.
\end{exercise} 

\begin{proposition}
\begin{enumerate}
\item If $z \in \mathcal{F}_{1}$, then $\mathrm{Stab}(z) = \{\pm \mathrm{Id}_{2}\}$
\item If $z = \exp(2 \pi i /3)=: \omega $, then $\mathrm{Stab}(\omega) = \langle \pm ST \rangle $.
\item If $z=i$, then $\mathrm{Stab}(i) = \langle \pm S \rangle $
\end{enumerate}
\end{proposition}

If $k$ is even, then $-\mathbf{Id}_{2}$ acts as identity on a modular form of weight $k$. Let $P \in \overline{\mathcal{F}}_{1}$. We define $$ n_{P}= \begin{cases} 1 & , P \not \sim i, \omega \\ 2 & , P \sim i \\ 3 &, P \sim \omega \end{cases}$$

\chapter{Lecture-3 (10th January, 2023): Valence formula and Eisenstein series}

\section{Valence formula}

\begin{definition}
Let $0 \neq f : \Hh \rightarrow \CC$ be a meromorphic function and $P \in \Hh$. The smallest integer $n$ such that $(z-P)^{-n} f(z)$ is holomorphic and non-vanishing at $P$ is called the order of $f$ at $P$, denoted by $\ord_{P}(f)$. We say that $f$ has a zero of order $n$ if $n$ is positive and pole of order $n$ if $n$ is negative.
\end{definition}


Recall that $M_{k}(\Gamma_{1})$ is the space of modular forms of weight $k$ and level $1$. It is also a vector space over $\CC$. 

\begin{theorem}
$\dim \Mm_{k}(\Gamma_{1}) = \begin{cases}[k/12]+1 & k\not \equiv 2\pmod{12} \\ [k/12] & k \equiv 2\pmod{12} \end{cases}$
\end{theorem}

\begin{lemma}
Let $0 \neq f$ be a modular function. Then there exists a $R < \infty$ such that $f$ is holomorphic and non-vanishing on $\mathfrak{I}z >R$
\end{lemma}

\begin{proof}
Since $f$ is holomorphic at $\infty$ therefore $g$ is holomorphic in the region $\{z \in \CC : 0 < |z| < R''\}$ for some $R>0$. Since $f$ is non-zero, therefore $g$ is also non-zero and cannot be an accumulation point of zeroes of $g$. Thus, there exists $R' >0$ such that $g$ is non-zero and holomorphic in the region $\{z \in \CC : 0 < |z| < R'\}$. This implies $f = g \circ q$ is holomorphic and non-zero in $\Hh$ if $|q(z)| < R' \Leftrightarrow |\exp(2\pi i z)| < R' \Leftrightarrow \exp(-2\pi \mathfrak{I}z) < R' \Leftrightarrow -2\pi \mathfrak{I}z < \log R' \Leftrightarrow \mathfrak{I}z > \frac{1}{2\pi} \log(1/R'):=R$
\end{proof}

\begin{proposition}
Let $f \in \Mm_{k}(\Gamma_{1})$. Then, $$\sum_{p \in \Gamma_{1}\backslash \Hh,\\ p\neq i,\omega}\mathrm{ord}_{p}(f) +  \frac{1}{2} \ord_{i}(f) + \frac{1}{3} \ord_{\omega}(f) + \mathrm{ord}_{\infty}(f) = \frac{k}{12}$$
Or, more generally $$\sum_{P \in \Gamma_{1}\bs \Hh} \frac{1}{n_{P}} \ord_{P}(f) + \ord_{\infty}(f) = \frac{k}{12}$$
\end{proposition}

\begin{proof}
Consider the contour $\mathcal{C}$ as follows: 

We just consider the case where $\mathcal{F}_{1}$ contains all its zeroes and poles inside the contour and 
\end{proof}

\begin{corollary}
$\dim \Mm_{k}(\Gamma_{1}) = \begin{cases} 0 & k < 0 \\ 0 & k \text{ is odd } \\ 1 & k=0 \\ \begin{cases}[k/12]+1 & k\not \equiv 2\pmod{12} \\ [k/12] & k \equiv 2\pmod{12} \end{cases} \end{cases}$
\end{corollary}

\begin{proof}
\begin{itemize}
\item If $k < 0$, then LHS is $>0$ but RHS is $<0$. A contradiction.
\item We have already seen
\item For $k=0$, take $f \in \Mm_{0}(\Gamma_{1})$. Clearly, $\Mm_{0}(\Gamma_{1}) \ni f(\infty)=c \Rightarrow f-c \in \Mm_{0}(\Gamma_{1})$. Since, the LHS is $>0$ and RHS $\equiv 0$ therefore $f \equiv c$.
\item Let $m = [k/12] + 1$ and take $f_{1}, \hdots , f_{m+1} \in \Mm_{k}(\Gamma_{1})$. If $P_{1}, \hdots , P_{m}$ are points in $\mathcal{F}_{1}$ not equal to $i, \omega , \omega + 1$ and consider $\{f_{i}(P_{j})\}_{1\le i \le m+1 , 1\le j \le m}$. \\
Next, consider the linear combination $f = \ds{\sum_{i=1}^{m+1}c_{i}f_{i}}$ with not all $c_{i}$s zero so that $f(P_{j})=0$ for $1 \le j \le m$. From the previous theorem, this means $f \equiv 0$ and hence $\{f_{i}\}$ is linearly independent and hence $\dim_{\CC}\Mm_{k}(\Gamma_{1}) \le m$. \\
For $k \equiv 2 \pmod{12}$, the relation holds only if there is atleast a simple pole at $i$ and a double zero at $\omega$. Therefore, 
\begin{align*}
\frac{k}{12} -\frac{1}{6} &= m-1
\end{align*} 
Therefore, we can repeat the argument.
\end{itemize}
\end{proof}

\begin{remark}
The above result only gives us an upper bound. To get lower bound, we will need a bit more machinery.
\end{remark}

\begin{theorem}
Let $f$ be a modular form of weight $k$ and level $1$ with $q$-expansion $\sum_{n=1}^\infty a_{n}q^n$. Suppose $$a_{n}=0 \; \text{ for } n =0, \hdots , [k/12]$$ then $f=0$.
\end{theorem}

\begin{proof}
Suppose $f\neq 0$. By our assumption, $\ord_{\infty}(f) \geq [k/12] + 1 > k/12$. By the valence formula, the LHS $> k/12$ which is a contradiction. Therefore $f=0$.
\end{proof}

\begin{corollary}
Suppose $f,g$ are modular forms of weight $k$, level $1$ with $q$ expansions $f(q)=\sum_{n=1}^\infty a_{n}q^n, g(q)=\sum_{n=1}^\infty b_{n}q^n$. If $$a_{n}=b_{n} \; \text{ for } n =0, \hdots , [k/12]$$ then $f=g$.
\end{corollary}


A slight notation. For $\gamma  = \begin{pmatrix}
a & b \\ c & d
\end{pmatrix} \in \mathrm{SL}_{2} (\ZZ)$ we set $f|_{\gamma} (z) = (cz+d)^{-k}f(\gamma \cdot z)$. \\
Thus, $1|_{\gamma}(z) = (cz+d)^{-k}$. If $1|_{\gamma}(z) =1 \Rightarrow c =0$. Conversely, if $c=0$, then $d^{-k}=1$. So, $1|_{\gamma}(z)=1 \Leftrightarrow c=0$. \\
$\Gamma_{\infty} = \left\{ \begin{pmatrix}
a & b \\ 0 & d 
\end{pmatrix} \in \mathrm{SL}_{2}(\ZZ) \right\} = \mathrm{stab}(\infty)$

\section{Eisenstein series}

\begin{definition}
The Eisenstein series $E_{k}(z)$ is defined to be $$E_{k}(z) = \sum_{\gamma \in \Gamma_{\infty} \backslash \Gamma_{1}} 1|_{\gamma}(z)$$
\end{definition}

\begin{proposition}
$$E_{k}(z) = \frac{1}{2} \sum_{(c,d) \in \ZZ^{2} \backslash \{(0,0)\}, \gcd(c,d)=1} \frac{1}{(cz+d)^k}$$
\end{proposition}

\begin{proof}
\begin{align*}
E_{k}(z) &= \sum_{\gamma \in \Gamma_{\infty} \backslash \Gamma_{1}} 1|_{\gamma}(z) \\
&= \sum_{\gamma \in \overline{\Gamma}_{\infty} \backslash \overline{\Gamma_{1}}} 1|_{\gamma}(z) \\
\end{align*}
If $(c,d)=1$ then there exists $a,b \in \ZZ$ such that $ad-bc=1 \rightarrow \begin{pmatrix}
a & b \\ c & d
\end{pmatrix} \in \SL_{2}(\ZZ)$. And, conversely if $\begin{pmatrix}
a & b \\ c & d
\end{pmatrix} \in \SL_{2}(\ZZ)$ then $(c,d)=1$. \\

If $\begin{pmatrix}
a' & b' \\ c & d
\end{pmatrix} \in \SL_{2}(\ZZ)$, then we want to show that there is an $\eta \in \Gamma_{\infty}$ such that $\eta\begin{pmatrix}
a & b \\ c & d
\end{pmatrix} \in \SL_{2}(\ZZ) = \begin{pmatrix}
a' & b' \\ c & d
\end{pmatrix} \in \SL_{2}(\ZZ)$\\

Note that $ad-bc=1=a'd-b'c \Rightarrow (a-a')d=(b-b')c \Rightarrow (a-a')=nc, b-b'=nd$. Thus, $\begin{pmatrix}
a & b \\ c & d
\end{pmatrix} = \begin{pmatrix}
a' + nc & b' + nd \\ c & d
\end{pmatrix} = S^n \begin{pmatrix}
a' & b' \\ c & d
\end{pmatrix}$
But, $T^n \in \Gamma_{\infty}$. This completes the proof.
\end{proof}

\begin{proposition}[DS, Exercise 1.1.4]
$$\sum_{(c,d) \in \ZZ^{2} \backslash \{(0,0)\}, \gcd(c,d)=1} \frac{1}{(cz+d)^k}$$ converges absolutely for $k>2$
\end{proposition}

\begin{proof}
\begin{itemize}
\item Consider the sum $\ds{\sum_{(a,b) \in \ZZ^2 - \{(0,0)\}} \sup \{|a|, |b|\}^{-k}}$. We shall prove that this sum converges.\\
First, we can sum this over the first quadrant only and quadruple the value later. So, let us focus on $$\sum_{(a,b) \in \ZZ_{>0}^2 - \{(0,0)\}} \sup \{|a|, |b|\}^{-k}$$
Look at the partial sums: 
\begin{align*}
\sum_{(a,b) \in \ZZ_{>0,le N }^2 - \{(0,0)\}} \sup \{|a|, |b|\}^{-k} &\le \sum N^{-k} = \frac{N^2-1}{N^k}
\end{align*}
\item Next, consider $\Omega = \{\tau \in \Hh : |\mathfrak{R}\tau|\le A, \mathfrak{I}\tau \geq B\}$.\\
\textbf{Claim}: $\exists c >0$ such that $|\tau + \delta|>C \sup\{1,|\delta|\}$
\begin{proof}
For $|\delta|<1$, $|\tau + \delta| > |\tau|> \mathfrak{I}\tau \geq B$. \\
For $|\delta|>2A$, $|\tau + \delta| > |\mathfrak{R}\tau + \delta|\geq |\delta |- |\mathfrak{R}\tau| \geq \delta - A \Rightarrow |\tau + \delta|\geq |\delta|-A > |\delta|/2 = 1/2 \sup\{1, |\delta|\}$. Therefore $C = \min\{1/2,B\}$ works.\\
For $1<|\delta|<2A$ and $\mathfrak{I}\tau \geq A$, $|\tau + \delta|>A > |\delta|/2$\\
For $1<|\delta|<2A$ and $B \le \mathfrak{I}\tau \geq A$, $|\tau + \delta|/|\delta|$ has a minimum $m$. Therefore, $|\tau + \delta| \geq m |\delta|$.
\end{proof}
\item Now, using the above two observations, we have 
\begin{align*}
E_{k}(\tau) &= \sum' \frac{1}{(c\tau +d)^k} \\
&= \sum_{d} \frac{1}{d^k} + \sum_{c\neq 0, d} \frac{1}{(c\tau +d)^k}\\
&= 2\zeta(k) + \sum_{c\neq 0, d} \frac{1}{(c\tau +d)^k}\\
\end{align*}
Now, $|c\tau +d| = |c||\tau + \delta|> |c|C_{1}\sup\{1, |\delta|\}$
\begin{align*}
\therefore \frac{1}{C_{1}^k} \sum_{c\neq 0, d} \frac{1}{(|c|^k\sup\{1, |\delta|\}^k}\\
\end{align*} converges for $\tau \in \Omega$ and since any compact set in $\Hh$ sits inside a suitable $\Omega$, our claim is proven.
\end{itemize}
\end{proof}

\begin{theorem}[DS, Exercise 1.1.4]
$E_{k}(z) \in \Mm_{k}(\Gamma_{1})$ for $k >2$.
\end{theorem}

\begin{proof}
Clearly, $E_{k}(\tau +1) = E_{k}(\tau)$ and $E_{k}(-1/\tau) = \tau^k E_{k}(\tau)$. Therefore, it is weakly modular of weight $k$ and level $1$. Previous proposition says that it is holomorphic on $\Hh$. We just need to check holomorphicity at $\infty$. \\
Let $\tau = i \nu$ and observe the sum as $\nu \rightarrow \infty$. Then, the sum is determined by the behaviour of $\sum_{d}d^{-k}$ and since $k\geq 3$, this sum is convergent and hence the sum is bounded and we are done. It is indeed a modular form of weight $k$ and level $1$.
\end{proof}

\begin{proposition}
$E_{k}(z) \not \equiv 0$ for $k>2$, even.
\end{proposition}

\begin{proof}
First of all notice  that $\begin{pmatrix}
a & b \\ c & d
\end{pmatrix} (m,n) = (am + bn,cm+dn) , a,b,c,d \in \ZZ$. Since the matrix is invertible, multiplication by $\gamma \in \SL_{2}(\ZZ)$ is a bijection from $\ZZ^2 - \{(0,0)\}$ to $\ZZ^2 - \{(0,0)\}$. \\

Next, observe that $$\frac{1}{(cz+d)^k} \rightarrow 0 , \mathfrak{I}(z) \rightarrow \infty , c\neq 0$$ and if $c = 0$, then $c=\pm 1$. Hence, $E_{k}(z) = 1 +$ bounded term as $\mathfrak{I}(z) \rightarrow \infty$. This implies $E_{k}(z) \not \equiv 0$ and $$E_{k}(z) = 1 + \sum_{n=1}^{\infty} a_{n}e^{2 \pi i z}$$
\end{proof}

Another way of looking at Eisenstein series is a function on a lattice. \\
Consider $G_{k}(z) = G_{k}(\ZZ z + \ZZ) = \frac{1}{2} \sum_{(c,d) \in \ZZ^2}^{'} \frac{1}{(cz+d)^k}$

\begin{proposition}
$G_{k}(z)$ converges absolutely for $k>2$.
\end{proposition} 

\begin{proof}
Same as above.
\end{proof}

\begin{proposition}
$G_{k}(z) = \zeta(k) E_{k}(z)$
\end{proposition}

\begin{proof}
\begin{align*}
G_{k}(z) &= \sum'_{(c,d) \in \ZZ^2} \frac{1}{(cz+d)^k} \\
&= \sum'_{(c,d)\in \ZZ^2, \gcd(c',d')=1} \frac{1}{\gcd(c,d)^k}\cdot \frac{1}{(c'z + d')^k} \\
&= \zeta(z)E_{k}(z)
\end{align*}
\end{proof}

\begin{proposition}
$\mathbb{G}_{k}(z) = \frac{(k-1)!}{(2 \pi i)^k} G_{k}(z) = -\frac{B_{k}}{2k} + \sum_{n=1}^{\infty} \sigma_{k-1}(n)q^n$ for $k>2$, even.
\end{proposition}

\chapter{Lecture-4 (12th January, 2023): Eisenstein series}

\section{Eisenstein series contd..}

Recall that $$\Mm(\Gamma_{1}) = \bigoplus_{k \in \ZZ}\Mm_{k}(\Gamma_{1})$$  is a graded ring.

\begin{proposition}
The graded ring $\Mm(\Gamma_{1})$ is freely generated by $E_{4},E_{6}$. This means that the map 
\begin{align*}
f: C[X,Y] &\rightarrow \Mm(\Gamma_{1})\\
X &\mapsto E_{4} \\
Y &\mapsto E_{6}
\end{align*}
is an isomorphism of graded rings. Here, $\deg X = 4, \deg Y=6$. 
\end{proposition}

\begin{proof}
We want to show that $E_{4}$ and $E_{6}$ are algebraically independent. We start by showing that $E_{4}^3$ and $E_{6}^2$ are linearly independent over $\CC$. Suppose $E_{6}(z)^2 = \lambda E_{4}(z)^{3}$. Consider $f(z) = E_{6}(z)/ E_{4}(z)$. Now observe that $f(z)^2 = \lambda E_{4}(z)$. This means that $f^2$ is holomorphic and thus $f$ is also holomorphic. But $f$ is weakly modular of weight $2$ which is a contradiction. So, our claim is proven. \\

\textbf{Claim}: Let $f_{1},f_{2}$ be two nonzero modular forms of same weight. If $f_{1},f_{2}$ are linearly independent, then they are algebraically independent as well. \\

Let $P(t_{1},t_{2}) \in \CC[t_{1},t_{2}] \backslash \{0\}$ be such that $P(f_{1},f_{2})=0$. Let $P_{d}(t_{1},t_{2})$ be the $d$ degree parts of $P$. Using the fact that modular forms of different weights are linearly independent, we get that $P_{d}(f_{1},f_{2})=0 \; \forall \; d \geq 0$. If $p_{d}(t_{1}/t_{2}) = P_{d}(t_{1},t_{2})/t_{2}^d$, then $p_{d}(f_{1}/f_{2})=0$. But this means that $f_{1}/f_{2}$ is a constant. But, $f_{1},f_{2}$ are linearly independent which implies that they are algebraically independent as well. \\

All of this implies that $E_{4},E_{6}$ are algebraically independent.
\end{proof}

\begin{corollary}
$\dim_{\CC} M_{k} (\Gamma_{1}) \geq \begin{cases}[k/12]+1 & k\not \equiv 2\pmod{12} \\ [k/12] & k \equiv 2\pmod{12} \end{cases} $
\end{corollary}

\begin{proof}
By the previous proposition, $E_{4}^aE_{6}^b$ is a basis for the space of modular forms. Clearly, $\{E_{4}^aE_{6}^b: \ZZ \ni a,b , 4a+6b=k\}$ forms a basis for $\Mm_{k}(\Gamma_{1})$. We just have to compute the size of this set. That is we want to find the size of the set $\{(a,b) \in \ZZ_{\geq 0} : 4a + 6b = k\}$. Since $k$ is also even, let us divide by $2$ and reduce to the set $\{(a,b) \in \ZZ_{\geq 0} : 2a + 3b = k\}$ with $k=6m + n$ and $k,n \in \{0,1,2,3,4,5\}$. We wish to show that $$\#\{(a,b) \in \ZZ_{\geq 0} : 2a + 3b = k\} = \begin{cases}m+1 & k\not \equiv 1\pmod{6} \\ m & k \equiv 1\pmod{6} \end{cases}$$
\begin{itemize}
\item $n=1$, then we have $2a = 6m + (1-3b) \geq 0 \Rightarrow 0 \le b \le 1/3 + 2m$. And, moreover $b$ has to be odd and hence we have $m$ choices.
\item $n=0$, then we have $2a = 6m -3b \geq 0 \Rightarrow 0 \le b \le 2m $ and $b$ is even therefore $m+1$ choices.
\end{itemize}
Similarly for the others.
\end{proof}

This completes the proof of the theorem 
\begin{theorem}
$\dim_{\CC} M_{k} (\Gamma_{1}) = \begin{cases}[k/12]+1 & k\not \equiv 2\pmod{12} \\ [k/12] & k \equiv 2\pmod{12} \end{cases} $
\end{theorem}



\subsection{Fourier expansions of $E_{k}(z)$}

\begin{proposition}[DS, Exercise 1.1.7]
$$\mathbb{G}_{k}(z) = \frac{(k-1)!}{(2 \pi i)^k} G_{k}(z) = -\frac{B_{k}}{2k} + \sum_{n=1}^{\infty} \sigma_{k-1}(n)q^n$$ for $k>2$, even and $B_{k}$ are Bernoulli numbers.
\end{proposition}

\begin{proof}
Use $$\frac{\pi}{\tan \pi z} = \sum_{n \in \ZZ} \frac{1}{z+n} = \lim_{M,N \rightarrow \infty , N-M < \infty} \sum_{-M}^{N} \frac{1}{z+n}$$ and 
$$\frac{\pi}{\tan \pi z} = \frac{\pi \cos \pi z}{\sin \pi z} = \pi i \frac{e^{\pi i z} + e^{-\pi i z}}{e^{\pi i z} - e^{-\pi i z}} = - \pi i \frac{1+q}{1-q} = - 2 \pi i \left( \frac{1}{2} + \sum_{r=1}^{\infty} q^r \right)$$ This leads to the equality 
$$\sum_{n\in \ZZ}\frac{1}{z+n} = -  2 \pi i \left( \frac{1}{2} + \sum_{r=1}^{\infty} q^r \right)$$
Differentiate both sides of equality $k-1$ times and divide by $(k-1)!$ to get $$\sum_{n \in \ZZ}\frac{1}{(z+n)^k} = \frac{(-2 \pi i)^k}{(k-1)!} \sum_{r=1}^{\infty} r^{k-1}q^r$$
Next, if we look at 
\begin{align*}
G_{k}(z) &= \frac{1}{2} \sum' \frac{1}{(mz+n)^k} \\
&= \frac{1}{2} \sum_{n \in \ZZ , n \neq 0} \frac{1}{n^k} + \frac{1}{2} \sum_{(m,n) \in \ZZ^2 , m \neq 0} \frac{1}{(mz + n)^k} \\
&= \zeta(k) + \sum_{m=1}^{\infty} \sum_{n=-\infty}^{\infty} \frac{1}{(mz+n)^k} \\
&= \zeta(k) + \frac{(2 \pi i)^k}{(k-1)!} \sum_{m=1}^{\infty} \sum_{r=1}^{\infty} r^{k-1} q^{mr} \\
&= \zeta(k) + \frac{(2 \pi i)^k}{(k-1)!} \sum_{m=1}^{\infty} \sum_{r=1}^{\infty} \sigma_{k-1}(n) q^{n}
\end{align*}
The expression of $\mathbb{G}_{k}(z)$ is trivial after noting $$\frac{(k-1)!}{(2 \pi i)^k} \zeta(k) = B_{k}$$
\end{proof} 

\begin{remark}[DS, Exercise 1.1.5]
\begin{align*}
\cot \pi \tau &= \frac{\cos \pi \tau}{\sin \pi \tau} \\
&= i \frac{e^{i \pi \tau } + e^{-i \pi \tau }}{e^{i \pi \tau} - e^{-i \pi \tau}} \\
&= i \frac{q+1}{q-1} \\
&= i + i \frac{2}{q-1}\\
&= i - 2i \sum_{m=0}^{\infty} q^m\\
\therefore \pi \cot \pi \tau &= \pi i - 2\pi i \sum_{m=0}^{\infty} q^m
\end{align*}
And, 
\begin{align*}
\sin \pi \tau &= \pi \tau \prod_{j=1}^{\infty} \left( 1 - \frac{\tau^2}{j^2} \right) \\
\log \sin \pi \tau &= \log \pi \tau + \sum_{j=1}^{\infty} \log \left( 1 - \frac{\tau^2}{j^2} \right) \\
\pi \frac{\cos \pi \tau }{\sin \pi \tau} &= \frac{1}{\tau} + \sum_{j=1}^{\infty} \frac{2 \tau}{1- \frac{\tau^2}{j^2}} \\
\pi \cot \pi \tau &= \frac{1}{\tau} + \sum_{j=1}^{\infty} \frac{\tau + j + \tau - j}{(\tau - j)(\tau + j)} \\
&= \frac{1}{\tau} + \sum_{d=1}^{\infty} \frac{1}{\tau + d} + \frac{1}{\tau -d}
\end{align*}
\end{remark}

\begin{remark}
\begin{enumerate}
\item $\mathbb{G}_{4}(z) = \frac{1}{240} + q + 9q^2 + 28q^3 + 73q^4 + \cdots $
\item $\mathbb{G}_{6}(z) = -\frac{1}{504} + q + 33q^2 + 244q^3 + \cdots $
\item $\mathbb{G}_{8}(z) = \frac{1}{480} + q + 129q^2 + 2188q^3 + \cdots $
\end{enumerate}
\end{remark}

\begin{proposition}[DS, Exercise 1.1.7]
$$\sum_{m=1}^{n-1} \sigma_{3}(m) \sigma_{3}(n-m) = \frac{\sigma_{7}(n) - \sigma_{3}(n)}{120}$$
\end{proposition}

\begin{proof}
Look at the space $\Mm_{8}(\Gamma_{1})$. This is $1$-dimensional. Therefore, $E_{4}^2$ and $E_{8}$ by virtue of being modular forms of weight $8$ must be linearly dependent. Thus, $E_{8}(\tau) = \lambda E_{4}^{2}(\tau)$. But, both have the same constant term and hence equal. After that, it is just comparing coefficients of the Fourier expansion.
\end{proof}

\subsection{Weight $2$ Eisenstein series}

\begin{definition}
\begin{align*}
\mathbb{G}_{2}(z) &= -\frac{1}{24} + \sum_{n=1}^{\infty} \sigma_{1}(n)q^n \\
&= -\frac{1}{24} + q + 3q^2 + 4q^3 + 7q^4 + \cdots 
\end{align*} 
\end{definition}

This converges rapidly on $\Hh$ and defines a holomorphic function. 

\begin{proposition}
$$G_{2}(z) = - 4 \pi^2 \mathbb{G}_{2}(z)$$
\end{proposition}

\begin{proof}
Since we know that $$G_{2}(z) = \sum_{(m,n) \in \ZZ^2 \backslash \{(0,0)\}}\frac{1}{(mz+n)^2}$$ does not converge absolutely, we define $$G_{2}(z) = \frac{1}{2} \sum_{n \in \ZZ , n \neq 0} \frac{1}{n^2} + \frac{1}{2} \sum_{m \neq 0} \sum_{n \in \ZZ} \frac{1}{(mz+n)^2}$$
This sum converges absolutely and we can show that this satisfies the functional equation as required.
\begin{lemma}

\end{lemma}

\begin{proof}

\end{proof}

\end{proof}

\begin{proposition}
For $\gamma = \begin{pmatrix}
a & b \\ c & d 
\end{pmatrix} \in \mathrm{SL}_{2}(\ZZ)$ we have $$G_{2}\left( \frac{az+b}{cz+d} \right) = (cz+d)^2 G_{2}(z) - \pi i c(cz+d)$$
\end{proposition}

$G_{2}$ is called a quasi modular form. \\

Introduce (due to Hecke): $$G_{2,s}(z) = \frac{1}{2} \sum_{(m,n)\in \ZZ^2 \backslash \{(0,0)\}} \frac{1}{(mz+n)^2 |mz+n|^{2s}} , \mathfrak{R}(s) >0$$

\section{Modular forms of higher level}

Let $N \in \ZZ_{\geq 1}$ 

$$\mathrm{SL}_{2}(\ZZ / N \ZZ) =\left\{ \begin{pmatrix}
a & b \\ c & d
\end{pmatrix} \in M_{2}(\ZZ / N \ZZ) \mid ad-bc \equiv 1 \pmod{N} \right\}$$

\begin{lemma}[DS, Exercise 1.2.2]
The map 
\begin{align*}
\mathrm{SL}_{2}(\ZZ) &\rightarrow \mathrm{SL}_{2}(\ZZ / N \ZZ) \\
\begin{pmatrix}
a & b \\ c & d
\end{pmatrix}  &\mapsto \begin{pmatrix}
\bar{a} & \bar{b} \\ \bar{c} & \bar{d}
\end{pmatrix}
\end{align*}
is a surjective group homomorphism.
\end{lemma}

\begin{proof}
\begin{itemize}
\item $\gcd(c,d,N)=1$. Indeed, if $\gcd(c,d,N)=g$, then $g \mid (ad-bc) =1\Rightarrow g\mid 1 \Rightarrow g =1$. 
\item Now, write $c=c_{1}c_{2}$ such that $\gcd(c,N)=c_{2}$. Then, $\gcd(c_{1},N)=1$. For, suppose if $p \mid N $, then $p \mid c_{2}$ necessarily. Now, there exists integers $r,s \in  \ZZ$ such that $1= c_{1}r + Ns \Rightarrow 1 \equiv Ns \pmod{c_{1}}$
\item Let $c'=c$ and $d'=d+mN$ where $m=s(1-d)$. Clearly, $d +mN \equiv 1\pmod{c_{1}} $. We claim that $\gcd(c',d')=1$. Indeed, if $p \mid c' , d' \Rightarrow p \mid c,d+mN$. Since $c=c_{1}c_{2}$ we have $p \mid c_{1}$ or $p \mid c_{2}$. If $p \mid c_{1}$, then we get a contradiction to $d + mN \equiv 1 \pmod{c_{1}}$. Hence, $p \mid c_{2}$ which then implies that $p \mid N \Rightarrow p \mid \gcd(c,d,N)$ a contradiction. 
\item Now, we can lift $\gamma$ to $\begin{pmatrix}
a + kN & b + l N \\
c' & d'
\end{pmatrix} \in \SL_{2}(\ZZ)$
\end{itemize}
Finally, we can conclude that this map is surjective.
\end{proof}

\begin{definition}
$$\Gamma (N) = \ker (\mathrm{SL}_{2}(\ZZ) \rightarrow \mathrm{SL}_{2}(\ZZ / N \ZZ))$$ is called the principal congruence subgroup.
\end{definition}

\begin{definition}
A subgroup $\Gamma$ of $\mathrm{SL}_{2}(\ZZ)$ is called a congruence subgroup if there exists $N$ such that $\Gamma (N) \subseteq \Gamma$.
\end{definition}

$$\Gamma_{0}(N) =\left\{ \begin{pmatrix}
a & b \\ c & d
\end{pmatrix} \in M_{2}(\ZZ / N \ZZ) \mid c\equiv 0 \pmod{N} \right\}$$
$$\Gamma_{1}(N) =\left\{ \begin{pmatrix}
a & b \\ c & d
\end{pmatrix} \in M_{2}(\ZZ / N \ZZ) \mid a\equiv d \equiv 1 \pmod{N} \right\}$$


\chapter{Lecture-5 (17th January, 2023): Congruence subgroups and $\Delta$ function}

\section{$\Delta$ function}

Consider $$\Delta (z) = \frac{1}{1729} (E_{4}^3(z) - E_{6}^2(z)) = q + q^2() + \cdots $$
Clearly, $\Delta(z)$ is a normalised cusp form of weight $12$ and level $1$. 

\begin{theorem}
$$\Delta (z) = q \prod_{n \geq 1} (1-q^{n})^{24}, q=e^{2\pi i z}$$
\end{theorem}

\begin{proposition}
$\Delta(z)$ has no zero in $\Hh$.
\end{proposition}

\begin{proof}
From the valence formula we have $$\sum_{p \in \Hh} \frac{1}{n_{p}} \ord_{p}(\Delta(z)) + \ord_{\infty} (\Delta(z)) = k/12 =1$$ Moreover, $\ord_{\infty} (\Delta(z))=1$. Hence, we can conclude that $\ord_{p}(\Delta(z)) =0 \; \forall \; p \in \Hh$.
\end{proof}

\textbf{Application}: We use $\Delta(z)$ to write any modular form as a polynomial in $E_{4},E_{6}$. \\

Take $f(z) \in M_{k} (\Gamma_{1})$ with $4a+6b , k \geq 4, a,b\geq 0$. The Fourier expansion of $f(z)$ can we written as $$f(z) = a_{0} + a_{1}q + \cdots $$
Clearly, $f(z) - a_{0}E_{4}^a(z)E_{6}^b(z) \in \Mm_{k}(\Gamma_{1}) \subseteq \Ss_{k}(\Gamma_{1})$.

Next, $$h(z) = \frac{f(z)-a_{0}E_{4}^a(z)E_{6}^b(z)}{\Delta(z)} \in \Mm_{k-12}(\Gamma_{1})$$ Recursively, we can now find expression for $f(z)$. 

\begin{proposition}
$$j(z) = \frac{E_{4}^3}{\Delta(z)} = q^{-1} + \cdots $$
\begin{align*}
j: \Gamma_{1}\bs \bar{\Hh} &\rightarrow \PP^1(\CC) \\
z &\mapsto j(z)
\end{align*}

is a bijection.
\end{proposition}

\begin{proof}
$E_{4}^3(z)$ and $\Delta(z)$ are linearly independent. For any $\lambda_{1},\lambda_{2} \in \CC$ both not zero, $\lambda_{1}E_{4}^3(z) + \lambda_{2}\Delta(z)$ has an unique zero in $\Gamma_{1}\bs \bar{\Hh}$. 
\end{proof}

\begin{remark}
This $j$ is called the $j$-invariant modular function. It attaches an elliptic curve in $\PP^1 (\CC)$ to any lattice in $\Lambda_{z} = \ZZ z + \ZZ$ and vice versa.
\end{remark}

Next, the Fourier series of $\Delta(z)$ is of the form $\Delta(z) = \sum_{n \geq 1} \tau(n)q^n$ where $\tau(n)$ satisfies the following properties: 
\begin{enumerate}
\item $\tau(pq) =\tau(p)\tau(q)$ if $p,q$ are dinstinct primes. 
\item $\tau(p^2) = \tau(p)^2 - p^{12-1}$.
\item $|\tau(p)| \le 2 p^{\frac{12-1}{2}}$.
\item 
\begin{align*}
\mathbb{G}_{12}(z) &= \Delta(z) + \frac{691}{156} \left( \frac{E_{4}^3(z)}{720} + \frac{E_{6}^2}{1008} \right) \\
\mathbb{G}_{12} &= -\frac{B_{12}}{24} + \sum_{n \geq 1} \sigma_{11}(n) q^n \\
&= \frac{691}{65520} + \sum_{n \geq 1} \sigma_{11}(n) q^n \\ 
\mathbb{G}_{12}(z) &\equiv \Delta(z) \pmod{691}
\end{align*}
To conclude $$\tau(n) = \sigma_{11}(n) \pmod{691}$$
(Related to the fact that $691 \mid \# \mathcal{C}(\QQ(\gamma_{691}))$)
\end{enumerate}

\section{Congruence subgroup}

\begin{proposition}
Let $N=p_{1}^{a_{1}}\cdots p_{r}^{a_{r}}$ be the prime factorisation. Then, $$\mathrm{SL}_{2} (\ZZ / N \ZZ) = \prod_{i=1}^{r} \mathrm{SL}_{2}(\ZZ / p^{a_{i}} \ZZ) $$ 
\end{proposition}

\begin{proof}
By CRT, we have $\ZZ/N\ZZ = \prod_{i=1}^{r} \ZZ/p^{a_{i}}\ZZ$. This completes the proof. 
\end{proof}

\begin{lemma}[DS, Exercise 1.2.3]
$$\# \mathrm{SL}_{2}(\ZZ / N \ZZ) = N^3 \prod_{p \mid N} \left( 1 - \frac{1}{p^2} \right)$$
\end{lemma}

\begin{proof}
\begin{itemize}
\item First, we need to find $\#\SL_{2}(\ZZ/p^e \ZZ)$. We will prove $\#\SL_{2}(\ZZ/p^e \ZZ) = p^{3e}(1-1/p^2)$ by induction.
\item $n=1 \Rightarrow \#\SL_{2}(\ZZ/p\ZZ) = p^3(1-1/p^2) = p(p^2-1)$
\item Suppose assertion is true for $n=k-1$
\item The map $\SL_{2}(\ZZ/p^k \ZZ) \xlongrightarrow{\phi} \SL_{2}(\ZZ/p^{k-1}\ZZ)$ be the surjective reduction map. 
\item The kernel of $\phi$ is $\{\begin{pmatrix}
a & b \\ c&d
\end{pmatrix}\in \SL_{2}(\ZZ/p^{k}\ZZ) : a,d \equiv 1 \pmod{p^{k-1}}, b,c \equiv 0 \pmod{p^{k-1}} \}$
\item Let $a =up^{k-1} + 1 , b = vp^{k-1} + 1 , c = rp^{n-1}, d=sp^{k-1}$
\item Then,
\begin{align*}
ad-bc &= usp^{2k-2} + up^{k-1} + sp^{k-1} + 1 - vrp^{2k-2} \\
&\equiv (u+s)p^{k-1} + 1 \pmod{p^k}\\
&\equiv 1 \pmod{p^k}\\
(u+s)p^{k-1} &\equiv 0 \pmod{p^{k}}
\end{align*}
This means $p \mid u+s$
\item Now, $(u,s)$ can be chosen in $p$ many ways and $v,r$ can be chosen in $p^2$ many ways.
\item To conclude, $\#\SL_{2}(\ZZ/p^k \ZZ) = p^3 p^{3(k-1)-2} (p^2 -1) = p^{3k-2}(p^2 -1)$
\item Now, using the previous proposition we can conclude what we want.
\end{itemize}
\end{proof}

\begin{remark}[DS, Exercise 1.2.3]
\begin{enumerate}
\item Consider the map $\Gamma_{1}(N) \xlongrightarrow{\phi} \ZZ/N\ZZ$ given by $\begin{pmatrix}
a & b \\ c&d
\end{pmatrix} \mapsto b \pmod{N}$. This map clearly surjects and the kernel is $\Gamma(N)$
\item The map $\Gamma_{0}(N) \xlongrightarrow{\phi} (\ZZ/N\ZZ)^{\times}$ given by $\begin{pmatrix}
a & b \\ c&d
\end{pmatrix} \mapsto b \pmod{N}$ surjects and has $\Gamma_{1}(N)$ as kernel.
\item Now, using the previous lemma and the above two remarks we have 
\begin{align*}
[\SL_{2}(\ZZ): \Gamma_{0}(N)] &= \frac{[\SL_{2}(\ZZ): \Gamma(N)]}{[\Gamma_{0}(N): \Gamma_{1}(N)][\Gamma_{1}(N):\Gamma(N)]} \\
&=\frac{N^3 \prod_{p \mid N} (1-1/p^2)}{N \cdot \varphi(N)}\\
&=\frac{N^3 \prod_{p \mid N} (1-1/p^2)}{N \cdot N \prod_{p \mid N} (1-1/p)}\\
&= N \prod_{p \mid N} (1 + 1/p)
\end{align*}
\end{enumerate}
\end{remark}

\begin{definition}
A subgroup $\Gamma \subseteq \mathrm{SL}_{2}(\ZZ)$ is called congruence subgroups if $\Gamma(N) \subseteq \Gamma$ for some $N \geq 1$.
\end{definition}

\begin{lemma}
A congruence subgroup has finite index in $\mathrm{SL}_{2}(\ZZ)$.
\end{lemma}

\begin{proof}
By definition $\Gamma (N) \subseteq \Gamma$ for some $N \geq 1$. This implies $\#(\SL_{2}(\ZZ)/\Gamma) \le \#(\SL_{2}(\ZZ)/\Gamma(N))$ But the RHS is exactly $\#(\SL_{2}(\ZZ/N\ZZ))$ which is finite. This completes the proof.
\end{proof}

\begin{remark}
There are non-congruence subgroups of finite index in $\mathrm{SL}_{2}(\ZZ)$.
\end{remark}

\textbf{Properties}:
\begin{enumerate}
\item $\mathrm{PSL}_{2}(\ZZ)$ is generated freely by an element of order $2$ and an element of order $3$.
\item $S_{7}$ is generated by an element of order $2$ and an element of order $3$. There is a surjection 
\begin{align*}
\mathrm{PSL}_{2}(\ZZ) &\xrightarrow{\pi} S_{7} \\
\pi^{-1}(\mathrm{Stab}_{1}) \subseteq \mathrm{PSL}_{2}(\ZZ) \\
\end{align*}
\item $\SL_{2}(\ZZ/p\ZZ)$ is a simple group for $p \geq 5$.
\end{enumerate}

\begin{remark}
$\Gamma$ is the smallest index subgroup that is non-congruence.
\end{remark}

\begin{definition}
A holomorphic function $f: \mathbb{H} \rightarrow \CC$ is a modular form of weight $k$ and level $\Gamma$ if 
\begin{enumerate}
\item $f\left(\frac{az+b}{cz+d}\right) = (cz+d)^k f(z)$ for all $\begin{pmatrix}
a & b \\ c & d
\end{pmatrix} \in \Gamma$
\item $f$ is holomorphic at all cusps.
\end{enumerate}
Cusps of $X(\Gamma)$ are just elements of $\Gamma \backslash \PP^1 (\QQ)$.
\end{definition}

\section{Fundamental Domain}


\chapter{Lecture-6 (19th January, 2023): Cusps, congruence modular forms and enhanced elliptic curves}

\section{Cusps}

Suppose $p$ is a prime.

\begin{proposition}
$$\mathrm{orbit}(\infty) = \{ \frac{a}{c}: p\mid c , p \nmid a\}, \mathrm{orbit}(1) = \{ \frac{a}{c}:p \nmid c\}$$
\end{proposition}
 
\begin{proposition}
$$\# \left( \Gamma_{0}(p)\bs\PP^1(\QQ) \right)=2$$
\end{proposition}

\begin{proof}
Take an element $t\in \QQ$. Write $t = a/c$ in the reduced form, then there exists integers $b,d$ such that $ad-bc=1$. Therefore, we have $\gamma = \begin{pmatrix}
a & b \\ c & d 
\end{pmatrix} \in \SL_{2}(\ZZ)$. Clearly, $\gamma \cdot \infty = t$. We now consider two cases: 
\begin{itemize}
\item If $p \mid c$, then $\begin{pmatrix}
a & b \\ c & d 
\end{pmatrix} \in \Gamma_{0}(p) \Rightarrow a/c \in [\infty]$. Conversely, if $\gamma \in \Gamma_{0}(p)$, then $p$ divides the denominator of $\gamma \cdot \infty$. Therefore $[\infty]$ is given by all rationals $a/c$ such that $p \mid c$
\item If $p \nmid c$, then $\begin{pmatrix}
a & b \\ c & d 
\end{pmatrix} \in \Gamma_{0}(p)$, then $ p \nmid d$ since $ad-bc=1$. Therefore, $p$ cannot divide the denominator of $\gamma \cdot 0$. Conversely, if $b/d $ is such that $\gcd(b,d)=1, p \nmid d$. Then, $\exists a,c$ such that $ad-bc=1$. We can replace $c$ with $c' = c + \lambda d, a' = a + \lambda d$ for some integer $\lambda$ such that $c' \equiv 0 \pmod{p}$. Then, $\begin{pmatrix}
a' & b \\ c' & d 
\end{pmatrix} \in \Gamma_{0}(p)$.  
\end{itemize}
\end{proof}

\begin{definition}
If $\Gamma$ is a congruence subgroup. Then the cusps of $\Gamma$ is the set of $\Gamma$ orbits in $\PP^1(\QQ)$, i.e., $$\mathrm{Cusps}(\Gamma) = \Gamma \bs \PP^1(\QQ) = \Gamma \bs \SL_{2}(\ZZ)/\SL_{2}(\ZZ)_{\infty}$$
Therefore there is a surjective map $$\Gamma \bs \SL_{2}(\ZZ) \twoheadrightarrow \mathrm{Cusps}(\Gamma)$$
\end{definition}

\begin{proposition}
Let $\Gamma$ be a congruence subgroup, then $\Gamma\backslash\PP^1 (\QQ)$ is finite. (called the cusps of level $\Gamma$). 
\end{proposition}

\begin{proof}
Since there exists $N \in \NN$ such that $\Gamma (N) \subseteq \Gamma$, we can just work with $\Gamma = \Gamma(N)$. Note that $\SL_{2}(\ZZ) = \sqcup_{i} g_{i} \Gamma$. \\
For any rational $q \in \PP^1(\QQ)$ we have a $\gamma \in \SL_{2}(\ZZ)$ such that $q = \gamma \cdot \infty$. And hence $\exists \; i $ such that $\gamma =g_{i} \gamma '$. Thus, $q = \\gamma ' (g_{i} \infty)$ OR equivalently $q$ is $\Gamma$ equivalent to some $g_{i}(\infty)$. This completes the proof since $[\SL_{2}(\ZZ): \Gamma] < \infty$.
\end{proof}

\begin{exercise}
$$\SL_{2}(\ZZ) = \bigsqcup_{j=0}^{p-1} \alpha_{j} \Gamma_{0}(p) \bigsqcup \alpha_{\infty} \Gamma_{0}(p)$$ where $\alpha_{j} = \begin{pmatrix}
1 & 0 \\ j & 1
\end{pmatrix}\; 0 
\le j \le p-1 ,  \alpha_{\infty} = \begin{pmatrix}
1 & -1 \\ 0 & 1
\end{pmatrix}$
\end{exercise}

\textbf{Notation:}Let $f: \Hh \rightarrow \CC$ be a function. Let $k \in \ZZ$ and $\gamma \in SL_{2}(\RR)$. We define 
\begin{align*}
f|_{[\gamma]_{k}} : \Hh &\rightarrow \CC \\
\text{ defined by } f|_{[\gamma]_{k}}(z) &= (cz+d)^{-k} f(\gamma \cdot z)
\end{align*}

With this notation: $f \in \Mm_{k}(\SL_{2}(\ZZ))$ if 
\begin{itemize}
\item $f$ is holomorphic on $\Hh$ and at $\infty$.
\item $f|_{[\gamma]_{k}}(z) = f(z) \; \forall \; \gamma \in \SL_{2}(\ZZ)$. 
\end{itemize}

\begin{definition}
Let $\Gamma$ be a congruence subgroup and $k \in \ZZ$. A modular form of weight $k$ and level $\Gamma$ is a function $f: \Hh \rightarrow \CC$ such that 
\begin{enumerate}
\item $f$ is holomorphic on $\Hh$.
\item $f|_{[\gamma]_{k}} = f \; \forall \; \gamma \in \Gamma$.
\item $f$ is holomorphic at all cusps. \\
Note that $\begin{pmatrix}
1 & h \\ 0 & 1
\end{pmatrix} \in \Gamma$ for some $h \in \ZZ_{>0}$, and if $\Gamma(N) \subseteq \Gamma$ then $h \mid N$. Thus, $$f(z+h) = f(z)$$ and $f $ admits the Fourier expansion $$f(z) = \sum_{n \in \ZZ} a_{n} \exp(2 \pi i n z/h)$$
$f$ is said to be holomorphic at $\infty$ if $a_{n} =0 \; \forall \; n \le -1$. Suppose $\alpha \in \PP^1(\QQ)$ is a cusp, then there exists $\gamma \in \SL_{2}(\ZZ)$ such that $\gamma \cdot \alpha = \infty$. $f$ is holomorphic at $\alpha$ if $f|_{[\gamma]_{k}}$ is holomorphic at $\infty$.
\end{enumerate}
\end{definition}

\begin{example}
$\Gamma_{1}(p)$ has cusps $0, \infty$. We just need to check 
\begin{itemize}
\item $f$ is holomorphic at $\infty$.
\item $f|_{[\gamma]_{k}}$ is holomorphic at $\infty$.
\end{itemize}
\end{example}

\textbf{Notation}: 
\begin{align*}
\Mm_{k}(\Gamma) &= \text{ the space of modular forms of weight $k$ and level $\Gamma$} 
\end{align*}

\begin{definition}
$f \in \Mm_{k}(\Gamma)$ is said to be a cusp form if $f$ vanishes at all cusps of level $\Gamma$, i.e., $f|_{[\gamma]_{k}}$ vanishes at $\infty \; \forall \; \gamma \in \SL_{2}(\ZZ)$. \\

By $\Ss_{k}(\Gamma)$ we denote the space of all cusp forms of weight $k$ and level $\Gamma$.
\end{definition}

$M(\Gamma) = \bigoplus_{k \in \ZZ} \Mm_{k}(\Gamma)$ is the graded ring of modular forms of level $\Gamma$. \\

If $\Gamma_{1} \subseteq \Gamma_{2}$ are two congruence subgroups, then $\Mm_{k}(\Gamma_{2}) \subseteq \Mm_{k}(\Gamma_{1})$. This implies $\Mm_{k}(\SL_{2}(\ZZ)) \subseteq \Mm_{k}(\Gamma)$ for any $\Gamma$.\\

Now, let $\Gamma$ be a congruence subgroup. Define: 
\begin{align*}
Y(\Gamma) &= \Gamma \backslash \Hh \\
X(\Gamma) &= \Gamma \backslash (\Hh \cup \PP^1 (\QQ))
\end{align*}
These are called modular curves. \\

We saw that $\SL_{2}(\ZZ) \backslash \Hh$ parametrises elliptic curves over $\CC$ (upto isomorphism).

\section{Interlude: Complex Tori and Elliptic curves}

\subsection{Complex Tori}
\begin{definition}
Take $\omega_{1}, \omega_{2} \in \CC$ linearly independent over $\RR$. We also make a normalising convention that $\omega_{1}/\omega_{2} \in \Hh$. Then, a lattice in $\CC$ is a set $$\Lambda = \omega_{1} \ZZ \oplus \omega_{2} \ZZ$$
\end{definition}

\begin{lemma}
Consider the two lattices $\Lambda = \omega_{1} \ZZ \oplus \omega_{2} \ZZ , \Lambda '= \omega_{1}' \ZZ \oplus \omega_{2}' \ZZ$ with $\omega_{1}/\omega_{2}, \omega_{1}'/\omega_{2}' \in \Hh$. Then, $\Lambda = \Lambda '$ iff $$\begin{pmatrix}
\omega_{1} ' \\ \omega_{2}'
\end{pmatrix} = \begin{pmatrix}
a & b \\ c & d
\end{pmatrix}\begin{pmatrix}
\omega_{1}  \\ \omega_{2}
\end{pmatrix} \text{ for some } \begin{pmatrix}
a & b \\ c & d
\end{pmatrix} \in \SL_{2}(\ZZ)$$
\end{lemma}

\begin{proof}
($'\Rightarrow '$)\\
Since $\omega_{1}, \omega_{2}$ is a basis, therefore $\omega_{1}' = a\omega_{1} + b\omega_{2}, \omega_{2}' = c\omega_{1} + d\omega_{2}$. Similary, since $\omega_{1}',\omega_{2}'$ is a basis, therefore $\omega_{1} = u\omega_{1} + v\omega_{2},\omega_{2} = w\omega_{1} + x\omega_{2}'$. Hence, $$\begin{pmatrix}
u & v \\ w & x
\end{pmatrix}\begin{pmatrix}
a & b \\ c & d
\end{pmatrix} = \begin{pmatrix}
1 & 0 \\ 0 & 1
\end{pmatrix}$$
Therefore, the determinant of each matrix is $1$ which proves this direction. \\

($'\Leftarrow'$)\\
The lattice $\Lambda '$ consists of all linear combinations of the form $$x(a\omega_{1} + b \omega_{2}) + y (c \omega_{1} + d \omega_{2}) = (xa+yc) \omega_{1} + (bx + dy)\omega_{2}$$ Therefore $\Lambda ' \subseteq \Lambda '$. And, for any $m \omega_{1} + n \omega_{2} \in \Lambda $ we can always find $(x,y)$ such that $$\begin{pmatrix}
a & b \\ c& d
\end{pmatrix} \begin{pmatrix}
x \\ y
\end{pmatrix} = \begin{pmatrix}
m \\ n
\end{pmatrix} $$
Thus, we can infer that $\Lambda \subseteq \Lambda '$. Hence, we conclude our proof.
\end{proof}

\begin{definition}
The parallelogram with vertices $z_{0}, z_{0} + \omega_{1}, z_{0} + \omega_{2} , z_{0} + \omega_{1} + \omega_{2}$ is called the fundamental parallelogram.
\end{definition}

\begin{lemma}
Any holomorphic map between compact Riemann surfaces is either a surjection or a map to one point.
\end{lemma}

\begin{proof}
Suppose $f: X\rightarrow Y$ be a holomorphic map between two compact Riemann surfaces $X,Y$. Since $f$ is continuous and $X$ is compact, therefore $f(X)$ is compact as well. Same for connected. Unless $f$ is constant, it is open by Open mapping theorem. Thus, $f(X)$ is clopen, (connected + compact $\Rightarrow$ closed) and connected. Hence, it is either a single point or the entirety of $Y$. 
\end{proof}

\begin{proposition}
Suppose $\phi: C/\Lambda \rightarrow \CC/\Lambda '$ is a holomorphic map between complex tori. Then there exists complex numbers $m,b$ with $m \Lambda \subseteq  \Lambda '$ such that $\phi(z + \Lambda) = mz + b \Lambda '$. The map is invertible iff $m \Lambda = \Lambda '$.
\end{proposition}

\begin{proof}

\end{proof}

\begin{corollary}
Suppose $\phi: C/\Lambda \rightarrow \CC/\Lambda '$ is a holomorphic map between complex tori with $\phi(z + \Lambda) = mz + b \Lambda ' , m \Lambda = \Lambda '$. TFAE: 
\begin{enumerate}
\item $\phi$ is a group homomorphism.
\item $b \in \Lambda '$, so $\phi(z + \Lambda) = mz + \Lambda '$.
\item $\phi(0)=0$
\end{enumerate}
\end{corollary}

\begin{proof}
$1 \Rightarrow 3 \Rightarrow 2 \Rightarrow 1$
\end{proof}

\subsection{Complex Tori as elliptic curves}

\begin{definition}
Given a lattice $\Lambda$, we define the Weierstrass $\wp$-function as 
\begin{equation}
\wp(z;\Lambda) = \frac{1}{\omega^2} + \sum_{\omega \in \Lambda - \{(0,0)\}} \left( \frac{1}{(z-\omega)^2} -\frac{1}{\omega^2} \right)
\end{equation}
$z \in \CC , z\not \in \Lambda$ \\

Eisenstein series generalises to 
\begin{equation}
G_{k}(\Lambda) = \sum_{\omega \in \Lambda - \{(0,0)\}} \frac{1}{\omega^k} , k>2 \text{ even }
\end{equation}
\end{definition}

\begin{theorem}[Sil, Exercise 6.2]
For any lattice $\Lambda$, the Weierstrass $\wp$-function converges absolutely and uniformly on every compact subset of $\CC\bs \Lambda$. The series defines a meromorphic function on $\CC$ having double pole with residue $0$ at each lattice point and no other poles.
\end{theorem}

\begin{proof}
\begin{itemize}
\item Observe the absolute value of a summand 
\begin{align*}
\left|\frac{1}{(z-\omega)}  - \frac{1}{\omega^2}\right| &= \frac{|z||2\omega - z|}{|\omega|^2 |z - \omega|^2} \\
&\le \frac{|z|(2 |\omega| + |z|)}{|\omega|^2(|z| -|\omega|)^2} 
\end{align*}
Suppose $|\omega| > 2|z|$, then by reverse triangle inequality $|\omega - z| > |\omega|/2$, and $2|\omega| + |z| < 5|\omega|/2$. Hence, 
\begin{align*}
\left|\frac{1}{(z-\omega)}  - \frac{1}{\omega^2}\right| &< \frac{|z|(2 |\omega| + |z|)}{|\omega|^2(|z| -|\omega|)^2} \\
&< 10\frac{|z|}{|\omega|^3}
\end{align*}
On a compact disc $|z|<R$ this bound becomes $10R/|\omega|^3$. \\

Thus, to prove convergence of the original series, we just need to check convergence of $\ds{\sum_{\omega} \frac{1}{\omega^3}}$
\item \textbf{Claim}: $\#\{\omega \in \Lambda : |\omega| \le R\} = \pi R^2/A(\Lambda) + \mathcal{O}(R)$ where $A(\Lambda)$ is the volume of fundamental parallelogram. \\
Let $\omega_{1},\omega_{2}$ be the basis of $\Lambda$. And say $c = \max\{|\omega_{1} + \omega_{2}|, |\omega_{1} - \omega_{2}|\}$. Also, assume $R>c$. We can move the fundamental parallelogram a bit so that there is only one lattice point inside each parallelogram. Say $B_{R}$ is the disc of radius $R$ and $K_{R}$ the intersection of all parallelograms that intersect $B_{R}$. Clearly, $B_{R} \subseteq K_{R} \subseteq B_{R+c}$. Therefore, $B_{R}$ contains atmost $\pi (R+c)^2/A(\Lambda)$ many lattice points. And, $B_{R-c} \subseteq K_{R-c} \subseteq B_{R}$ which implies there are atleast $\pi (R-c)^2/A(\Lambda)$ many lattice points in $B_{R}$. To summarise $$\frac{\pi (R-c)^2}{A(\Lambda)} < \text{ lattice points } < \frac{\pi (R+c)^2}{A(\Lambda)}$$ 
And hence we conclude what we want.
\item We can conclude that there exists a constant $c=c(\Lambda)$ such that $$\#\{\omega \in \Lambda : N \le |\omega|< N+1\} < cN$$
\item Thus, $$\sum_{\omega \in \Lambda} \frac{1}{\omega^3} \le \sum_{N=1}^3 \frac{\#\{\omega \in \Lambda : N \le |\omega|< N+1}{N^3}\le \sum_{N} \frac{c}{N^2}<\infty$$
\end{itemize}
This concludes the proof.
\end{proof}

\begin{proposition}
Let $\wp$ be the Weierstrass function wrt to lattice $\Lambda$. 
\begin{enumerate}
\item The Laurent expansion of $\wp$ is 
\begin{equation}
\wp(z) = \frac{1}{z^2} + \sum_{n\equiv 0 \pmod{2}} (n+1)G_{n+2}(\Lambda)z^n
\end{equation}
for all $z$ such that $0 < |z| < \inf \{|\omega|: \omega \in \Lambda - \{(0,0)\}\}$
\item The functions $\wp, \wp'$ satisfy the relation 
\begin{equation}
(\wp'(z))^2 = 4(\wp(z))^3 - g_{2}(\Lambda)\wp(z) - g_{3}(\Lambda)
\end{equation}
where $g_{2}(\Lambda) = 60G_{4}(\Lambda), g_{3}(\Lambda) = 140 G_{6} (\Lambda)$
\item Let $\Lambda =\omega_{1}\ZZ \oplus \omega_{2}\ZZ$ and $\omega_{3} =\omega_{1} + \omega_{2}$. Then the cubic equation satisfied by $\wp , \wp '$, $y^2 = 4x^3 -g_{2}(\Lambda)x - g_{3}(\Lambda)$, is $$y^2 = 4(x-e_{1})(x-e_{2})(x-e_{3})$$ where $e_{i} = \wp (\omega_{i}/2), i=1,2,3$
\end{enumerate}
\end{proposition}

\begin{proof}
\begin{enumerate}
\item Observe
\begin{align*}
\frac{1}{(z-\omega)^2} - \frac{1}{\omega^2} &= \frac{1}{\omega^2} \left( \frac{1}{(1-z/\omega)^2} -1\right)
\end{align*}
Since we took $|z|< \inf \{|\omega|: \omega \in \Lambda - \{(0,0)\}\} \Rightarrow |z/\omega|<1 $ and thus we can use binomial expansion to get 
\begin{align*}
\frac{1}{(z-\omega)^2} - \frac{1}{\omega^2} &=\frac{1}{\omega^2} \sum_{n=1}^{\infty} (n+1)(z/\omega)^2\\
\therefore \wp(z) &= \frac{1}{z^2} + \sum'_{\omega}\sum_{n=1}^{\infty} (n+1)\frac{z^n}{\omega^{n+2}} \\
&= \frac{1}{z^2} + \sum_{n=1}^{\infty} (n+1)z^n\sum'_{\omega}\frac{1}{\omega^{n+2}}\\
&= \frac{1}{z^2} + \sum_{n} (n+1)G_{n+2}(\Lambda)z^n
\end{align*}
The interchange of sums can be done due to absolute convergence. The odd terms in the last sum vanish therefore we obtain what we want. 
\item \begin{align*}
\wp(z) &= \frac{1}{z^2} + 3G_{4}(\Lambda)z^2 + 5G_{6}(\Lambda)z^4 + \mathcal{O}(z^6) \\
\wp'(z) &= \frac{-2}{z^3} + 6G_{4}(\Lambda)z + 20G_{6}(\Lambda)z^3 + \mathcal{O}(z^5) 
\end{align*}
After some laborious calculation, we note that $(\wp'(z))^2$ and $4(\wp(z))^3 - g_{2}(\Lambda)\wp(z) - g_{3}(\Lambda)$ evaluate to $\mathcal{O}(z^2)$. Hence, as $z\rightarrow 0$ the difference of the two mentioned in the previous line goes to zero. Moreover, the difference is periodic, holomorphic and thus bounded and constant. This gives the desired equality.
\item Note that $\wp'(z)$ is odd, it has zeroes of order $2$. Indeed, if $z \equiv -z \pmod{\Lambda}$ then $\wp'(z) =\wp'(-z) =- \wp'(z)\Rightarrow \wp'(z)=0$. Letting $\Lambda = \omega_{1}\ZZ \oplus \omega_{2}\ZZ$, the order two points are $\omega_{i}/2, i=1,2$ and $\omega_{3}/2 = (\omega_{1}+\omega_{2})/2$. Clearly, $\wp'(\omega_{i}/2)=0, i=1,2,3$. From the previous part, $\wp(\omega_{i}/2)$ are roots of the polynomial $4x^3 -g_{2}(\Lambda)x - g_{3}(\Lambda)$ and hence the said factorisation works. 
\end{enumerate}
\end{proof}

\begin{remark}
\begin{itemize}
\item $\wp(z)$ is an even function.
\item The roots $e_{i}'s$ in the previous proposition are distinct. Note that $\wp(z)-e_{1}$ vanishes at $z=\omega_{1}/2$. This is a double zero since $\wp'(\omega_{1}/2)=0$. Similarly, $\wp(z)-e_{2}$ has double zero at $z= \omega_{2}/2$. If $e_{1}=e_{2}$ then $\wp(z)-e_{1}$ would have zero of order $4$ a contradiction. Therefore $e_{1}\neq e_{2}$. Similarly, $e_{1}\neq e_{3}, e_{2} \neq e_{3}$.
\end{itemize}
\end{remark}

\subsection{Isogenies}

\begin{definition}
A non-zero holomorphic homomorphism between complex tori is called an isogeny.
\end{definition}

\section{Enhanced elliptic curves}
Let $N \in \ZZ_{\geq 1}$
\begin{definition}
\begin{enumerate}
\item An enhanced elliptic curve of level $\Gamma_{0}(N)$ is a pair $(E,C)$ where $E$ is an elliptic curve and $C$ is an order $N$ cyclic subgroup of $E(\CC)$. \\
Morphism between $(E,C)$ and $(E',C')$ is a homomorphism $$\varphi : E \rightarrow E'$$ such that $\varphi (C) = C'$
\item An enhanced elliptic curve of level $\Gamma_{1}(N)$ is a pair $(E,Q)$ such that  $E$ is an elliptic curve and $Q$ is an order $N$ point on $E(\CC)$. \\
Morphism between $(E,Q)$ and $(E',Q')$ is a homomorphism $$\varphi : E \rightarrow E'$$ such that $\varphi (Q) = Q'$
\item An enhanced elliptic curve of level $\Gamma(N)$ is a triplet $(E,Q_{1},Q_{2})$ such that $E$ is an elliptic curve and $Q_{1},Q_{2}$ are points of order $N$ and $$\langle Q_{1} , Q_{2} \rangle = E(\CC)[N] = \{ x \in E(\CC) \mid Nx =0\}$$
\end{enumerate}
\end{definition}

\begin{proposition}
\begin{enumerate}
\item $Y(\Gamma_{0}(N))$ parametrizes enhanced elliptic curves of level $\Gamma_{0}(N)$. The map $z\in \Hh \mapsto \left( \CC/\Lambda_{z} , \left( 1 + \Lambda_{z} \right)/\Lambda_{z}\right)$ gives a bijection between 
$$Y (\Gamma_{0}(N)) \leftrightarrow \{\text{ isomorphism classes of enhanced elliptic curves of level } \Gamma_{0}(N)\}$$
\item $Y(\Gamma_{1}(N))$ parametrizes enhanced elliptic curves of level $\Gamma_{0}(N)$. The map $z\in \Hh \mapsto \left( \CC/\Lambda_{z} , \frac{1}{N} \right)$ gives a bijection between 
$$ (\Gamma_{1}(N)) \leftrightarrow \{\text{ isomorphism classes of enhanced elliptic curves of level } \Gamma_{1}(N)\}$$
\item $Y(\Gamma(N))$ parametrizes enhanced elliptic curves of level $\Gamma_{0}(N)$. The map $z\in \Hh \mapsto \left( \CC/\Lambda_{z} , \frac{1}{N} , \frac{1}{N} \cdot z \right)$ gives a bijection between 
$$Y (\Gamma(N)) \leftrightarrow \{\text{ isomorphism classes of enhanced elliptic curves of level } \Gamma(N)\}$$
\end{enumerate}
\end{proposition}

\begin{proof}

\end{proof}

\chapter{Lecture-7 (24th January, 2023): Modular curves as Riemann surfaces}

\section{Riemann surfaces}

\begin{proposition}
The action of $\SL_{2}(\ZZ)$ on $\Hh$ is properly discontinuous, i.e., for $z_{1},z_{2} \in \Hh$ there exists neighbourhoods $U_{i}$ of $z_{i}$ such that if $\gamma \in U_{1}\cap U_{2} \neq 0$, then $\gamma \cdot z_{1} = z_{2}$.
\end{proposition}

\begin{proof}

\end{proof}

\begin{corollary}
$Y(\Gamma) = \Gamma \bs \Hh$ with the quotient topology is Hausdorff.
\end{corollary}

\begin{proof}

\end{proof}

\begin{proposition}
$X(\Gamma)$ is compact.
\end{proposition}

\begin{proof}

\end{proof}



\begin{definition}
A Riemann surface consists of the following data: 
\begin{enumerate}
\item $X$ is a topological space (Hausdorff + second countable).
\item $(U_{i},V_{i}, \phi_{i})$ with $V_{i}$ open in $X$, $U_{i}$ open ball in $\CC$ and $\phi: U_{i} \rightarrow V_{i}$ a homeomorphisms such that whenever $V_{i} \cap V_{j} \neq \emptyset$ we have $$\phi_{j}^{-1} \circ \phi_{i} : U_{i} \cap \phi_{i}^{-1}(V_{i} \cap V_{j}) \rightarrow U_{j} \cap \phi_{j}^{-1} (V_{i} \cap V_{j})$$ to be homeomorphisms. 
\end{enumerate}
\end{definition}

\textbf{\textcolor{BrickRed}{GOAL}:} To make $X(\Gamma)$ a Riemann surface. That is we want to construct charts on $X(\Gamma)$.

\begin{definition}
A point $P \in Y(\Gamma)$ is called an elliptic point if for any lift $z$ of $P$ in $\Hh$, we have $\Stab_{\Gamma}(z) / \left(  \Stab_{\Gamma}(z) \cap \{\pm I_{2}\} \right)$ is nontrivial. That is to say $\Stab_{\bar{\Gamma}}(z)$ is nontrivial, where $\bar{\Gamma}$ is the image of $\Gamma$ in $\mathrm{PSL}_{2}(\ZZ)$.
\end{definition}

$P$ is an elliptic point in $Y(\Gamma)$ only if it lifts to a point equivalent to $i = \exp(2 \pi i /4)$ or $\omega  = 2 \pi  i /6$. If $P$ is an elliptic point, then $\Stab_{\bar{\Gamma}}(z)$ has order $2$ or $3$.

\begin{theorem}
$Y(\Gamma)$ is a Riemann surface.
\end{theorem} 

\begin{proof}

\end{proof}

\begin{theorem}
$X(\Gamma)$ is a Riemann surface.
\end{theorem}

\begin{proof}

\end{proof}

\subsection{Local charts on $Y(\Gamma)$}

\begin{enumerate}
\item $P \in Y(\Gamma)$ is not an elliptic point. \\

Let $z\in \HH$ be a lift of $P$ and $U_{1},U_{2}$ be neighbourhoods of $z$. Put $U=U_{1} \cup U_{2}$. If $\gamma U \cap U \neq \emptyset$, then the image set of $Y$ in $\bar{\Gamma}$ is identity. 
$$\pi : \HH \rightarrow Y(\Gamma)$$
Put $\gamma   = \pi (U)$. Then, $\pi|_{U} : U \rightarrow V$ is a homeomorphism. 

\item Let $P \in Y(\Gamma)$ be an elliptic point. Let $z\in \HH$ such that $\pi(z) = P$. Same as previous case get $U_{1},U_{2}$ and define $U$ as the the union of the two. \\
Now, if $\gamma U \cap U \neq \emptyset$ then $\gamma \in \Stab_{\bar{\Gamma}}(z)$. \\
Set $V= \pi (U)$. Notice that here $\pi|_{U}: U \rightarrow V$ need not be a homeomorphism. We instead have 
\[\begin{tikzcd}
	U & {U'=U/\mathrm{Stab}_{\bar{\Gamma}}(z)} \\
	V
	\arrow["{\pi|_{U} }"', from=1-1, to=2-1]
	\arrow[from=1-1, to=1-2]
	\arrow["\simeq", from=1-2, to=2-1]
\end{tikzcd}\]

\item We next want to extend this to cusps of $X(\Gamma)$. \\
For $\SL_{2}(\ZZ)$ we have already seen a local chart $z\mapsto \exp(2 \pi i z)$. In general, take any cusp $P$ of $X(\Gamma)$. Take $\gamma \in \SL_{2}(\ZZ)$ such that $\gamma P = \infty$. Use the local charts at $\infty$.
\end{enumerate}
Hence, $X(\Gamma)$ is a compact Riemann surface. \\

Next, we wish to compute genus of $X(\Gamma)$. \\
Genus $\mathfrak{g}$ of $X(\Gamma)$ is an integer such that 
\begin{align*}
H^1(X(\Gamma), \ZZ) &\cong \ZZ^{2g} \\
H_{1}(X(\Gamma), \ZZ) &\cong \ZZ^{2g}
\end{align*}



\chapter{Lecture-8 (2nd February, 2023): More Riemann surfaces}

\section{Riemann surfaces contd..}


\chapter{Lecture-9 (7th February, 2023): Riemann surfaces and Riemann Roch theorem}

\begin{corollary}
Let $\Gamma$ be a congruence subgroup of $\SL_{2}(\ZZ)$. Then, $$g(X(\Gamma)) = 1 + \frac{N}{12} - \frac{\epsilon_{2}}{4} - \frac{\epsilon_{3}}{4} - \frac{\epsilon_{\infty}}{2}$$ where $N = [\mathrm{PSL}_{2}(\ZZ): \bar{\Gamma}]$, $\epsilon_{2}$ is the number of elliptic points of order $2$, $\epsilon_{3}$ is the number of elliptic points of order $3$ and $\epsilon_{\infty}$ is the number of cusps of $\Gamma = \#(\Gamma \bs \PP^{1}(\QQ))$.
\end{corollary}

\begin{proof}
Consider the map $$X(\Gamma) \xlongrightarrow{f} X(\SL_{2}(\ZZ))$$ with $\deg f = n$. Then, $$2g(X(\Gamma)) -2 = N(-2) + \sum_{Q \in X(\Gamma)} (e_{Q}-1)$$
If $f(Q)$ is not equivalent to $i$ or $\omega$, $Q$ is unramified. \\

If $f(Q)$ is equivalent to $i$, then we have two cases: 
\begin{enumerate}
\item $Q$ is elliptic implies $Q$ is unramified.
\item $Q$ is not elliptic implies $e_{Q}=2$.
\end{enumerate}
$$\sum_{Q \in f^{-1}([i])} e_{Q}=N , \sum_{Q \in f^{-1}([i])} (e_{Q}-1) = \frac{N-\epsilon_{2}}{2}$$

If $f(Q) \sim \omega$, then $$\sum_{Q \in f^{-1}([i])} (e_{Q}-1) = \frac{2(N-\epsilon_{3})}{3}$$

Finally, let us talk about ramification at the cusps. Recall from the charts at cusps that Local coordinate at a cusp $Q$ is $e^{2\pi i /h}$ for some integer $h \geq 1$. In this case $e_{Q}=h$ and $$\sum_{Q \in f^{-1}([i])} (e_{Q}-1) = \left(\sum_{Q}e_{Q}\right) - \epsilon_{\infty} = N - \epsilon_{\infty}$$
Therefore, 
\begin{align*}
2g(X(\Gamma)) &= -2N + \frac{N - \epsilon_{2}}{2} + 2\frac{N-\epsilon_{3}}{3} + N -\epsilon_{\infty} \\
2g(X(\Gamma)) &= \frac{N}{12} - \frac{\epsilon_{2}}{4} - \frac{\epsilon_{3}}{6} - \frac{\epsilon_{\infty}}{2} + 1
\end{align*}
\end{proof}

\begin{exercise}
$g(X(\Gamma_{0}(p)))=0$ iff $p \in \{2,3,5,6,7,13\}$
\end{exercise}

\textcolor{Blue}{Goal of the day} $$\dim_{\CC} M_{k}(\Gamma) = (k-1)(g(X(\Gamma))-1) + \frac{k}{2} \epsilon_{\infty} + \left[\frac{k}{4} \right]\epsilon_{2} + \left[ \frac{k}{3} \right]\epsilon_{3}$$ and $$\dim_{\CC} S_{k}(\Gamma) = (k-1)(g(X(\Gamma))-1) - \frac{k}{2} \epsilon_{\infty} + \left[\frac{k}{4} \right]\epsilon_{2} + \left[ \frac{k}{3} \right]\epsilon_{3}$$

To proceed we will need some algebraic geometry language. 
\begin{align*}
X &- \text{ is a smooth projective curve over } \CC \\
\mathcal{O}_{X} &- \text{structure sheaf} \\
\mathcal{O}_{X}(U) &- \{f: U \rightarrow \CC \text{ regular for open } U\}\\
\mathcal{F}(U) &- \mathcal{O}_{X}\text{- module sheaves}
\end{align*}

We are concerned with invertible $\mathcal{O}_{X}$-module sheaves, i.e., locally free of rank $1$. These are called invertible sheaves. 

\begin{example}
For a number field $F$, take $\mathcal{O}_{F}$ as the structure sheaf. Any nonzero fractional ideal is invertible hence we can think of that as the invertible sheaves.
\end{example}

Invertible sheaves form a group under $\otimes_{\mathcal{O}_{X}}$. \\

$H^{0}(X, \mathcal{F})=$ global sections. 

\begin{remark}
As $X$ is a smooth projective curve, $H^{0}(X, \mathcal{F})=\CC$. This is like Liouville's theorem.
\end{remark}

We can take a look at meromorphic sections of $\mathcal{F}$ $$\mathcal{F}(U) \otimes_{\mathcal{O}_{X}(U)} \mathrm{Frac}(\mathcal{O}_{X}(U))$$ where $\mathrm{Frac}(\mathcal{O}_{X}(U))$ are the meromorphic functions. 

\begin{theorem}[Riemann existence theorem]
An invertible sheaf on a compact Riemann surface has a non-constant global meromorphic section.
\end{theorem}

\begin{definition}
Let $\mathcal{F}$ be an invertible sheaf on $X$. Take a non-constant global section $s$ of $\mathcal{F}$. We define $\deg(\mathcal{F})$ to be the sum of orders of zeros of $s$. 
\end{definition}

\begin{remark}
Note that this definition does not depend on the choice of $s$. If we take a different $s'$, then $s = fs'$ where $f$ is a global meromorphic section of $\mathcal{O}_{X}$ ($\deg f = 0 = \text{ sum of orders of zeros}$)
\end{remark}

\begin{proposition}
Properties of the degree function:
\begin{enumerate}
\item $\deg(\mathcal{F} \otimes_{\mathcal{O}_{X}} \mathcal{G}) = \deg \mathcal{F} + \deg \mathcal{G}$
\item $\deg \mathcal{F}^{-1} = - \deg \mathcal{F}$
\end{enumerate}
\end{proposition}

\begin{theorem}[Riemann-Roch theorem]
\begin{enumerate}
\item $H^{0}(X, \mathcal{F})$ is a finite dimensional $\CC$-vector space.
\item Put $h^{0}(X,\mathcal{F}) = \dim_{\CC}H^{0} (X, \mathcal{F})$. Then, $$h^{0}(X, \mathcal{F}) - h^{0}(X, \Omega_{X}^{1} \otimes \mathcal{F}^{-1}) = 1- g(X) + \deg \mathcal{F}$$ where $\Omega = \Omega_{X}^{1}$ is the sheaf of holomorphic differentials on $X$.
\end{enumerate}
\end{theorem}

\begin{remark}[Facts]
\begin{enumerate}
\item If $\deg \mathcal{F} < 0$, then $H^{0}(X, \mathcal{F})=0$ 
\item If $\deg \mathcal{F} >>0$, then $H^{0}(X, \Omega \otimes \mathcal{F}^{-1})=0$
\end{enumerate}
\end{remark}

\begin{lemma}
\begin{enumerate}
\item $h^{0}(X, \Omega) = g(X)$
\item $\deg \Omega = 2g-2$
\end{enumerate}
\end{lemma}

\begin{proof}
\begin{enumerate}
\item Take $\mathcal{F}  = \mathcal{O}_{X}$. Then 
\begin{align*}
h^{0}(X, \mathcal{O}_{X}) - h^{0}(X, \Omega \otimes_{\mathcal{O}_{X}} \mathcal{O}_{X}^{-1}) &= 1 - g(X) + \deg \mathcal{O}_{X} \\
1 - h^{0}(X, \Omega) = 1- g(X) + 0\\
g(X) &= h^{0}(X,\Omega)
\end{align*}
\item Take $\mathcal{F} = \Omega$. Then, 
\begin{align*}
h^{0}(X, \Omega) - h^{0}(X, \Omega_{X}) &= 1 - g(X) + \deg \Omega \\
g(X) - 1 &= 1-g + \deg \Omega
\end{align*}
\end{enumerate}
\end{proof}

\begin{lemma}

\end{lemma}

\begin{definition}[Katz sheaf]
Let $X(\Gamma), k \in \ZZ$ be as usual. For $V$ open in $X(\Gamma)$, we define a sheaf $\omega_{k}$ as $$\omega_{k} = \{f: \pi^{-1}(V) \subseteq \Hh \rightarrow \CC\}$$ with $f$ is holomorphic and $f(\gamma \cdot z) = (cz+d)^k f(z) \; \forall \; \gamma \in \Gamma, \forall \; z \in \pi^{-1}(V)$ and $\pi: \Hh^{*} = \Hh \cup \PP^{1}(\QQ) \rightarrow X(\Gamma)$ 
\end{definition}

\begin{remark}
$H^{0}(X(\Gamma), \omega_{k})$ gives modular forms of weight $k$ and level $\Gamma$. 
\end{remark}

\begin{theorem}
\begin{enumerate}
\item $\omega_{k}$ is an invertible sheaf.
\item $\omega_{2} = \Omega$ (cusps).
\end{enumerate}
\end{theorem}

Let $\mathcal{L}$ be an invertible sheaf, $D =\sum_{i} n_{i}P_{i}$ divisor of $X$ ($P_{i}$ is a point of $X$). If $x$ is a meromorphic section of $\mathcal{L}$, then $$\mathrm{div}(x) := \sum_{P \in X} \ord_{P}(x)$$ and $$\mathcal{L}(D) = \{x, \text{ a meromorphic section of }\mathcal{L}: \mathrm{div}(x) + D \geq 0\}$$

\begin{example}
If $P$ is a point of $X$. Then, 
\begin{align*}
\mathcal{L}(-P) &= \text{ meromorphic section $x$ of $\mathcal{L}$ with atleast a simple zero of $P$}\\
\mathcal{L}(P) &= \text{meromorphic section $x$ of $\mathcal{L}$ with atmost a simple pole at $P$}
\end{align*}
\end{example}

\begin{definition}
$$\text{cusps} =\sum_{P \in \Gamma \bs \PP^1(\QQ)} P$$
\end{definition}

If $\Gamma$ is a congruence subgroup. Then, $$\Gamma_{\infty} =\{g \in \Gamma: g \cdot \infty = \infty\}$$ has one of the following forms $\{\pm I_{2}\}, \left\langle \begin{pmatrix}
1 & h \\ 0 & 1 
\end{pmatrix} \right\rangle, \left\langle -\begin{pmatrix}
1 & h \\ 0 & 1 
\end{pmatrix} \right\rangle$

\begin{definition}
\begin{enumerate}
\item $\infty$ is called irregular cusp for $\Gamma$ if $\Gamma_{\infty} = \left\langle -\begin{pmatrix}
1 & h \\ 0 & 1 
\end{pmatrix} \right\rangle$
\item A cusp $s$ is called irregular if $\infty$ is irregular for $\alpha \Gamma \alpha^{-1}$ where $s= \alpha \cdot \infty (\alpha \in \SL_{2}(\ZZ))$
\end{enumerate}
\end{definition}

\begin{definition}
Let $r$ be the least common multiple of integers in the following set $$\{\mathrm{Stab}_{\Gamma}(P), P \in \mathbb{H}\} \cup \{2 \text{ if there exists an irregular cusp}\}$$
\end{definition}

\begin{exercise}
$1\le r \le 12$
\end{exercise}

\begin{definition}
$\Gamma$ is called neat if $r=1$.
\end{definition}

\begin{exercise}
\begin{enumerate}
\item $\Gamma_{0}(N)$ is neat for $N \geq 5$.
\item If $\Gamma$ is neat, then it has no elliptic points and $-I \not \in \Gamma$
\end{enumerate}
\end{exercise}

\begin{theorem}
If $\Gamma$ is neat, then for any $k \geq 0$, we have $$\dim_{\CC} M_{k}(\Gamma) = (k-1)(g(X(\Gamma))-1) + \frac{k}{2} \epsilon_{\infty}$$ and $$\dim_{\CC} S_{k}(\Gamma) = (k-1)(g-1) + \left[ \frac{k}{4} \right] \epsilon_{2} + \left[ \frac{k}{3} \right] \epsilon_{3} + \frac{k-1}{2} \epsilon_{\infty}$$
\end{theorem}

\begin{remark}[Fact]
For an integer $r$ as above, we have $\omega_{k + r} = \omega_{k} \otimes \omega_{r}$ \\

In particular, if $r=1$, then $\omega_{r} = \omega_{1}^{\otimes r}  = \omega^{\otimes r}$
\end{remark}

\begin{proof}[Proof of theorem]
\begin{align*}
\deg \omega &= \frac{1}{2} \deg \omega^{\otimes 2} \\
&= \frac{1}{2} \deg (\Omega(\text{cusps})) \\
&= \frac{1}{2} (\deg \Omega + \epsilon_{\infty}) \\
&= g-1 + \frac{\epsilon_{\infty}}{2}
\end{align*}
$\therefore$
\begin{align*}
h^{0}(X(\Gamma), \omega_{k}) - h^{0}(X, \Omega \otimes (\omega^{\otimes k})^{-1}) &= \deg (\omega^{\otimes (-k)}) - g + 1 \\
&= (k-1)(g-1) + \frac{k}{2} \epsilon_{\infty}
\end{align*}
\end{proof}


\chapter{Lecture-10 (9th February, 2023): Automorphic forms, $j$-invariant and Riemann-Roch}

\section{Automorphic forms}

Let $\widehat{\CC}$ be the compactified Riemann sphere $\CC \cup \{\infty\}$. \\

Let $V \subseteq \CC$ be open. $f: V \rightarrow \widehat{\CC}$ is meromorphic if either $f \equiv 0$ or $f$ has a Laurent series expansion 
\begin{equation}
f(z) = \sum_{n=m}^{\infty} a_{n} (z  - \tau)^n \text{ for every } \tau \in V , a_{m} \neq 0
\end{equation}
The order of vanishing of $f$ at $\tau$ is $v_{\tau}(f) = m$. \\

Recall that $f$ is weakly modular if $f: \mathbb{H} \rightarrow \widehat{\CC}$ is meromorphic and $f(\gamma \cdot z) = (cz+d)^k f(z) \; \forall \; \gamma \in \Gamma$ \\

$f$ has a $q$-expansion at $\infty$, namely 
\begin{equation}
f(z) = \sum_{-\infty}^{\infty} a_{n}q_{h}^n  \text{  where }q_{h} = \exp(w\pi z/h)
\end{equation}

$f$ is meromorphic at $\infty$ if $f(z) = \sum_{n=m}^{\infty} a_{n}q_{h}^n$ with $a_{m} \neq 0$ and $v_{\infty}(f) = m$

\begin{definition}
$\Gamma \subseteq \SL_{2}(\ZZ)$ be a congruence subgroup and $k \in \ZZ$. A function $f: \HH \rightarrow \widehat{\CC}$ is an automorphic form  of weight $k$ wrt $\Gamma$ if 
\begin{enumerate}
\item $f$ is meromorphic 
\item $f(\gamma z) = (cz+d)^{k} f(z) \; \forall \; \gamma \in \Gamma , \forall \; z \in \HH$
\item $f|_{k, \gamma}$ is meromorphic at $\infty$ for all $\gamma \in \SL_{2}(\ZZ)$
\end{enumerate}

We denote the space of automorphic forms of weight $k$ and level $\Gamma$ by $\mathcal{A}_{k}(\Gamma)$
\end{definition}

\begin{example}
\begin{equation}
j(z)= 1728 \frac{g_{2}^3}{\Delta (z)}
\end{equation}
Recall that $g_{2}(z)$ is the Eisenstein series of weight $4$ and level $\SL_{2}(\ZZ)$ and $\Delta (z)$ is the cusp form of weight $12$ and level $\SL_{2}(\ZZ)$. 
\end{example}

$\mathcal{A}_{0}(\Gamma)$ is the field of automorphic forms on $X(\Gamma)$. We have already seen that $X(\SL_{2}(\ZZ)) \cong \PP^1 (\CC)$ and the function field of $\PP^1 (\CC) \cong \CC(X)$

\begin{proposition}
$\mathcal{A}_{0}(\SL_{2}(\ZZ)) = \CC(j)$
\end{proposition}

\begin{proof}
Suppose $j \in \mathcal{A}_{0}(\SL_{2}(\ZZ)), \CC(j) \subseteq \mathcal{A}_{0} (\SL_{2}(\ZZ))$. \\
Let $g \in \mathcal{A}_{0}(\SL_{2}(\ZZ))$ and $z_{1}, \hdots , z_{r}$ be zeroes of $g$ and $p_{1}, \hdots , p_{s}$ be the poles of $g$ (with multiplicities). Then define $$f(z) = \frac{\prod_{i=1}^r (j(z) - j(z_{1}))}{\prod_{i=1}^s (j(z) - j(p_{i}))} \in \CC(j)$$
Clearly, $f$ has the same number of zeroes as $g$ on $\HH$. Hence, $f,g$ also have zeros of same order at $\infty$ (for a meromorphic function on $X(\SL_{2}(\ZZ))$ the sum of zeros at all points is $0$). This implies that $g=cf \in \CC(j)$.
\end{proof}

\begin{remark}[Fact]
$\mathcal{A}_{0}(\Gamma_{0}(N)) = \CC(j,j_{N})$ where $j_{N}(z) = j(Nz)$
\end{remark}

We define the order of vanishing of function on $X(\Gamma)$ as follows: \\

Let $\tau \in \HH$ and 
\[\begin{tikzcd}
	{\pi: \mathbb{H}^{*}} & {X(\Gamma)} \\
	& {\mathbb{C}}
	\arrow[from=1-1, to=1-2]
	\arrow["f", from=1-2, to=2-2]
	\arrow["g"', dashed, from=1-1, to=2-2]
\end{tikzcd}\]
Then, 
\begin{eqnarray}
v_{\pi}(\tau) (g) &= \frac{v_{\tau}(f)}{h} \in \QQ
\end{eqnarray}
where $h$ is the order of stabiliser of $\tau$ and this equals $\# (\{\pm I_{2} \Gamma_{\tau}/\{\pm I_{2}\}\})$\\

If $\tau$ is a cusp, $\Gamma_{\infty} = \{I_{2}\} \langle $ and thus we define \\

$v_{\pi (\infty)}(f) = \begin{cases} v_{\infty} (f)/2 & \Gamma_{\infty} = \langle -\begin{pmatrix}
1 & h \\ 0 & 1
\end{pmatrix}  \\ v_{\infty}(f) & \text{ otherwise }\end{cases}$

\section{Riemann-Roch theorem}

Back to Riemann-Roch again. \\

Let $X$ be a compact Riemann surface. A divisor on $X$ is a finite linear combination $\sum_{x \in X(\Gamma)} n_{x}x , n_{x} \in \ZZ$ and $n_{x}=0$ for almost all $x$. The divisors $\mathrm{Div}(X)$ forms a free abelian group on $X$. \\

\begin{eqnarray}
\deg : \mathrm{Div}(X) &\longrightarrow \ZZ \\
\sum n_{x}x &\mapsto \sum n_{x}
\end{eqnarray}

The kernel of this map is denoted by $\mathrm{Div}^{0}(X)$ \\

Next, 
\begin{eqnarray}
\mathrm{div} : \CC(X)^{\times} &\longrightarrow \mathrm{Div}(X) \\
f &\mapsto \sum_{x} v_{x}(f) x
\end{eqnarray}

\begin{exercise}
In fact, $\deg(\mathrm{div}(f))=0$
\end{exercise}

Let $D\in \mathrm{Div}(X)$. We define \begin{equation}
L(D) = \{f \in \CC(X): f=0 \text{ or } \mathrm{div}(f) + D \geq 0\}
\end{equation}

Define $\ell (D) = \dim_{\CC} L (D)$. 

\begin{remark}
A fact to be noted is that $\ell (D) < \infty$.
\end{remark}

\begin{theorem}[Riemann-Roch theorem]
$X$ is a compact Riemann surface of genus $g$. Let $\lambda$ be an $\mathcal{O}_{X}$ module generator of $\Omega_{X}^{1}$ ($\mathrm{div}(\lambda)$ is called the canonical divisor). Then, for any $D \in \mathrm{Div}^{0}(X)$ we have 
\begin{equation}
\ell (D)  - \ell(\mathrm{div}(\lambda) - D) = \deg D - g + 1
\end{equation}
\end{theorem}



\begin{theorem}
Let $k$ be an even integer. Then, 
\begin{equation}
\dim_{\CC} M_{k}(\Gamma) = \begin{cases} (k-1)(g-1) + \left[ \frac{k}{4} \right] \epsilon_{2} + \left[ \frac{k}{3} \right] \epsilon_{3} + \frac{k-1}{2} \epsilon_{\infty} & k \geq 2 \\ 1 & k=0 \\ 0 & k<0 \end{cases}
\end{equation}

\begin{equation}
\dim_{\CC}S_{k}(\Gamma) = \begin{cases} \dim_{\CC} M_{k}(\Gamma) - \epsilon_{\infty}  & k \geq 4 \\ g & k =2 \\ 0 & k \le 0 \end{cases}
\end{equation}
\end{theorem}

\begin{proof}
Note that $\mathrm{Div}_{\QQ}(X(\Gamma))$ is a $\QQ$ vector space generated by points of $X(\Gamma)$. \\

For $\sum n_{x} x \in \mathrm{Div}_{\QQ}(X(\Gamma))$ put 
\begin{equation}
\left[ \sum n_{x} x \right]  = \sum [n_{x}]x
\end{equation}

Take $\omega \in \Omega^{\otimes k/2} (X(\Gamma))$ and $D = \mathrm{div}(\omega)$
\end{proof}

\begin{theorem}
Let $k$ be an odd integer. If $-I_{2} \in \Gamma$, then $M_{k}(\Gamma) = S_{k}(\Gamma)=0$. \\

Assume $-I_{2} \not \in \Gamma$, then 
\begin{equation}
\dim_{\CC} M_{k}(\Gamma) = \begin{cases} (k-1)(g-1) + \left[ \frac{k}{3} \right] \epsilon_{3} + \frac{k}{2} \epsilon_{\infty}^{\mathrm{reg}} + \frac{k-1}{2} \epsilon_{\infty}^{\mathrm{irr}} & k \geq 3 \\ 0 & k<0 \end{cases}
\end{equation} 
and 
\begin{equation}
\dim_{\CC}S_{k}(\Gamma) = \begin{cases} \dim_{\CC} M_{k}(\CC) - \epsilon^{\mathrm{reg}}_{\infty}  & k \geq 3 \\ 0 & k <2  \end{cases}
\end{equation}
\end{theorem}

This leaves us with the case $k=1$.

\chapter{Lecture-11 (14th February, 2023): Cusps of Congruence subgroups}

\textcolor{BrickRed}{Goal}: We want to find the number of cusps of level $\Gamma$ for various congruence subgroups $\Gamma$. \\

Recall that cusps of level $\Gamma$ is the set $\gamma \bs \PP(\QQ)$. We work with $\QQ \times \QQ$ instead of $\PP^{1}(\QQ)$

\begin{lemma}
If $s=a/c , s'=a'/c'$ are two elements of $\QQ \cup \{\infty \}$. Then, $$\begin{pmatrix}
a' \\ c' 
\end{pmatrix} = \pm \gamma \begin{pmatrix}
a \\ c
\end{pmatrix} \Leftrightarrow  s' = \gamma \cdot s$$
Here, the rationals are taken in reduced form.
\end{lemma}

\begin{remark}
Action of $\SL_{2}(\ZZ)/\{\pm \mathbf{Id}_{2}\}$ on $\PP^1(\QQ)$ corresponds to left multiplication by elements of $\SL_{2}(\ZZ)$ on columns $\begin{pmatrix}
a \\ c
\end{pmatrix} \in \ZZ^2$ such that $\gcd(a,c)=1$
\end{remark}

\begin{proposition}
Let $s= a/c , s' = a'/c' \in \PP^1(\QQ)$. Then, 
\begin{enumerate}
\item $\Gamma(N) \cdot s = \Gamma \cdot s' \Leftrightarrow \pm  \begin{pmatrix}
a \\ c
\end{pmatrix} \equiv \begin{pmatrix}
a' \\ c'
\end{pmatrix} \pmod{N}$
\item $\Gamma_{1}(N) \cdot s = \Gamma_{1} \cdot s' \Leftrightarrow \begin{pmatrix}
a+ jc \\ c
\end{pmatrix} \equiv \begin{pmatrix}
a' \\ c'
\end{pmatrix} \pmod{N}$ for some $j$
\item $\Gamma_{0}(N) \cdot s = \Gamma_{0} \cdot s' \Leftrightarrow \pm \begin{pmatrix}
a + jc \\ yc
\end{pmatrix} \equiv \begin{pmatrix}
ya' \\ c'
\end{pmatrix} \pmod{N}$
\end{enumerate}
\end{proposition}

\begin{lemma}
Let $a,c \in \ZZ $ and $\bar{a}, \bar{c}$ be their images in $\ZZ/N\ZZ$. TFAE: 
\begin{enumerate}
\item $\begin{pmatrix}
\bar{a} \\ \bar{c}
\end{pmatrix} \in (\ZZ/N\ZZ)^2$ has a lift $\begin{pmatrix}
a' \\ c'
\end{pmatrix}\in \ZZ^2  $ with $\gcd(a',c')=1$
\item $\gcd(a,c,N)=1$
\item $\begin{pmatrix}
\bar{a} \\ \bar{c}
\end{pmatrix}$ has order $N$ in $(\ZZ/N\ZZ)^2$.
\end{enumerate}
\end{lemma}

\begin{proof}
$1 \Leftrightarrow 2$ has been done earlier. \\
Now let us show $3 \Leftrightarrow 2$.\\
Suppose $k\begin{pmatrix}
a \\ c
\end{pmatrix}  \equiv \begin{pmatrix}
0 \\ 0
\end{pmatrix} \pmod{N}$. Then, $ka \equiv 0 \equiv kc \pmod{N} \Rightarrow k\gcd(a,c)\equiv 0 \pmod{N}$. We know that $N\mid k$ and this can happen iff no non-trivial factors of $N$ divides $\gcd(a,c)$ or equivalently $\gcd(a,c,N)=1$. 
\end{proof}

\begin{proposition}
\begin{eqnarray*}
\#\mathrm{Cusps}(\Gamma(N)) &= \begin{cases} \frac{1}{2} \sum_{d \mid N} \frac{N}{d} \varphi(d)\varphi(N/d) \\
= \frac{1}{2} \prod_{p \mid N} \left( 1 - \frac{1}{p^2} \right) & \text{ if } N>2 \\ 3 & \text{ if } N=2  \end{cases}
\end{eqnarray*}
\end{proposition}

\begin{proof}

\end{proof}

\chapter{Lecture-12 (16th February, 2023): More about cusps}






\part{Elliptic Curves}

\chapter{Lecture-1 (3rd January): Introduction }

\chapter{Lecture-2 (5th January, 2023): Affine varieties }

\section{Affine Varieties}

Suppose $k$ is a perfect field (every extension is separable) like $Q$ any field of characteristic $0$ or any finite field. \\
Let $G_{\bar{k}/k}$ be the Galois group of the extension $\bar{k}/k$ where $\bar{k}$ is the algebraic closure of $k$.

\begin{definition}
An afffine $n$-space over $k$ denoted by $\Aa^n(\bar{k})$ or $\Aa^n$ is the set 
\begin{equation}
\Aa^n = \Aa^n(\bar{k}) = \{P=(x_{1}, \hdots , x_{n}): x_{i} \in \bar{k}\}
\end{equation}
The set of $k$-rational points of $\Aa^n$ is the set 
\begin{equation}
\Aa^n = \Aa^n(k) = \{P=(x_{1}, \hdots , x_{n}): x_{i} \in k\}
\end{equation}
\end{definition} 

The Galois group $G_{\bar{k}/k}$ acts on $\Aa^n$ as follows: Take $\sigma \in \mathrm{Gal}(\bar{k}/k)$ then 
\begin{equation}
\sigma P = (\sigma x_{1}, \hdots , \sigma x_{n})
\end{equation}

Therefore, $k$-rational points can also be realised as 

\begin{eqnarray}
\Aa^n (k) = \{P \in \Aa^n : \sigma P = P \; \forall \; \sigma \in G_{\bar{k}/k}\}
\end{eqnarray}

\begin{definition}
Let $\bar{k}[X_{1}, \hdots , X_{n}]$ be the polynomial ring and $I$ be an ideal in the ring. Then, we associate a set $I$ to this ideal as follows: 
\begin{equation}
V_{I} = \{P \in \Aa^n : f(P)=0 \; \forall \; f\in I\}
\end{equation}
An affine algebraic set is a set of the form $V_{I}$ for some ideal $I$.
\end{definition}

\begin{theorem}[Hilbert Basis theorem]
All ideals of $k[\mathbf{X}]$ is finitely generated.
\end{theorem}

\begin{definition}
If $V$ is an algebraic set, then we associate an ideal $I(V)$ as 
\begin{equation}
I(V) = \{f \in \bar{k}[\mathbf{X}]: f(P)=0 \; \forall \; P \in V\}
\end{equation}
An algebraic set is said to be defined over $k$ if its ideal $I(V)$ is generated by polynomials in $k[\mathbf{X}]$ and we denote this by $V/k$. If $V$ is defined over $k$, then the set of $k$-rational points is just $$V(k) = V \cap \Aa^n (k)$$
\end{definition}

\begin{remark}
Let $V$ be an algebraic set and consider the ideal $I(V/k)$ defined by 
\begin{equation}
I(V/k) = \{f \in k [\mathbf{X}]: f(P)=0 \; \forall\; P \in V\} = I(V) \cap k[\mathbf{X}]
\end{equation}
Then we see that $V$ is defined over $k$ iff $$I(V) = I(V/k)\bar{k}[\mathbf{X}]$$ 
\end{remark}

If $V$ is defined over $k$ and $f_{1}, \hdots , f_{r} \in k[\mathbf{X}]$ be the generators of $I(V/k)$. Then, $V(k)$ is precisely the set of solutions $P=(x_{1}, \hdots , x_{n})$ to the simultaneous polynomial equations 

\begin{equation}
f_{1}(P) = \cdots = f_{r}(P)=0 \text{ where } x_{i} \in k
\end{equation}

Note that if $f(\mathbf{X}) \in k[\mathbf{X}] , P =(x_{1}, \hdots , x_{n})$ and $\sigma \in G_{\bar{k}/k}$, then 
\begin{equation}
f(\sigma P) = a_{0} + a_{1} \sigma P + \cdots + a_{m} (\sigma P)^m = \sigma f(P)
\end{equation}

Hence, if $V$ is defined over $k$, then the action of $G_{\bar{k}/k}$ on $\Aa^n$ induces an action on $V$ and clearly, 
\begin{equation}
V(k) = \{P \in V : \sigma P = P \; \forall \; \sigma \in G_{\bar{k}/k}\}
\end{equation}

\begin{example}
Let $V= \{(x,y)\in \Aa^2 : x^2 -y^2=1\}$. Then, $I(V) = \langle X^2 - Y^2 -1 \rangle$ and therefore $V$ is defined over $k$. 
\end{example}

\begin{example}
The algebraic set $V: X^n + Y^n =1$ is defined over $\QQ$.
\end{example}

\begin{example}
The algebraic set $V: Y^2 = X^{3} + 17$ has many $\QQ$-rational points (infact infinitely many). 
\end{example}

\begin{definition}
An affine algebraic set $V$ is called an affine variety if $I(V)$ is prime ideal in $\bar{k}$. 
\end{definition}
\begin{remark}
It is not enough to check that $I(V/k)$ is prime in $k[\mathbf{X}]$ to conclude whether $I(V)$ is prime in $\bar{k}[\mathbf{X}]$. For example, consider the ideal $\langle X^2 - 2Y^2 \rangle $ in $\QQ[X,Y]$
\end{remark}

\begin{definition}
Let $V/k$ be a variety defined over $k$. Then the affine coordinate ring of $V/k$ is defined by 
\begin{equation}
k[V] = \frac{k[\mathbf{X}]}{I(V/k)}
\end{equation}
The ring $k[V]$ is an integral domain. Its quotient field is denoted by $k(V)$ and is called the function field of $V/k$. Similarly, $\bar{k}(V)$ and $k(V)$ are defined by replacing $k$ by $\bar{k}$.
\end{definition}


\begin{definition}
Let $V$ be a variety. The dimension of $V$, denoted by $\dim (V)$ is the transcendence degree of $\bar{k}(V)$ over $\bar{k}$.
\end{definition}

\begin{example}
The dimension of $\Aa^n$ is $n$ since $\bar{k}(\Aa^n) = \bar{k}(X_{1},\hdots , X_{n})$. Similarly, if $V \subseteq \Aa^n$ is given by a single polynomial $f$, then the dimension of $V$ is $n-1$,
\end{example}

\begin{definition}
Let $V$ be a variety, $P \in V$ and $f_{1}, \hdots , f_{m} \in \bar{k}[\mathbf{X}]$ a set of generators for $I(V)$. Then, $V$ is nonsingular (or smooth) at $P$ if the $m \times n$ matrix $$\left( \frac{\partial f_{i}}{\partial X_{j}} \right)_{1 \le i \le m \\ 1 \le j \le n}$$ has rank $n-\dim V$. If $V$ is non-singular at every point, then we say that $V$ is non-singular or smooth.
\end{definition}

\begin{example}
Let $V$ be a variety corresponding to a single nonconstant polynomial $f$. Then, $\dim V =n-1$ and $P \in V$ is regular iff 
\begin{equation}
\frac{\partial f}{\partial X_{1}}(P) = \cdots = \frac{\partial f}{\partial X_{n}}(P) = 0
\end{equation}
\end{example}

\begin{example}
Consider the two varieties $V_{1}: Y^2 = X^3 + X , V_{2}: Y^2 = X^3 + X^2$. Then, from previous example the singular points on $V_{1},V_{2}$ are given by $V_{1}^{\text{sing}}: 3X^2 + 1 = 2Y=0$ and $V_{2}^{\text{sing}}: 3X^2 + 2X = 2Y=0$. Clearly, $V_{1}$ is non-singular and $V_{2}$ has one singular point $(0,0)$.
\end{example}

We also have another way to determine smoothness using the functions on the variety $V$. 

\begin{definition}
For each $P \in V$, we define an ideal $M_{P}$ of $\bar{k}[V]$ by $$M_{P} = \{f \in \bar{k}[V]: f(P)=0\}$$ This ideal is a maximal ideal as it is the kernel of the map 
\begin{align*}
\bar{k}[V] &\rightarrow \bar{k} \\
f &\mapsto f(P)
\end{align*}
\end{definition}

The quotient $M_{P}/M_{P}^2$ is a finite dimensional $\bar{k}$-vector space. 
\begin{proposition}
Let $V$ be a variety. A point $P \in V$ is non-singular iff $\dim_{\bar{k}} M_{P}/M_{P}^2 = \dim V$. 
\end{proposition}

\begin{example}
Consider the point $P = (0,0)$ and the two varieties $V_{1},V_{2}$ as in previous example. Then in both the cases $M_{P}$ is generated by $X,Y$ which implies $M_{P}^2$ is ideal generated by $X^2 , XY, Y^2$. \\

For $V_{1}$, $X=Y^2 - X^3 \equiv 0 \pmod{M_{P}^2}$ and thus $M_{P}/M_{P}^2$ is generated by $Y$ alone. \\

For $V_{2}$, there is no nontrivial relationship between $X,Y$ modulo $M_{P}^2$ therefore $M_{P}/M_{P}^2$ requires both $X,Y$ as generators. Since $\dim V_{i} = 1$ therefore $V_{1}$ is smooth but $V_{2}$ is not. 
\end{example}

\begin{proposition}
The local ring of $V$ at $P$, denoted by $\bar{k}[V]_{P}$ is the localisation of $\bar{k}[V]$ at $M_{P}$. 
\begin{equation}
\bar{k}[V]_{P} = \{F \in \bar{k}(V): F = f/g \text{ for some } f,g \in \bar{k}[V], g(P)\neq 0\}
\end{equation}
The functions in $\bar{k}[V]_{P}$ are called regular functions at $P$.
\end{proposition}

\chapter{Lecture-3 (10 January, 2023): Projective varieties}

\section{Projective varieties}

\begin{definition}
A Projective $n$-space over $k$ denoted by $\mathbb{P}^n$ or $\mathbb{P}^{n}(\bar{K})$ is the set $\mathbb{A}^{n+1} \backslash \{(0,\hdots , 0)\} / \sim$ with  
$$(x_{0}, \hdots ,x_{n}) \sim (y_{0},\hdots ,y_{n})$$ iff $\exists \lambda \in \bar{k}^{\times}$ such that $(y_{0},\hdots ,y_{n}) = (\lambda x_{0},\hdots ,\lambda x_{n})$\\
The equivalence class $(x_{0},\hdots ,x_{n+1})$ is denoted by $[x_{0},\hdots ,x_{n}]$\\
The set of $k$-rational points of $\mathbb{P}^n$ is $$\mathbb{P}^n = \{[x_{0}, \hdots , x_{n}]\; \mid \; x_{i} \in k\}$$
\end{definition}

\textcolor{red}{Caution}: If $p = [x_{0},\hdots ,x_{n}] \in \mathbb{P}^n(k)$ and $x_{i} \neq 0$ for some $i$, then $x_{j}/x_{i} \in k \forall j$

\begin{definition}
Let $p = [x_{0},\hdots ,x_{n}] \in \mathbb{P}^n(\bar{k})$. The minimal field of definition for $p$ is the field $$k(p) = k(x_{0}/x_{i},\hdots ,x_{n}/x_{i}) \text{ for any } i \text{ such that } x_{i} \neq 0$$
$k(p) \frac{x_{i}}{x_{j}} = k(x_{0}/x_{j},\hdots ,x_{n}/x_{j})$ is the same as $k(p)$ as $x_{i}/x_{j} \in k(p)$
\end{definition}  

For $\sigma \in G(\bar{k}/k)$ and $p = [x_{0},\hdots ,x_{n}] \in \mathbb{P}^n$, we have the following action $$\sigma(p) = [\sigma(x_{0}),\hdots ,\sigma(x_{n})]$$
This action is well defined as $$\sigma(\lambda p) = [\sigma(\lambda)\sigma(x_{0}),\hdots ,\sigma(\lambda) \sigma(x_{n})] \sim [\sigma(x_{0}),\hdots ,\sigma(x_{n})]$$

\begin{definition}
A polynomial $f \in \bar{k}[X_{0}, \hdots , X_{n}]$ is homogenous of degree $d$ if $$f(\lambda x_{0} , \hdots , \lambda x_{n})  = \lambda^d f(x_{0} , \hdots ,x_{n}) \forall \lambda \in \bar{k}$$
\end{definition}

\begin{definition}
An ideal $I \subseteq \bar{k}[X_{0} , \hdots , X_{n}]$ is called a homogenous ideal if it is generated by homogenous polynomial.  
\end{definition}

\begin{definition}
Let $I \subseteq \bar{k}[X_{0} , \hdots , X_{n}]$ be a homogenous ideal. Then, $$V(I) = \{p \in \mathbb{P}^n (\bar{k}) \mid f(p)=0 \; \forall \; f\in I\}$$
\end{definition}

\begin{definition}
\begin{itemize}
\item A projective algebraic set is any set of the form $V(I)$ for some homogenous ideal $I$. 
\item If $V$ is a projective algebraic set, the homogenous ideal of $V$, denoted by $I(V)$ is the ideal of $\bar{k}[X_{0} \hdots , X_{n}]$ generated by $\{f \in \bar{k}[X_{0} \hdots , X_{n}]\mid f \text{ is homogenous and }f(p)= 0 \; \forall \; p\in V\}$
\item Such a $V$ is defined over $k$, denoted by $V/k$ if its ideal $I(V)$ can be generated by homogenous polynomials $k[X_{0} \hdots , X_{n}]$. 
\item If $V$ is defined over $k$, then the set of $k$-rational points of $V$ is $$V(k) = V \cap \mathbb{P}^{n}(k) = \{p \in V \mid \sigma(p)=p \; \forall \; \sigma \in G(\bar{k}/k)\}$$
\end{itemize}
\end{definition}

\begin{example}
A line in $\mathbb{P}^2$ is given by the equation $aX + bY + cZ=0$ with $a,b,c \in \bar{k}$ and not all $0$ simultaneously. \\
If $c\neq 0$, then such a line is defined over a field containing $a/c, b/c$. \\
More generally, a hyperplane in $\mathbb{P}^n$ is given by an equation $a_{0}X_{0} + \cdots + a_{n}X_{n} = 0$ with all $a_{i} \neq 0$ simultaneously.
\end{example}

\begin{example}
Let $V$ be the projective algebraic set in $\mathbb{P}^2$ given by $X^2 + Y^2 = Z^2$. 
\begin{align*}
\mathbb{P}^1 & \overbrace{\rightarrow}^{\sim} V \\
[s,t] & \mapsto [s^2 - t^2 : 2st : s^2+t^2]
\end{align*}
\end{example}

\begin{remark}
For $p \in \mathbb{P}^n(\QQ)$ you can clear the denominators and then divide by common factor so that $x_{i} \in \ZZ$ and $\gcd(x_{0} , \hdots ,x_{n})=1$. So, $I =(f_{1}, \hdots , f_{m})$ and finding a rational point of $V_{I}$ is same as finding coprime integer solutions to $f_{i}'s$. 
\end{remark}

\begin{example}
$V \subseteq \mathbb{P}^2$ such that $X^2 + Y^2 = 3Z^2$ over $\QQ$. To find $V(\QQ)$, we just need to find integers $a,b,c$ such that $a^2 +b^2 = 3c^2$
\end{example}

\begin{example}
$V : 3X^3 + 4Y^3 + 5Z^3 = 0$. $V(\QQ) = \emptyset$ but for all prime $p$ we have $V(\QQ_{p}) \neq \emptyset$
\end{example}

\begin{definition}
A projective algebraic set is called a projective variety if its homogenous ideal $I(V)$ is prime $\bar{k}[X_{0}, \hdots , X_{n}]$
\end{definition}

Relation between affine and projective varieties: \\

For $ 0 \le i \le n$
\begin{center}
\begin{align*}
\phi_{i}: \mathbb{A}^n &\rightarrow \mathbb{P}^n \\
(Y_{1}, \hdots , Y_{n}) &\mapsto [Y_{1}, \hdots , Y_{i-1}, 1, Y_{i+1} , \hdots , Y_{n}]
\end{align*}
\end{center}
$\mathrm{Im}(\phi) = U_{i} = \{p \in \mathbb{P}^n \mid \; p =[x_{0}: \hdots  : x_{n}] \text{ with } x_{i} \neq 0\} = \mathbb{P}^n \backslash H_{i}$. \\
This process can also be reversed by the following map : 
\begin{center}
\begin{align*}
\phi_{i}^{-1}: U_{i} &\rightarrow \mathbb{A}^n \\
[x_{0}: \hdots : x_{n}) &\mapsto [x_{0}/x_{i}, \hdots , x_{i-1}/x_{i}, x_{i+1}/x_{i} , \hdots , x_{n}/x_{i}]
\end{align*}
\end{center}

Let $V$ be a projective algebraic set with homogenous ideal $I(V) \subseteq \bar{k}[X_{0} ,\hdots , X_{n}]$. Then, $$V \cap \mathbb{A}^n = \phi_{i}^{-1}(V \cap U_{i}) \text{ for fixed } i $$ is an affine algebraic set with $I(V \cap \mathbb{A}^n ) \subset \bar{k}[X_{0} , \hdots , X_{i-1},X_{i+1}, \hdots , X_{n}]$

\begin{definition}
Let $V \subseteq \mathbb{A}^n$ be an affine algebraic set with ideal $I(V)$ and consider $V \subseteq \mathbb{P}^n$ and $\phi_{i}$ defined as before. \\
The projective closure of $V$ is $\bar{V}$ is the projective algebraic set whose homogenous ideal $I(V)$ is generated by $\{f^* \mid f \in I(V)\}$. \\
Here, for $f \in k[X_{0}, \hdots , X_{i-1}, X_{i+1}, \hdots , X_{n}]$ we define $$f^* (X_{0} , \hdots , X_{n}) = X_{i}^d (f(X_{0}/X_{i}, \hdots , X_{i-1}/X_{i}, X_{i+1}/X_{i}, \hdots , X_{n}/X_{i}))$$ with $d = \deg (f)$.
\end{definition}

\begin{definition}
Dehomogenization of $f(X_{0}, \hdots , X_{n})$ with respect to $i$ is $f(X_{0}, \hdots , X_{i-1}, 1 , X_{i+1}, \hdots ,X_{n})$
\end{definition}

\begin{proposition}
\begin{enumerate}
\item Let $V$ be an affine variety. Then $\bar{V}$ is a projective variety and $V = \bar{V} \cap \mathbb{A}^n$. 
\item Let $V$ be a projective variety. Then, $V \cap \mathbb{A}^n$ is an affine variety and either $V \cap \mathbb{A}^n = \emptyset$ or $V = \overline{V \cap \mathbb{A}^n}$. 
\item If an affine (resp. projective) variety $V$ is defined over $k$, then $\bar{V}$ (resp. $V \cap \mathbb{A}^n$) is also defined over $k$.
\end{enumerate}
\end{proposition}

\begin{proof}
\begin{enumerate}
\item 

\item

\item
\end{enumerate}
\end{proof}

\begin{example}
$V : Y^2 = X^3 + 17 \subseteq \mathbb{A}^2 \rightarrow \mathbb{P}^2$ with $(X,Y) \mapsto [X: Y: 1]$. Here, $\overline{V} : Y^2Z =X^3 +17Z^3$ and $\overline{V} \backslash V = \{[0:1:0]\}$
\end{example}

\chapter{Lecture-4 (12th January, 2023): Projective varieties and maps between varieties}

\section{Projective varieties contd..}

\begin{definition}
\begin{itemize}
\item Let $Y / k$ be a projective variety and choose $\Aa^n \subseteq \PP^n$ such that $V \cap \Aa^n \neq \emptyset$. The dimension of $V$ is just dimension of $V \cap \Aa^n$.
\item The function field of $V$, $\bar{k}(V) = \bar{k}(V \cap \Aa^n)$ is the function field for $V \cap \Aa^n$ over $\bar{k}$.
\item Similarly, $k(V) = k(V \cap \Aa^n)$
\begin{align*}
\phi_{i}: \Aa^n \rightarrow \PP^n &\I(V \cap \Aa_{i}^n) \\
\phi_{j}: \Aa^n \rightarrow \PP^n &\I(V \cap \Aa_{j}^n) \\ 
\end{align*}
For different $\phi_{i}$ we obtain $k(V)$s but they are canonically isomorphic to each other. This is because we can just switch $x_{i},x_{j}$ are dehomogenise accordingly. 
\end{itemize}
\end{definition}

\begin{definition}
Let $V$ be a projective variety and $p \in V$. Choose $\Aa^n \subseteq \PP^n$ with $p \in \Aa^n$. Then, $V$ is non-singular (or smooth) at $p$ if $V\cap \Aa^n$ is non-singular at $p$. \\

The local ring of $V$ at $p$, $\bar{k}[V]_{p}$ is just the local ring of $\bar{k}[V \cap \Aa^n]_{p}$
\end{definition}

\begin{remark}
Function field of a projective variety $V$ is field of rational functions $f(X)/g(X)$ such that 
\begin{enumerate}
\item $f,g$ are homogenous of same degree. 
\item $g \in \I(V)$.
\item $f_{1}/g_{1} = f_{2}/g_{2}$ iff $f_{1}g_{2} - f_{2}g_{1} \in \I(V)$ 
\end{enumerate}
Equivalently, take $f,g \in \bar{k}[X]/I(V)$ satisfying $1,2$.\\

\textcolor{BrickRed}{Here, $X$ is just a short form for $(X_{0}, \hdots , X_{n})$}
\end{remark}

\section{Maps between varieties}

\begin{definition}
Let $V_{1},V_{2}\in \PP^n$ be projective varieties. A rational map $$\phi : V_{1} \rightarrow V_{2}$$ $\phi = [f_{0} : \cdots : f_{n}]$ where $f_{i} \in \bar{k}(V_{1})$ such that $\forall p \in V_{1}$ at which $f_{i}$ are defined, we have $$\phi(p)= [f_{0}(p): \cdots : f_{n}(p)]$$
\end{definition}

If $V_{1},V_{2}$ are defined over $k$, we have a Galois action. For $\sigma \in G(\bar{k}/k)$ we have $$\sigma(\phi)(p) = [\sigma(f_{0}): \cdots : \sigma(f_{n})(p)]$$
We can check that $\sigma(\phi(p)) = \sigma(\phi)(\sigma(p))$.\\

\begin{definition}
If $\exists \lambda \in \bar{k}^{\times}$ such that $\lambda f_{i} \in k(V_{1})$, then $\phi$ is said to be defined over $k$.
\end{definition}

\begin{proposition}
$\phi$ is defined over $k$ iff $\phi =\sigma(\phi) \; \forall \; \sigma \in G(\bar{k}/k)$.
\end{proposition}

\begin{definition}
A rational map $\phi: V_{1} \rightarrow V_{2}$ is said to be regular if there exists a function $g \in \bar{k}(V_{1})$ such that 
\begin{enumerate}
\item Each $gf_{i}$ is regular at $p$.
\item There exists some $i$ such that $(gf_{i})(p) \neq 0$
\end{enumerate}
If such a $g$ exists, then we set $$\phi(p) = [(gf_{0})(p) : \cdots : (gf_{n})(p)]$$
\end{definition}

\begin{definition}
A rational map is called a morphism if it is regular everywhere.
\end{definition}

\begin{remark}
Let $V_{1},V_{2} \in \PP^n$ be projective varieties. \\
$\bar{k}(V_{1}) = $ quotient of homogenous polynomials in $\bar{k}[X]$ of same degree.\\ 
A rational map $\phi = [f_{0}, \hdots , f_{n}]$ can be multiplied by a homogenous polynomial to clear denominators and get $\phi = [\phi_{0} , \hdots ,  \phi_{n}]$ such that 
\begin{enumerate}
\item $\phi_{i} \in \bar{k}[X]$ homogenous polynomials not all in $\I(V_{1})$ and have same degree.
\item For all $f \in \I(V_{2})$ we have $f(\phi_{0}(X), \hdots , \phi_{n}(X)) \in \I(V_{1})$. 
\end{enumerate}
\end{remark}

\begin{definition}
A rational map $\phi = [\phi_{0} , \hdots , \phi_{n}]: V_{1} \rightarrow V_{2}$ as  above is regular at $p \in V_{1}$ if there exists homogenous polynomials $\psi_{0},\hdots , \psi_{n} \in \bar{k}[X]$ such that 
\begin{enumerate}
\item $\psi_{i}$s have the same degree 
\item $\phi_{i}\psi_{j} \equiv \phi_{j}\psi_{j} \pmod{\I(V_{1})}$ for all $0  \le i,j \le n$
\item $\psi_{i}(p) \neq 0$ for some $i$.
\end{enumerate}
If this happens, we set $$\phi(p) = [\psi_{0}(p), \hdots , \psi_{n}(p)]$$
\end{definition}

\begin{remark}
Let $\phi= [\phi_{0}, \hdots , \phi_{n}]: \PP^m \rightarrow \PP^n$ be a rational map. $\phi_{i}$s are homogenous polynomials having same degree. We can cancel common factors to assume $\gcd(\phi_{0} , \hdots , \phi_{n})=1$. \\

And, $\phi$ is regular at a point $p \in \PP^n$ iff $\phi_{i}(p) \neq 0$ for some $i$.  \\
So, $\phi$ is a morphism if $\phi_{i}$s have no common zeros in $\PP^n$. 
\end{remark}

\begin{definition}
Let $V_{1},V_{2}$ be two projective varieties. We say that $V_{1},V_{2}$ are isomorphic if there are morphisms $$\phi: V_{1}\rightarrow V_{2}, \psi : V_{2} \rightarrow V_{1}$$ such that $\phi \circ \psi = \mathrm{id}_{V_{2}}, \psi \circ \phi = \mathrm{id}_{V_{1}}$. \\
$V_{1}/k$ and $V_{2}/k$ are isomorphic over $k$ if both maps are defined over $k$.
\end{definition}

\begin{example}
$\mathrm{char}(k) \neq 2$, $V: X^2 + Y^2 = Z^2$. 
\begin{align*}
\phi: V &\rightarrow \PP^2 \\
[X:Y:Z] &\mapsto [X+Z:Y]
\end{align*}
$\phi$ is regular everywhere except $[1:0:1]$\\
Since $(X+Z)(X-Z) \equiv -Y^2 \equiv \pmod{\I(V)}$, we have $[X+Z:Y] = [X^2-Z^2 : Y(X-Z)] = [-Y^2 : Y(X-Z)] = [-Y: X-Z] = \psi$ \\
\begin{align*}
\psi: \PP^1 &\rightarrow V \\
[s:t] &\rightarrow [s^2-t^2 : 2st : s^2 +t^2]
\end{align*}
$\psi\circ \phi$ and $\phi \circ \psi$ are both identity maps.
\end{example}

\begin{example}
\begin{align*}
\phi: \PP^2 &\rightarrow \PP^2 \\
[X:Y:Z] &\mapsto [X^2 : YZ : Z^2]
\end{align*}
 is regular everywhere but $[0:1:0]$ and this cannot be salvaged.
\end{example}

\begin{example}
$V: Y^2Z = X^3 + X^2 Z$
\begin{align*}
\psi: \PP^1 &\rightarrow V \\
[s:t] &\mapsto [(s^2 -t^2)t : (s^2 -t^2)s : t^3]
\phi: V &\rightarrow \PP^1 \\
[X:Y:Z] &\mapsto [X:Y]
\end{align*}
$\phi$ is not regular at $[0:0:1]$. $[0:0:1]$ is a singular point of $V$ which implies $\phi$ cannot be extended. So $\phi \circ \psi$ and $\psi \circ  \phi$ are identities  when they are defined.
\end{example}

\begin{example}
$V_{1}: X^2 + Y^2 = Z^2 , V_{2}: X^2 + Y^2 = 3Z^2$. $V_{1} \not \cong V_{2}$ over $\QQ$ but $V_{1}\cong V_{2}$ over $\QQ(\sqrt{3})$.
\end{example}

\chapter{Lecture-5 (17th January, 2023): Algebraic curves}

\section{Curves}

\begin{definition}
A curve is a projective variety of dimension $1$.
\end{definition}

\begin{example}
Vanishing set of an irreducible polynomial in $\PP^2$.
\end{example}

\begin{proposition}
Let $C$ be a curve and $p \in C$ be a smooth (non-singular) point. Then, $\bar{k}[C]_{p}$ is a discrete valuation ring.
\end{proposition}

\begin{proof}
$p \in C$ smooth implies $M_{p}/M_{p}^2$ is one dimensional over $\bar{k}[C]_{p}/M_{p} = \bar{k}$. Now, Nakayama will give us $M_{p}$ is a principal ideal. \\

\textbf{Claim}:
$\bigcap_{n} M_{p}^n = 0$.

\begin{proof}
If $\alpha \in \bigcap_{n} M_{p}^n$, then $\alpha = a_{1}t = a_{2} t^2 = a_{3} t^3 = \cdots $. This implies $a_{1} = a_{2}t = a_{3}t^2 = \cdots $. But this gives us a chain $$\langle a_{1} \rangle \subseteq \langle a_{2} \rangle \subseteq \langle a_{3} \rangle \subseteq \cdots $$ that must terminate at some point. This implies $t$ is an unit which is a contradiction. Hence, we are done.
\end{proof}
\end{proof}

\begin{definition}
Let $C$ be a curve and $p \in C$ a smooth point. The normalised valuation on $\bar{k}[C]_{p}$ is 
\begin{align*}
\mathrm{ord}_{p}: \bar{k}[C]_{p} &\rightarrow \NN \cup \{0, \infty\} \\
f &\mapsto \sup \{d \in \ZZ \mid f \in M_{p}^d \} \\
\mathrm{ord}_{p}(\frac{f}{g}) &= \mathrm{ord}_{p}(f) - \mathrm{ord}_{p}(g)
\end{align*}
Thus we can define $$\mathrm{ord}_{p} : \bar{k}[C] \rightarrow \ZZ \cup \{\infty\}$$
\end{definition}

\begin{definition}
An uniformiser for $C$ at $p$ is any function $t \in \bar{k}(C)$ with $\mathrm{ord}_{p}(t) = 1$ that is the generator of $M_{p}$
\end{definition}

\begin{remark}
If $C$ is defined over $k$, we can find a unit $t \in k(C)$.
\end{remark}

\begin{definition}
Let $C$ be a curve and $p \in C$ a smooth point, $f \in \bar{k}(C)$, $\ord_{p}(f) =$ order of $f$ at $p$.

\begin{enumerate}
\item If $\ord_{p}(f)> 0$, then $f$ has a zero at $p$.
\item If $\ord_{p}(f)< 0$, then $f$ has a pole at $p$.
\item If $\ord_{p}(f)\geq 0$, then $f$ is regular at $p$.
\end{enumerate}
\end{definition}

\begin{proposition}
Let $C$ be a smooth curve and $0 \neq f \in \bar{k}(C)$. Then, there are only finitely many points in $C$ at which $f$ has a pole or $0$. If $f$ has no poles, then $f\in \bar{k}$.
\end{proposition}

\begin{proof}
A standard exercise in Riemann surfaces.
\end{proof}

\begin{example}
Suppose $C_{1}: Y^2 = X^3 + X , C_{2}: Y^2 = X^3 + X^2$. $C_{1}$ is smooth everywhere but $C_{2}$ is smooth everywhere except $p = [0:0:1]$.\\

In $\bar{k}[C_{1}]_{p}$, $M_{p} = \langle X,Y \rangle$ and $X \in M_{p}^2$.
\end{example}

\begin{proposition}
Let $C/k$ be a curve and $p \in C$ be a smooth point, and $t \in k(C)$ an uniformiser at $p$. Then, $k(C)$ is a finite separable extension of $k(t)$.
\end{proposition}

\begin{proof}
$k(C)$ is a finite algebraic extension as it is finitely generated over $k$ and has transcendence degree $0$ over $k(t)$ as $t$ is not algebraic over $k$ (it is a local coordinate of $C$ at $p$). \\

Now, take $x\in k(C)$ and let $\Phi(T,X) = \sum a_{ij}T^{i}X^{j}$ be the minimal polynomial at $x$ over $k(t)$. Say $q = \mathrm{char}(k)$. If $\Phi (T,X)$ is not separable, then $\frac{\partial \Phi(T,X)}{\partial X} = 0$ as $\Phi(T,X)$ is irreducible. 

\begin{align*}
\Phi(T,X) &= \Psi (T,X^p) \\
&= \sum_{k=0}^{q-1} \left( \sum_{i,j} b_{ijk} T^{iq} X^{iq} \right) T^k \\
&= \sum_{k=0}^{q-1} \left( \Phi_{k}(T,X) \right)^q T^k \text{     since $k$ is perfect, every element is a $q$-th power} \\
\sum_{k=0}^{q-1} \left( \Phi_{k}(t,x) \right)^q t^k &=0\\
\ord_{p} (\Phi_{k}(t,x)^q t^k) &\equiv k \pmod{q}
\end{align*}
This implies that every term in the final sum has distinct order at $p$. And, hence $$\Phi_{0}(t,x) = \Phi_{1}(t,x) = \cdots = \Phi_{q-1}(t,x) = 0$$ Atleast one of the $\Phi_{i}$s should have a nonzero power of $X$ and $X - \deg \Phi_{i} < X - \deg \Phi$ and hence $\Phi_{k}(t,x) =0$ which contradicts minimality of $\Phi$. Hence, we are done.
\end{proof}

\section{Morphism between curves}

\begin{proposition}
Let $C$ be a curve, $V \subseteq \PP^n$ be a variety, $p \in C$ a smooth point and $$\phi : C \rightarrow V$$ a rational map. Then, $\phi$ is regular at $p$. In particular, if $C$ is smooth, then $\phi$ is a morphism.
\end{proposition}

\begin{proof}
Suppose $\phi = [f_{0} : \cdots : f_{n}]$ with $f_{i} \bar{k}(C)$ and $t \in \bar{k}(C)$ an uniformiser for $C$ at $p$. Let $$n = \min \ord_{p} f_{i}$$ Then, $\ord_{p}(t^{-n}f_{i}) \geq 0 \; \forall \; i$ and $\ord_{p}(t^{-n}f_{j})=0$ for some $j$. But then this means $t^{-n}f_{i}$ are regular at $p$, $t^{-n}f_{j}(p) \neq 0$ and thus $\phi$ is regular at $p$.
\end{proof}

\begin{remark}
This proposition is not true if either $\dim(C) > 1$ or $p$ is singular 
\begin{enumerate}
\item $\phi: \PP^n \rightarrow \PP^n$ be $[X:Y:Z] \mapsto [X^2 : YZ : Z^2]$ is not regular at $p = [0:1:0]$.
\item Suppose $V : Y^2 Z = X^3 + X^2 Z$ and $V \rightarrow \PP^1$ be given by $[X:Y:Z] \mapsto [Y:X]$ is not regular at $[0:0:1]$.
\end{enumerate}
\end{remark}

\begin{example}
\begin{enumerate}
\item $V: X^2 + Y^2 = Z^2$
\end{enumerate}
\end{example}


\chapter{Lecture-6 (19th January, 2023): Morphisms between curves, ramification and Frobenius map}

\section{Morphisms between curves contd..}

\begin{theorem}
If $\phi: C_{1} \rightarrow C_{2}$ are morphisms of curves, then $\phi$ is either surjective or constant.
\end{theorem}

Let $C_{1},C_{2}/k$ be two curves over $k$ and $\phi: C_{1},C_{2}$ nonconstant rational map defined over $k$. Then, we obtain a map 
\begin{equation*}
\phi^* : k(C_{2}) \rightarrow k(C_{1}) \text{ such that } f \mapsto f \circ \phi
\end{equation*} 

\begin{theorem}
Let $C_{1},C_{2}$ be two curves defined over $k$. 
\begin{enumerate}
\item Suppose $\phi: C_{1} \rightarrow C_{2}$ be two nonconstant rational maps defined over $k$. Then $k(C_{1})$ is a finite extension of $\phi^{*}(k(C_{2}))$. 
\item Suppose $i : k(C_{2}) \rightarrow k(C_{1})$ is an injection of function fields. Then there exists an unique rational map $\phi : C_{1} \rightarrow C_{2}$ defined over $k$ such that $\phi^{*} = i$.
\item Suppose $\mathbb{K} \subseteq k(C_{1})$ be a subfield containing $k$ and of finite index. Then there exists a smooth curve $C'/k$ unique upto isomorphism and a non-constant rational map $\phi: C_{1} \rightarrow C'$ defined over $k$ such that $\phi^* (k(C_{1})) = \mathbb{K}$
\end{enumerate}
\end{theorem}
\begin{definition}
Let $\phi: C_{1} \rightarrow C_{2}$ be a morphism of curves defined over $k$. If $\phi$ is constant, then $\deg \phi :=0$. Otherwise, $$\deg \phi = [k(C_{1}): \phi^* (k(C_{2}))]$$ 
\begin{itemize}
\item $\phi$ is separable if $k(C_{1})/\phi^* (k(C_{2}))$ is separable.
\item $\phi$ is inseparable if $k(C_{1})/\phi^* (k(C_{2}))$ is inseparable.
\item $\phi$ is purely inseparable if $k(C_{1})/\phi^* (k(C_{2}))$ is purely inseparable.
\end{itemize}
Denote the separable and inseparable degree of $k(C_{1})/\phi^*(k(C_{2}))$ by $\deg_{s}\phi $ and $\deg_{i} \phi$ respectively.
\end{definition}

\begin{definition}
Let $\phi: C_{1} \rightarrow C_{2}$ be a morphism of curves defined over $k$. Then $k(C_{1})$ is a finite extension of $\phi^* (k(C_{2}))$, We wish this to construct a map \begin{equation}
\phi_{*} : k(C_{1}) \rightarrow k(C_{2}) \;\;\;\;\; \phi_{*} = \phi^* \circ \mathrm{Nm}_{k(C_{1})/\phi^*(k(C_{2}))}
\end{equation}
\end{definition}

\begin{proposition}
Let $f(X) \in k[X]$ be of degree $4$ with $\mathrm{disc}(f) \neq 0$. Then there is a smooth projective curve in $\PP^3$ satisfying 
\begin{enumerate}
\item $C \cap \Aa^3 \cong$ affine curve $Y^2 = f(X)$
\item $f(X)=a_{0}X^4 + \cdots + a_{4}$. Then $C \cap \{X_{0}=0\}$ consists of two points $[0,0, \pm \sqrt{a_{0}}, 1]$
\end{enumerate}
\end{proposition}

\section{Ramification}

\begin{definition}
Let $\phi: C_{1} \rightarrow C_{2}$ be a non-constant map of smooth curves and $P \in C_{1}$. Ramification index of $\phi$ at $P$ is defined by $$e_{\phi}(P) : = \ord_{P}(\phi^* (t_{\phi(P)}))$$ where $t_{\phi(P)} \in k(C_{2})$ is a uniformizer at $\phi (P)$. 
\begin{itemize}
\item $\phi$ is unramified at $P$ if $e_{\phi}(P) = 1$
\item If $\phi$ is unramified at all points of $C_{1}$, then we say $\phi$ is unramified.
\end{itemize}
\end{definition}

\begin{proposition}
Let $\phi: C_{1} \rightarrow C_{2}$ be a non-constant morphism of smooth curves. 
\begin{enumerate}
\item For all $Q \in C_{2}$ 
\begin{equation}
\sum_{P \in \phi^{-1}(Q)} e_{\phi}(P) = \deg \phi
\end{equation}

\item For all but finitely many $Q \in C_{2}$
\begin{equation}
\# \phi^{-1}(Q) = \deg_{s} \phi
\end{equation}

\item Let $\psi : C_{2} \rightarrow C_{3}$ be another non-constant morphism of smooth curves. Then for all $P \in C_{1}$ 
\begin{equation}
e_{\psi \circ \phi} (P) = e_{\phi}(P) e_{\psi}(\phi(P))
\end{equation}
\end{enumerate}
\end{proposition}

\begin{corollary}
A map $\phi: C_{1} \rightarrow C_{2}$ in unramified iff $\# \phi^{-1} (Q) = \deg \phi \; \forall \; Q \in C_{2}$
\end{corollary}

\section{Frobenius map}

Let $\mathrm{char} (k)=p >0$ and $q=p^n$. For $f(X) \in k[X]$ and let $f^{(q)}$ be the polynomial obtained from $f$ by raising each coefficient of $f$ to the $q$th power. \\

If $C/k$ is a curve, $C^{(q)}/k$ is defined to be the curve whose homogenous ideal is generated by $\{f^{(q)} \mid f \in I(C)\}$. Furthermore, there is a natural map 
\begin{align*}
\phi : C &\rightarrow C^{(q)} \\
[X_{0} : \cdots : X_{n}] &\mapsto [X_{0}^{q} : \cdots : X_{n}^{q}]
\end{align*}


\begin{proposition}
Suppose $C/k$ is a curve and $\phi : C \rightarrow C^{(q)}$ is the $q$th power Frobenious morphism. 
\begin{enumerate}
\item $\phi^* (k(C^{(q)})) = k(C)^{(q)} = \{f \in k(C)\}$
\item $\phi$ is purely inseparable.
\item $\deg \phi = q$
\end{enumerate}
\end{proposition}


\begin{corollary}
Every map $\phi: C_{1} \rightarrow C_{2}$ which is a morphism of smooth curves over a field of positive characteristic $p$ factors as $$C_{1} \xlongrightarrow{\phi} C_{1}^{(q)} \xlongrightarrow{\lambda} C_{2}$$ with $q=\deg_{i}(\phi)$ where $\phi$ is the $q$th power Frobenius and $\lambda$ is separable.
\end{corollary}


\chapter{Lecture-7 (24th January, 2023): Weierstrass equation}

\section{Weierstrass equation}

Consider 
\begin{equation}
Y^2Z + a_{1}XYZ + a_{3}YZ^2 = X^3 + a_{2}X^2Z + a_{4}XZ^2 + a_{6}Z^3
\end{equation}

Let $E \subseteq \PP^2$ be the variety given by this Weierstrass equation. \\

Dehomogenize wrt $Z$ and obtain 
\begin{equation}
E : y^2 + a_{1}xy + a_{3}y = x^3 + a_{2}x^2 + a_{4}x + a_{6}
\end{equation} 
with $[0:1:0]$ as a potential point at infinity. \\

If $a_{i} \in k$, then $E$ is defined over $k$. \\

\begin{itemize}
 \item If $\mathrm{char}(k) \neq 2$, then substituting $y$ for $1/2 (y -a_{1}x - a_{3})$ gives us 
 \begin{equation}
 E: y^2 = 4x^3 + b_{2}x^2 + 2b_{4}x + b_{6}
 \end{equation}
 \item If $\mathrm{char}(k) \neq 2,3 $ then 
 \begin{equation}
 E: y^2 = x^3 - 27c_{4}x - 54c_{6}
 \end{equation}
\end{itemize}

\begin{example}

\end{example}

\begin{definition}
An elliptic curve is a smooth curve in $\PP^2$ given by the Weierstrass equation.
\end{definition}

\begin{definition}
$\Delta$ is called the discriminant of the Weierstrass equation and $j$ is called the $j$ invariant of an elliptic curve.
\end{definition}

Let $P=(x_{0},y_{0})$ be a singular point satisfying the Weierstrass equation $f(x,y)$. Then, $$\frac{\partial f}{\partial X} (P)= \frac{\partial f}{\partial Y} (P) = 0$$


\chapter{Lecture-8 (31st January, 2023): Group Law}

\section{Group Law and definition of Elliptic Curve}

%insert the diagram of Y^2=X^3-X

\begin{proposition}
If $E$ is a curve given by the Weierstrass cubic $f(x,y,z)=0$ and $L$ be a line. Then, number of points of $L \cap E$ counted with multiplicity is $3$.
\end{proposition}

\begin{proof}
Suppose $P \in L \cap E$. Multiplicity of intersection of $L \cap E$ at $P$ is given by $\dim_{\bar{k}}K[E]_{P}/L = \dim_{\bar{k}}\bar{k}[X,Y]_{\m_{P}}/\langle \tilde{L}, \tilde{f} \rangle$ where $\tilde{-}$ is the dehomogenisation depending on the affine chart.
\begin{example}
$P=(0,0), L :x=0$. Multiplicity of $L\cap E$ at $P$ equals $\dim_{\bar{k}}[X,Y]_{\m_{P}}/\langle Y^2 - X^3 + X, X\rangle \simeq \dim_{\bar{k}} \bar{k}[Y]/Y^2 = 2$. \\

If $L': Y-X=0$, then $\dim_{\bar{k}}[X,Y]_{\m_{P}}/\langle Y^2 - X^3 + X-Y, X\rangle \simeq \dim_{\bar{k}} \bar{k}[Y]/\langle Y(Y - Y^2 -1) = \dim_{\bar{k}}\bar{k}=1$ since $Y-Y^2 -1$ is an unit.
\end{example}
All of this is a special case of Bezout's theorem which is stated after this proof.\\

Back to the proof. \\

Suppose $L : aX+bY+cZ=0$.\\ 
\textbf{Case-1:} $b\neq 0$ so $O = [0:1:0] \not \in L$. Dehomogenize with respect to $Y$ to get $aX+bY+c=0$. For $P \in L \cap E$, $$\frac{\bar{k}[E]_{P}}{\langle L,f \rangle } = \frac{\bar{[X,Y]_{\m_{P}}}}{\langle f, aX+bY+c \rangle} = \frac{\bar{k}[X,Y]}{\langle g(X) \rangle} (X-X(P))$$ where $g(X)$ is obtained by substituting $Y = -(aX+c)/b$ to $f$. Therefore $\dim_{\bar{k}}\bar{k}[E]_{P}/ \langle L , f \rangle =$ multiplicity of $X(P)$ as a root of $g(X)$. This implies the number of points of $L \cap E$ with multiplicity is the number of roots of $g(X)$ with multiplicity which is $3$. \\

\textbf{Case-2:} $b=0$
\begin{enumerate}
\item Suppose $a \neq 0 \Rightarrow L: X-cZ=0$. \\
Multiplicity of $L \cap E$ at $O$ is $\dim_{\bar{k}}\bar{k}[X,Y]_{\m_{O}}/ \langle \tilde{f} , X-cZ \rangle \simeq \dim_{\bar{k}} \bar{k} = 1$. \\
For $P \in \Aa^2 \cap E$, multiplicity of $L\cap E$ at $P$ is $\dim_{\bar{k}}\bar{k}[X,Y]_{\m_{P}} \langle f, X-cZ \rangle  = \dim_{\bar{k}}(\bar{k}[Y]/h)(Y-Y(P))$ with $h$ obtained from $f$ by substituting $X=c$. \\
Therefore, the total multiplicity is $1 + \#\text{ roots with multiplicity of } h = 1+2 =3$
\item $L : z=0$, then $L \cap E = \{0\}$. In the Weierstrass equation, after homogenization, this will mean that just $X^3$ survives. Therefore, $\dim_{\bar{k}} \bar{k}[X,Z]_{\m_{P}}/ \langle \tilde{f} , Z \rangle = \dim_{\bar{k}} \bar{k}[X]/\langle X^3 \rangle = 3$
\end{enumerate}
This concludes the proof.
\end{proof}

\begin{theorem}
IF $F,G$ are coprime homogenous polynomials in $k[X,Y,Z]$ and $V=\V(F), W=\V(G) \subseteq \PP^2$. Then the number of points of $V\cap W$ with multiplicity equals $mn$ where $m=\deg F, n = \deg G$.
\end{theorem}

\subsection{Composition Law of $E$}
Suppose $P,Q \in E$ and $L$ be the line passing through $P,Q$ (if $P=Q$, then $L$ is the tangent line at $P$) and let $R$ be the third point in the intersection of $L$ with $E$. Let $L'$ be the line joining $R$ and $O$ the point at infinity. $L'$ intersects $E$ at $R,O$ and a third point. We denote this point by $P \oplus Q$

\begin{proposition}
The composition law above makes $E$ into an abelian group, i.e., 
\begin{enumerate}
\item If $L$ intersects $E$ in $P,Q,R$, then $(P \oplus Q) \oplus R = 0$
\item $P \oplus O = P \; \forall \; P$
\item $P \oplus Q = Q \oplus P \; \forall \; P,Q$
\item If $P \in E$, then there exists $-P \in E$ such that $P \oplus (-P) = 0$
\item If $P,Q,R \in E$, then $(P \oplus Q ) \oplus R = P \oplus (Q \oplus R)$
\item If $E$ is defined over $k$, then $$E(k) = \{(x,y) \in k^2 : f(x,y)=0\} \cup \{0\}$$ is a subgroup of $E$.
\end{enumerate}
\end{proposition} 

\begin{proof}
s
\end{proof}

\textbf{Notation}: Suppose $P \in E$
$ [m]P = \begin{cases}
0 & \\
P \oplus \cdots \oplus P &, m>0 \\
-P \oplus \cdots \oplus -P &, m<0
\end{cases} $

Now, we can explicitly compute the coordinates of the points. Suppose $P = (x_{0},y_{0}) \in E$. Line passing through $P$ and $O$: $X-x_{0}Z$. Dehomogenize with respect to $Z$ to get $X-x_{0}=0$

\begin{align*}
f(X,Y) &= Y^2 + a_{1}XY + a_{3}Y - (X^3 + a_{2}X^2 + a_{4}X - a_{6})\\
\Rightarrow X(-P) &= x_{0}\\
\Rightarrow Y(-P) &\text{ is another root of } f(Y,x_{0}) \\
\Rightarrow Y(-P) &= -y_{0} - a_{1}x_{0} -a_{3} \\
\Rightarrow -P &= (x_{0}, -y_{0}-a_{1}x_{0} - a_{3}) 
\end{align*}

Suppose $P_{i} = (x_{i},y_{i}), i=1,2$. \\

We want to find $P_{1} \oplus P_{2}$'s coordinates. \\

If $x_{1}=x_{2}$ and $y_{1} + y_{2} + ax_{2} + a_{3}=0$, then $P_{1} \oplus P_{2} = 0$. Assume that this is not the case. This means that the line passing through $P_{1}$ and $P_{2}$ does not go through $O$ and thus $L: Y = \lambda x + \nu$. By high school methods, \\

If $x_{1}\neq x_{2}$, we get 
\begin{align*}
\lambda &= \frac{y_{1}-y_{2}}{x_{1}-x_{2}} \\
\nu &= \frac{y_{1}x_{2} - y_{2}x_{1}}{x_{2}-x_{1}} 
\end{align*}

If $x_{1} = x_{2}$, we have 
\begin{align*}
\lambda &= \frac{3x_{1}^2 + 2a_{2}x_{1} + a_{4} - a_{1}y_{1}}{2y_{1} + a_{1}x_{1} + a_{3}} \\
\nu &= \frac{-x_{1}^3 + a_{4}x_{1} +2a_{6} -a_{3}y_{1}}{2y_{1} + a_{1}x_{1} + a_{3}}
\end{align*}

This implies $f(x, \lambda x + \nu)$ has $3$ roots say $x_{1},x_{2},x_{3}$. Let $P_{3}=(x_{3}, y_{3} = \lambda x_{3} + \nu)$

\begin{align*}
\therefore f(X,\lambda X + \nu -(X-x_{1})(X-x_{2})(X-x_{3}) ) &= 0\\
\Rightarrow x_{1} + x_{2} + x_{3} &= \lambda^2 + a_{1}\lambda - a_{2} \\
\Rightarrow X(P_{1}\oplus P_{2}) &= \lambda^2 + a_{1}\lambda - a_{2} - x_{1} - x_{2} \\
Y(P_{1} \oplus P_{2}) &= -(\lambda + a_{1})x_{3} - \nu -a_{3}
\end{align*}

\textbf{Duplication formula}: $X([2]P) = \ds{\frac{x^4 -b_{4}x^2 -2b_{6}x - b_{8}}{4x^3 + b_{2}x^2 + 2b_{4}x + b_{6}}}$

\begin{example}
$E: Y^2 = X^3 + 17$. Some points on the curve are $P_{1} = (-2,3), P_{2} = (-1,4), P_{3} = (2,5), P_{4} = (4,9) , P_{5}=(8,23), P_{6}= (43, 282), P_{7} = (52, 375), P_{8} = (5234, 378661)$ \\
$\therefore P_{5} = [-2]P_{1}, P_{4} = P_{1}-P_{3} , P_{7} = [3]P_{1} - P_{3}. [2]P_{1} = \left( \frac{137}{64}, \frac{-2651}{512} \right) , P_{2} + P_{3} = \left(\frac{-8}{9}, \frac{-169}{27} \right)$ \\

The point being that $E(\QQ) = \ZZ P_{1} \oplus \ZZ P_{3}$ (Mordell-Weil) \\
$E(\ZZ) = \{\pm P_{1}, \hdots , \pm P_{8}\}$ (Siegel's theorem)
\end{example}

\begin{corollary}
Suppose $f\in \bar{k}(E) = \bar{k}(X,Y)$. Then $$f \text{ is even } \Leftrightarrow f \in \bar{k}(X)$$
\end{corollary}

\begin{proof}
'$\Leftarrow$' Let $P=(x_{0},y_{0})$ then $-P=(x_{0}, -y_{0} - a_{1}x_{0} - a_{3})$. This implies and element of $\bar{k}(X)$ is even. \\

'$\Rightarrow$' Suppose $f \in \bar{k}(X,Y)$ is even. Using Weierstrass equation $$f(x,y) = g(x) + yh(x), g,h \in \bar{k}(X)$$
$f$ even implies $f(x,y) = f(x, -y - a_{1}x - a_{3})$

\begin{align*}
\therefore g(x) + yh(x) &= g(x) + h(x)(-y - a_{1}x - a_{3}) \\
\Rightarrow (2y + a_{1}x + a_{3})h(x) &= 0 \forall (x,y) \in E \\
\Rightarrow h(x) = 0 &\text{ or } a_{1}=a_{3}=0
\end{align*}
In the latter case we have $\Delta = 0 \Rightarrow E = 0 \Rightarrow \Leftarrow$. Therefore, $h(x)=0 \Rightarrow f(x,y) = g(x) \in \bar{k}(E)$
\end{proof}

\section{Group Law for singular Weierstrass equation}

Let $E$ be a curve given by the Weierstrass equation $E_{ns}=$ set of non-singular points of $E$. \\

If $P\in E$ is a singular point. The multiplicity of $L \cap E$ at $P$ equals $\dim_{\bar{k}} \bar{k}[E]_{P}/L >1$. This is because $\bar{k}[E]_{P}$ is not a $DVR$ and going modulo $1$ relation does not give us $\bar{k}$ and thus dimension over $\bar{k}$ is atleast $2$.\\

Law of composition on $E_{ns} =$ same as the one we give to the non-singular Weierstrass equation.

\begin{proposition}
Let $E$ be a curve given by a singular Weierstrass equation. The law of composition makes $E_{ns}$ into an abelian group and, 
\begin{enumerate}
\item If $E$ has a node, then 
\begin{align*}
E_{ns} &\rightarrow \bar{K}^{\times} \\
(x,y) &\mapsto \frac{Y-\alpha_{1} X - \beta_{1}}{Y- \alpha_{2}X - \beta_{2}}
\end{align*} 
is an isomorphism of abelian groups, 
where $X = \alpha_{2} X + \beta_{2} , Y = \alpha_{1}X + \beta_{1}$

\item If $E$ has a cusp, $Y = \alpha X + \beta$ is a tangent at $S$, then 
\begin{align*}
E_{ns} &\rightarrow \bar{k}^{\times} \\
(x,y) \mapsto \frac{Y- X(S)}{Y - \alpha X - \beta}
\end{align*}
is an isomorphism of abelian groups.
\end{enumerate}
\end{proposition} 

\begin{definition}
An elliptic curve is a pair $(E,O)$ where $E$ is a nonsingular curve of genus $1$ and $O \in E$. The elliptic curve $E$ is defined over $k$ written $E/k$ if $E$ is defined over $k$ as a curve and $O \in E(k)$.
\end{definition}

To make sense of this definition we will have to employ more algebraic geometry as seen in next portion.



\chapter{Lecture-9 (2nd February, 2023): Group Law and more algebraic geometry}

\section{Algebraic geometry}

\subsection{Divisors}

Consider the formal linear combination $$D = \sum_{P \in C} n_{P} P$$ with $n_{P} = 0$ for all but finitely many $P \in C$. This formal linear combination is called a divisor of $C$. CLearly, this forms a free abelian group over the points of $C$. We denote this group by $\mathrm{Div}(C)$. \\

The degree of a divisor $D$ is defined as $$\deg D = \sum_{P \in C} n_{P}$$

The divisors of degree $0$ form a subgroup of $$\mathrm{Div}^{0}(C) = \{D \in \mathrm{Div}(C): \deg D =0\}$$

If $C$ is defined over $k$, we get a Galois action on $\mathrm{Div}^{0}(C), \mathrm{Div}(C)$ as follows: $$\sigma D = \sum_{P \in C} n_{P} \sigma P$$
Then we say $D$ is defined over $k$ iff $\sigma D=D$ for all $\sigma \in G$. We denote the group of divisors defined over $k$ by $\mathrm{Div}_{k}(C)$ and similarly $\mathrm{Div}_{k}^{0}(C)$\\

Next, suppose $f \in \bar{k}(C)^{\times}$. Then we can associate a divisor to this by letting $$\mathrm{div}(f) =\sum_{P \in C} \ord_{P}(f) P$$ This is indeed a divisor from a previous observation that we made. \\

Notice that if $\sigma \in G$, then $$\mathrm{div}(\sigma f) = \sigma (\mathrm{div}(f))$$ Or equivalently if $f \in \bar{k}(C)$ then $\mathrm{div}(f) \in \mathrm{Div}_{k}(C)$. \\

Since each $\ord_{P}$ is a valuation, the map $\mathrm{div}: \bar{k}(C) \rightarrow \mathrm{Div}(C)$ is a homomorphism of abelian groups. 

\begin{definition}
A divisor $D$ is called principal divisor if it has the form $D = \mathrm{div}(f)$ for some $f \in \bar{k}(C)^{\times}$. Two divisors are equivalent $D_{1} \sim D_{2}$ if $D_{1} - D_{2}$ is a principal divisor. \\

The divisor class group or Picard group of $C$ denoted by $\mathrm{Pic}(C)$ is the quotient of $\mathrm{Div}(C)$ modulo the principal divisors. We let $\mathrm{Pic}_{k}(C)$ be the subgroup of $\mathrm{Pic}(C)$ fixed by $G$. 
\end{definition}

\begin{proposition}
Let $C$ be a smooth curve and let $f \in \bar{k}(C)^{\times}$. 
\begin{enumerate}
\item $\mathrm{div}(f) = 0$ if and only if $f \bar{k}^{\times}$.
\item $\deg(\mathrm{div}(f))=0$
\end{enumerate}
\end{proposition}

\begin{definition}
From the previous definition, it is clear that the principal divisors form a subgroup of $\mathrm{Div}^{0}(C)$. The quotient of $\mathrm{Div}^{0}(C)$ by the subgroup of principal divisors is denoted by $\mathrm{Pic}^{0}(C)$. Similarly, we define $\mathrm{Pic}^{0}_{k}(C)$ to be the subgroup of $\mathrm{Pic}^{0}(C)$ fixed by $G$.
\end{definition}

All the information above can be summarised in the following exact sequence 

Let $\phi : C_{1} \rightarrow C_{2}$ be a non-constant map of smooth curves. We have already seen that this induces the two maps 
\begin{align*}
\phi^{*} : \bar{k}(C_{2}) \rightarrow \bar{k}(C_{1}) & \phi_{*} : \bar{k}(C_{1}) \rightarrow \bar{k}(C_{2}) 
\end{align*}

Similarly, we define maps of divisor groups as follows: 
\begin{align*}
\phi^{*} : \mathrm{Div}(C_{2}) &\rightarrow  \mathrm{Div}(C_{1}) & \phi_{*} : \mathrm{Div}(C_{1}) &\rightarrow \mathrm{Div}(C_{2}) \\
(Q) &\mapsto \sum_{P \in \phi^{-1}(Q)}e_{\phi}(P) P & (P) &\mapsto (\phi(P))
\end{align*}
Extend these maps $\ZZ$ linearly to arbitrary divisors. 

\begin{proposition}
Let $\phi : C_{1} \rightarrow C_{2}$ be a non-constant map of smooth curves. 
\begin{enumerate}
\item $\deg (\phi^{*} D) = (\deg \phi)(\deg D)$ for all $D \in \mathrm{Div}(C_{2})$
\item $\phi^* (\mathrm{div}(f)) = \mathrm{div}(\phi^* f)$ for all $k \in \bar{k}(C_{2})^{\times}$
\item $\deg(\phi_{*}(D)) = \deg D$ for all $D \in \mathrm{Div}(C_{1})$
\item $\phi_{*}(\mathrm{div}(f)) = \mathrm{div}(\phi_{*}(f))$ for all $f \in \bar{k}(C_{1})^{*}$
\item $\phi_{*} \circ \phi^*$ acts as multiplication by $\deg \phi$ on $\mathrm{Div}(C_{2})$
\item If $\psi : C_{2} \rightarrow C_{3}$ is another such map, then 
\begin{align*}
(\psi \circ \phi)^* = \phi^* \circ \psi^* &\text{ and } (\psi \circ \phi)_{*} = \psi_{*} \circ \phi_{*}
\end{align*}
\end{enumerate}
\end{proposition}


\subsection{Differentials}

\begin{definition}
Let $C$ be a curve. The space of meromorphic differential forms on $C$, denoted by $\Omega_{C}$ is the $\bar{k}$ vector space generated by symbols of the form $dx$ for $x \in \bar{k}(C)$ modulo the relations : 
\begin{enumerate}
\item $d(x+y) = dx + dy$ for all $x,y \in \bar{k}(C)$
\item $d(xy) = xdy + ydx$ for all $x,y \in \bar{k}(C)$
\item $dx=0$ for all $x \in \bar{k}$
\end{enumerate}  
\end{definition}

Let $\phi : C_{1} \rightarrow C_{2}$ be a non-constant map of smooth curves. The associated function field map $\phi^* : \bar{k}(C_{2}) \rightarrow \bar{k}(C_{1})$ induces a map of differentials : 
\begin{align*}
\phi^* : \Omega_{C_{2}} &\rightarrow \Omega_{C_{1}} \\ \phi^* \left( \sum_{i} f_{i} dx_{i} \right) &= \sum_{i} (\phi^* f_{i}) d(\phi^* x_{i})
\end{align*}

\begin{proposition}
Let $C$ be a curve. 
\begin{enumerate}
\item $\Omega_{C}$ is $1$-dimensional $\bar{k}(C)$ vector space. 
\item Let $x\in \bar{k}(C)$. Then $dx$ is a $\bar{k}(C)$ basis for $\Omega_{C}$ iff $\bar{k}(C)/ \bar{k}(x)$ is a finite separable extension.
\item Let $\phi: C_{1} \rightarrow C_{2}$ be a non-constant map of curves. Then $\phi$ is separable iff the map $\phi^*$ is injective. 
\end{enumerate}
\end{proposition}

\begin{proposition}
Let $C$ be a curve, $P \in C$ and $t \in \bar{k}(C)$ an uniformiser at $P$. 
\begin{enumerate}
\item For every $\omega \in \Omega_{C}$ there exists an unique function $g \in \bar{k}(C)$ depending on $\omega$ and $t$ satisfying $$\omega = g dt$$
\item Let $f \in \bar{k}(C)$ with $\omega \neq 0$. The quantity $$\ord_{P} (\omega / dt)$$ depends on $\omega$ and $P$, independent of the choice of uniformiser $t$. We call this value the order of $\omega$ at $P$ and denote it by $\ord_{P}(\omega)$.
\item Let $x,f \in \bar{k}(C)$ with $x(P)=0$ and let $\mathrm{char}(k)=p$. Then, 
\begin{align*}
\ord_{P}(fdx) &= \ord_{P}(f) + \ord_{P}(x) - 1 &\text{ if } p=0 \text{ or } p \nmid \ord_{P}(x) \\
\ord_{P}(fdx) &\geq \ord_{P}(f) + \ord_{P}(x) &\text{ if } p > 0 \text{ and } p \mid \ord_{P}(x)
\end{align*}
\item Let $\omega \in \Omega_{C}$ with $\omega \neq 0$. Then $\ord_{P}(\omega)=0$ for all but finitely many $P \in C$.
\end{enumerate}
\end{proposition}

\begin{definition}
Let $\omega \in \Omega_{C}$. The divisor associated to $\omega$ is 
\begin{equation*}
\mathrm{div}(\omega) = \sum_{P \in C} \ord_{P}(\omega) P \in \mathrm{Div}(C)
\end{equation*}
The differential $\omega$ is regular or holomorphic if 
\begin{eqnarray}
\ord_{P}(\omega) \geq 0 &\forall \; P \in C
\end{eqnarray}
It is nonvanishing if 
\begin{eqnarray}
\ord_{P}(\omega) \le 0 &\forall \; P \in C
\end{eqnarray}
\end{definition}

\begin{remark}
If $\omega_{1}, \omega_{2} \in \Omega_{C}$ are nonzero differentials, then there exists a function $f \in \bar{k}(C)^{\times}$ such that $\omega_{1} = f\omega_{2}$. Therefore $$\mathrm{div}(\omega_{1}) = \mathrm{div}(f) + \mathrm{div}(\omega_{2})$$
\end{remark}

\begin{definition}
The canonical divisor class on $C$ is the image in $\mathrm{Pic}(C)$ of $\mathrm{div}(\omega)$ for any non-zero differential $\omega \in \Omega_{C}$. Any choice in this divisor class is called a canonical divisor.
\end{definition}

\chapter{Lecture-10 (7th February, 2023): Riemann-Roch theorem }

\begin{definition}
\begin{enumerate}
\item A divisor $D$ is positive, denoted by $D\geq 0$ if $n_{P} \geq 0 \; \forall \; P \in C$.
\item $D_{1},D_{2} \in \mathrm{Div}(C)$, then $D_{1} \geq D_{2}$ if $D_{1}-D_{2} \geq 0$.
\item For $D \in \mathrm{Div}(C)$ 
\begin{align*}
\mathcal{L}(D) &= \{f \in \bar{k}(C)^{*} : \mathrm{div}(f) \geq -D\} \cup \{0\}\\
\ell(D) &= \dim_{\bar{k}}\mathcal{L}(D)
\end{align*}
\end{enumerate}
\end{definition}

\begin{proposition}
Suppose $D \in \mathrm{Div}(C)$
\begin{enumerate}
\item If $\deg D <0$, then $\mathcal{L}(D) = \{0\}$ and $\ell(D) = 0$.
\item $\mathcal{L}(D)$ is finite dimensional over $\bar{k}$.
\item If $D' \in \mathrm{Div}(C)$ and $D \sim D'$, then $\mathcal{L}(D) \cong \mathcal{L}(D')$ and $\ell(D)=\ell(D')$
\end{enumerate}
\end{proposition}

\begin{proof}
\begin{enumerate}
\item If $\mathrm{div}(f) \geq -D$ then $\deg (\mathrm{div}(f)) \geq \deg (-D) = -\deg D$. If $\deg D <0$, then $\deg (-D) >0$ and thence no such $f$ exists.
\item For $D' \le D \Rightarrow \mathcal{L}(D) \subseteq \mathcal{L}(D')$\\

\textbf{Claim}: $\dim_{\bar{k}}(\mathcal{L}(D+P)/\mathcal{L}(D)) \le 1$. 
\begin{proof}
Consider the map $\phi_{P}: \mathcal{L}(D+P) \rightarrow \bar{k}$ by $f \mapsto t^{r+1}$
\end{proof}
\end{enumerate}
\end{proof}

\begin{theorem}[Riemann-Roch]
Let $C$ be a smooth curve and let $K_{C}$ be a canonical divisor on $C$. There is an integer $g \geq 0$ called the genus of $C$ such that for every divisor $D \in \mathrm{Div}(C)$ we have 
\begin{equation*}
\ell(D) - \ell(K_{C} - D) = \deg D - g + 1
\end{equation*}
\end{theorem}

\begin{corollary}
\begin{enumerate}
\item $\ell(K_{C}) = 0$
\item $\deg K_{C} = 2g-2$
\item If $\deg D > 2g-2$, then $$\ell(D) = \deg D - g +1$$
\end{enumerate}
\end{corollary}

Now, we will use this theory to show that every elliptic curve can be written as a plane cubic and conversely any smooth Weierstrass equation is an elliptic curve. 

\begin{proposition}
Let $E$ be an elliptic curve defined over $k$. 
\begin{enumerate}
\item There exists functions $x,y \in k(E)$ such that the  map $$\phi: E \rightarrow \PP^2 \text{ is given by } \hspace{5mm} \phi = [x,y,1]$$
gives an isomorphism of $E/k$ onto the curve given by a Weierstrass equation with $a_{i} \in k$ satisfying $\phi(O) = [0,1,0]$
\item Any two Weierstrass equations for $E$ as in previous part are related by a linear change of variables of the form $$X = u^2 X' + r \hspace{ 1cm} Y = u^2 Y' + su^2 X' + t$$ with $u \in k^{\times}$ and $r,s,t \in k$
\item Conversely, every smooth cubic curve $C$ given by the Weierstrass equation as in first part is an elliptic curve defined over $k$ with base point $O = [0,1,0]$
\end{enumerate}
\end{proposition}

\begin{corollary}
Let $E/k$ be an elliptic curve with Weierstrass coordinate functions $x,y$. Then, 
\begin{align*}
k(E) = k(x,y) &\text{ and } & [k(E): k(x)] = 2
\end{align*}
\end{corollary}

\chapter{Lecture-11 (9th February, 2023): Isogenies}

\begin{lemma}
Let $C$ be a curve of genus $1$ and let $P,Q \in C$. Then 
\begin{equation}
(P) \sim (Q) \Leftrightarrow P = Q
\end{equation}
\end{lemma}

\begin{proposition}
Let $(E,O)$ be an elliptic curve. 
\begin{enumerate}
\item For every $0$-degree divisor $D \in \mathrm{Div}^{0}(E)$ there exists an unique point $P \in E$ satisfying 
\begin{equation}
D \sim (P)-(O)
\end{equation}
Define 
\begin{equation}
\sigma : \mathrm{Div}^{0}(E) \rightarrow E
\end{equation}
to be the map that sends $D$ to its associated $P$
\item The map $\sigma $ is surjective.
\item Let $D_{1},D_{2} \in \mathrm{Div}^{0}(E)$. Then, 
\begin{equation}
\sigma (D_{1}) = \sigma (D_{2}) \Leftrightarrow D_{1} \sim D_{2}
\end{equation}
Thus $\sigma$ induces a bijection of sets 
\begin{equation}
\sigma : \mathrm{Pic}^{0}(E) \xlongrightarrow{\sim} E
\end{equation}
\item The inverse to $\sigma$ is the map 
\begin{equation}
\kappa : E \xlongrightarrow{\sim} \mathrm{Pic}^{0}(E) \hspace{1cm} P \mapsto (\text{ divisor class of } (P) - (O))
\end{equation}
\item If $E$ is given by a Weierstrass equation, then the geometric group law on $E$ and the algebraic group law are the same.
\end{enumerate}
\end{proposition}

\begin{corollary}
Let $E$ be an elliptic curve and $D \in \mathrm{Div}(E)$. Then $D$ is a principal divisor iff 
\begin{equation}
\sum_{P \in E} n_{P} = 0 \text{ and } \sum_{P \in E} [n_{P}]P= O
\end{equation}
\end{corollary}

\section{Isogenies}

\chapter{Lecture-12 (14th February, 2023): Isogenies continued}
\section{Isogenies}


\chapter{Lecture-13 (15th February, 2023): }




\part{Basic Algebraic Geometry}

\chapter{Lecture-1 (5th January): Introduction}

\chapter{Lecture-2 (10 January, 2023): Ideals and Zariski topology}

\section{Ideals}
For $I,J$ ideals $$I+J = \{x+y \mid x\in I, y\in J\}$$ $$IJ = \{\sum x_{i}y_{i} \mid x_{i} \in I, y_{i} \in J\}$$

\begin{itemize}
\item $IJ \subseteq I \cap J$.
\item If $I+J = R$, then $I^2 + J^2 = R$. This is because, say $I^2 + J^2 \neq R$, then there is a maximal ideal $\mathrm{m}$ such that $I^2 + J^2 \subseteq \mathfrak{m}$. This means $I^2, J^2 \subseteq \mathfrak{m}$. But $\mathfrak{m}$ is prime ideal, therefore $I,J \subseteq \mathfrak{m} \Rightarrow I+J \subseteq \mathfrak{m}$ which is a contradiction. Thus, we are done. 
\item If $\mathfrak{p}$ is a prime ideal and $IJ \subseteq \mathfrak{p}$. Then, $I \subseteq \mathfrak{p}$ or $J \subseteq \mathfrak{p}$. Suppose not, then there exists $x \in I \backslash \mathfrak{p} , y \in I \backslash \mathfrak{p}$. But then $xy \in IJ \subseteq \mathfrak{p}$.
\item $\mathfrak{p} \supseteq I \cap J \Leftrightarrow IJ \subseteq \mathfrak{p}$.
\end{itemize}

\section{Zariski topology}

\begin{definition}
\begin{itemize}
\item For an ideal $I$, let $$V(I) = \{\mathfrak{p} \text{ prime ideal }\mid I \subseteq \mathfrak{p}\}$$
\item $\mathrm{Spec}(R)=\{ \text{ collection of all prime ideals of }R\}$
\end{itemize}
\end{definition}

\begin{definition}[Zariski Topology]
It is the topology defined on $\mathrm{Spec}(R)$ such that the closed sets are $V(I)$.
\end{definition}

Verification that this indeed is a topology. 
\begin{enumerate}
\item $V(0) = \mathrm{Spec}(R), V(R) = \emptyset$.
\item $V(I) \cup V(J) = V(I\cap J) = V(IJ)$.
\item $\bigcap_{k \in k} V_{k} = V(\sum_{k \in K} I_{k})$. This is because $\mathfrak{p} \supseteq I_{k} \Leftrightarrow \mathfrak{p} \supseteq \sum_{k \in K}I_{k}$
\end{enumerate}

Let us now look at the open sets of this topology. The basis for the open sets is given by $$D(f \in R) = \{ \text{ all prime ideals not containing } f\}$$
Clearly, $$(V(I))^c = \bigcup_{f \in I} D(f)$$ and moreover, each $D(f)$ is open since $D(f) = (V(\langle f \rangle))^c$

\begin{theorem}
$\mathrm{Spec}(R)$ is quasi-compact.
\end{theorem}

\begin{proof}
We wish to prove that every open cover has a finite subcover. This is equivalent to saying every cover by $D(f_{i})$ has a finite subcover. Say $$\spec(R) = \bigcup_{i \in I} D(f_{i})$$ Take $J$ to be the ideal generated by $f_{i}'s$. Either $J = R$ or $J \subseteq \mathfrak{m}$. Suppose $J \subseteq \m$, then $f_{i} \in \m \in \spec(R) \Rightarrow \m \not \in D(f_{i}) \; \forall \; i \Rightarrow D(f_{i})$ does not cover $\m$. A contradiction. Therefore, $J=R$ and this implies $1 =$ some linear combination of $f_{i}$ and notice that this sum is finite. So, just consider these finitely many $f_{i}'s$ (say the indexing set is $K$). These cover $J$. Suppose that $\{ D(f_{k}), k \in K\}$ do not cover $\spec(R)$. Then, there is a prime ideal $\pr \not \in \bigcup_{k\in K}D(f_{k}) \Rightarrow \pr \ni f_{k} \; \forall\; k \in K \Rightarrow R \subseteq \pr \Rightarrow \Leftarrow$. Hence, it covers all of $\spec(R)$ as required. \\

\textbf{Another proof:}\\
Suppose $\spec(R) = \bigcup_{j \in J} U_{j} = \bigcup_{j \in J} \spec(R) \backslash \V(I_{j}) = \spec(R) \backslash \bigcap_{j \in J} \V(I_{j}) =\spec(R) \backslash \V(\sum_{j \in J} I_{j})$. This is equivalent to saying that $\V(\sum_{j \in J} I_{j}) = \emptyset$. So, we conclude that $\sum_{j \in J}I_{j} = R \Rightarrow \sum_{k \in K}a_{k} = 1$ for some finite set $K$. We claim that $\{U_{k}: k \in K\}$ covers $\spec(R)$. This is because 
\begin{align*}
\V(\sum_{k\in K} I_{k}) &= 0 \\
\Rightarrow \spec(R) &= \spec(R) \backslash \V(\sum_{k\in K} I_{k})  \\
&= \bigcup_{k \in K} \spec(R) \backslash \V(I_{k}) \\
&= \bigcup_{k \in K} U_{k}
\end{align*}
This completes the proof. 
\end{proof}

\begin{proposition}
Each $D(f)$ is quasi-compact.
\end{proposition}

\begin{proof}
Suppose $$D(f) = \bigcup D(g_{i})$$ and let $J$ be the ideal generated by $g_{i}'s$. Take $\mathfrak{p} \supseteq J$. Then, each $g_{i} \in J \subseteq \mathfrak{p} \Rightarrow \mathfrak{p} \not \in D(g_{i}) \Rightarrow \mathfrak{p} \not \in D(f) \Rightarrow f \in \mathfrak{p} \Rightarrow f \in \bigcap_{\mathfrak{p} \supseteq J} \mathfrak{p}$. \textcolor{BrickRed}{Before completing this proof, we need to understand this intersection much better. Refer to following content on nilpotent elements and come back.}\\
Now, we know that $f \in \mathrm{rad}(J)$ which implies $\exists n$ such that $f^n \in J$. We get $$f^n = \sum_{\text{finite}} r_{i}g_{i}$$ Finally, we claim that these $D(g_{i})$s cover $D(f)$. 
\end{proof}

\begin{definition}
$x\in R$ is nilpotent if $x^n =0$ for some $n \in \NN$.
\end{definition}

\begin{remark}
Any nilpotent element ($x^n = 0$ for some $n$ ) is clearly in every prime ideal ($0 \in \mathfrak{p}$) and thus in the intersection of all prime ideals. This can be recorded as $$\bigcap_{\pr \in \spec(R)} \pr \supseteq \nil(R)$$
\end{remark}

\begin{proposition}
$$\bigcap_{\pr \in \spec(R)} \pr \subseteq \nil(R)$$
\end{proposition}

\begin{proof}
Take an element $x \in R \backslash \nil(R)$ (not nilpotent) and consider the set $$\Sigma = \{ I \unlhd R \mid x^n \not \in I \; \forall \; n >0\}$$
Notice that $\Sigma$ is a poset with respect to inclusion. And every chain $I_{1} \subseteq I_{2} \subseteq I_{3} \subseteq \cdots $ has an upper bound (union of all the ideals). Thus, we can apply Zorn's lemma to get a maximal element $\pr$ which we claim is prime. Indeed, if $ab \in \pr$ but $a\not \in \pr, b \not \in \pr$ then $\pr + \langle a \rangle, \pr + \langle b \rangle$ are ideals strictly containing $\pr$ contradicting maximality of $\pr$. Therefore, we can conclude that $x \not \in \pr \Rightarrow x \not \in \bigcap_{\pr \supseteq J} \pr$ or rather not nilpotent implies not in intersection and hence we have proved the required inclusion. 
\end{proof}

$$\nil (R) = \bigcap_{\pr \in \spec (R)} \pr = \bigcap_{\pr \subseteq \{0\}} \pr$$
$$\{x \mid x^n \in J\} = \mathrm{rad}(J) = \bigcap_{\pr \supseteq J} \pr $$

\chapter{Lecture-3 (12th January): Zariski topology }

\section{Zariski topology contd..}

\begin{definition}
If $J = \mathrm{rad}(J)$, then $J$ is called radical ideal.
\end{definition}

\textbf{Properties:}
\begin{enumerate}
\item Every radical ideal is an intersection of prime ideals.
\item $\V(J) = \V(\mathrm{rad}(J))$ 
\item $\V(J) = \V(J')$ implies $\mathrm{rad}(J) = \mathrm{rad}(J')$
\end{enumerate}

Suppose $S\subseteq R$ such that 
\begin{itemize}
\item $1\in S, 0 \not \in S$
\item If $x,y \in S \Rightarrow xy \in S$
\end{itemize}

\begin{proposition}
Take an ideal maximal wrt not intersecting $S$. Then, it is prime.
\end{proposition}

\begin{proof}
Suppose $\m$ is the ideal in question. Next, suppose $\m$ is not prime which implies  $\exists a,b \in R \text{ such that } ab \in \m$ but $a,b \not \in \m$. Then, $\m + \langle a \rangle \supsetneq \m , \m + \langle b \rangle \supsetneq \m$. But, this means $(\m + \langle a \rangle) \cap S \neq \emptyset \Rightarrow m + ra \in S$ for some $m\in \m , r  \in R$. Similarly, $n + sb \in S$ for some $n \in \m , s \in R$. But, $S$ is multiplicative therefore $(m+ra)(n + sb) \in S \Rightarrow mn + ran + msb + rsab \in S \Rightarrow ((\langle ab \rangle + \m)= \m) \cap S \neq \emptyset$. This is a contradiction. Hence, we are done.  
\end{proof}

\begin{proposition}
Say $J$ is maximal wrt not being principal. Then, $J$ is prime.
\end{proposition}

\begin{proof}
Suppose $\m$ is the ideal in question. Next, suppose $\m$ is not prime which implies  $\exists a,b \in R \text{ such that } ab \in \m$ but $a,b \not \in \m$. Next, we can consider the ideal $I = \m + \langle a \rangle$. By maximality of $\m$, we have $I = \langle c \rangle$ for some $c\in R$. Now, consider $J = \{x \in R \mid xc \in \m\}$. Clearly, $I \subseteq J$. Notice that $c = m + ar$ for some $m \in \m , r \in R$. 
\begin{align*}
bc &= b(m+ar) \\
&= bm + (ba)r \\
\Rightarrow bc &\in \m \\
\Rightarrow b &\in J 
\end{align*} 
This means $b \in J \backslash \m$. Therefore $V$ is also principal and hence $V = \langle d \rangle$. Since $\m \in I$, therefore $m = cr$ for some $r\in R$. But this means that $r\in V \Rightarrow r = r'd$ for some $r' \in R$. Hence, $m = cd r' \in \langle cd \rangle \Rightarrow \m \subseteq \langle cd \rangle$. For the other direction, since $d\in V \Rightarrow cd \in U$. All of these tells us that $\m = \langle cd \rangle$ a contradiction to our assumption. Therefore, $\m$ must be prime. 
\end{proof}


\begin{proposition}
Say $J$ is maximal wrt not being finitely generated. Then, $J$ is prime.
\end{proposition}

\begin{proof}
Suppose $\m$ is the ideal in question. Next, suppose $\m$ is not prime which implies  $\exists a,b \in R \text{ such that } ab \in \m$ but $a,b \not \in \m$. \\

If we now look at $\m + \langle a \rangle$, by our assumption, this ideal is finitely generated by say $u_{1}, \hdots , u_{m}$. 
\end{proof}

\begin{exercise}
Suppose $J$ is maximal wrt not being generated by a cardinal number of generators. Then, $J$ is prime.
\end{exercise}

\begin{definition}
A topological space $X$ is said to be irreducible if it cannot be written as the union of proper closed subsets of $X$
\end{definition}

\section{Identify closed irreducible subsets of $\spec(R)$}

\begin{proposition}
The sets $\V(\pr)$ are exactly the irreducible components of $\spec(R)$.
\end{proposition}

\begin{lemma}
Let $I \subseteq R$ be a radical ideal. If $\V(I)$ is irreducible, then $I$ is prime.  
\end{lemma}

\begin{proof}
Suppose $I$ is not prime. Then there exists $a,b$ such that $ab \in I$ but $a\not \in I$ and $b \not \in I$. Consider a prime ideal $\pr$ that contains $I$, it will also contain $ab$ and thus $\pr$ contains either $a$ or $b$. This is summarised as $$\V(I) = (\V(I)\cap \V(a)) \cup (\V(I) \cap \V(b))$$ Thus $\V(I)$ is union of closed sets. It remains to be shown that the sets are proper in order to conclude that $\V(I)$ is not irreducible. Since $\V(I) \cap \V(a) = \V(I + \langle a \rangle)$ and $a \not \in I$ therefore $\V(I +\langle a \rangle)$ is a proper closed subset of $I$ and same for $b$. This is a contradiction to our hypothesis. So, we are done. 
\end{proof}

\begin{lemma}
$\V(\pr)$ is an irreducible closed subset for $\pr$ prime.
\end{lemma}

\begin{proof}
Suppose $\V(\pr) = V_{1} \cup V_{2}$ with $V_{1},V_{2}$ proper closed subsets of $V(\pr)$. Then there exists ideals $I,J$ such that $\V(\pr) = \V(I) \cup \V(J)$. Since $\pr \in \V(\pr)$ this implies $\pr \in \V(I)$ or $\pr \in \V(J)$. Suppose $\pr \in \V(I)$, then $I \subseteq \pr \Rightarrow \V(\pr) \subseteq \V(I) \Rightarrow \V(\pr) = \V(I)$. This is a contradiction to our assumption and hence we are done. $\V(\pr)$ is irreducible.
\end{proof}

\begin{proposition}
Every irreducible closed subset of $\spec(R)$ has an unique generic point.
\end{proposition}

\begin{proof}
Notice that any irreducible closed subset is of the form $\V(\pr)$. Now, $\V(\pr)$ is the closure of $\pr$. This is because $\mathrm{cl}(\pr)$ is a closed set and hence of the form $\V(I)$ for some ideal $I$. Moreover $\pr \supseteq I$. The biggest ideal $I$ such that $I \subseteq \pr$ is $\pr$ and this gives us what we want because $\V$ reverses inclusions. Therefore, $\mathrm{cl}(\pr) = \V(\pr)$. And, such a generic point is unique for suppose $\V(\pr) = \V(\mathfrak{q})$ then clearly $\pr \subseteq \frak{q}$ and $\frak{q} \subseteq \pr$. So, we are done. 
\end{proof}

To summarise, Zariski topology has the following properties: 
\begin{enumerate}
\item $\spec(R)$ is quasi-compact
\item $\spec(R)$ has a basis of quasi-compact opens which is closed under intersection.
\item Every irreducible closed subset has a generic point.
\end{enumerate}

\begin{theorem}[Hochster]
Any topological space with the $3$ properties is the spectrum of some commutative ring.
\end{theorem}



Suppose $X$ is spectral. Define a new space $X^{*}$ with open sets as finite union of quasi-compact open sets in $X$. This new space is called the Hochster dual.

\begin{theorem}
$X^{*}$ is also spectral.
\end{theorem} 

\begin{proof}

\end{proof}


\chapter{Lecture-4 (17th January, 2023): Noetherian spaces}

\section{Noetherian spaces}

First, let us try to remember all the equivalent definitions of a ring being Noetherian. 

\begin{proposition}
The following are equivalent:
\begin{enumerate}
\item Every ideal is finitely generated. 
\item Every ascending chain of ideals $$I_{1} \subseteq I_{2} \subseteq I_{3} \subseteq \cdots $$ stabilises. 
\item Every non-empty family of ideals has a maximal element. 
\end{enumerate}
\end{proposition}

Nowhere do we use Zorn's lemma, so in some sense, these properties are essentially about some "finite-ness" property. Thus, Noetherian means strong finiteness in some sense. 

\begin{definition}

\end{definition}

\begin{definition}

\end{definition}

\begin{theorem}
A module $M$ over $R$ is Noetherian iff the module is finitely generated and finitely presented. 
\end{theorem}

\begin{proof}

\end{proof}

\begin{proposition}
The direct sum of projective modules is projective. 
\end{proposition}

\begin{proposition}
The direct product of injective modules is injective.
\end{proposition}

A question we can ask is when is the direct sum of injective modules injective. 

\begin{proposition}
Direct sum of injective modules is injective iff the module is Noetherian.
\end{proposition}

\chapter{Lecture-5 (19th January 2023):}

Suppose $A$ is a commutative ring and $M$ an $A$-module. \\
Define $\mathrm{Sub}(M)=$ to be the set of all submodules of $M$. For any finite collection $m_{1}, \hdots , m_{k} \in M$, we next define 
\begin{align*}
\textbf{V}(m_{1}, \hdots , m_{k}) &= \text{ collection of submodules containing } m_{1}, \hdots , m_{k} \\
\textbf{D}( m_{1}, \hdots , m_{k}  ) &= \mathrm{Sub}(M) \backslash \textbf{V}(m_{1}, \hdots , m_{k}) 
\end{align*}
Using these $\textbf{D}(m_{1}, \hdots , m_{k})$'s as open sets (sub-basis of open sets), we generate a topology. 

\begin{proposition}[\href{https://arxiv.org/pdf/1605.03105.pdf}{read this here}]
The above mentioned topology is the same as Zariski topology OR the space is spectral.
\end{proposition}

\begin{remark}
The takeaway point being this is also another way to get a spectral space.
\end{remark}

\begin{exercise}
Suppose $X$ is spectral, $Y\subseteq X$ be a quasi-compact open subset. Then, $Y$ is spectral.
\end{exercise}

\section{Localisation}

\begin{definition}
A multiplicatively closed set $S$ is one that has the following properties:
\begin{enumerate}
\item $1 \in S, 0 \not \in S$.
\item $x,y \in S \Rightarrow xy \in S$.
\end{enumerate}
\end{definition}

\begin{example}
\begin{enumerate}
\item Invertible elements of a ring.
\item 
\end{enumerate}
\end{example}

\textcolor{Brown}{Objective}: We wish to construct a new ring in which each $s\in S$ is invertible. \\

If $S$ was the collection of invertible elements, then localisation is just $A$. \\

Our objective can be summed up as follows: 

\[\begin{tikzcd}
	A && B \\
	& R
	\arrow["\psi"', from=1-1, to=1-3]
	\arrow["\phi", from=1-1, to=2-2]
	\arrow["{\exists ! \pi}"', dashed, from=2-2, to=1-3]
\end{tikzcd}\]

\begin{enumerate}
\item $\phi(s)$ is invertible in $R$ for each $s\in S$
\item for any $\psi : A \rightarrow B$ such that each $\psi (s)$ is invertible, there is an unique map $\pi : R \rightarrow B$ that makes the diagram above commute.
\end{enumerate}

\begin{definition}
The localisation of $A$ with respect to $S$, denoted by $S^{-1}A$ is the set of equivalence classes $$\frac{a}{s} \;\;\;, a \in A , s \in S$$ with $$\frac{a}{s} \sim \frac{a'}{s'} \text{ if and only if } \exists \; t \in S \text{ such that } t (as' - sa')=0$$ 
\end{definition}

The ring addition and multiplication are the same as adding and multiplying fractions. Need to check it is well-defined!\\

Now, back to $$A_{f} = \{\text{ localisation of $A$ at }f\}$$ What we want to do is we essentially want to turn $f$ into an unit. Take $S$ to be all powers of $f$. Then, $S^{-1}A = A_{f}$. \\

This can also be realised as $$\frac{A[X]}{\langle fX-1 \rangle}$$

Now, the question is why are the two notions equivalent. 
\[\begin{tikzcd}
	A && B \\
	& {A_{f}}
	\arrow["\psi", from=1-1, to=1-3]
	\arrow["\phi"', from=1-1, to=2-2]
	\arrow["\pi"', dashed, from=2-2, to=1-3]
\end{tikzcd}\]

with $$\pi\left( \frac{a}{f^k} \right) = \frac{\psi (a)}{\psi (f)^k}$$

And, 

\[\begin{tikzcd}
	A && B \\
	& {\frac{A[X]}{\langle fX - 1 \rangle}}
	\arrow["\psi", from=1-1, to=1-3]
	\arrow["\phi"', from=1-1, to=2-2]
	\arrow["\pi"', dashed, from=2-2, to=1-3]
\end{tikzcd}\]

with $\pi(X) = \psi(f)^{-1}$ \\

\subsection{Prime ideals of $A_{f}$}

\begin{theorem}
The prime ideals of $A_{f}$ are precisely $D(f)$ the set of primes not containing $f$.
\end{theorem}

Consider $A_{S}$ and look at the ideals of $A_{S}$. They are precisely of the form $$\left \{ \frac{x}{s} \mid \;  x \in I \unlhd A, s  \in S \right\}$$

There is a bijection between $$\{ \text{ prime ideals of } A_{S} \} \leftrightarrow \{ \text{prime ideals of } A \text{ not intersecting S} \}$$

Say $\mathfrak{P}$ is a prime ideal of $A_{S}$. Then, $\mathfrak{P} = \frac{\pr}{s}$ with $\pr$ prime in $A$.


\chapter{Lecture-6 (24th January, 2023): Localisation of modules, exact sequences}

\section{Localisation contd..}

Suppose $M$ is an $A$-module. And $S\subseteq A$ be a multiplicative set. Then, the localisation $$M_{S} = \{\text{equivalence classes of all elements of the form } \frac{m}{s}\}$$ with $\frac{m}{s} \sim \frac{m'}{s'}$ if there exists $t\in S$ such that $t(s'm - m's)=0$. This can be made into a module by standard operations. 

\begin{lemma}
$M_{S}$ is an $A_{S}$-module.
\end{lemma}

\begin{proof}

\end{proof}

Some natural questions to ask are if $I \subseteq A$ is an ideal, whether 
\begin{align*}
\left( \frac{A}{I} \right)_{S} &\overset{?}{\cong} \frac{A_{S}}{I_{S}} \\
\text{More generally } \left( \frac{M}{M'} \right)_{S} &\overset{?}{\cong} \frac{M_{S}}{M_{S}'}
\end{align*}


We will need to introduce exact sequences to answer these questions.

\section{Exact sequences}
Suppose $$f: M \rightarrow N$$ Then, 
\begin{align*}
\ker(f) &= \{m \in M : f(m)=0\} \\
\coker &= N/\iM(f)
\end{align*}

This can be captured in the following diagram : 
\[\begin{tikzcd}
	{\mathrm{Ker}(f)} & M & N & {\mathrm{Coker}(f)} \\
	& P & Q
	\arrow["i", from=1-1, to=1-2]
	\arrow["\pi", from=1-3, to=1-4]
	\arrow["f", from=1-2, to=1-3]
	\arrow["g", tail reversed, no head, from=1-2, to=2-2]
	\arrow["{\exists \; !}", dashed, from=2-2, to=1-1]
	\arrow["h", from=1-3, to=2-3]
	\arrow[dashed, from=1-4, to=2-3]
\end{tikzcd}\]

Here, $$M/\ker(f) \cong \iM(f)$$ is equivalent to saying $\coker(i) = \ker(\pi)$. This leads to the definition
\begin{definition}
$$M' \xrightarrow{f} M \xrightarrow{g} M''$$ is exact at $M$ if $\iM(f) = \ker(g)$.
\end{definition}

\begin{lemma}
\begin{enumerate}
\item $$0 \rightarrow M' \xrightarrow{f} M$$ being exact means $f$ is injective.
\item $$M \xrightarrow{g} M'' \rightarrow 0$$ being exact means $g$ is surjective.
\end{enumerate}
\end{lemma}

\begin{proof}
\begin{enumerate}
\item $\ker f = \mathrm{Im}(0 \rightarrow M')$
\item $\mathrm{Im}(g) = \ker (M'' \rightarrow 0) = M$
\end{enumerate}
\end{proof}

\begin{definition}
A short exact sequence is a sequence of the form $$0 \rightarrow A \xrightarrow{f} B \xrightarrow{g} C \rightarrow 0$$ is exact everywhere.
\end{definition}

\begin{proposition}
If  $$0 \rightarrow M' \xrightarrow{f} M \xrightarrow{g} M'' \rightarrow 0$$ is exact, then $$0 \rightarrow M_{S}' \xrightarrow{f_{S}} M_{S} \xrightarrow{g_{S}} M_{S}'' \rightarrow 0$$ is exact
\end{proposition}

\begin{proof}
\textbf{Claim}: $\ker(f)_{S} = \ker(f_{S})$\\
$\subseteq $ is clear since $\frac{f(m)}{s} = 0$. Suppose $\frac{m}{s} \in \ker(f_{S}) \Rightarrow f(m)/s = 0 \in M_{S}$. This means that there is a $t\in S$ such that $tf(m)=0=f(tm)\Rightarrow tm \in \ker(f) \Rightarrow \frac{tm}{ts} = \frac{m}{s} \in \ker(f)_{S}$. This gives us $"\supseteq"$. \\

Similarly, $\mathrm{Coker}(f)_{S} = \mathrm{Coker}(f_{S})$. This completes the proof.
\end{proof}

Next, take $\pr \subseteq A$ be a prime ideal and $A\backslash \pr$ be the multiplicative set $S$. We denote $M_{S}$ by $M_{\pr}$.
\begin{itemize}
\item If $M=0$, then $M_{\pr} =0$ for all prime ideals $\pr$. This implies $M_{\m} =0$ for all maximal ideals $\m$.
\item If $M_{\m} =0 \; \forall \; \m \Rightarrow M=0$. Take an element $m \in M$ such that $\frac{m}{1}=0 \in M_{\m}$ for each maximal ideal $\m $ in $A$. Suppose $\mathrm{Ann}(m) \neq A$, then $\mathrm{Ann}(A) \subsetneq \m'$ for some maximal ideal $\m'$. But then we will have $sm=0$ for some $s\in A \bs \mathrm{Ann}(m)$ which is a contradiction. Hence, $\mathrm{Ann}(m)=A$ and $m=0$. This completes the claim.
\item If $M \xrightarrow{f} N$ is an isomorphism iff $M_{\m} \xrightarrow{f_{\m}} N_{\m}$ is an isomorphism for all maximal ideals $\m$.
\end{itemize}
We can summarise in the following theorems
\begin{theorem}
Let $M$ be an $A$-module and $m \in M$. Then TFAE:
\begin{enumerate}
\item $m=0$.
\item $\frac{m}{1}=0$ in $M_{\pr}$ for all prime ideals $\pr$ of $A$.
\item $\frac{m}{1}=0$ in $M_{\m}$ for all maximal ideals $\m$ of $A$. 
\end{enumerate}
\end{theorem}

\begin{theorem}
Let $M$ be an $A$-module. Then TFAE:
\begin{enumerate}
\item $M=0$.
\item $M_{\pr}=0$ for all prime ideals $\pr$ of $A$.
\item $M_{\m}=0$ for all maximal ideals $\m$ of $A$. 
\end{enumerate}
\end{theorem}

\begin{theorem}
Let $\phi: M \rightarrow N$ be an $R$-module homomorphism. Then, TFAE:
\begin{enumerate}
\item $\phi$ is injective. 
\item $\phi_{\pr}: M_{\pr} \rightarrow N_{\m}$ is injective for all prime ideals $\pr$.
\item $\phi_{\m} : M_{\m} \rightarrow N_{\m}$ is injective for all maximal ideals $\m$.
\end{enumerate} 
\end{theorem}

\begin{proof}
From the exactness of the sequence as per a proposition mentioned above we have $1 \Rightarrow 2, 1 \Rightarrow 3$. Moreover, $2 \Rightarrow 3$. We wish to show that $3 \Rightarrow 1$. Let $M' = \ker (\phi)$. Then we have the following exact sequence $$0 \rightarrow M' \rightarrow M \rightarrow N $$ By the proposition above, we have $$0 \rightarrow M'_{\m} \rightarrow M_{\m} \rightarrow N_{\m} $$ exact. This implies $M'_{\m} = \ker (\phi_{\m}) = 0$ since $\phi_{\m}$ is injective by hypothesis. Therefore, $M'_{\m} = 0$ for all maximal ideals $\m$. Now, the result follows from previous theorem.
\end{proof}

The same theorem can be repeated with injective replaced with surjective. This leads us thereby to the last conclusion in the points mentioned before these theorems.
\begin{definition}
Suppose $$M \xrightarrow{f} N$$ Then, $f$ is a monomorphism means 
\[\begin{tikzcd}
	T & M & N
	\arrow["f", from=1-2, to=1-3]
	\arrow["g", from=1-1, to=1-2]
	\arrow[from=1-1, to=1-2]
	\arrow["h"', shift right=2, from=1-1, to=1-2]
\end{tikzcd}\] such that $f\circ g = f\circ h \Rightarrow g=h$. \\

Epimorphism is the dual of this.
\end{definition}

\begin{remark}
For sets, these mean injection and surjection but monomorphism and epimorphism need not mean isomorphism in a random category. 
\end{remark}

\chapter{Lecture-7 (7th February, 2023): Hilbert-Basis Theorem}

\section{Hilbert basis theorem}

\begin{definition}
An ideal $I$ is irreducible if $$I = J \cap K$$ implies $I \supseteq J$ or $I \supseteq K$ (could just say $I=J$ or $I = K$).
\end{definition}

\begin{theorem}
In a Noetherian ring, every ideal is finite intersection of irreducible ideals.
\end{theorem}

\begin{proof}
Say $S$ is the collection of ideals that are not finite intersections of irreducible ideals. $S$ is non-empty since maximal ideals are 
\end{proof}

\begin{proposition}
If $R$ is Noetherian, then $R/I$ is also Noetherian for any ideal $I$.
\end{proposition}


\begin{theorem}[Hilbert Basis theorem]
Let $R$ be a Noetherian ring, then $R[X]$ is also Noetherian.
\end{theorem}

\begin{proof}

\end{proof}


\begin{definition}
A $R$-module $M$ is Noetherian iff it satisfies one of the following equivalent conditions: 
\begin{enumerate}
\item Every submodule $N$ of $M$ is finitely generated.
\item Every collection of submodules has a maximal element.
\item Satisfies a.c.c. 
\end{enumerate}
\end{definition}

\begin{proposition}
If $M_{1}, M_{2}$ are Noetherian $R$-modules, then $M_{1} \oplus M_{2}$ is also Noetherian. 
\end{proposition}

\begin{proof}

\end{proof}

\begin{corollary}
If $R$ is Noetherian ring, then $R^n$ is a Noetherian module.
\end{corollary}

\begin{proposition}
Every finitely generated module on a Noetherian ring is Noetherian.
\end{proposition}


\chapter{Lecture-8 (8th February, 2023): Affine Varieties }

\section{Zariski topology }

Let $k$ be an algebraically closed field. \\

$\Aa^n (k) = k^n$ be the affine space. \\

$k[X_{1}, \hdots , X_{n}=: \mathbf{X}]$ be the polynomial ring that is Noetherian by Hilbert basis theorem. \\

Take an ideal $I \subseteq k [\mathbf{X}]$. We define $$\V (I) = \{\mathbf{x} :=(x_{i}) \in \Aa^n: f(\mathbf{x})=0 \; \forall \; f\in I \} = \bigcap_{f_{1}, \hdots , f_{r} \text{ generates } I} \V(f_{i})$$


\textbf{Properties:}
\begin{enumerate}
\item If $I \subseteq J \Rightarrow \V(I) \supseteq \V(J)$.
\item If $I,J$ are ideals, then $\V(I) \cup \V(J) = \V(I \cap J) = \V(IJ)$. 
\item If $\{I_{\alpha}\}_{\alpha \in A}$ is a collection of ideals, then $\V(\sum_{\alpha \in A} I_{\alpha}) = \ds{\bigcap_{\alpha \in A} \V(I_{\alpha})}$.
\item $\V(I) = \V(\sqrt{I})$. Clearly, $\V(I) \supseteq \V(\sqrt{I})$.
\end{enumerate} 

The first $3$ properties clearly give a topology on the ideals of $k[\mathbf{X}]$ by taking $\V(I)$ to be the closed sets. This topology is called the Zariski topology.\\

Say, we are given subset $S$ of $\Aa^n(k)$. We wish to associate an ideal to this subset by defining $$\I(S)= \{f \in k [\mathbf{X}] : f(\mathbf{x})=0\}$$

\textbf{Properties}:
\begin{enumerate}
\item It is clearly an ideal. Actually, it is a radical ideal. 
\item 
\end{enumerate}

\begin{align*}
\{\text{ideals in } k[\mathbf{X}]\} &\rightarrow \{ \text{subsets of }  \Aa^n\} &\rightarrow \{ \text{ideals in }  k[\mathbf{X}]\}\\
I &\rightarrow \V(I) &\rightarrow \I(V(I))
\end{align*}

Clearly, $I \subseteq \I(\V(I)) \Rightarrow \sqrt{I} \subseteq \I(\V(I))$. So, instead of asking whether $I = \I(\V(I))$, it is much more interesting to ask whether $\sqrt{I} = \I(\V(I))$ \\


Take an ideal $I \in k [\mathbf{X}]$ and say $\{f_{1}, \hdots , f_{r} \}$ generates $I$. Then, $\V(I)$ is the set of points where all $f_{i}'s$ vanish. Say $g$ vanishes where $f_{i}'s$ vanish. Then, $g \in \sqrt{I} \Rightarrow g^n \in I \Rightarrow g^n = \sum_{i} r_{i}f_{i}$. \\

Next, consider the ideal $J$ generated by $f_{1}, \hdots , f_{r}, gX_{n+1}-1 \subseteq k[\mathbf{X},X_{n+1}]$. If $(a_{1}, \hdots , a_{n+1}) \in \V(J)$, then all $f_{i}s$ vanish on this point and thus $g$ also vanishes which implies polynomials in $J$ cannot simultaneously vanish and thus $\V(J)=\emptyset$. So, naturally we can ask whether $J=R$ OR $\V(J)= \emptyset \Leftrightarrow J=R$ (a deep theorem, we will prove later). Suppose we have proven this already and proceed further. Take $g$ as before. \\

Now, this means 
\begin{align*}
1 &= \sum_{i} r_{i}f_{i} + r_{n+1}(gX_{n+1}-1) \\
g^N &= \sum_{i} s_{i}f_{i}  + s_{n+1}(g-Y) \text{  for some large value of }N \text{ and } Y = 1/X_{n+1}
\end{align*} 
If we set $G=Y$, then clearly $g^N \sum_{i} s_{i}f_{i} \Rightarrow g^N \in I \Rightarrow g \in \sqrt{I}$. This leads us to the famous Hilbert Nullstellansatz. (This is just a sketch of what will happen, so spare the details for now)


\chapter{Lecture-9 (9th February, 2023): Tensor products}

Defined tensor product. Not writing it down. Universal property, blah blah 

\begin{theorem}
Suppose $R\rightarrow S$ is a ring homomorphism. Then, $R \rightarrow S$ is an epimorphism iff $S \otimes_{R} S = S$
\end{theorem}

\begin{proof}

\end{proof}

\chapter{Lecture-10 (14th February, 2023): More Tensor products}

We have shown that $R \rightarrow S$ is a ring epimorphism iff $S \otimes_{R} S = S$. Therefore, if we localise $A$ wrt $T$, we know that $A\rightarrow A_{T}$ is an epimorphism iff $A_{T} \otimes_{A} A_{T} = A_{T}$. 



\chapter{Lecture-11 (16th February, 2023): }





\chapter{Shaferavich Alg geo rant}
This chapter will contain anything that requires justification and also solutions to exercises. 
\section{Schemes}
\subsection{The Spec of a ring}





\part{Algebraic Geometry I}

\chapter{Lecture-1 (9th January, 2023): Topological properties and Zariski Topology}
\section{Topological properties}

Consider a topological space $X$. 
\begin{definition}
\begin{enumerate}
\item We say $X$ is quasi-compact if every open cover of $X$ admits a finite subcover. 
\item If $f: X \rightarrow Y$ is continuous, we call $f$ quasi-compact if $f^{-1}(V)$ is quasi-compact for all quasi-compact open $V \subseteq Y$.
\end{enumerate}
\end{definition} 

\begin{exercise}
Composition of quasi-compact maps is quasi-compact.\\

Consider the two maps $f: X \rightarrow Y$ and $g: Y \rightarrow Z$. Next, look at the composition $g \circ f : X \rightarrow Z$. For all quasi-compact open $V \subseteq Z$, $(g \circ f)^{-1} (V) = f^{-1} \circ g^{-1} (V)$. Since $g$ is quasi-compact and continuous, $g^{-1}(V)$ is also quasi-compact and open. Similarly, $f$ is also quasi-compact and continuous, therefore $f^{-1}(g^{-1}(V))$ is also quasi-compact and we are done.
\end{exercise} 

\begin{lemma}
$X$ quasi-compact and $Y \subseteq X$ is closed implies $Y$ is quasi-compact.
\end{lemma}

\begin{proof}
Let $\{U_{i}\}_{i \in I}$ be an open cover of $Y$. Set $U = X - Y$. Since $U_{i}$ is open in $Y$, we have $U_{i} = Y \cap V_{i}$ where $V_{i}$ is open in $X$. Now we note that $\{V_{i}\}_{i \in I} \cup U$ covers $X$ but $X$ is quasi-compact and we obtain a finite subcover $\{V_{i}\}_{i \in J} \cup U$ where $J$ is finite. The corresponding $U_{i}, i \in J$ must therefore cover $Y$ and we are done.
\end{proof}

\begin{proposition}
If $X$ is quasi-compact and Hausdorff, then $E \subseteq X$ is quasi-compact iff $E$ is closed.
\end{proposition}

\begin{proof}
$\Leftarrow$ direction is done. \\
$\Rightarrow$ direction is what we need to prove. \\
Take $x\in X \backslash E$. For each $y \in E$, due to Hausdorff-ness we have two disjoint open sets $U_{y}$ and $U_{y}$ containing $x$ and $y$ respectively. Do this for all $y \in E$. The collection $\{U_{y}\}_{y \in E}$ covers $E$ but it is quasi-compact thus we get a finite subcover $\{U_{y_{i}}\}_{i \in I}$ with $I$ finite. Now, let $$U = \bigcap_{i \in I} U_{y_{i}}$$ $U$ is clearly open, contains $x$ and is disjoint from $E$. Since $x$ was chosen arbitrarily, $X\backslash E$ must be open. 
\end{proof}

\begin{lemma}
Any finite union of quasi-compact spaces is quasi-compact.
\end{lemma}

\begin{proof}
Suppose $X_{i}, i =1,2,\hdots ,n$ are the spaces in question. We want to show that $$X = \bigcup_{i=1}^n X_{i}$$ is also quasi-compact. Take any cover $\{U_{i}\}_{i \in I}$  be an open cover of $X$. Then for each $i=1,2, \hdots , n$ it is clear that $\{U_{i}\}_{i \in I}$ also covers $X_{i}$. Using quasi-compactness of $X_{i}$ we can get a finite subcollection $\{U_{i_{j}}: j =1 , \hdots , n_{i}\}$. This can be done for all $i$. Now, consider $\bigcup_{i=1}^{n} \bigcup_{j=1}^{n_{i}} U_{i_{j}}$. This union covers $X$ and is finite. So, we are done. 
\end{proof}

\begin{lemma}
Suppose $f: X \rightarrow Y$ is continuous, if $X$ is quasi-compact then so is $f(X)$.
\end{lemma}

\begin{proof}
Let $\{U_{i}\}_{i \in I}$ be an open cover of $f(X)$. Now, $\{f^{-1}(U_{i})\}_{i \in I}$ covers $X$ and by continuity, each of them are open. Use quasi-compactness of $X$ to get a finite subcover that covers $X$.
\begin{align*}
X &= \bigcup_{i=1}^n f^{-1}(U_{i}) \\
\because f(f^{-1}(U_{i})) &\subseteq U_{i} \\
\therefore f(X) &\subseteq \bigcup_{i=1}^n U_{i} 
\end{align*}
\end{proof}

Suppose $\Sigma$ is a poset. $\Sigma$ satisfies acc if every ascending chain $$x_{1} \le x_{2} \le \cdots $$ is stationary.

\begin{lemma}
The following are equivalent: 
\begin{enumerate}
\item $\Sigma$ satisfies acc. 
\item Every non-empty subset of $\Sigma$ has maximal element.
\end{enumerate}
\end{lemma}

\begin{proof}
$1 \Rightarrow 2$. Suppose $S \subseteq \Sigma$ has no maximal element. \\
Then choose $x_{0} \in S$ non-maximal, then we can find a $x_{1}$ such that $x_{0} \lneq x_{1}$. By induction we can construct an infinite chain $x_{0} \lneq x_{1} \lneq \cdots \neq x_{i} \lneq \cdots$ which does not terminate which is a contradiction to our hypothesis. Thus, $S$ must have a maximal element.\\
$2 \Rightarrow 1$. Suppose $x_{1} \le x_{2} \le \cdots \le x_{i} \le $ is an infinite ascending chain, then $S = \{x_{i}\mid i \geq 1\}$ has no maximal element. 
\end{proof}

\begin{definition}
A topological space is called Noetherian if set of all closed subsets of $X$ satisfies dcc.
\end{definition}

\begin{lemma}
$X$ Noetherian implies $X$ is quasi-compact.
\end{lemma}

\begin{proof}
Let $\mathcal{U}=\{U_{i}\}_{i \in I}$ be an open cover of $X$ that does not have a finite subcover. Consider the collection $\mathcal{F}$ of union of finite number of elements of $\mathcal{U}$. Since being Noetherian is equivalent to saying any finite subset of open subsets has a maximal element, we know that $\mathcal{F}$ has a maximal element. Suppose that maximal element is $U_{i_{1}} \cup \hdots \cup U_{i_{n}}$. If this does not cover $X$, take an element $x$ in the complement of the maximal element. Since $\mathcal{U}$ covers $X$, there is an $i \in I$ such that $x \in U_{i}$. Notice that now $U_{i_{1}} \cup \hdots \cup U_{i_{n}} \subseteq U_{i_{1}} \cup \hdots \cup U_{i_{n}} \cup U_{i}$ which contradicts the maximality. Thus, we are done.  
\end{proof}

\begin{remark}
The converse need not be true. Consider $[0,1]$ covered by $[1/2^n , 1]$.
\end{remark}

\begin{lemma}
If $X_{1}, \hdots , X_{n}$ are Noetherian subspaces of $X$, then so is $X=X_{1} \cup X_{2} \cup \hdots \cup X_{n}$
\end{lemma}

\begin{proof}
Let $Y_{i}$s be closed in $X$ that forms the chain $$X \supseteq Y_{1} \supseteq Y_{2} \supseteq Y_{3} \supseteq \cdots $$ For each $i$, we get a chain of closed sets in $X_{i}$ by intersecting with $X_{i}$. This gives us $$X_{i} \supseteq Y_{1}\cap X_{i} \supseteq Y_{2}\cap X_{i} \supseteq Y_{3}\cap X_{i} \supseteq \cdots $$ Since $X_{i}$ is Noetherian, this chain terminates at say $r_{i}$. Now, take $r = \max_{i} r_{i}$. The original chain will terminate after this point. Suppose $y \in Y_{i}$ with $i \le r$, there is an $j$ such that $y \in X_{j}$. This means $y \in X_{j} \cap Y_{i} = X_{j} \cap Y_{r}$. Hence, $y \in Y_{r}$ and we are done.
\end{proof}

\begin{definition}
Locally Noetherian means every point $x\in X$ has a neighbourhood $U$ which is Noetherian wrt subspace topology.
\end{definition}

\begin{lemma}
Quasi-compact and locally Noetherian implies Noetherian.
\end{lemma}

\begin{proof}
Since $X$ is locally Noetherian, for each $x\in X$ we have a nbd. $U_{x}$ that is Noetherian. $\{U_{x}\}_{x \in X}$ is an open cover of $X$. Quasi-compactness gives us a finite subcover $\{U_{x_{i}}\}_{i=1}^n$, i.e., $$X = \bigcup_{i=1}^n U_{x_{i}}$$ $X$ is Noetherian from previous lemma.
\end{proof}

\begin{exercise}
Give an example of a ring $R$ such that $\spec(R)$ is Noetherian but $R$ is not. \\

Consider the ring $R=k[X_{1}, X_{2}, \hdots , ]$ and the ideal $I = \langle X_{1}^2 , X_{2}^2 , \hdots , \rangle$. Now, look at $R'=R/I$. $\spec(R')$ is a singleton. 
\end{exercise}

\begin{definition}
A topological space $X$ is called irreducible if it cannot be written as finite union of proper closed subsets. \\

A closed subset $Y \subseteq X$ is called irreducible component of $X$ if it is a maximal irreducible closed subset of $X$.
\end{definition}

\begin{lemma}
If $X$ is Noetherian and $Y \subseteq X$ is a subspace, then $Y$ is Noetherian. 
\end{lemma}

\begin{proof}
Let $Y_{i}$s be closed in $Y$ that forms the chain $$Y \supseteq Y_{1} \supseteq Y_{2} \supseteq Y_{3} \supseteq \cdots $$ For each $i$, we have a closed set in $X$ such that $Y_{i} = Y \cap X_{i}$. This gives us $$Y \supseteq X_{1}\cap Y \supseteq X_{2}\cap Y \supseteq X_{3}\cap Y \supseteq \cdots $$
\end{proof}

\begin{lemma}
Let $X$ be Noetherian. Then, $X$ has finitely many irreducible components. 
\end{lemma}

\begin{proof}
More generally, we will show that every closed subset for $X$ has finitely many irreducible components. \\

Suppose that this is false. Let $\Sigma$ be the collection of closed subsets of $X$ that does not satisfy our condition. Order this as follows: $A \le B$ if $A \supseteq B$. If $\{C_{i}\}$ is a chain in $\Sigma$, then it must eventually stabilise since $X$ is Noetherian. This $C_{\alpha}$ is an upper bound for this chain. Therefore, by Zorn's lemma, there is a maximal element $Y$. Since $Y \in \Sigma$, therefore it is not irreducible. Suppose $Y = Y_{1} \cup Y_{2}$ with $Y_{1},Y_{2}$ proper closed subsets of $Y$. $Y \le Y_{1}, Y \le Y_{2}$. Since $Y \in \Sigma$, $Y$ is not a finite union of irreducible components. Hence, either $Y_{1}$ or $Y_{2}$ is not irreducible. If $Y_{1}$ is not irreducible but $Y_{1} \in \Sigma$, since $Y$ is maximal in $\Sigma$ and $Y \le Y_{1}$, therefore $Y =Y_{1}$ a contradiction that $Y_{1}$ is a proper subset of $Y$. Thus, $\Sigma$ must be empty and the claim is proven. 
\end{proof}

\begin{lemma}
$X$ is Noetherian implies there exists an unique expression $X = X_{1} \cup \cdots \cup X_{n}$ where $X_{i}'s$ are irreducible components of $X$.
\end{lemma}

\begin{proof}
Suppose $$X = X_{1} \cup \cdots \cup X_{n} = X_{1}' \cup \cdots \cup X_{m}'$$ Clearly $X_{1}' \subseteq X$, this means $X_{1}' = \bigcup_{i=1}^n X_{1}' \cap X_{i}$. Since $X_{1}'$ is irreducible, there must be a $i_{1}$ such that $X_{1}' = X_{i_{1}} \cap X_{1}'$. Thus, $X_{1}' \subseteq X_{i_{1}}$. We can choose $i_{1}$ to be $1$ to get $X_{1}' \subseteq X_{1}$. Similarly, $X_{1} \subseteq X_{j_{1}}'$. Since $X_{1}' \subseteq X_{j_{1}}'$ and our assumption that $X_{i} \not \in X_{j}$ for $i \neq j$ we conclude that $j_{1}=1$. Finally, we conclude that $X_{1}=X_{1}'$. Let $Z$ be the closure of $X - X_{1}$, then $Z = X_{2} \cup \cdots \cup X_{n} = X_{2}' \cup \cdots \cup X_{m}' $. We can argue inductively and conclude that $X_{i} = X_{i}'$ and $n=m$. 
\end{proof}

\begin{lemma}
Suppose $X$ is Noetherian and $X_{1} \subseteq X$ an irreducible component. Then, $X_{1}$ contains a non-empty open set in $X$.
\end{lemma}

\begin{proof}
Consider $U = X \backslash X_{2} \cup \cdots \cup X_{n}$. Clearly, $U$ is non-empty and open. Moreover, $U \subseteq X_{1}$ and we are done. 
\end{proof}

\begin{definition}
Let $X$ be a topological space. We say that $X$ is a spectral space if the following holds: 
\begin{enumerate}
\item $X$ is quasi-compact.
\item $X$ is $T_{0}$.
\item $X$ has a basis of quasi-compact open sets.
\item Every irreducible closed subset of $X$ has a generic point ($\exists x \in Y$ such that $\overline{\{x\}}  = X$)
\end{enumerate}
\end{definition}

\section{Zariski Topology}
Let $A$ be a commutative ring with identity and $X = \spec(A)$. \\

Zariski topology is the unique topology such that a subset $Y\subseteq X$ is closed iff $Y = \V (I)$ for some ideal $I \unlhd A$. Here, $$\V(I) = \{\pr \in X \mid \pr \supseteq I\}$$

\begin{theorem}
$\spec(A)$ is always spectral.
\end{theorem}

\begin{proof}
\begin{enumerate}
\item $X$ is $T_{0}$ \\
For all $f \neq 0$ in $A$,  let $A_{f} = S^{-1}A$ be the localisation of $A$ at $f$ where $A_{f} = \{f^n \mid n \geq 0\}$. Next, let $V_{f} = X \backslash V(f) = \spec (A_{f})$. This forms a basis for the Zariski topology. \\
Now, let $\pr, \mathfrak{P}$ be two distinct primes. 
\begin{itemize}
\item Suppose $\pr \not \subseteq \mathfrak{P}$. \\
$Y = V(\pr)$ is closed set and $\mathfrak{P} \not \in V(\pr)$. Take $Y^c$. Then $\mathfrak{P} \in Y^c$ and $\pr \not \in Y^c$. 
\item If $\pr \subseteq \mathfrak{P}$ \\
Then consider $\V(\mathfrak{P})$. Clearly, $\pr \not \in \V(\mathfrak{P})$. Take $U = \V(\mathfrak{P})^c$, then $\pr \in U$ but $\mathfrak{P} \not\in U$. 
\end{itemize}
\item $X$ is quasi-compact. \\
Let $\{U_{i}\}$ be an open cover of $X$. WLOG, we can assume that $U_{i} = \spec(A_{f_{i}}), f \neq 0$. Let $I$ be the ideal generated by these $f_{i}s$. \\
\textbf{Case-1}: Suppose that $I \neq A$. Then there exists a maximal ideal $\m \supseteq I \Rightarrow \V(\m) \subseteq \V(I) \Rightarrow X \backslash \V(\m) \supseteq X\backslash \V(I) = X \backslash \bigcap_{i \in I}\V(f_{i}) = \bigcup U_{i} = X$ which is absurd. Hence, we conclude that $I=A$. Next, 
\begin{align*}
1 &= \sum_{i=1}^n a_{i}f_{i} &\text{ for some } a_{i} \in A \\
\Rightarrow \bigcup_{i=1}^n U_{i} &= \bigcup_{i=1}^n X \backslash \V(f_{i})
\end{align*}
And, we get the required refinement. 
\item $X$ has a basis of quasi-compact open sets follows from the above.
\item Let $Y \subseteq X$ be an irreducible closed subset. Then, $Y = \spec(A/I)$. WLOG, we can assume $X$ is irreducible. Next, observe that $\spec(A) = \spec(A_{\mathrm{red}}) = \spec(A/ \nil(A))$. Since $A$ is irreducible and reduced, we conclude that $A$ is an integral domain. We are now done since $0$ is a generic point in that case.
\end{enumerate}
\end{proof}

\chapter{Lecture-2 (11th January, 2023): Zariski topology and affine schemes}

\section{Zariski topology contd..}

\begin{theorem}[Hochster]
Every spectral space is homeomorphic to $\spec(A)$ for some commutative ring $A$.
\end{theorem}

\textbf{Notation:} $\mathrm{\textbf{Ring}}$ be the category of commutative rings, $\mathrm{\textbf{Top}}$ be the category of topological spaces.

\begin{theorem}
There is a contravariant functor 
\begin{align*}
sp: \mathrm{\textbf{Ring}} &\rightarrow \mathrm{\textbf{Top}} \\
\spec(B) &\mapsto \spec(A)
\end{align*}
\end{theorem}

\begin{proof}
Consider $f: A \rightarrow B$. This induces a map $$f_{\#} : \spec(B) \rightarrow \spec(A)$$ such that $f_{\#}(\pr) = f^{-1}(\pr)$. \\
\textbf{Well-defined:} Suppose $xy \in f^{-1}(\pr) \Rightarrow f(xy) = f(x)f(y) \in \pr \Rightarrow$ either $x$ or $y$ lies in $f^{-1}(\pr)$ which completes our check. \\
We claim that $f_{\#}$ is continuous. This can be seen as follows: \\
Take a basic open set $D(a), a\in A$. Enough to show for these sets since $D(a)$ forms a basis for the topology on $\spec(A)$. Now, $$\pr \in f_{\#}^{-1}(D(a))\Leftrightarrow f_{\#}(\pr) \in D(a) \Leftrightarrow a \not \in f^{-1}(\pr)$$ But this means $$a \not \in f^{-1}(\pr) \Leftrightarrow f(a) \not \in \pr \Leftrightarrow \pr \in D(f(a))$$
\end{proof}

\section{Affine schemes}

\begin{definition}
$\spec(A)$ will be called an affine "scheme" (we will see this properly later on).
\end{definition}

\begin{definition}
Let $X= \spec(A), Y = \spec(B)$. Let $f: Y \rightarrow X$ be a continuous map. We call such a map $f$ regular (holomorphic) if there is a ring homomorphism $g: A \rightarrow B$ such that $f = g_{\#}$
\end{definition}

\begin{example}
Take $\spec(\ZZ)$ and consider the constant map. This cannot be regular because any ring homomorphism must take $1$ to $1$ and as a consequence fixes every element. 
\end{example}

\begin{proposition}
If $X = \spec(A)$. A regular function on $X$ is a regular map from $X$ to $\spec(\ZZ[t])$.
\end{proposition}

\begin{proof}

\end{proof}

\begin{remark}
On an affine scheme, the set of all regular maps is the ring $A$ itself since, the map $\ZZ[t] \rightarrow A$ is determined by where $t$ is sent to. 
\end{remark}

\begin{lemma}
Every affine scheme has a closed point.
\end{lemma}

\begin{proof}
Every commutative ring has a maximal ideal. 
\end{proof}

\begin{definition}
Open in affine is called quasi-affine.
\end{definition}

\begin{example}
Take $A$ a local integral domain with $\m$ the maximal ideal. Suppose that all prime ideals of $A$ are of the form $$\langle 0 \rangle \subset \pr_{1} \subset \pr_{2} \subset \cdots \subset \{ \m \}$$ Consider $X = \spec(A) \backslash \m$. $X$ is open in affine scheme but has no closed point. \\

An example of such a ring is $$\Gamma = \ZZ x_{1} \oplus \ZZ x_{2} \oplus \cdots $$ Give an ordering: $\sum a_{i}x_{i} \geq 0$ if the first nonzero term is $>0$ or all $a_{i}=0$\\
$\Gamma$ is a totally ordered abelian group and hence there exists a valuation ring $A$ with value group $\Gamma$ and the prime ideals of $\Gamma$ are in $1-1$ correspondence with prime ideals of $A$.
\end{example}

\begin{exercise}
Let $A = k[X_{1}, X_{2}, \hdots ], B= A_{\m}, X = \spec(B) \backslash \m , \m = \langle X_{1}, X_{2}, \hdots , \rangle$. Claim is that $X$ has no closed point.
\end{exercise}

\subsection{Fiber products of affine schemes}

Suppose $A$ is a commutative ring, $B,C$ are $A$-algebras. Let $X= \spec(A), Y = \spec(B), Z = \spec(C)$. Next, suppose we have 
\[\begin{tikzcd}
	A & B \\
	C
	\arrow["f", from=1-1, to=1-2]
	\arrow["g"', from=1-1, to=2-1]
\end{tikzcd}\]

\textbf{Universal property of fiber products}:
\[\begin{tikzcd}
	{W'} \\
	& Y \times_{X} Z & Z \\
	& Y & X
	\arrow["{f_{\#}}"', from=3-2, to=3-3]
	\arrow["{g_{\#}}", from=2-3, to=3-3]
	\arrow[from=2-2, to=2-3]
	\arrow[from=2-2, to=3-2]
	\arrow[curve={height=12pt}, from=1-1, to=3-2]
	\arrow[curve={height=-12pt}, from=1-1, to=2-3]
	\arrow["{\exists \; ! }", dashed, from=1-1, to=2-2]
\end{tikzcd}\]

\begin{definition}
If a $W$ exists such that the universal property is satisfied, then $W$ is called the fiber product of $Y,Z$ over $X$ and we write $W = Y \times_{X} Z$
\end{definition}

\begin{theorem}
$\mathrm{\textbf{Aff}_{\ZZ}}=$ category of affine schemes admits fiber products.
\end{theorem}

\begin{proof}
Consider the following data: 
\[\begin{tikzcd}
	A & B \\
	C
	\arrow["f", from=1-1, to=1-2]
	\arrow["g"', from=1-1, to=2-1]
\end{tikzcd}\]

Let $D = B \otimes_{A} C$. We have the natural maps $f_{1}: B \rightarrow  B \otimes_{A} C$ sending $b \mapsto b \otimes 1$ and $f_{2}: C \rightarrow B \otimes_{A} C$ sending $c \mapsto 1 \otimes c$. Both are ring homomorphisms and fit into the following diagram due to the nature of tensor product 
 \[\begin{tikzcd}
	A & B \\
	C & {B\otimes_{A} C}
	\arrow["f", from=1-1, to=1-2]
	\arrow["g"', from=1-1, to=2-1]
	\arrow["{f_{1}}", from=1-2, to=2-2]
	\arrow["{g_{1}}"', from=2-1, to=2-2]
\end{tikzcd}\]

Now, let $W = \spec(B \otimes_{A} C)$ and we claim that this satisfies the universal property of fibre product. Apply $\spec(-)$ functor to the diagram to get 
\[\begin{tikzcd}
	A & B \\
	C & {\mathrm{Spec}(B\otimes_{A} C)}
	\arrow["{g_{1\#}}"', tail reversed, no head, from=2-1, to=2-2]
	\arrow["{f_{1\#}}"', from=2-2, to=1-2]
	\arrow["{f_{\#}}"', from=1-2, to=1-1]
	\arrow["{g_{\#}}", from=2-1, to=1-1]
\end{tikzcd}\]

From the universal property of tensor product we have the following diagram

\[\begin{tikzcd}
	A & B \\
	C & {B \otimes_{A}C} \\
	&& U
	\arrow["f", from=1-1, to=1-2]
	\arrow["g"', from=1-1, to=2-1]
	\arrow["{f_{1}}", from=1-2, to=2-2]
	\arrow["{g_{1}}"', from=2-1, to=2-2]
	\arrow[curve={height=18pt}, from=2-1, to=3-3]
	\arrow[curve={height=-18pt}, from=1-2, to=3-3]
	\arrow["{\exists \; ! }",dashed, from=2-2, to=3-3]
\end{tikzcd}\]

Again, apply the $\spec(-)$ functor. 

\[\begin{tikzcd}
	X & Y \\
	Z & {\mathrm{Spec}(B\otimes_{A} C)} \\
	&& {\mathrm{Spec}(U)}
	\arrow["{g_{1\#}}"', tail reversed, no head, from=2-1, to=2-2]
	\arrow["{f_{1\#}}"', from=2-2, to=1-2]
	\arrow["{f_{\#}}"', from=1-2, to=1-1]
	\arrow["{g_{\#}}", from=2-1, to=1-1]
	\arrow[curve={height=-24pt}, from=3-3, to=2-1]
	\arrow[curve={height=18pt}, from=3-3, to=1-2]
	\arrow["{\exists \; ! }",dashed, from=3-3, to=2-2]
\end{tikzcd}\]

This completes the proof.
\end{proof}

\chapter{Lecture-3 (16th January, 2023): Category theory brushup}

Suppose we have a ring homomorphism $f : A \rightarrow B$ and $X = \spec(A), Y = \spec(B)$. This induces a map $f_{\#}: Y\rightarrow X$. From, the previous discussion, there is a fiber product $Y \times_{X} Y$ such that the following diagram makes sense

\[\begin{tikzcd}
	X & Y \\
	Y & Y \times_{X} Y \\
	&& Y
	\arrow["{p_{1}}"', tail reversed, no head, from=2-1, to=2-2]
	\arrow["{p_{2}}"', from=2-2, to=1-2]
	\arrow["{f_{\#}}"', from=1-2, to=1-1]
	\arrow["{f_{\#}}", from=2-1, to=1-1]
	\arrow[curve={height=-24pt}, from=3-3, to=2-1]
	\arrow[curve={height=18pt}, from=3-3, to=1-2]
	\arrow["{\exists \; ! \; \Delta_{Y} }",dashed, from=3-3, to=2-2]
\end{tikzcd}\]

Here, $p_{1} \circ \Delta_{Y} = p_{2} \circ \Delta_{Y} = \mathrm{id}$ where $$\Delta_{Y} : Y \rightarrow Y\times_{X} Y$$ is called the relative diagonal of $Y/X$. 

\begin{definition}
Say $X_{1},X_{2}$ are affine schemes. $X_{1} \rightarrow X_{2}$ is a closed immersion iff $A_{1} \rightarrow A_{2}$ is a surjective. Here, $\spec(A_{i}) = X_{i}, i =1,2$. 
\end{definition}

\begin{lemma}
$\Delta_{Y}$ is a closed immersion.
\end{lemma}

\begin{proof}
$B \otimes_{B} B \rightarrow B$ is a surjection. 
\end{proof}

\begin{example}
Take $A = \ZZ, B = \ZZ[t]/ \langle t^n \rangle$ for some $n \geq 2$. There is a canonical inclusion $f: A \rightarrow B$. This induces a map $Y = \spec(B) \rightarrow X = \spec(A)$ which is an identity map in terms of sets. Thus, it is a closed inclusion but not a closed immersion.  
\end{example}

\begin{remark}
We know that diagonal is closed iff the space is Hausdorff. This seems to contradict our assumptions! But we are fine because this claim is true only when the topology is the product topology. Here, the topology we have is not the product topology. 
\end{remark}

\begin{definition}
A regular map $f: X \rightarrow Y$ is called separated morphism if the relative diagonal of $Y$ over $X$ is closed in $Y \times_{X} Y$.
\end{definition}

\begin{lemma}
Let $X = \spec(A)$. Suppose $U_{1},U_{2}$ are two open affine subsets of $X$. Then, $U_{1}\cap U_{2}$ is also affine.
\end{lemma}

\begin{proof}
We have two natural injections $$U_{1} \xhookrightarrow{j_{1}} X , U_{2} \xhookrightarrow{j_{2}} X$$ then we naturally have the following $$U_{1} \times_{Z} U_{2} \xrightarrow{j_{1} \times j_{2}} X \times_{Z} X $$ where $Z = \spec(\ZZ)$ (if it is blank, just assume $Z$ by default). \\
From previous discussion we get 
\[\begin{tikzcd}
	{U_{1}\times_{Z}U_{2}} & {} & {X\times_{Z}X} \\
	{} && {} \\
	&& X
	\arrow["{j_{1}\times j_{2}}", from=1-1, to=1-3]
	\arrow["{\Delta_{X}}"', from=3-3, to=1-3]
\end{tikzcd}\]
Since each term is affine, we can take the fiber product of $U_{1}\times_{Z}U_{2}$ and $X$. Say the fiber product is $W$.
\[\begin{tikzcd}
	& {U_{2}} && X \\
	{U_{1}} & {U_{1}\times_{Z}U_{2}} & {} & {X\times_{Z}X} & X \\
	& {} & \square & {} \\
	& W && X
	\arrow["{\Delta_{X}}"', from=4-4, to=2-4]
	\arrow["{j_{1}\times j_{2}}", from=2-2, to=2-4]
	\arrow["{\Delta'}", from=4-2, to=2-2]
	\arrow["j"', from=4-2, to=4-4]
	\arrow["{p_{1}}"', from=2-4, to=1-4]
	\arrow["{p_{2}}"', from=2-4, to=2-5]
	\arrow["{q_{2}}"', from=2-2, to=2-1]
	\arrow["{q_{1}}"', from=2-2, to=1-2]
\end{tikzcd}\]

Then, we claim that \\

\textbf{Claim}: $W = U_{1} \cap U_{2}$

\begin{proof}
Suppose $x\in W$, then 
\end{proof}

It now remains to show that $W$ is affine but it is clear from the definition of fiber products.

\end{proof}

\begin{remark}
If $\Delta_{X}$ is a closed immersion then so is $\Delta '$. That is, closed immersions are preserved under fiber products. Follows from right exactness of tensor product. 
\end{remark}

Now, that we have discussed intersection, we naturally ask : What happens to $U_{1} \cup U_{2}$. Is it still affine ? \\

The answer turns out to be NO. To see this,

\begin{example}[NON-example]
Consider $k$ be an algebraically closed field. $A  = k [t_{1},t_{2}]$ and $X =\spec(A)$. Let $U_{i}  = \{x \mid t_{i}(x) \neq 0\} = X \backslash \V(t_{i})$. Clearly, $U_{i}$ is open and affine ($ = \spec(A_{t_{i}})$). But $U_{1} \cup U_{2} $ is not affine. 
\[\begin{tikzcd}
	&& {} \\
	&& {} \\
	{} & {} && {} & {t_{2}=0} \\
	\\
	&& {t_{1}=0} \\
	&& {}
	\arrow[tail reversed, from=1-3, to=5-3]
	\arrow[tail reversed, from=3-5, to=3-1]
\end{tikzcd}\]
$U_{1}$ is complement of the horizontal axis and $U_{2}$ of the vertical axis. But $U_{1}\cup U_{2}$ is the complement of origin. The question is asking if the complement of origin is affine or not. A highly NON-TRIVIAL question to answer.
\end{example}

\begin{exercise}[not trivial but do think about it]
Suppose $X = \spec(A)$ and $U \hookrightarrow X$ is affine open. Does this imply $U = \spec(S^{-1}A)$ for some multiplicatively closed set $S \subseteq A$?
\end{exercise}

\begin{definition}
Suppose $S = \spec(A)$ and $x\in X$. Let $K(A) = S^{-1}(A)$ where $S$ is the set of all nonzero divisors in $A$. Here, we have $A \hookrightarrow S^{-1}(A)=$ the ring of all meromorphic functions on $X$. Then, $$\mathcal{O}_{X,x} = \{f \in K(A) \mid f \text{ is regular in a nbd of } x \}$$ is called the germ of regular function. 
\end{definition}

\begin{lemma}
$$\mathcal{O}_{X,x} = A_{\pr}$$ where $\pr =x$.
\end{lemma}

\begin{proof}
Suppose $f$ is regular in a nbd of $\pr$ iff there exists $b \not \in \pr $ such that $f \not \in \V(b)$. But this means $f \not \in A_{b}$ which in turn implies $f \in \bigcup_{b \not \in \pr} A_{b} = A_{\pr}$.
\end{proof}

\begin{definition}
The germs of analytic functions at $x$ is the completion of $\mathcal{O}_{X,x}$, denoted by $\mathcal{O}_{X,x}^{\wedge}$ with respect to its maximal ideal. 
\end{definition}

\begin{remark}
We have the natural map $\mathcal{O}_{X,x} \rightarrow \mathcal{O}_{X,x}^{\wedge}$ but if $\mathcal{O}_{X,x}$ is Noetherian then this map is also injective. 
\end{remark}

\section{Categories and functors}

A category $\mathcal{C}$ consists of a collection $\mathrm{ob}(\mathcal{C})$ and for all $X,Y \in \mathrm{ob}(\mathcal{C})$, there is a set $\mathrm{Hom}_{\mathcal{C}}(X,Y)$ and a map $$\mathrm{Hom}_{\mathcal{C}}(X,Y) \times \mathrm{Hom}_{\mathcal{C}}(Y,Z) \rightarrow \mathrm{Hom}_{\mathcal{C}}(X,Z)$$ satisfying 
\begin{enumerate}
\item $\forall \; X \in \mathrm{ob}(\mathcal{C}) \; \exists \; 1_{X} \in \mathrm{Hom}_{\mathcal{C}}(X,Y)$ such that $f \circ 1_{X} = 1_{X} \circ f = f$
\item $f \circ (g \circ h) = (f \circ g)\circ h$
\end{enumerate}

A functor (contravariant) $\mathcal{F}: \mathcal{C}_{1} \rightarrow \mathcal{C}_{2}$ is a function $\mathcal{F}: \mathrm{ob}(\mathcal{C}_{1}) \rightarrow \mathrm{ob}(\mathcal{C}_{2})$ and a map of sets $\mathcal{F}: \mathrm{Hom}_{\mathcal{C}_{1}}(X,Y)\rightarrow \mathrm{Hom}_{\mathcal{C}_{2}}(\mathcal{F}(X),\mathcal{F}(Y))$ such that 
\begin{enumerate}
\item $f(1_{X}) = 1_{\mathcal{F}(X)}$
\item $\mathcal{F}(f \circ g) = \mathcal{F}(f) \circ \mathcal{F}(g)$
\end{enumerate}

To each category $\mathcal{C}$, we associate a category $\mathcal{C}^{\mathrm{op}}$ such that $$\mathrm{ob}(\mathcal{C}) = \mathrm{ob}(\mathcal{C}^{\mathrm{op}})$$ and $$\mathrm{Hom}_{\mathcal{C}^{\mathrm{op}}}(X,Y) = \mathrm{Hom}_{\mathcal{C}}(Y,X)$$

Suppose $\mathcal{F},\mathcal{F}' : \mathcal{C} \rightarrow \mathcal{C}'$ be two functors. Then, a natural transformation is $T: \mathcal{F} \rightarrow \mathcal{F}'$ consisting of the following data: 
\begin{enumerate}
\item $\forall \; X \in \mathcal{C} , \exists \; T_{X} : \mathcal{F}(X) \rightarrow \mathcal{F}'(X)$ ,i.e., $T_{X} \in \mathrm{Hom}_{\mathcal{C}'}(\mathcal{F}(X),\mathcal{F}' (X))$ such that for all $f: X \rightarrow Y$, the diagram commutes\[\begin{tikzcd}
	{\mathcal{F}(X)} & {} & {\mathcal{F}'(X)} \\
	{} & \circlearrowright & {} \\
	{\mathcal{F}(Y)} && {\mathcal{F}'(Y)}
	\arrow["{T_{X}}", from=1-1, to=1-3]
	\arrow["{\mathcal{F}(f)}"', from=1-1, to=3-1]
	\arrow["{\mathcal{F}'(f)}", from=1-3, to=3-3]
	\arrow["{T_{Y}}"', from=3-1, to=3-3]
\end{tikzcd}\]
\end{enumerate}

Given, $\mathcal{C}, \mathcal{C}'$ then $F(\mathcal{C}, \mathcal{C}')=$ all functors from $\mathcal{C}$ to $\mathcal{C}'$ is a category and $\mathrm{Hom}_{F(\mathcal{C}, \mathcal{C}')}(F_{1},F_{2})=$ all natural transformations from $F_{1}$ to $F_{2}$ 



\chapter{Lecture-4 (20th January, 2023): Category theory}


\section{Category theory contd..}
\subsection{Equivalence of categories}

Two categories $\mathcal{C},\mathcal{C}'$ are equivalent if there exists functors $$\mathcal{F}: \mathcal{C} \rightarrow \mathcal{C}'  \text{ and } \mathcal{G}: \mathcal{C}' \rightarrow \mathcal{C}$$ and natural transformations $$T : \mathrm{id}_{\mathcal{C}} \rightarrow \mathcal{G} \circ \mathcal{F} \text{ and } T' : \mathrm{id}_{\mathcal{C}'} \rightarrow \mathcal{F} \circ \mathcal{G}$$ which are isomorphisms.

\begin{example}
\begin{enumerate}
\item The category of categories with all morphisms being identity is equivalent to the category of sets.
\item The category 
\item Consider the category of $A$-modules and let $B = M_{n}(A)$. We claim that $\mathrm{\textbf{Mod}_{A}}$ and $\mathrm{\textbf{Mod}_{B}}$ are equivalent. This is also known as Morita equivalence.
\end{enumerate}
\end{example}

\subsection{Products and Co-products}


In partially ordered sets, neither product nor co-product might exist. \\

\section{Pre-sheaves and Yoneda lemma}

Suppose $\mathcal{C}$ is a category. Then a presheaf on $\mathcal{C}$ is a contravariant functor $$\mathcal{F} : \mathcal{C} \rightarrow \mathbf{Sets} ( \text{ or }\mathbf{Ab})$$
The category of presheaves on $\mathcal{C}$ is denoted by $\mathbf{Presh(\mathcal{C})}$\\

Suppose $X \in \mathrm{ob}(\mathcal{C})$. Then we can construct $h_{X} \in \mathbf{Presh}(\mathcal{C})$ such that $h_{X}(Y) = \mathrm{Hom}_{\mathcal{C}}(Y,X)$. Hence, we have a functor $$h : \mathcal{C} \rightarrow \mathbf{Presh}(\mathcal{C})$$ that sends $X \mapsto h_{X}$. This $h$ is called the Yoneda functor. 

\begin{lemma}[Yoneda Lemma]
For every pre-sheaf $F$ on $\mathcal{C}$ and for all $X \in \mathrm{ob}(\mathcal{C})$, there exists a natural bijection $$\theta_{X}: \mathrm{Hom}_{\mathbf{Presh}(\mathcal{C})}(h_{X}, F) \rightarrow F(X)$$
\end{lemma}

\begin{proof}
Suppose we are given $f: h_{X} \rightarrow F$. This is a natural transformation and thus we obtain $$f(X): h_{X}(X) \rightarrow F(X)$$ but $h_{X}(X) = \mathrm{Hom}_{\mathcal{C}}(X,X)$ and $\mathrm{id}_{X} \in \mathrm{Hom}_{\mathcal{C}}(X,X)$. This implies $f(X)(\mathrm{id}_{X}) \in F(X)$ and therefore $\theta_{X}(f) = f(X)(\mathrm{id}_{X}) \in F(X)$. \\
Now, let us construct the inverse. Construct $$\psi_{X} : F(X) \rightarrow \mathrm{Hom}_{\mathbf{Presh}(\mathcal{C})}(h_{X}, F)$$ Let $\alpha \in F(X)$, we want to define $$h_{X}(Y) \rightarrow F(Y) \; \forall \; Y \in \mathcal{C}$$ But then $f \in h_{X}(Y)=\mathrm{Hom}_{\mathcal{C}}(Y,X)$ implies $F(X) \xrightarrow{F(f)} F(Y) \Rightarrow F(f)(\alpha) \in F(Y)$. \\
We can easily check that these two maps are inverses which completes the proof.
\end{proof}

Suppose $\mathcal{F}: \mathcal{C} \rightarrow \mathcal{C}'$ is a functor. Then, $\mathcal{F}$ is called faithful if $$\mathrm{Hom}_{\mathcal{C}}(X,Y) \hookrightarrow \mathrm{Hom}_{\mathcal{C}'} (\mathcal{F}(X), \mathcal{F}(Y)) \; \forall \; X,Y \in \mathrm{ob}(\mathcal{C})$$
We say that $\mathcal{F}$ is full if this map is an epimorphism and $\mathcal{F}$ is an embedding if $\mathcal{F}$ is fully faithful.

\begin{lemma}
Yoneda functor is an embedding.
\end{lemma}

\begin{proof}
By Yoneda lemma we have $$\mathrm{Hom}_{\mathbf{Presh}(\mathcal{C})}(h_{X}, h_{Y}) = h_{Y}(X)$$ But since $h_{Y}(X) = \mathrm{Hom}_{\mathcal{C}}(X,Y)$, the proof is complete.
\end{proof}

\subsection{Adjoint functors}

Suppose we have two functors $$\mathcal{F}: \mathcal{C} \rightarrow \mathcal{C}'  \text{ and } \mathcal{G}: \mathcal{C}' \rightarrow \mathcal{C}$$
The pair $(\mathcal{F}, \mathcal{G})$ is an adjoint pair if for all $X \in \mathrm{ob}(\mathcal{C})$ and $Y \in \mathrm{ob}(\mathcal{C}')$, there exists a natural transformation $$\mathrm{Hom}_{\mathcal{C}}(X, \mathcal{G}(Y)) \xlongrightarrow{\sim} \mathrm{Hom}_{\mathcal{C}'}(\mathcal{F}(X),Y)$$

\begin{example}
\begin{enumerate}
\item Take $\mathcal{C,D}$ to be the category $\mathbf{Mod}_{R}$ of $R$-modules. Define the functors 
\begin{align*}
F: \mathcal{C} &\rightarrow \mathcal{D} & & G: \mathcal{D} \rightarrow \mathcal{C} \\
F(A) &= A \otimes_{R} N & & G(B) = \Hom_{R}(N,B)
\end{align*}
Consider $\Hom_{R}(A \otimes_{R} N , B)$ and $\Hom_{R}(A, \Hom_{R}(N,B))$. These are both in bijective correspondence, in fact they are isomorphic as $R$-modules. Hence, $(F,G)$ is an adjoint pair. This is also called the Hom-tensor adjunction.
\item $(F,G)$ with $F$ the free functor and $G$ the forgetful functor is also an adjoint pair. 
\end{enumerate}
\end{example}

\begin{proposition}
Left adjoint and right adjoint have to be unique (if they exist).
\end{proposition}

Suppose we have an adjoint pair $(\mathcal{F}, \mathcal{G})$. Then, for every $X \in \mathrm{ob}(\mathcal{C})$ we have (follows from adjoint-ness) $$\mathrm{Hom}_{\mathcal{C}}(X, \mathcal{G}\circ \mathcal{F}(Y)) \xlongrightarrow{\sim} \mathrm{Hom}_{\mathcal{C}'}(\mathcal{F}(X),\mathcal{F}(Y))$$
This implies there is a canonical map $$u_{X}: X \rightarrow \mathcal{G} \circ \mathcal{F}(X)$$ and this in turn implies the existence of a natural transformation $$u: \mathrm{id}_{\mathcal{C}} \rightarrow \mathcal{G} \circ \mathcal{F}$$ called the unit of adjunction. \\
Similarly, for all $Y \in \mathrm{ob}(\mathcal{C}')$ we have 
$$\mathrm{Hom}_{\mathcal{C}}(\mathcal{F}\circ \mathcal{G}(X), Y) \xlongrightarrow{\sim} \mathrm{Hom}_{\mathcal{C}'}(\mathcal{G}(Y),\mathcal{G}(Y))$$
This implies the existence of a natural transformation $$\epsilon : \mathrm{id}_{\mathcal{C}'} \rightarrow \mathcal{G} \circ \mathcal{F}$$ called the co-unit of adjunction.


\begin{definition}
It is a category $\mathcal{C}$ such that 
\begin{enumerate}
\item it admits finite coproduct.
\item it has a zero product (both final and initial object).
\item $\mathrm{Hom}_{\mathcal{C}}(X,Y) \in \mathbf{Ab}$ such that $$\mathrm{Hom}_{\mathcal{C}}(X,Y) \times \mathrm{Hom}_{\mathcal{C}}(Y,Z) \rightarrow \mathrm{Hom}_{\mathcal{C}}(X,Z)$$ is bilinear.
\end{enumerate}
\end{definition}

\begin{definition}
It is an additive category such that every map $f: X \rightarrow Y$ has a kernel and a cokernel.
\end{definition}


\chapter{Lecture-5 (23rd January, 2023): Etale morphisms}

Let $A$ be a commutative ring and $M$ be a $A$-module. 

\begin{definition}
$M$ is flat if $$N \hookrightarrow N' \Rightarrow N \times M \hookrightarrow N' \times M$$
\end{definition}

\begin{definition}
$M$ is faithfully-flat if $M$ is flat and  $$N =0 \Leftrightarrow N \times_{A} M =0$$
\end{definition}

\begin{definition}
$M$ is projective if it is a direct summand of a free $A$-module. Or equivalently,

\begin{align*}
N \twoheadrightarrow N' &\Rightarrow \Hom(M, N) \twoheadrightarrow \Hom(M, N')\\
\Leftrightarrow \mathrm{Ext}_{A}^{i} (M,N) &= 0 \; \forall i >0 \; \forall \; N
\end{align*}
\end{definition}

\begin{lemma}
Suppose $A$ is Noetherian and $M$ is a finitely generated $A$-module. TFAE: 
\begin{enumerate}
\item $M$ is projective.
\item $M$ is flat.
\item $M_{\m}$ is flat for all maximal ideals $\m$.
\item $M_{\m}$ is free for all maximal ideals $\m$.
\end{enumerate}
\end{lemma}

\begin{proof}
$1 \Rightarrow 2$ is obvious. $2\Rightarrow 3$ is a local property. $3 \Rightarrow 4$ is done in commutative algebra. $4 \Rightarrow 1$: Note that $4 \Rightarrow 3 \Rightarrow 2$. So we just prove that $2 \Rightarrow 1$. Thus, enough to show that 
\begin{align*}
\mathrm{Ext}_{A}^i (M,N) &=0 & \forall \; N  \forall \; i>0 \\
\Leftrightarrow \mathrm{Ext}_{A}^i(M,N)_{\m} &= 0 &\forall \m \text{ maximal ideals } \\
\Leftrightarrow \mathrm{Ext}_{A_{\m}}^i (M_{\m}, N_{\m}) &= 0
\end{align*}
This completes the proof.
\end{proof}

Let $k$ be a field and $A$ a $k$-algebra.

\begin{definition}
$A$ is called separable over $k$ if $A \otimes_{k} k'$ is reduced for all field extensions $k' /k$.
\end{definition}

\begin{lemma}
$A$ is separable over $k$ iff every finitely generated subalgebra is separable over $k$.
\end{lemma}

\begin{proposition}
Assume that $A$ is finite dimensional over $k$. TFAE:
\begin{enumerate}
\item $A$ is separable over $k$.
\item $\overline{A}: = A \otimes_{k} \overline{k} = \prod_{i=1}^n \bar{k}$.
\item $A = \prod_{i=1}^n k_{i}$ where $k_{i}/k$ is a finite separable field extension.
\item The trace form $A \times A \rightarrow k ((w,w') \mapsto \mathrm{Tr}_{A/k}(ww'))$ is non-degenerate.
\end{enumerate}
\end{proposition}

\begin{proof}
($1 \Rightarrow 2$)
$$ \bar{A} = \prod_{i=1}^{n} A_{\m_{i}} = \prod_{i=1}^n \bar{k}$$
($2 \Rightarrow 3$) 
$$\frac{A}{\mathrm{Nil}(A)} = \prod_{i=1}^{n} k_{i}$$ where $k_{i}$ is finite field extension of $k$. \\
$\bar{A}$ is reduced $\Rightarrow \nil(A)=0 \Rightarrow A \simeq \ds{\prod_{i=1}^n k_{i}}$. \\
Now, say that $A$ is a finite field extension of $k$. Say $A = k'$. We have the following inclusions \[\begin{tikzcd}
	k & {k''} & {k'} && {}
	\arrow[hook, from=1-1, to=1-2]
	\arrow[hook, from=1-2, to=1-3]
\end{tikzcd}\]
where $k'$ is the maximal purely inseparable extension of $k$ inside $k'$. Need to show that $k'' = k$. \\
Can make $$k '' = \frac{k[t]}{t^{p^n} - \alpha} , \alpha \in k$$
Therefore $$k ' \otimes_{k} \bar{k} = \frac{\bar{k}[t]}{t^{p^n} - \beta^{p^{n}}} = \frac{\bar{k}[t]}{(t-\beta)^{p^n}}$$ for some $\beta\in \bar{k}$. Since the last quotient is not reduced therefore $k'' =k$.\\

($3 \Rightarrow 4$) done in comm. alg. \\

($4 \Rightarrow 1$)
\begin{align*}
\varphi : A\times A &\rightarrow k \\
(w,w') &\mapsto \mathrm{Tr}_{A/k}(ww')
\end{align*}
is non-degenerate. \\
Let $\{w_{1}, \hdots , w_{n}\}$ be a $k$-basis of $A$. \\
Consider $B = (\mathrm{Tr}(w_{i}w_{j}))$ and $\mathrm{disc}_{k}(A) = \det(B) \neq 0 \Rightarrow \mathrm{disc}_{\bar{k}}(\bar{A}) \neq 0$. \\
Suppose that $\nil(\bar{A}) \neq 0$. Suppose $\{w_{1}, \hdots , w_{m}\}$ be a $\bar{k}$-basis of $\bar{A}$. Extend this to a basis $\{w_{1}, \hdots , w_{m}, w_{m+1}, \hdots , w_{n}\}$ of $A$ such that $w_{i}w_{j}$ is nilpotent for all $i,j$. This implies $\det(\mathrm{Tr}(w_{i}w_{j})) = 0$ which is a contradiction. Therefore $\nil(\bar{A})=0$.
\end{proof}

\section{Kahler Differentials}

Some reference materials. 
\begin{itemize}
\item \href{http://math.uchicago.edu/~amathew/chsmoothness.pdf}{advanced1} \href{http://therisingsea.org/notes/Section2.8-Differentials.pdf}{advanced2} \href{https://anagrams-seminar.github.io/hdr/kahler.pdf}{advanced3}
\item \href{http://people.math.sfu.ca/~kyeats/teaching/math800/MATH%20800%20-%20Final%20Project.pdf}{basic}
\item Matsumura book on Commutative algebra and Commutative ring theory
\item Chapter 4 of T.A. Springer's Linear Algebraic Groups.
\end{itemize}

Let $A$ be a commutative ring and $M$ an $A$-module. A derivative $D : A \rightarrow M$ is an abelian group homomorphism such that $$D(ab) = aD(b) + D(a)b$$
Let $$\mathrm{Der}(A,M) = \text{ the set of derivations from } A \text{ to } M$$

If $A$ is a $k$-algebra where $k$ is a commutative ring, then we say that $D$ is a $k$-derivation if $D(k)=0$ \\

More notation: Let $\mathrm{Der}_{k}(A,M)$ be the set of all $k$-derivations and $\mathrm{Der}_{k}(A) = \mathrm{Der}_{k}(A,A)$\\

We can make $\mathrm{Der}(A,M)$ is an $A$-module so that $$(a\cdot D)(b) = aD(b)$$

Suppose $$D: A \rightarrow M$$ then 
\begin{align*}
D(\ZZ) &= 0 \\
\mathrm{Der}(A,M) &= \mathrm{Der}_{\ZZ}(A,M)
\end{align*}

Take $D,D' \in \mathrm{Der}_{k}(A)$, then we can define the bracket $[-,-]$ as $$[D,D'] = DD' - D'D$$ This converts $\mathrm{Der}_{k}(A)$ into a Lie algebra 

\begin{remark}
\begin{enumerate}
\item $d(a^n) = na^{n-1}d(a)$
\item $d^{n}(ab) = \sum_{i=0}^{n} \binom{n}{i} d^{i} a d^{n-i} b$
\end{enumerate}
In particular, if $\mathrm{char}(A)=p > 0$ then 
\begin{enumerate}
\item $d^{p}(ab) = ad^{p}b + bd^{p}(a) \Rightarrow d^p$ is a $k$-derivation.
\item $d^{p}(a+b) = d^p a + d^p b$ 
\end{enumerate}
\end{remark}

Clearly, $\mathrm{Der}_{k}(A, -): A-\mathrm{mod} \rightarrow A-\mathrm{mod}$ is a covariant functor.

\begin{proposition}
$\mathrm{Der}_{k}(A,-)$ is a representable functor.
\end{proposition}

\begin{proof}

\end{proof}

\chapter{Lecture-6 (25th January, 2023):Kahler Differentials}

\section{Differentials and Derivations}

\begin{theorem}
There exists an unique $A$-module (upto isomorphism) $\Omega_{A/k}'$ with a $k$-derivation $d_{A/k}: A \rightarrow \Omega_{A/k}'$ such that for all $A$-modules $M$ and a $k$-derivation $D: A \rightarrow M$, $\exists ! A$-linear map $\varphi : \Omega_{A/k}' \rightarrow M$ such that $D = \varphi \circ d_{A/k}$ $$\mathrm{Der}_{k}(A,M) \xrightarrow{\sim} \Hom_{A}(\Omega_{A/k}' , M)$$
\end{theorem}

\begin{proof}
Take $\Omega_{A/k}^1$ to be the free $A$-module generated by symbols $\{da: a\in A\}$ modulo the relations $d(a+b) - d(a) - d(b) =0$ and $d(ab) = ad(b) + bd(a) \; \forall \; a,b \in A$.
\end{proof}

\begin{definition}
A square zero extension of $k$-algebras is a surjection of $k$-algebras $g: B \twoheadrightarrow C$ such that $M^2 = 0$ where $M = \ker(g)$. \\
We can think of $B$ as the manifold $C$ plus some other tangent directions. $B$ is some kind of thickening of $C$ in the spec level.
\end{definition}

\begin{example}
Suppose $M \in \mathbf{Mod}_{A}$ and $B = A \oplus M$. Addition and multiplication are defined as follows: 
\begin{align*}
(a,m) + (a',m') &= (a+a', m + m')\\
(a,m) \cdot (a',m') &= (aa' , am' + a'm)
\end{align*}

Here, 
\[\begin{tikzcd}
	0 & M & B & A & 0 \\
	&& {(a,m)} & a
	\arrow[from=1-1, to=1-2]
	\arrow[from=1-2, to=1-3]
	\arrow["\varphi", from=1-3, to=1-4]
	\arrow[from=1-4, to=1-5]
	\arrow[maps to, from=2-3, to=2-4]
\end{tikzcd}\]
is a square zero extension. We wish to ask when does the following lift exist. 
 \[\begin{tikzcd}
	0 & M & B & C & 0 \\
	&& A
	\arrow[from=1-1, to=1-2]
	\arrow[from=1-2, to=1-3]
	\arrow["\varphi", from=1-3, to=1-4]
	\arrow[from=1-4, to=1-5]
	\arrow["g", from=2-3, to=1-4]
	\arrow["{?}", dashed, from=2-3, to=1-3]
\end{tikzcd}\]
Suppose we are given a lift $h: A \rightarrow B$ of $g$ and $h'$ is another lift of $g$. Then, 
\[\begin{tikzcd}
	{D:= h-h' :A} & B \\
	& M
	\arrow[from=1-1, to=1-2]
	\arrow[from=1-1, to=2-2]
	\arrow[hook, from=2-2, to=1-2]
\end{tikzcd}\]
$M$ is a $C$-module ($M = M/M^2 = M \otimes B/M = C$ is a $C$-module). So, $M$ is also an $A$-module via the map $g$. \\

\textbf{Claim}: $D \in \mathrm{Der}_{k}(A,M)$ \\
\begin{proof}

\end{proof}
Conversely, if $D \in \mathrm{Der}_{k}(A,M)$, then $h' = h + D$ is also a lift. 
\end{example}

\begin{proof}[Proof of main theorem]
\begin{enumerate}
\item Consider the map $A\otimes_{k} A \xrightarrow{\mu} A$ such that $a \otimes b \mapsto ab$. $\mu$ is a surjective $k$-algebra homomorphism. Let $I = \ker(\mu)$ and $B = A \otimes_{k} A/I^2$. We obtain the following square zero extension
\[\begin{tikzcd}
	0 & {I/I^2} & B & A & 0
	\arrow[from=1-1, to=1-2]
	\arrow[from=1-2, to=1-3]
	\arrow["\varphi", from=1-3, to=1-4]
	\arrow[from=1-4, to=1-5]
\end{tikzcd}\]
Let $\Omega_{A/k}' := I/I^2$ is the module of Kahler differentials. \\
$\Omega_{A/k}$ the canonical sheaf of diagonal embedding of $X \hookrightarrow X \times X$\\
Define 
\begin{align*}
\alpha_{1}: A \rightarrow B & \alpha_{1}(a) = a\otimes 1 \pmod{I^2} \\
\alpha_{2}: A \rightarrow B & \alpha_{2}(a) = 1\otimes a \pmod{I^2}
\end{align*}
We obtain the following diagram that commutes
\[\begin{tikzcd}
	0 & {\Omega_{A/k}'} & B & A & 0 \\
	&& A
	\arrow[from=1-1, to=1-2]
	\arrow[from=1-2, to=1-3]
	\arrow["\varphi", from=1-3, to=1-4]
	\arrow[from=1-4, to=1-5]
	\arrow["{\alpha_{i}}", from=2-3, to=1-3]
	\arrow["{\mathrm{id}}"', from=2-3, to=1-4]
\end{tikzcd}\]
Next, define $d_{A/k} = \alpha_{1} - \alpha_{2}$.\\
Let $M \in \mathbf{Mod}_{A}$ and $D: A \rightarrow M$ a $k$-derivation. \\
Now, define
\begin{align*}
\theta: A \otimes_{k} A &\rightarrow A *M (= A \oplus M) \text{ a square zero extension}\\
a\otimes b &\mapsto (ab, aDb)
\end{align*}
\textbf{Claim}:  $\theta(I) \hookrightarrow M$
\begin{proof}
Suppose $\sum x_{i} \otimes y_{i} \in I \Rightarrow \sum x_{i}y_{i} = 0 \in A$. This implies $\theta(\sum x_{i}\otimes y_{i}) = (\sum x_{i}y_{i}=0 , \sum x_{i} Dy_{i}) \in M$. Therefore, $$\theta (I/I^2) \hookrightarrow M/M^2 = M$$
Thus $\theta$ descends to a map $$\tilde{\theta} : I/I^2 \rightarrow M$$ or $\tilde{\theta} : \Omega_{A/k}' \rightarrow M$
\end{proof}
\textbf{Claim}:  $\tilde{\theta}$ is unique such that $\tilde{\theta} \circ d_{A/k} = D : A \rightarrow M$
\begin{proof}
Suffices to show that $\Omega_{A/k}'$ is generated by $\langle da : a \in A \rangle$ as an $A$-module. 
\begin{align*}
a\otimes a' &= (a \otimes 1)(-a' \otimes 1 + 1 \otimes a') + aa' \otimes 1 & &\\
\alpha \in \Omega_{A/k}' &\Rightarrow \alpha = \sum x_{i} \otimes y_{i} &\text{ such that } \sum x_{i}y_{i} = 0 \\
&\Rightarrow \alpha = \sum x_{i}dy_{i}
\end{align*}
\end{proof}
\end{enumerate}
\end{proof}

\begin{corollary}
$$\Omega_{A/k}' = \frac{\text{ free module on } da}{\text{ additivity + Leibnitz rule}}$$
In particular, $\mathrm{Der}_{k}(A) = \Omega_{A/k}^{'*} = \Hom(\Omega_{A/k}',A) = T_{A/k}$ (tangent space)
\end{corollary}


\begin{definition}
We say that $A$ is formally smooth over $k$ if given any square zero extension \[\begin{tikzcd}
	0 & M & B & A & 0
	\arrow[from=1-1, to=1-2]
	\arrow[from=1-2, to=1-3]
	\arrow["\varphi", from=1-3, to=1-4]
	\arrow[from=1-4, to=1-5]
\end{tikzcd}\] and a diagram of $k$-algebras, there exists a lifting $\tilde{g}$ of $g$ 
\[\begin{tikzcd}
	k & C \\
	B & A
	\arrow[from=1-1, to=2-1]
	\arrow["f", from=1-1, to=1-2]
	\arrow["\varphi"', from=2-1, to=2-2]
	\arrow["g", from=1-2, to=2-2]
	\arrow["{\tilde{g}}", dashed, from=1-2, to=2-1]
\end{tikzcd}\]
\end{definition}

\begin{definition}
We say that $A$ is formally unramified over $k$ if $g$ has atmost one lift. \\
We say that $A$ is formally \'{e}tale over $k$ if $A$ is formally smooth and formally unramified.
\end{definition}

\begin{definition}
We say that $A$ is smooth (resp. unramified, \'{e}tale) if it is formally smooth (unramified, \'{e}tale) and finite type over $k$.
\end{definition}

\begin{exercise}
$A= k[X_{1}, \hdots , X_{n}], \Omega_{A/k}' = ?$ \\
\textbf{Claim}: There is a canonical isomorphism of $A$-modules $$\theta : \underbrace{AdX_{1} \oplus \cdots \oplus AdX_{n}}_{F} \xrightarrow{\sim} \Omega_{A/k}'$$ 
\begin{lemma}
Suppose $U \subseteq A$ is any set that generates $A$ as $k$-algebra. Then, $\Omega_{A/k}'$ is generated by $\{da: a\in A\}$ as $A$-module. 
\end{lemma}
\begin{proof}

\end{proof}
This lemma implies the map is surjective. \\
Next, define $D_{i}: A \rightarrow A$ such that $f \mapsto \ds{\frac{\partial d}{\partial x_{i}}(f)}$. \\
This gives an unique $A$-linear map $\psi_{i}: \Omega_{A/k}' \rightarrow A$. Define $\psi: \Omega_{A/k}' \rightarrow F$ such that $\psi = \sum_{i=1}^n \psi_{i}$. This implies $\psi \circ \theta = \mathrm{id}_{F}$. Hence, $\psi$ is injective. 
\end{exercise}


\chapter{Lecture-7 (30th January, 2023): Module of differentials}

\begin{lemma}
$A$ is formally unramified iff $\Omega_{A/k}^{1} = 0$.
\end{lemma}

\begin{proof}
Suppose that $\Omega_{A/k}^{1} =0$. Let 
\[\begin{tikzcd}
	0 & M & B & C & 0
	\arrow[from=1-1, to=1-2]
	\arrow[from=1-2, to=1-3]
	\arrow["\varphi", from=1-3, to=1-4]
	\arrow[from=1-4, to=1-5]
\end{tikzcd}\] be a square zero extension. We had seen that all liftings of $f: A \rightarrow C$ differ by $\mathrm{Der}_{k}(A,M) = \Hom_{A}(\Omega_{A/k}^{1}, M)$. This implies there is atmost one lifting of $f$ and this concludes what we want. \\

For the other direction, suppose $A$ is formally unramified over $k$. Recall $$\mu: A \otimes_{k} A \rightarrow A$$ $$I = \ker(\mu)$$ and 
\[\begin{tikzcd}
	0 & I/I^2 = \Omega_{A/k}^{1} & B = (A \otimes_{k} A)/I^2 & C & 0
	\arrow[from=1-1, to=1-2]
	\arrow[from=1-2, to=1-3]
	\arrow["\varphi", from=1-3, to=1-4]
	\arrow[from=1-4, to=1-5]
\end{tikzcd}\]
We had two liftings from $A$ to $B$, $\alpha_{1},\alpha_{2}$ namely $\alpha_{1}(a) = a\otimes 1, \alpha_{2}(a) = 1 \otimes a$ and $d_{A/k} = \alpha_{1} - \alpha_{2}$. Since $A$ is formally unramified, $d_{A/k} = 0$ which implies $\Omega_{A/k}^{1} = 0$.
\end{proof}

\begin{exercise}
Suppose $K/k$ be a finite separable extension. We claim that this extension is formally unramified. 
\end{exercise}

\begin{lemma}
If $k \xrightarrow{f} A $ is of finite type ($k$ is a ring), then $\Omega_{A/k}^{1}$ is a finitely generated $A$-module.
\end{lemma}

\begin{proof}

\end{proof}

\begin{example}
If $A$ is a commutative ring, $S$ a multiplicatively closed subset of $A$, $B = S^{-1}A$. Then, $A \rightarrow B$ is formally etale. \\

\textbf{Formally unramified:} Enough to show that $$d_{A/k}(fg^{-1}) = 0 \; \forall \; f \in A , g\in S$$ But this means 
\begin{align*}
d_{B/A}(fg^{-1}) &= fd_{A/B} (g^{-1}) + g^{-1}d_{B/A}(f) \\
&= fd_{A/B}(g^{-1}) \\
gd_{A/B}(g^{-1}) + g^{-1}d_{A/B}(g) &= 0 \\
\Rightarrow gd_{A/B}(g^{-1}) &= 0 \\
\Rightarrow d_{B/A}(g^{-1}) &= 0
\end{align*}
Next, \\
\textbf{Formally unramified:}
\[\begin{tikzcd}
	0 & M & C & {C'} & 0 \\
	\\
	& A & B
	\arrow["g", from=3-2, to=1-3]
	\arrow["{\tilde{f}}"', dashed, from=3-3, to=1-3]
	\arrow["\theta"', from=3-2, to=3-3]
	\arrow[from=3-2, to=3-3]
	\arrow[from=1-1, to=1-2]
	\arrow[from=1-2, to=1-3]
	\arrow["\varphi"', from=1-3, to=1-4]
	\arrow[from=1-4, to=1-5]
	\arrow["f"', from=3-3, to=1-4]
\end{tikzcd}\]
\begin{align*}
\tilde{f} \text{ exists } &\Leftrightarrow g(s) \in C^{\times} \;\forall \; g \in S \\
&\Leftrightarrow \varphi g(s) = f\theta(s) \in C'^{\times}
\end{align*}
Then use the lemma stated after this example.
\end{example}

\begin{lemma}
If \[\begin{tikzcd}
	0 & I & C & C' & 0
	\arrow[from=1-1, to=1-2]
	\arrow[from=1-2, to=1-3]
	\arrow["\varphi", from=1-3, to=1-4]
	\arrow[from=1-4, to=1-5]
\end{tikzcd}\] is an extension of rings such that $I$ is nilpotent. Then $a\in C^{\times} \Leftrightarrow \varphi(a) \in C'^{\times}$.
\end{lemma}

\begin{proof}

\end{proof}

\begin{theorem}[First fundamental theorem for module of differentials]
Let \[\begin{tikzcd}
	k & A & B
	\arrow["f"', from=1-1, to=1-2]
	\arrow["g"', from=1-2, to=1-3]
\end{tikzcd}\] be ring homomorphisms. Then, \[\begin{tikzcd}
	{\Omega_{A/k}^{1} \otimes_{A} B } & {\Omega_{B/k}^{1}} & {\Omega_{B/A}^{1}} & 0
	\arrow["\alpha", from=1-1, to=1-2]
	\arrow["\beta", from=1-2, to=1-3]
	\arrow[from=1-3, to=1-4]
\end{tikzcd}\] is exact. Moreover, it is split exact if $B$ is formally smooth over $A$. Here, $\alpha(ad_{A/k}a' \otimes b') = bad_{B/k}(a'), \beta(ad_{B/k}b) = ad_{B/A}(b)$.
\end{theorem}

\begin{proof}
We know that a sequence of $B$-modules \[\begin{tikzcd}
	{N'} & N & {N''}
	\arrow[from=1-1, to=1-2]
	\arrow[from=1-2, to=1-3]
\end{tikzcd}\] is exact iff \[\begin{tikzcd}
	{\mathrm{Hom}_{B}(N'',M)} & {\mathrm{Hom}_{B}(N,M)} & {\mathrm{Hom}_{B}(N',M)}
	\arrow[from=1-1, to=1-2]
	\arrow[from=1-2, to=1-3]
\end{tikzcd}\] is exact for all $B$-module $M$. \\

Thus, we just need to check that \[\begin{tikzcd}
	{\mathrm{Hom}_{B}(\Omega_{B/A}^{1},M)} & {\mathrm{Hom}_{B}(\Omega_{B/k}^{1},M)} & {\mathrm{Hom}_{B}(\Omega_{A/k}^{1} \otimes_{A} B,M)} = {\mathrm{Hom}_{A}(\Omega_{A/k}^{1},M)}
	\arrow[from=1-1, to=1-2]
	\arrow[from=1-2, to=1-3]
\end{tikzcd}\] is exact. But this is equivalent to checking \[\begin{tikzcd}
	{\mathrm{Der}_{A}(B,M)} & {\mathrm{Der}_{k}(B,M)} & {\mathrm{Der}_{k}(A,M)}
	\arrow["{\beta^*}", from=1-1, to=1-2]
	\arrow["{\alpha^*}", from=1-2, to=1-3]
\end{tikzcd}\] is exact. \\

Next, assume that $B$ is formally smooth over $A$. We need to show that $\alpha^*$ is surjective. Let $D\in \mathrm{Der}_{k}(A,M)$. We know that the diagram \[\begin{tikzcd}
	B && B \\
	\\
	A && {B*M = B \oplus M}
	\arrow["{\mathrm{id}}", from=1-1, to=1-3]
	\arrow["{p_{1}}"', from=3-3, to=1-3]
	\arrow["g", from=3-1, to=1-1]
	\arrow["\varphi"', from=3-1, to=3-3]
\end{tikzcd}\] commutes. But $B * M$ is a square zero extension. Thus, we get a map $B \rightarrow B *M$ such that diagram \[\begin{tikzcd}
	B && B \\
	\\
	A && {B*M = B \oplus M}
	\arrow["{\mathrm{id}}", from=1-1, to=1-3]
	\arrow["{p_{1}}"', from=3-3, to=1-3]
	\arrow["g", from=3-1, to=1-1]
	\arrow["\varphi"', from=3-1, to=3-3]
	\arrow["\theta", from=1-1, to=3-3]
\end{tikzcd}\]
commutes. Here, $\varphi(a) = (ga, Da)$. We write $\theta (b) = (b, D'b)$. \\

\textbf{Claim}: $D'$ is a $k$-derivation from $B$ to $M$. \\

It is clear that $D' \circ g = D$. This is equivalent to a $B$-linear map $\alpha ' : \Omega_{B/k}^{1} \rightarrow M$. Define 
\begin{align*}
D: A &\rightarrow \Omega_{A/k}^{1} \otimes_{A} B \\
D(a) &= d_{A/k}(a) \otimes 1
\end{align*}

Check that $D\in \mathrm{Der}_{k}(A, \Omega_{A/k}^{1} \otimes_{A} B)$. This implies the existence of an extension $D' : B \rightarrow \Omega_{A/k}^{1} \otimes_{A} B$ such that $D' \circ g = D$ iff a $B$-linear map $\alpha ' : \Omega_{B/k}^{1} \rightarrow \Omega_{A/k}^{1} \otimes_{A} B$ such that $\alpha' \circ g = \alpha$. \\

\textbf{Claim}: $\alpha' \circ \alpha = \mathrm{id}$\\

This concludes the proof.
\end{proof}

Suppose \[\begin{tikzcd}
	{k } & A & B
	\arrow["g", two heads, from=1-2, to=1-3]
	\arrow["f", from=1-1, to=1-2]
\end{tikzcd}\] 

From the previous theorem, we get \[\begin{tikzcd}
	{\Omega_{A/k}^{1} \otimes_{A} B } & {\Omega_{B/k}^{1}} & {\Omega_{B/A}^{1}}=0 & 0
	\arrow["\alpha", from=1-1, to=1-2]
	\arrow["\beta", from=1-2, to=1-3]
	\arrow[from=1-3, to=1-4]
\end{tikzcd}\] is exact. Or rather \[\begin{tikzcd}
	{\Omega_{A/k}^{1} \otimes_{A} B } & {\Omega_{B/k}^{1}} &  0
	\arrow["\alpha", from=1-1, to=1-2]
	\arrow[from=1-2, to=1-3]
\end{tikzcd}\] is exact. What is the kernel of this map? 

\begin{theorem}[Second fundamental theorem of module of differentials]
Let $I = \ker(A \twoheadrightarrow B)$. Then, there exists an exact sequence \[\begin{tikzcd}
	{I/I^2} & {\Omega_{A/k}^{1} \otimes_{A} B} & {\Omega_{B/k}^{1}} & 0
	\arrow[two heads, from=1-2, to=1-3]
	\arrow["\delta", from=1-1, to=1-2]
	\arrow[from=1-3, to=1-4]
\end{tikzcd}\] where $\delta(a) = d_{A/k} (a) \otimes 1$. Moreover, this sequence is split exact if $B$ is formally smooth over $k$.
\end{theorem}

\begin{example}
Let $B = k[X_{1},X_{2}, \hdots , X_{n}]/\langle f_{1}, \hdots , f_{n} \rangle$. Then, what is $\Omega_{B/k}^{1}$. \\

If $A = k[X_{1}, X_{2}, \hdots , X_{n}]$. Then, $$\Omega_{A/k}^{1} = Adx_{1} \oplus \cdots \oplus Adx_{n}$$ 
$X= \spec(A), Y = \spec(B = A/I)$
\end{example}


\chapter{Lecture-8 (1st February, 2023):Differentials}

contd..
\begin{theorem}[Second fundamental theorem of module of differentials]
Let $I = \ker(A \twoheadrightarrow B)$. Then, there exists an exact sequence \[\begin{tikzcd}
	{I/I^2} & {\Omega_{A/k}^{1} \otimes_{A} B} & {\Omega_{B/k}^{1}} & 0
	\arrow[two heads, from=1-2, to=1-3]
	\arrow["\delta", from=1-1, to=1-2]
	\arrow[from=1-3, to=1-4]
\end{tikzcd}\] where $\delta(a) = d_{A/k} (a) \otimes 1$. Moreover, this sequence is split exact if $B$ is formally smooth over $k$.
\end{theorem}

\begin{proof}
Suffices to show that for any $B$-module $M$, the sequence 
\[\begin{tikzcd}
	{\Hom_{B}(\Omega_{B/k}^{1},M)} & {\Hom_{B}(\Omega_{A/k}^{1} \otimes_{A} B,M)} & {\Hom_{B}(I/I^2,M)} & {(*)}
	\arrow["\delta^*", from=1-2, to=1-3]
	\arrow["\alpha^*", from=1-1, to=1-2]
\end{tikzcd}\] is exact. \\
\textbf{Well-definedness} of $\delta$: 
\begin{align*}
\delta(ab) &= d_{A/k}(ab) \otimes 1 \\
&=(ad_{A/k}(b) + bd_{A/k}a) \otimes 1 \\
&= ad_{A/k}b \otimes 1 + bd_{A/k} \otimes 1 \\
&= d_{A/k}b \otimes g(a) + d_{A/k} \otimes g(b) \text{ since }\\
&= 0+0 \\
&= 0
\end{align*}
Now, we have 
\[\begin{tikzcd}
	{\mathrm{Der}_{k}(B,M)} & {\Hom_{B}( \mathrm{Der}_{k}(A,M)} & {\Hom_{B}(I/I^2,M)} & {=(*)}
	\arrow["\delta^*", from=1-2, to=1-3]
	\arrow["\alpha^*", from=1-1, to=1-2]
\end{tikzcd}\]
Take $D\in \mathrm{Der}_{k}(A,M)$, then $D(a)=0 \; \forall \; a \in I$. Since, 
\[\begin{tikzcd}
	A & B \\
	M
	\arrow[two heads, from=1-1, to=1-2]
	\arrow[from=1-1, to=2-1]
\end{tikzcd}\]
Therefore there exists $D' \in \mathrm{Der}_{k}(B,M)$ such that $D' \circ g = D$. Thus, sequence is exact. \\

Now, assume that $B$ is formally smooth over $k$. Look at the exact sequence of $B$-modules which is a square $0$ extension of $B$. 
\[\begin{tikzcd}
	0 & {I/I^2} & {A/I^2} & B & 0 \\
	&&& B
	\arrow[from=1-1, to=1-2]
	\arrow[from=1-2, to=1-3]
	\arrow["{g'}", from=1-3, to=1-4]
	\arrow[from=1-4, to=1-5]
	\arrow["{\mathrm{id}}"', from=2-4, to=1-4]
	\arrow["{\exists \;h}", dashed, from=2-4, to=1-3]
\end{tikzcd}\]
This implies the existence of a $k$-algebra homomorphism $h: B \rightarrow A/I^2$ such that $g\circ h = \mathrm{id}_{B}$. Now, consider the map $$h\circ g' : A/I^2 \rightarrow A/I^2$$ that kills $I/I^2$ so that $g'(1-hg')=0$. \\
Let $D' = 1-hg': A/I^2 \rightarrow A/I^2$. Check that this is a $k$-derivation of $A/I^2$. \\
Let $\psi \in \Hom_{B}(I/I^2, M)$ and consider the maps 
\[\begin{tikzcd}
	{D:=A} & {A/I^2} & {A/I^2} & M
	\arrow[from=1-1, to=1-2]
	\arrow["{D'}", from=1-2, to=1-3]
	\arrow["\psi", from=1-3, to=1-4]
\end{tikzcd}\]
Since $D'$ is a derivation, check that $D\in \mathrm{Der}_{k}(A,M)$. This means we get an $A$-linear map $\varphi : \Omega_{A/k}^{1} \rightarrow M$ which is equivalent to getting a map $\varphi : \Omega_{A/k}^1\otimes_{A} B \rightarrow M$ such that $\delta*(\phi)=\psi$. \\
Finally, take $M = I/I^2$ and $\psi =\mathrm{id}$. \\
$\Rightarrow \; \exists \;$ a map from $\varphi : \Omega_{A/k}^1 \rightarrow I/I^2 \Leftrightarrow \varphi : \Omega_{A/k}^1\otimes_{A} B \rightarrow I/I^2$ and $\varphi \circ \delta = \id_{I/I^2}$. Finally, 
\[\begin{tikzcd}
	0 & {I/I^2} & {\Omega_{A/k}^{1} \otimes_{A} B} & {\Omega_{B/k}^{1}} & 0
	\arrow[from=1-1, to=1-2]
	\arrow["\delta", from=1-2, to=1-3]
	\arrow[from=1-3, to=1-4]
	\arrow["\varphi", shift left=1, curve={height=-12pt}, from=1-3, to=1-2]
	\arrow[from=1-4, to=1-5]
\end{tikzcd}\]
or the sequence splits. This concludes the proof.
\end{proof}

\begin{corollary}
Let $B = k[X_{1}, \hdots , X_{n}]/\langle f_{1}, \hdots , f_{r} \rangle$. Thus, $$\Omega_{B/k}^1 = \frac{Bd\bar{x}_{1} + \cdots + Bd\bar{x}_{n}}{\langle df_{1} , \cdots , df_{r} \rangle }$$ with $df_{i} = \ds{\sum_{j=1}^n \frac{\partial f_{i}}{\partial x_{j}}}$ modulo $I^2$.
\end{corollary}

\begin{corollary}
Suppose $k \xlongrightarrow{f} A \xtwoheadrightarrow{g} B$ are $k$-algebra homomorphisms such that $A,B$ are formally smooth over $k$. Then,
\[\begin{tikzcd} 
	0 & {I/I^2} & {\Omega_{A/k}^{1} \otimes_{A} B} & {\Omega_{B/k}^{1}} & 0
	\arrow[from=1-1, to=1-2]
	\arrow["\delta", from=1-2, to=1-3]
	\arrow[from=1-3, to=1-4]
	\arrow[from=1-4, to=1-5]
\end{tikzcd}\]
\end{corollary}

\begin{definition}
\begin{enumerate}
\item Let $k \xlongrightarrow{f}A$ be a ring homomorphism. We say that $A$ is unramified over $k$ if it is formally unramified and of finite type.
\item Let $q\in \spec(A)$. Then, we say that $A$ is unramified over $k$ at $q$ if there exists $g\in A\bs k$ such that the map $f: K \rightarrow A_{g}$ is unramified.
\item We say that $A$ is locally unramified over $k$ if it is unramified at every $q\in \spec(A)$.
\end{enumerate}
\end{definition}

The question is whether the $1,3$ conditions are equivalent. $1\Rightarrow 3$ is known. So we need to check if $3 \Rightarrow 1$. Suppose that for all $q \in \spec(A)$ there exists $g \not \in q$ such that $k \rightarrow A_{g}$ is unramified. Let $U_{q} = \spec(A_{g})$. \\
Since $X$ is locally unramified, we must have $X = \ds{\bigcup_{q} U_{q}}$ but remember that $X$ is spectral and thus quasi-compact. This implies that there is a finite subcover $X = \ds{\bigcup_{i=1}^n U_{q_{i}}} \Rightarrow A = \langle g_{1} , \hdots , g_{n} \rangle \Rightarrow A$ is of finite type over $k$. \\

Why is $\Omega_{A/k}^1 = 0$?\\

Notice that $\Omega_{A_{g}/k}^{1} =0$ and we have an exact sequence
\[\begin{tikzcd}
	{\Omega_{A/k}^{1} \otimes_{A} A_{g}} & {\Omega_{A_{g}/k}^{1}} & {\Omega_{A_{g}/A}^{1}} & 0
	\arrow[from=1-1, to=1-2]
	\arrow[from=1-2, to=1-3]
	\arrow[from=1-3, to=1-4]
\end{tikzcd}\]
But $\Omega_{A_{g}/A}^{1}$ is formally \'{e}tale and is therefore $0$. This transforms the above sequence to 
\[\begin{tikzcd}
	0 & {\Omega_{A/k}^{1} \otimes_{A} A_{g}} & {\Omega_{A_{g}/k}^{1}} & 0 & {}
	\arrow[from=1-2, to=1-3]
	\arrow[from=1-3, to=1-4]
	\arrow[from=1-1, to=1-2]
\end{tikzcd}\]
Hence, $(\Omega_{A/k}^1)_{g} = \Omega_{A_{g}/k}^{1} \; \forall \; g \Rightarrow \Omega_{A/k}^{1} =0$. This finishes the proof.

\begin{proposition}
\begin{enumerate}
\item Unramified maps are preserved under base change.
\item Unramified maps are preserved under composition.
\item Principal localisations are unramified.
\item Any surjection is unramified.
\end{enumerate}
\end{proposition}

\begin{proof}
Let us prove $3$. Look at 
\[\begin{tikzcd}
	k & A &&&& {} \\
	B & {A\otimes_{k}B=C}
	\arrow["f", from=1-1, to=1-2]
	\arrow["g"', from=1-1, to=2-1]
	\arrow["{f'}"', from=2-1, to=2-2]
	\arrow["{g'}", from=1-2, to=2-2]
\end{tikzcd}\]
More generally, we have 
\begin{lemma}
$\Omega_{C/B}^1 \simeq \Omega_{A/k}^1 \otimes_{k}B$
\end{lemma}

\begin{proof}
Look at the maps as a consequence of first fundamental exact sequence
\[\begin{tikzcd}
	{\Omega_{A/k}^1} & {\Omega_{C/k}^1} & {\Omega_{C/B}^1} &&& {}
	\arrow[from=1-1, to=1-2]
	\arrow[two heads, from=1-2, to=1-3]
\end{tikzcd}\]
This gives the map $\Omega_{A/k}^1 \otimes_{k} B \xlongrightarrow{\alpha} \Omega_{C/B}^1$. Now, we wish to construct an inverse map $\Omega_{C/B} \rightarrow \Omega_{A/k}^1 \otimes_{k} B$. Look at the map $d_{A/k} : A \rightarrow \Omega_{A/k}^1$. This gives a map 
\begin{align*}
d' : C= A\otimes_{k} B &\rightarrow \Omega_{A/k}^1 \otimes_{k} B \\
a\otimes b &\mapsto d_{A/k} a \otimes b
\end{align*}
Check that $d'$ is a $B$-derivation. This implies the existence of a map $\beta : \Omega_{C/B}^1 \rightarrow \Omega_{A/k}^1 \otimes_{k} B$. Check that $\alpha \circ \beta = \beta \circ \alpha = \id$. The proof is complete.
\end{proof}
\end{proof}

\begin{lemma}
Let $f \xrightarrow{f} A$ be a finite type morphism. Let $\pr \in \spec(A)$. Then, $A$ is unramified over $k$ at $\pr$ if $\Omega_{A/k}^1 \otimes_{A} k(\pr) = 0$ where $k(\pr) = A_{\pr} / \m_{\pr}$.
\end{lemma}

\begin{proof}
It suffices to show that $(\Omega_{A/k}^1)_{\pr} =0$ since $A$ is of finite type. We can deploy Nakayama lemma to conclude what we want. We just need to check that localisation of finitely generated implies finitely generated.
\end{proof}


\chapter{Lecture-9 (6th February, 2023): Differentials}

\section{Differentials contd...}
Recall that given a ring homomorphism $k \xlongrightarrow{A}$, $A$ is unramified iff $A$ is of finite type and $\Omega_{A/k}^{1} =0$. \\

Also, for a finite type algebra $A$, if $\pr \in \spec(A)$ such that $\Omega_{A/k}^{1} \otimes_{A} k (\pr) = 0$ then $A$ is unramified at $\pr$. \\

In particular, if $\Omega_{A/k}^{1} \otimes_{A} k(\pr) = 0 \; \forall \pr \in \spec(A)$, then $A$ is unramified.

\begin{lemma}
\begin{enumerate}
\item Say $A$ is a commutative ring and $I \subseteq A$ is a finitely generated ideal such that $I=I^2$. Then there exists an idempotent $e \in A$ such that $I= \langle e \rangle$.
\item $A/I = A_{e'}$ where $e'(1-e)=0$.
\item $\V(I)$ is open in $\spec(A)$.
\end{enumerate}
\end{lemma}

\begin{proof}
Since $I=I^2$ and $I$ is f.g. then Nakayama lemma implies the existence of an $a\in I$ such that $(1+a)I=0$. Set $f = 1+a$. Then $f^2 = ff = f(1+a)=f + af =f$. \\

Next, take $e' = f$ and $e = 1-f$. Then $\forall \; b \in I$ we have $b=(1-f)b=eb \Rightarrow I =\langle e \rangle$. \\

In particular, the map $$A \rightarrow \frac{A}{\langle e \rangle} \times \frac{A}{\langle e' \rangle}$$ is an isomorphism. This helps us conclude that $A/ \langle e \rangle = A_{e'}$. \\

And, $V(I) = \spec(A/I) = \spec(A_{e'}) \hookrightarrow \spec(A)$ is open. 
\end{proof}

\begin{corollary}
If $k \xlongrightarrow{f} A$ is unramified, then the diagonal map $\Delta_{X}: X \hookrightarrow X \times_{Y} X$ is a closed and open immersion where $X =\spec(A)$ and $Y = \spec(k)$.
\end{corollary}

\begin{proof}
Recall that by definition $\Omega_{A/k}^{1} = I/I^2$ where $I$ is the ideal of $X$ inside $X \times_{Y} X$. The module of differential is of f.g. since $A$ is of finite type.\\

\textcolor{BrickRed}{Red Flag}: $I$ need not be f.g. as required in our previous lemma. So, WE HAVE TO change the definition of unramified by replacing finitely presented instead of finite type. So, work with this definition. \\

Working with the new definition, we know that $\Omega_{A/k}^{1} = 0 \Rightarrow I=I^2$ and hence from previous lemma, we know that this immersion is open.  
\end{proof}

\begin{lemma}
Let $(A, \m)$ be a local ring which is a $k$-algebra for some field $k$ such that $A/\m = k$. Then, $\m / \m^2 \simeq \Omega_{A/k}^{1} \otimes_{A} k $
\end{lemma}

\begin{proof}
We have an exact sequence 
\[\begin{tikzcd}
	&&&&& {} \\
	0 & {\m/\m^2} & {\Omega_{A/k}^{1} \otimes_{A} k } & {\Omega_{k/k}^{1}} & 0
	\arrow[from=2-1, to=2-2]
	\arrow[from=2-2, to=2-3]
	\arrow[from=2-3, to=2-4]
	\arrow[from=2-4, to=2-5]
\end{tikzcd}\]
Here, $k$ sits inside $A$ and acts on $A$ as well, hence the sequence splits and is formally smooth. Now, using a result from before, we know that $\Omega_{k/k}^{1} = 0$ and hence the proof is complete.
\end{proof}

\begin{lemma}
Suppose $A$ is as in prev. lemma such that $A$ is essentially of finite type (localisation of a finite type) over $k$. Then, $\Omega_{A/k}^{1}$ is a free $A$-module of rank$= \dim(A)$ iff $A$ is a regular ring.  
\end{lemma}

\begin{proof}
$( \Rightarrow )$ By previous lemma, $\m / \m^2$ is free $k$-module of rank $n = \dim(A)$. Now apply Nakayama to observe that $\m$ is generated by $n$ generators which is the minimal number required. Hence, $A$ is regular. \\
Caution: Here, $\m$ is f.g. requires Noetherian-ness which is guaranteed by the essentially finite-ness. \\

($\Leftarrow$)  Suppose $A$ is regular. Therefore $\m = \langle x_{1}, x_{2}, \hdots , x_{n}\rangle$ where $n = \dim A$. By previous lemma, $\dim_{k} (\Omega_{A/k}^{1} \otimes_{A} k ) = n$. \\

Let $K$ be the quotient field of $A$ (makes sense because a regular local ring is integral domain). This implies $\dim_{K}(\Omega_{K/k}^{1}) =\dim_{K}( \Omega_{A/k}^{1} \otimes_{A} K )= tr.\deg_{k}(K) = \dim(A) = n$. \\

\textcolor{BrickRed}{PLEASE STOP}: The third equality is because of Noether Normalisation theorem. The second equality is some other very non-trivial fact. Just take it for granted for the time being. We will justify it later. 

Detour: \\

\begin{lemma}
Let $A$ be a local integral domain with fraction field $K$ and residue field $k$. Let $M$ be a f.g. $A$-module such that $\dim_{k}(M \otimes_{A} k) = \dim_{K}(M \otimes_{A} K) = n < \infty$. Then, $M$ is a free module of rank $n$. 
\end{lemma}

\begin{proof}
By Nakayama, can find a surjection $\phi: F = A^n \twoheadrightarrow M$. Let $N  = \ker (\phi)$. This gives us the exact sequence
\[\begin{tikzcd}
	0 & N & F & M & 0
	\arrow[from=1-1, to=1-2]
	\arrow[from=1-2, to=1-3]
	\arrow["\phi", from=1-3, to=1-4]
	\arrow[from=1-4, to=1-5]
\end{tikzcd}\]
which implies \[\begin{tikzcd}
	0 & N_{K} & F_{K} & M_{K} & 0
	\arrow[from=1-1, to=1-2]
	\arrow[from=1-2, to=1-3]
	\arrow["\phi", from=1-3, to=1-4]
	\arrow[from=1-4, to=1-5]
\end{tikzcd}\] is exact. \\

We know that $F_{k} \rightarrow M_{K}$ is a surjection of vector spaces of same dimension ($n$). Hence, $N_{K}=0$. But, $N$ is torsion free therefore $N \hookrightarrow N_{K} =0$.
\end{proof}

The proof is now complete using this lemma.
\end{proof}

\begin{corollary}
If $\Omega_{A/k}^{1} = 0$, then $A=k$.
\end{corollary}

\begin{lemma}
Let $k \hookrightarrow L$ be an algebraic extension of fields which is separable. Then, $L/k$ is formally unramified.
\end{lemma}

\begin{proof}
Can assume that $L/k$ is finite, and 
\begin{lemma}
If $A = \varinjlim_{i \in I} A_{i} \Rightarrow \Omega_{A/k}^{1} = \varinjlim_{i \in I} \Omega_{A_{i}/k}^{1}$
\end{lemma}

\begin{proof}

\end{proof}
Can choose a primitive element $\alpha \in L$. Let $f(X)$ be the minimal polynomial of $\alpha \in k[X]$ such that $f(\alpha)=0$ and $f'(\alpha) \in L^{\times}$. Now, write $A=k[X]$ and $I = \langle f(X)\rangle \in A \Rightarrow L = k[X]/ \langle f(X) \rangle$. By the second fundamental exact sequence, we get that 
\[\begin{tikzcd}
	{I/I^2} & {\Omega_{A/k}^{1} \otimes_{A} L} & {\Omega_{L/k}^{1}} & 0
	\arrow["\delta", from=1-1, to=1-2]
	\arrow["\phi", from=1-2, to=1-3]
	\arrow[from=1-3, to=1-4]
\end{tikzcd}\]
But $\Omega_{A/k}^{1} \otimes_{A} L = L d\bar{X}$ \\

And, $\delta(f) = \ds{\frac{\partial }{\partial X}f|_{L} d\bar{X}= f'(\alpha) d\bar{X}} $. This implies $\delta$ is an isomorphism (takes basis to basis). $\Rightarrow \Omega_{L/k}^{1} = 0$

\end{proof}


\chapter{Lecture-10 (8th February, 2023): Unramified morphisms }

Recall the lemma
\begin{lemma}
If $k'/k$ is an algebraic separable field extension, then $k'/k$ is formally unramified.
\end{lemma}

\begin{proposition}
Let $k$ be a field and $A/k$ an unramified $k$-algebra. Then, $A$ is a finite product of finite separable field extensions of $k$.
\end{proposition}

\begin{lemma}
$A$ is finite type over $k$. 
\end{lemma}

\begin{remark}
We can replace $k$ by $\bar{k}$ since basis goes to basis.
\end{remark}

\begin{proof}
It suffices to show that $\dim(A)=0$ since it is Noetherian and $\dim (A)=0$ which implies Artinian. (all primes are maximal ideal and use some ideas about minimal primes stuff + CRT) to conclude. LOTs of details went over my head. Anyway, assume $A$ has one prime ideal and complete.\\

Now let us show $\dim (A)=0$. Suppose $\dim (A) > 0$, i.e., there is a maximal ideal $\m$ such that $\mathrm{ht}(\m) >0 \Leftrightarrow \dim(A_{\m})>0$. But, we are given $\Omega_{A_{\m}/k}^{1}=0$. But from corollary in last class, $A_{\m}=k \Rightarrow \Leftarrow$. (The conditions are satisfied due to Hilbert Nullstellansatz).
\end{proof}

\begin{proof}[Proof of proposition]
Using the lemma, just apply one of the $4$ equivalent conditions of separability and the corollary in previous class.
\end{proof}

\begin{lemma}
Let $f: k \rightarrow A$ be a ring homomorphism and let $\mathfrak{q}\in \spec(A)=X$ be a prime. Let $\pr = f^{-1}(\mathfrak{q})\in \spec(k)=Y$. Assume that $f$ is unramified at $\mathfrak{q}$. Then, 
\begin{enumerate}
\item $\pr A_{\mathfrak{q}} = \mathfrak{q}A_{\mathfrak{q}}$
\item $k(\mathfrak{q})$ is a finite separable extension of $k(\pr)$
\end{enumerate} 
\end{lemma}

\begin{proof}
Since $k \rightarrow A$ is unramified at $\mathfrak{q}$, it follows that there exists a $g \in A\bs \mathfrak{q}$ such that $k \rightarrow A_{g}$ is unramified. In particular, we can assume that $A/k$ is unramified. This means that $k(\pr) \rightarrow k(\pr) \otimes_{k} k(\pr)=:B$ is also unramified as it is just a base change. But this tensor product is still finite type and by previous proposition $B$ is a product of $k_{i}$ with $k_{i}/k$ finite separable field extension. One of these $k_{i}s$ must be $k(\mathfrak{q})$. Therefore, $(2)$ is proven. $(1)$ also follows from this. 
\end{proof}

\begin{proposition}
Let $f: k \rightarrow A$ be a finite type ring extension and let $\mathfrak{q} \in X = \spec(A)$ and $\pr = f^{-1}(\mathfrak{q}) \in \spec(k)$. Suppose 
\begin{enumerate}
\item $\pr A_{\mathfrak{q}} = \mathfrak{q}A_{\mathfrak{q}}$
\item $k(\mathfrak{q})$ is a finite separable extension of $k(\pr)$
\end{enumerate}
Then, $f$ is unramified at $\mathfrak{q}$.
\end{proposition}

\begin{proof}
We need to show that $exists g \in A\bs \mathfrak{q}$ such that $\Omega_{A_{g}/k}^{1}=0$. For this, it suffices to show that $\Omega_{A_{\mathfrak{q}}/k}^{1}=0$ since $\Omega_{A/k}^{1}$ is a f.g. $A$-module and use result from previous class (if stalk at a point is $0$, then it must be globally $0$).\\

Let us prove $B:=\Omega_{A_{\mathfrak{q}}/k}^{1} \otimes_{k} k(\pr)=0$. \\
We know that $\Omega_{B/k(\pr)}^{1} = \Omega_{A_{\mathfrak{q}}/k}^{1} \otimes_{k} k(\pr)$. Hence, we can assume that $B$ is a localisation if a finite type $k$-algebra where $k$ is a field. In this case, our hypothesis says that $B$ is a finite separable field extension of $k$. By the lemma we recorded in the beginning, we have $B$ is unramified. This completes the proof.
\end{proof}

Thus, we have a characterisation of a map being unramified at a point. Let us record it in the following theorem
\begin{theorem}
Let $f: k \rightarrow A$ be a finite type ring extension and let $\mathfrak{q} \in X = \spec(A)$ and $\pr = f^{-1}(\mathfrak{q}) \in \spec(k)$. Then, 
\begin{enumerate}
\item $\pr A_{\mathfrak{q}} = \mathfrak{q}A_{\mathfrak{q}}$
\item $k(\mathfrak{q})$ is a finite separable extension of $k(\pr)$
\end{enumerate}
if and only if $f$ is unramified at $\mathfrak{q}$.
\end{theorem}

\begin{definition}
Let $f: k \rightarrow A$ be a ring homomorphism. Say $\pr \in \spec(A)$, then we say $f$ is quasi-finite at $\pr$ if $\spec(A\otimes_{k} k(\pr))$ is finite. \\

We say $f$ is quasi finite if it is so at every point of $Y$.
\end{definition}

To understand it better, consider $k \xlongrightarrow{f} A$ and $X \xlongrightarrow{f} Y$. Then, for $y \in Y$ look at $f^{-1}(y)$. The ring $A \otimes_{k} k(\pr)$ is just $A/\pr A$. Therefore the $\spec(A \otimes_{k} k(\pr))$ is just the prime ideals containing $\pr$. Set theoretically, this is just the fiber of $y$ under $f$. 

\begin{corollary}
Every finite morphism is quasi-finite. 
\end{corollary}

\begin{proof}

\end{proof}

\begin{corollary}
An unramified morphism is quasi-finite.
\end{corollary}

\begin{proof}
To check this at point $\pr$, we can replace $k$ by $k(\pr)$ and do base change. Now, the result follows from the proposition proved today. 
\end{proof}

\textcolor{Magenta}{PLEASE READ Artinian Rings,  Integral extensions, minimal primes business}


\chapter{Lecture-11 (13th February, 2023): Smoothness}

\section{Dimension Theory}

For an affine scheme $X$ and a point $x \in X$, we let $$\dim_{x}(X) := K\dim(\mathcal{O}_{X,x})$$ where $\mathcal{O}_{X,x}$ is the localisation of $A$ at $\pr$ if $X = \spec(A)$ and $x = \{\pr \}$ \\

From now on we will assume that our rings to be Noetherian. \\

Recall that if $A$ is a Noetherian ring, then $$K\dim = \sup\{ n \mid \exists \text{ a chain of prime ideals } \pr_{0} \subsetneq \pr_{0} \subsetneq \pr_{1} \subsetneq \cdots \subsetneq \pr_{n} \text{ of length } n \}$$

Also, $\dim (X)  :=  \dim (A)$ if $X = \spec (A)$

If $\pr \in X$, then $$\mathrm{ht} (\pr) = K \dim (A/{\pr})$$

We also defined co-height of $\pr $ as $$ \mathrm{co-ht}(\pr)= K \dim (A_{\pr})$$ 

Note that $\mathrm{ht} + \mathrm{co-ht} \le \dim (A)$

\begin{theorem}
Let $A$ be an integral domain which is of finite type over a field. Then, 
\begin{enumerate}
\item $\dim (A) = tr.\deg(\mathrm{Frac}(A)/k)$
\item $\forall \; \pr \in \spec(A)$ we have $\Ht(\pr) + \coht(\pr) = \dim (A)$
\end{enumerate}
\end{theorem}

Proof can be seen in the commutative algebra part of the notes.

Now, for any ideal $I \subseteq A$, we define height of $I$ as 
\begin{equation*}
\Ht(I) = \inf \{ \Ht(\pr)\mid I \subseteq \pr \}
\end{equation*}

\begin{theorem}[Krull dimension theorem]
Suppose there exists $f_{1},\hdots , f_{r} \in \pr$ such that $\pr$ is a minimal prime if $I = \langle f_{1} , \hdots , f_{r}\rangle$. Then, 
\begin{equation}
\Ht(\pr) \le r 
\end{equation}
Moreover, if $f_{1}, \hdots , f_{r}$ is a regular sequence, then 
\begin{equation}
\Ht(I) = r
\end{equation}
\end{theorem}

\begin{corollary}
The height of an ideal is always finite.
\end{corollary}

\begin{proof}
Since the ring is Noetherian, therefore every ideal is f.g and hence by previous theorem $\Ht(I) < \infty$.
\end{proof}

\begin{corollary}
If $A$ is semi-local, then $\dim (A)< \infty$
\end{corollary}

\begin{proof}

\end{proof}

\begin{remark}[due to Nagata]
A Noetherian ring in general might not have finite dimension. For example, take $A = k[X_{1}, X_{2} , \hdots ]$. Choose a sequence of integers (positive) $m_{1} < m_{2} <\cdots $ such that $m_{i+1} - m_{i} > m_{i} - m_{i-1} \; \forall \; i$ \\

Now, define $\pr_{i} = \langle X_{m_{i}} , X_{m_{i} + 1} , \hdots , X_{m_{i+1}} \rangle$ \\

Take $S = \bigcup \pr_{i} \subseteq A$ and let $B = S^{-1} A$. \\

Clearly, $\Ht(\pr_{i}) = m_{i+1} - m_{i}$. This implies $\dim (B)=\infty$. \\

Remains to be shown that $B$ is Noetherian. (Noetherian local ring has finite Krull dimension, and some other result )
\end{remark}

\begin{proposition}
If $\Ht(\pr)=r$, then there exists $f_{1}, \hdots , f_{r} \in \pr$ such that $\pr$ is a minimal prime of $\langle f_{1}, \hdots , f_{r} \rangle$.
\end{proposition}

\begin{definition}
IF $A$ is local and $\m$ is the minimal prime of $\langle f_{1} , \hdots , f_{r} \rangle$ such that $\dim (A)=r$, then $\{f_{1}, \hdots , f_{r}\}$ is called a system of parameters of $A$.
\end{definition}

\begin{proposition}
$\dim (A[X_{1}, \hdots ,X_{n}]) = \dim (A) + n$
\end{proposition}

\begin{proof}
We can always assume that $n=1$. \\


\textbf{Case-1}: Suppose $\dim (A) = \infty $. This implies $\dim (A[x])< \infty $. \\

\textbf{Case-2}: Suppose $\dim (A) < \infty $. We prove by induction. \\

\textbf{Case-2a}: Suppose $\dim (A)=0$, then it is either a field of a product of fields when modded out by the Nilpotent radical (does not change dimension). This implies $A[X]/ \nil(A)$ is a finite product of PIDs and hence we are done.\\

\textbf{Case-2b}: Assume $\dim (A) > 0$. Let $\pr$ be a prime ideal of $A$ of height $r$. Let $	q = \langle \pr[X], X \rangle $. Then, $q \in \spec(A[X])$. Moreover $\Ht(q) \geq \Ht(\pr) + 1$. This implies $\dim(A[X]) \geq \dim (A) + 1$. \\

For the other direction, let $\dim (A)=n$ and $\m$ be a maximal ideal of $A[X]$. Enough to show that $\Ht(\m) \le n+ 1$. \\
Take $\pr = \m \cap A$. Recall that $\Ht(\m) = \Ht(A_{\pr}[X]_{\m})$. Hence, can assume that $(A, \pr) $ is local. Now, 



This implies there exists $f\in \m$ such that $\m = \langle f \rangle \Rightarrow \langle \pr[X],f \rangle = \m$. Since, $\dim (A) = n$, we get that $\Ht(\pr) \le n$. Now, apply the previous proposition to observe that $\pr$ is a minimal prime of $\langle f_{1}, \hdots , f_{r} \rangle$ and thus $\m$ is minimal prime of $\langle f_{1}, \hdots , f_{r}, f\rangle$ (needs to be checked). Finally, $\Ht(\m) \le n+1$.
\end{proof}

\section{Geometric intuition of flatness}

Suppose $\spec(B) = X \xrightarrow{f} Y = \spec(A)$ and $f$ is flat. \\

\textcolor{Blue}{Blabber (useful but blabber nonetheless)}

\begin{theorem}
Let $f: Y \rightarrow X$ be a flat morphism of affine schemes. Let $y \in Y$ and $x = f(y)$. Then,   
\begin{equation}
\dim_{x}(Y_{x}) = \dim_{y}(Y) - \dim_{x}(X)
\end{equation}

where $Y_{x} = \spec(B \otimes_{A} k(x))$ if $X=\spec(A)$ is the scheme theoretic fibre.
\end{theorem}

\begin{proof}
We can assume that $A$ is local since the fibre remains the same whether we work with $A$ or $A_{\pr}$. \\

If $\dim (A)=0$, then it is a field and hence $\dim_{x}(X)=0$ and $Y_{x} = \spec(B \otimes_{A} k(x)) = \spec(B)$ and hence we are done. \\

Assume $\dim (A) > 0$. \\

We can further assume that $A$ is reduced and in a reduced ring, every associated prime is also minimal prime. 
\end{proof}


\chapter{Lecture-12 (15th February): \'{E}tale morphisms}

\begin{definition}
Let $f: Y \rightarrow X$ be a morphism of Noetherian affine schemes. If $y \in Y$ and $x=f(y)$. Then, $f$ is said to be smooth at $y$ if the following holds: 
\begin{enumerate}
\item $f$ is finite type at $x$
\item $f$ is flat at $y$
\item $B_{\mathfrak{q}} \otimes_{A_{\pr}} \overline{k(x)}$ is regular
\end{enumerate}
$X = \spec(A), Y = \spec(B), x= \pr , y = \mathfrak{q}$\\

We say that $f$ is smooth if it is smooth at every point $y \in Y$
\end{definition}

\part{Topics in Analytic Number Theory}

\chapter{Lecture-1: Hardy-Littlewood proof of infinitely many zeros on the line $\mathfrak{R}(s) = 1/2$}

\chapter{Lecture-2: }

\chapter{Lecture-3 (10th January, 2023): Siegel's theorem }

\begin{theorem}[Siegel]
Let $\chi(q)$ be a real Dirichlet character modulo $q\geq 3$. Given any $\epsilon >0$, we have $$L(1, \chi) \geq \frac{C_{\epsilon}}{q^{\epsilon}}$$
\end{theorem}

A trivial lower bound: $L(1, \chi) \gg q^{-1/2}$

\begin{proof}[Goldfeld's proof]
Consider $$f(s) = \zeta(s)L(s,\chi_{1})L(s,\chi_{2})L(s,\chi_{1}\chi_{2})$$ with $\chi_{i}, i=1,2$ primitive quadratic characters. Notice that $f(s) = \sum_{n} b_{n}n^{-s}$ with $b_{1} =1 , b_{n} \geq 0$. Let $\lambda = \mathrm{Res}_{s=1}f(s) = L(1,\chi_{1})L(1,\chi_{2})L(1,\chi_{1}\chi_{2})$

\begin{lemma}
Given any $\epsilon >0$, one can find $\chi_{1}(q_{1})$ and $\beta$ with $1-\epsilon < \beta < 1$ such that $f(\beta) \le 0$, independent of what $\chi_{2}(q_{2})$ is. 
\end{lemma}

\begin{proof}
\textbf{Case-1:} If there are no real zeros of $L(s, \psi)$ for any primitive quadratic character in $(1-\epsilon,1)$, then $f(\beta) < 0$ for any $\beta \in (1 - \epsilon,1)$. This is because $$f(\beta) = \underbrace{\zeta(\beta)}_{<0} \underbrace{L(s,\chi_{1})L(s,\chi_{2})L(s,\chi_{1}\chi_{2})}_{>0}$$ as $L(1,\chi)>0$ and $L$ is continuous so any change of sign will lead to a zero which is a contradiction. \\
\textbf{Case-2:} If we cannot find such a $\psi$, then just set $\chi_{1}=\chi$ and let $\beta$ be the real zero. Then, $f(\beta)=0$. We are done. 
\end{proof}

Next, consider the integral 

\end{proof}

\begin{corollary}
\begin{align*}
h(-d) &= \frac{L(1,\chi_{d}) \sqrt{|d|} \;\omega}{2 \pi} \\
&= \frac{L(1,\chi_{d})}{\log \epsilon_{d}}
\end{align*}
\end{corollary}

\begin{theorem}[Y. Zhang]
$$L(1, \chi) \geq \frac{c}{(\log q)^{2022}}$$
\end{theorem}

\begin{theorem}
If $\chi(q)$ does not have a Siegel zero, then $L(1, \chi) \gg \frac{1}{\log q}$
\end{theorem}


\chapter{Lecture-4 (12th January, 2023): PNT for Dirichlet characters and APs}

\begin{lemma}
If $\rho = \beta + i \gamma$ runs through nontrivial zeros of $L(s, \chi)$, then $$\sum_{\rho} \frac{1}{1 + (T - \gamma)^2} = \mathcal{O} (\log q(|T| + 2)) \forall T \in \RR $$
\end{lemma}

\begin{lemma}
$$N(T+1, \chi) - N(T , \chi) = \mathcal{O} (\log q (|T| + 2))$$
\end{lemma}

\begin{lemma}
$$\sum_{\rho: |\gamma - t|\le 1} \frac{1}{s - \rho} + \mathcal{O}(\log qt) = \frac{L'}{L}(s,\chi)$$ for $-1 \le \sigma \le 2 , |t|\geq 2, L(s,\chi) \neq 0$
\end{lemma}

\begin{lemma}
Let $\chi(q)$ be primitive, $q \geq 3, T \geq 2$. Then, there exists $T_{1} \in [T,T+1]$ such that $\frac{L'}{L}(\sigma \pm iT_{1}, \chi) \ll (\log qT)^2 , -1 \le \sigma \le 2$.
\end{lemma}

\begin{lemma}
Put $a=1$ if $\chi$ is even and $0$ otherwise. $$\mathcal{A}(a) := \{ s \in \CC \mid \sigma \le -1 , |s + 2n - a| \geq \frac{1}{4} \; \forall \; n \geq 1\}$$ Then, $$\frac{L'}{L} (s, \chi) \ll \log (q(|s|+1))$$ on $\mathcal{A}(a)$
\end{lemma}

These are all the ingredients needed to prove the the explicit formula for $\psi_{0}(x, \chi)$. 

\begin{theorem}
$$\psi(s,\chi) = \sum_{n \le x} \Lambda(n) \chi(n)$$ $$\psi_{0}(x,\chi)  = \frac{1}{2}(\psi(x^{+}, \chi) + \psi(x^{-}, \chi)) = - \sum_{\rho : |\gamma|\le t} \frac{x^{\rho}}{\rho} - \frac{1}{2} \log (x-1) - \frac{\chi(-1)}{2} \log (x+1) + C_{\chi} + R_{\chi}(T)$$ where $C_{\chi} = \frac{L'}{L} (1, \overline{\chi}) + \log \frac{q}{2 \pi} - \gamma$ and $R_{\chi}(T) \ll (\log x)\min(1, x/T <x>) + \frac{x}{T} (\log (qxT))^2$. Letting $T \rightarrow \infty$ we see that $R_{\chi}(T) \rightarrow 0$.
\end{theorem}

\begin{theorem}[Brun-Titsmarsh inequality]
Let $x\geq 0, y \geq 2q$. Then, $$\pi(x+y; q,a) - \pi (x; q,a) \le \frac{2y}{\phi(q) \log (\frac{y}{q})} \left( 1 + \mathcal{O}(\frac{1}{\log (\frac{y}{q})}) \right)$$
\end{theorem}

\textcolor{BrickRed}{Remind him to prove this later; uses Sieve theoretic methods}

\begin{theorem}[PNT for Dirichlet characters]
There exists a $c_{1} \geq 0$ such that for all $q \le \exp (c_{1} \sqrt{\log x})$, we have $$\psi(x, \chi) = \sum_{n \le x}\Lambda (n) \chi (n) = \begin{cases} E_{0}(x) + \mathcal{O} (x\exp(-c_{1}\sqrt{\log x})) & \chi \text{ has no Siegel zero} \\ -\frac{x^{\beta_{1}}}{\beta_{1}} + \mathcal{O} (x\exp(-c_{1}\sqrt{\log x})) & \chi \text{ has Siegel zero} \end{cases}$$ Here, $E_{0}(\chi) = 1$ if $\chi = \chi_{0}$ and $0$ otherwise. 
\end{theorem}

Recall from MA317 that $L(x,\chi) \neq 0$ when $\sigma \geq  1 - \frac{c}{\log q\tau}$ for some constant $c>0$ with the exception of atmost one real zero ($\beta_{1}$ the Siegel zero)

\begin{proposition}
Let $c$ be as above and assume that $\sigma \geq 1 - \frac{c}{2 \log q \tau}$. Then, 
\begin{enumerate}
\item If $L(s,\chi)$ has no Siegel zero or if $\beta_{1}$ is a Siegel zero (thus $\chi$ quadratic) but $|s- \beta_{1}| \geq \frac{1}{\log q}$, then $$\frac{L'}{L}(s, \chi) \ll \log q \tau$$ $$|\log L(s,\chi)| \ll \log \log q\tau + \mathcal{O}(1)$$ $$\frac{1}{L(s,\chi)} \ll \log q\tau$$
\item If $\beta_{1}$ is a Siegel zero and $|s - \beta_{1}| \le \frac{1}{\log q}$, then $$\frac{L'}{L}(s, \chi) = \frac{1}{s- \beta_{1}} + \mathcal{O}(\log q)$$ $$|\mathrm{arg} L(s,\chi)| \le \log \log q + \mathcal{O}(1)$$ $$|s- \beta_{1}| \ll |L(s,\chi)| \ll |s-\beta_{1}|(\log q)^2$$
\end{enumerate}
\end{proposition}


















\part{Commutative Algebra}

\chapter{Tensor Products}

If $M,N$ are abelian groups, what is $M \otimes_{\ZZ} N$. We will try to address this question. A bilinear map $\phi : M\times N \rightarrow P$ ($P$ is an abelian group) is a function which is linear in both $M,N$. Tensor products are universal objects for all Bilinear maps from $M \times N$. In other words, 

\begin{theorem}
There exists pair $(T,g) \in \mathrm{Bilin}(M,N)$ satisfying the following properties: 
\begin{enumerate}
\item Given any pair $(P, \phi)$ there exists an unique homomorphism $\tilde{\phi} : T \rightarrow P$ such that $\tilde{\phi} \circ g = \phi$.
\item If $(T',g')$ is another pair which satisfies $1$, then there exists an unique isomorphism $\theta : T \rightarrow T'$ such that $\phi \circ g = g'$.
\end{enumerate}
\end{theorem}

\begin{proof}
\begin{itemize}
\item \textbf{Uniqueness}: Suppose there are two pairs $(T,g)$ and $(T',g')$ satisfying $1$. Then, we have two following two diagrams 
\[\begin{tikzcd}
	& {M\times N} &&& {M\times N} \\
	{T'} & T && T & {T'}
	\arrow["g", from=1-2, to=2-2]
	\arrow["{g'}"', from=1-2, to=2-1]
	\arrow["{\exists \;!\; \tilde{g'}}", dashed, from=2-2, to=2-1]
	\arrow["{g}", from=1-5, to=2-5]
	\arrow["{g'}"', from=1-5, to=2-4]
	\arrow["{\exists \;!\;\tilde{g}}", dashed, from=2-5, to=2-4]
\end{tikzcd}\]
Therefore
\[\begin{tikzcd}
	& {M\times N} &&& {M\times N} \\
	T & T && {T'} & {T'}
	\arrow["g", from=1-2, to=2-2]
	\arrow["g"', from=1-2, to=2-1]
	\arrow["{\tilde{g}\circ \tilde{g'}}", dashed, from=2-2, to=2-1]
	\arrow["{g'}", from=1-5, to=2-5]
	\arrow["{g'}"', from=1-5, to=2-4]
	\arrow["{\tilde{g'} \circ \tilde{g}}", dashed, from=2-5, to=2-4]
\end{tikzcd}\]
with $\tilde{g} \circ \tilde{g'} = \mathrm{id}_{T}$ and $\tilde{g'} \circ \tilde{g} = \mathrm{id}_{T'}$. This completes the first part. The uniqueness of isomorphism $\theta$ is also similar.

\item \textbf{Existence}: Take $C$ to the free abelian group on $M \times N$. Define $D \hookrightarrow C$ to be the subgroup generated by the relations:
\begin{enumerate}
\item $(m + m' , n) - (m,n) - (m',n) $
\item $(m, n+ n') - (m,n) - (m,n')$
\item $(am,n) - a(m,n)$
\item $(m,an) - a(m,n)$
\end{enumerate}
with $m,m' \in M , n,n' \in N , a\in \ZZ$ \\
Let $T := C/D$ and $g: M \times N \rightarrow T$ is defined by $g(m,n) = (m,n) \pmod{D}$. \\
The conditions imposed implies that $g$ is bilinear. Next, suppose $(P, \phi) \in \mathbf{Bilin}(M,N)$, then we get a homomorphism $\tilde{\phi} : C \rightarrow P$ (by the universal property of free abelian groups). \\
$\because \phi$ is bilinear, $\tilde{\phi}$ kills all of $D$. \\
$\therefore$ it descends to a map $\tilde{\phi} : C/D \rightarrow P$ that is $\tilde{\\phi : T \rightarrow P}$ 
\end{itemize}
This completes the proof.
\end{proof}

\begin{example}
$\ZZ/2\ZZ \otimes_{\ZZ} \ZZ / 3\ZZ = 0$ since $1 \otimes 1 = 1 \otimes 4 = 2(1 \otimes 2) = 2 \otimes 2 = 0$
\end{example}

\begin{exercise}
\begin{enumerate}
\item If $f : G \rightarrow H$ is a bijective group homomorphism, then $f$ is an isomorphism. 

\item $\ZZ / m ZZ \otimes_{\ZZ} \ZZ / n \ZZ = \ZZ / (m,n) \ZZ$
\item If $M,N$ are free abelian groups, then $M \otimes N$ is a free abelian group. Also, find the rank of $M \otimes N$ in terms of rank of $M,N$
\item $M,N$ torsion implies $M \otimes N$ is torsion
\end{enumerate}
\end{exercise}

\begin{example}
Consider $M' = 2\ZZ \xhookrightarrow{\iota} M  =\ZZ$ and $N = \ZZ / 2 \ZZ$. Take $\alpha = 2 \in M' , \beta = 1 \in N$. Then,
\begin{equation}
M' \otimes N \xlongrightarrow{\iota \otimes \mathrm{id}_{N}} M \otimes N
\end{equation} 
says that $\alpha \otimes \beta =0$ in $M \otimes N$. However, we claim that $\alpha \otimes \beta $ is nonzero in $M' \otimes N$. This is because $\iota$ is injective but the induced map $\iota \otimes \mathrm{id}_{N}$ is not injective. In other words, tensor products need not preserve monomorphisms but it must preserve epimorphisms.
\end{example}

\begin{theorem}
Let $M,N,P$ be abelian groups. Then there exists unique isomorphisms: 
\begin{enumerate}
\item $M \otimes N \xlongrightarrow{\alpha} N \otimes M$
\item $(M \otimes N)\otimes P \xlongrightarrow{\beta} M \otimes (N \otimes P)$
\item $\ZZ\otimes M \xlongrightarrow{\gamma} M $
\item $(M \oplus N) \otimes P \xlongrightarrow{\delta} (M \otimes P) \oplus (N \otimes P)$
\end{enumerate}
\end{theorem}

\begin{proof}
\begin{enumerate}
\item 
\end{enumerate}
\end{proof}

\begin{definition}
An $R$-module is an abelian group $M$ together with a group homomorphism 
\begin{equation}
\phi_{M} : R \otimes M \longrightarrow M
\end{equation}
 such that 
 \begin{enumerate}
 \item \[\begin{tikzcd}
	{R \otimes R \otimes M} && {R\otimes M} \\
	\\
	{R\otimes M} && M
	\arrow["{\mu_{R} \otimes \mathrm{id}_{M}}", from=1-1, to=3-1]
	\arrow["{\mathrm{id}_{R}\otimes \phi_{M}}", from=1-1, to=1-3]
	\arrow["{\phi_{M}}", from=1-3, to=3-3]
	\arrow["{\phi_{M}}"', from=3-1, to=3-3]
\end{tikzcd}\]
 
 \item $\phi_{M}(1 \otimes m) = m$ for all $m \in M$
 \end{enumerate}
\end{definition}


\begin{remark}
$(1)$ gives us associativity, i.e., 
\begin{eqnarray*}
\phi: R \times M &\longrightarrow M \text{ bilinear with } \\
\phi(a, bm) &= (ab)m \; \forall \; m \in M , a,b \in R
\end{eqnarray*}
and $(2)$ shows that $\phi(1,m) = m \; \forall \; m \in M$
\end{remark}


\begin{remark}
Abelian groups are just $\ZZ$-modules
\end{remark}

Suppose $M,N$ are $R$-modules. Then an $R$-linear map is just a group homomorphism $f: M \longrightarrow N$ such that $f(am) = af(m) \; \forall \; a\in R , m \in M$

\begin{remark}
\begin{enumerate}
\item Every abelian group homomorphism is a $\ZZ$-linear map. 
\item $f$ is $R$-linear iff $f$ commutes with the action of $R$ on $M$. 
\item $R$-linear maps are also called module homomorphisms.
\end{enumerate}
\end{remark}

A submodule of $M$ is a subgroup $M'$ of $M$ such that the actions are preserved ($RM' = M'$)\\

If $f: M\rightarrow N$ is a module homomorphism, then $\ker(f)$ is a submodule of $M$ and $\mathrm{Im}(f)$ is a submodule of $N$. \\

If $M \hookrightarrow N$, then $N/M$ is an $R$-module homomorphism with the canonical projection $N \xlongrightarrow{\pi} N/M$ being $R$-linear.

\begin{remark}
The category of $R$-modules is usually denoted by $\mathbf{Mod}_{R}$ or $R-\mathbf{mod}$\\

Suppose $M,N$ are $R$-modules. Then, $\mathbf{Mod}_{R}$ has operations $\oplus $ and $\otimes_{R}$ such that 
\begin{enumerate}
\item $M \otimes_{R} N \simeq N \otimes_{R} M$
\item $(M \oplus M') \otimes_{R} N \simeq (M \otimes_{R} N) \oplus (M' \otimes_{R} N )$
\item $R \otimes_{R} M \simeq M$
\end{enumerate}
\end{remark}

\begin{remark}
Define $\Hom_{R}(M,N) = \{R\text{-linear maps from} M \rightarrow N\}$. It can be seen that $\Hom_{R}(M,N)$ is also a $R$-module such that 
\begin{enumerate}
\item $(f+g)(m) = f(m) + g(m)$ for all $m\in M, f,g \in \Hom_{R}(M,N)$
\item $(af)(m) = af(m)$ for all $a\in R, m\in M, f \in \Hom_{R}(M,N)$
\end{enumerate}
\end{remark}

\begin{definition}
Say $M$ is a $R$-module. Then we define 
\begin{equation}
M_{tor} = \{m \in M \mid am=0 \text{ for some } 0 \neq a \in R\}
\end{equation}
\end{definition}

\begin{exercise}
\begin{enumerate}
\item Prove or disprove that $M_{tor}$ is a $R$-submodule. 
\item $M$ is a torsion abelian group iff $M \otimes_{\ZZ} \QQ = 0$
\end{enumerate}
\end{exercise}

\chapter{Ideals}

An ideal $I$ of $R$ is a $R$-submodule of $R$ such that $IR = I = RI$

\begin{proposition}
Let $I_{1},I_{2}$ be two ideals of $R$.
\begin{enumerate}
\item $I_{1} \cap I_{2}$ is an ideal. 
\item $I_{1}\cup I_{2}$ might not be an ideal. 
\item $I_{1}I_{2}$ is an ideal. 
\item $I_{1}I_{2} \subseteq I_{1} \cap I_{2}$.
\end{enumerate}
\end{proposition}

\begin{exercise}
If $I_{1},I_{2}$ are ideals such that $I_{1} + I_{2} = R$, then $I_{1}\cap I_{2} = I_{1}I_{2}$
\end{exercise}

\begin{lemma}
For ideals $I,J,K$ of $R$, we have $I(J+K) = IJ + IK$
\end{lemma}

For $I \subseteq R$ an ideal, we define the quotient ring $R/I$ such that 
\begin{itemize}
\item $(a+I) + (b+I) = (a+b) + I$
\item $(a+I)(b+I) = ab + I$
\item $b(a+I) = ab + I$
\end{itemize}

Let $I_{1}, \hdots ,I_{n} \subseteq R$ be ideals and say we have a ring homomorphism 
\begin{eqnarray*}
R &\xlongrightarrow{Q} \frac{R}{I_{1}} \times \cdots \times \frac{R}{I_{n}}\\
a &\longmapsto (a+ I_{1}, \hdots , a+I_{n})
\end{eqnarray*}

\begin{exercise}
Show that $\ker(Q) = \bigcap_{j=1}^n I_{j}$
\end{exercise}

Now, 
\begin{equation}
\frac{R}{\ker Q} \xlongrightarrow{Q} \frac{R}{I_{1}} \times \cdots \times \frac{R}{I_{n}}
\end{equation}
So, it is injective. We ask when is it surjective. Recall
\begin{theorem}[Chinese Remainder Theorem]
If $I_{i} + I_{j} = R$ for all $i \neq j$, then 
\begin{enumerate}
\item $\ds{\bigcap_{j=1}^n I_{j} = \prod_{j=1}^n I_{j}}$
\item $\ds{R/\prod_{j=1}^n I_{j}  \simeq \prod_{j=1}^n R/I_{j}}$
\end{enumerate}
\end{theorem}

\begin{definition}
\begin{enumerate}
\item $\pr \subseteq R$ is a prime ideal if $ab \in \pr \Rightarrow a \in \pr$ or $b \in \pr$
\item $\m \subseteq R$ is a maximal ideal if it is a proper ideal and for all proper ideals $\m'$ such that $\m \subsetneq \m'$ implies $\m = \m'$
\end{enumerate}
\end{definition}

\begin{proposition}
\begin{enumerate}
\item $\pr$ is a prime ideal iff $A/\pr$ is an integral domain. 
\item $\m$ is a maximal ideal iff $A/\m$ is a field.
\end{enumerate}
\end{proposition}

\begin{theorem}
Every proper ideal is contained in a maximal ideal.
\end{theorem}

\begin{definition}
\begin{enumerate}
\item $a\in R$ is called nilpotent if $a^n = 0$ for some $n \in\ZZ_{\geq 0}$
\item $\nil(R) = \{a \in R : a \text{ is nilpotent }\}$
\end{enumerate}
\end{definition}

\begin{theorem}
\begin{equation}
\nil(R) = \bigcap_{\pr : \text{ prime }} \pr
\end{equation}
\end{theorem}

\begin{definition}
Jacobson ideal of $R$ = intersection of all maximal ideals. 
\end{definition}

In general, $\nil(R) \subseteq \mathrm{Jac}(R)$

\begin{exercise}
In $R[X]$, $\mathrm{Jac}(R[X]) = \nil(R[X])$
\end{exercise}

\begin{theorem}[Theorem of prime avoidance]
Let $I \subseteq R$ be an ideal and $\pr_{1}, \hdots , \pr_{r}$ be prime ideals such that $I \not \subseteq \pr_{i} \;\forall \; i$. Then, $I \not \subseteq \bigcup_{i=1}^{r} \pr_{i}$
\end{theorem}


\begin{theorem}
If $I_{1}, \hdots ,I_{n} \subseteq R$ are ideals such that $\bigcap_{j=1}^n I_{j} \subseteq \pr$ where $\pr$ is a prime ideal, then $I_{i} \subseteq \pr$ for some $j$.
\end{theorem}

\begin{exercise}
Prove or disprove that a finite intersection of distinct prime ideals cannot be a prime ideal.
\end{exercise}

\begin{definition}
Let $I \subseteq R$ be an ideal. Then, the radical of $I$ is defined as 
\begin{equation}
\sqrt{I} = \{a \in R : a^n \in I \text{ for some } n \in \ZZ_{\geq 1}\}
\end{equation}
\end{definition}

\begin{remark}
\begin{enumerate}
\item $I \subseteq \sqrt{I}$
\item If $I = \{0\}$, then $\sqrt{I}$ is the nilradical. 
\item $R \xlongrightarrow{f} R/I$ implies $\sqrt{I} = f^{-1}(\nil(R/I))$
\end{enumerate}
\end{remark}

\begin{exercise}
\begin{enumerate}
\item Suppose $\sqrt{I} + \sqrt{J} = R$. Then, $I + J = R$. 
\item Show that $x \in \mathrm{Jac}(R) \Leftrightarrow 1-xy \in R^{\times} \; \forall \; y \in R$
\item 1,2,4,6,7,8,14 from chapter $1$ Atiyah McDonald
\end{enumerate}
\end{exercise}

\chapter{Modules}

\begin{theorem}[Cayley-Hamilton Theorem]
Let $I \subseteq R$ be an ideal and $M$ a f.g. $R$-module. Suppose $\phi$ is an endomorphism of $M$ such that $\phi(M) \subseteq IM$. Then $\phi$ satisfies a polynomial equation 
\begin{equation}
\phi^n + a_{1}\phi^{n-1} + \cdots + a_{n-1}\phi + a_{n} = 0
\end{equation}
where $a_{i} \in I \; \forall \; i$
\end{theorem}

\begin{corollary}
Let $M$ be a f.g. $R$-module. Suppose $I \subseteq R$ is an ideal such that $IM=M$. Then there exists $a\in R$ such that $(1) a\equiv 1 \pmod{I}$ and $(2) aM=0$.
\end{corollary}

\begin{lemma}[Nakayama Lemma]
Suppose $M$ is a f.g. $R$-module. Suppose $I \subseteq \mathrm{Jac}(R)$ is an ideal such that $IM=M$, then $M=0$.
\end{lemma}

\begin{definition}
Let $R$ be a commutative ring. $R$ is called local if it has only one maximal ideal. We denote a local ring by $(R, \m , k)$ where $\m$ is the maximal ideal and $k$ is the residue field.
\end{definition}

\begin{example}
fields, $\ZZ/n\ZZ$ where $n  = p^f$
\end{example}

\begin{example}
Fix a prime $p$. Set $\ZZ_{(p)} = \{\frac{m}{n}: p \nmid n\}$. Then, $\ZZ_{(p)}$ is a local ring with $p \ZZ_{(p)}$ as the only maximal ideal. \\

More generally, for an integral domain $R$ and a prime ideal $\pr$ of $R$, $R_{\pr} = \{\frac{m}{n}: n \not \in \pr\}$ is also a local ring 
\end{example}

\begin{remark}
Every element outside the maximal ideal is an unit in a local ring. The converse is also true. 
\end{remark}

\begin{corollary}
Suppose $R$ is a local ring with residue field $k$. If $M \otimes_{R} k = 0$, then $M=0$.
\end{corollary}

Consider the categories 
\begin{align*}
\mathbf{Mod}_{R} &\xlongrightarrow{\otimes_{R} k} k-\text{vector spaces}\\
M &\longmapsto M \otimes_{R} k 
\end{align*}

Here, $M = 0 \Leftrightarrow M \otimes_{R} k = 0$ by the previous field. Hence, it is a conservative functor. 

\begin{definition}
Let \begin{equation}
\mathcal{C} \xlongrightarrow{F} \mathcal{D}
\end{equation}
We say $F$ is conservative if $C=0 \Leftrightarrow F(C)=0$ where $C \in \mathrm{Ob}(\mathcal{C}), F(C) \in \mathrm{Ob}(\mathcal{D})$.
\end{definition}

If $V$ is a vector space over $\mathbb{F}$ and let $V$ be a finite dimensional. Then, $$\phi : V \twoheadrightarrow V$$ implies $\phi$ is an isomorphism. If we remove finite dimensionality, then this assertion is not true. For take $V=\ZZ[X]$ and $\phi$ be the polynomial derivative.

\begin{theorem}
Let $M$ be a $R$-module. If $\phi : M \twoheadrightarrow M$ is a surjection, then $\phi$ is an isomorphism.
\end{theorem} 

\begin{lemma}[Snake Lemma]

\end{lemma}

\begin{lemma}
Suppose we have a sequence 
\begin{equation}[*]
M' \xlongrightarrow{u} M \xlongrightarrow{v} M'' \longrightarrow 0
\end{equation}
Then, $*$ is exact iff the sequence 
\begin{equation}[*]
0 \longrightarrow \Hom_{R}(M '' , P) \xlongrightarrow{v^*} \Hom_{R}(M,P) \xlongrightarrow{u^*} \Hom_{R}(M',P)
\end{equation}
is exact for all $R$-modules $P$.
\end{lemma}

\begin{lemma}
Suppose we have a sequence 
\begin{equation}[*]
M' \xlongrightarrow{u} M \xlongrightarrow{v} M'' \longrightarrow 0
\end{equation}
Then, $*$ is exact iff the sequence 
\begin{equation}[*]
0 \longrightarrow \Hom_{R}(P,M') \xlongrightarrow{v_*} \Hom_{R}(P,M) \xlongrightarrow{u_*} \Hom_{R}(P,M'')
\end{equation}
is exact for all $R$-modules $P$.
\end{lemma}

\begin{remark}
$\Hom_{R}(P, -)$ is a left exact functor. We claim that $\Hom_{R}(- , P)$ is not a right exact functor. Because consider 
\begin{equation}
0 \longrightarrow M' \xlongrightarrow{u} M 
\end{equation}
Then \begin{equation}
0 \longrightarrow \Hom_{R}(M,P) \xlongrightarrow{u^*} \Hom_{M',P}
\end{equation}
need not be exact. For example, 

take  $R=\ZZ$
\end{remark}

\begin{theorem}
Suppose we have a sequence 
\begin{equation}[*]
M' \xlongrightarrow{u} M \xlongrightarrow{v} M'' \longrightarrow 0
\end{equation}
Then, $*$ is exact iff the sequence 
\begin{equation}[*]
M' \otimes_{R} P \longrightarrow M \otimes_{R} P \longrightarrow M'' \otimes_{R} P \longrightarrow 0
\end{equation}
is exact for all $R$-modules $P$.
\end{theorem}

\chapter{Projective, Injective, Flat modules}

\section{Flat and faithfully flat}

\begin{definition}
Let $N$ be a $R$-module. Then, $N$ is flat if $\otimes_{R} N : \mathbf{Mod}_{R} \rightarrow \mathbf{Mod}_{R}$ is exact.
\end{definition}

\begin{lemma}
TFAE: 
\begin{enumerate}
\item $N$ if flat 
\item $M' \hookrightarrow M \Rightarrow M' \otimes N \hookrightarrow M \otimes N$
\item $M' \hookrightarrow M $ with $M,M'$ f.g. implies $M'\otimes N \hookrightarrow M \otimes N$
\end{enumerate}
\end{lemma}

\begin{example}
\begin{enumerate}
\item every free module is flat.
\item 
\end{enumerate}
\end{example}

\begin{exercise}
Let $R$ be an integral domain with fraction field $k$. Then, $k$ is $R$-flat.
\end{exercise}

\begin{definition}
$F$ is called faithfully flat (ff) if for all sequences 
\begin{equation}[*]
M' \longrightarrow M \longrightarrow M''
\end{equation}
$*$ is exact iff 
\begin{equation}[**]
M' \otimes_{R} F \longrightarrow M \otimes_{R} F \longrightarrow M'' \otimes_{R} F 
\end{equation}
is exact.
\end{definition}

\begin{remark}
$ff \Rightarrow $ flat
\end{remark}

\begin{example}
Abeliean $\Rightarrow ff$. Is the converse true? If $F= \QQ $ ff as a $\ZZ$-module?\\

Notice that if $M$ is finite abelian group, and $$0 \rightarrow M \rightarrow 0$$ Then, $$0 \rightarrow M \otimes \QQ \rightarrow$$ which implies $$0 \rightarrow 0 \rightarrow 0$$ Therefore $\QQ$ is flat but not ff.
\end{example}

\begin{theorem}
Let $R$ be an integral domain with fraction field $K$. Suppose $$R \hookrightarrow S \hookrightarrow K$$ are inclusions of rings. Then, 
\begin{enumerate}
\item $S$ is flat over $R$.
\item $S$ is ff over $R$ iff $S=R$.
\end{enumerate}
\end{theorem}

\begin{exercise}
Suppose $F$ has the property that $M \otimes F = 0 \Leftrightarrow M=0$. Is $F$ ff?
\end{exercise}

\begin{theorem}
$M$ is flat iff $I \otimes M \rightarrow M$ is injective for all ideal $I \subseteq R$.
\end{theorem}

\begin{exercise}
\begin{enumerate}
\item $R[X]$ is flat over $R$.
\item Direct sum(summand) is flat (if flat)
\item Tensor product of flat modules are flat.
\item Suppose $R$ is an integral domain. Then, flat implies torsion-free.
\item Does torsion free implies flat? 
\end{enumerate}
\end{exercise}

\begin{theorem}
Let $R\rightarrow S \rightarrow T$ be a ring homomorphism. Then $T$ is flat over $S$ and $S$ flat over $R$ implies $T$ is flat over $R$.
\end{theorem}

\begin{exercise}
Let $R\rightarrow S$ be a ring homomorphism which is ff, $M \in \mathbf{Mod}_{R}$. Then, $M$ is flat over $R$ iff $M \otimes_{R} S$ is flat over $S$.
\end{exercise}

\begin{remark}
Let $R \rightarrow S$ be a ring homomorphism. Then, $M$ is $R$-flat implies $M\otimes_{R} S $ is $S$-flat. Is the converse true? \\

Take $R=\ZZ, M = \ZZ / n\ZZ$. M is not flat over $R$ since it is not torsion free but $M \otimes_{\ZZ} \ZZ / n\ZZ = \ZZ / n \ZZ$
\end{remark}

\section{Projective Modules}

Say $F$ is a free $R$-module. Then we can ask two questions: 
\begin{itemize}
\item Is the submodule free ? 
\item Is the direct summand free if $F$ is free ?
\end{itemize}

\begin{definition}
A $R$-module $P$ is called projective if it is a direct summand of a free $R$-module.
\end{definition}

\begin{exercise}
\begin{enumerate}
\item Let $R = k[X,Y], M=R$. Take $I = \langle X,Y \rangle$. Show that $I$ is not injective.  
\item Take $R = \RR[X_{1},X_{2},X_{3}]/ \langle X_{1}^2 + X_{2}^2 + X_{3}^2 -1 \rangle$. Then, 
\[\begin{tikzcd}
	0 & {P } & {R^3} & R \\
	&& {\{e_{1},e_{2},e_{3}\}} & {e_{i} \mapsto X_{i}}
	\arrow[from=1-1, to=1-2]
	\arrow[from=1-2, to=1-3]
	\arrow[from=1-3, to=1-4]
\end{tikzcd}\]
We see that $R^3 = R \oplus P$. Therefore, $P$ is projective module by definition. But, it is an \textcolor{BrickRed}{OPEN} question whether $P$ is free or not.
\end{enumerate}
\end{exercise}

\begin{theorem}
$P$ is projective iff $\Hom_{R}(P,-)$ is an exact functor of $\mathbf{Mod}_{R}$ to $\mathbf{Mod}_{R}$
\end{theorem}

\begin{remark}
Projective implies flat.
\end{remark}

\begin{theorem}
Suppose $R$ is local, $P$ a f.g. projective $R$-module. Then $P$ is free. 
\end{theorem}

\section{Injective Modules}

\begin{definition}
We say that a $R$-module $E$ is injective if $\Hom_{R}(-,E)$ is exact functor.
\end{definition}

\begin{example}
$\ZZ$ is not injective as $\ZZ$-module 
\end{example}

\begin{remark}
To show that $E$ is injective, it suffices to show that 
\begin{equation}[*]
0 \rightarrow M' \rightarrow M
\end{equation}
exact iff 
\begin{equation}[**]
\Hom_{R}(M,E) \rightarrow \Hom_{R}(M',E) \rightarrow 0
\end{equation}
is exact. \\

i.e., $M' \hookrightarrow M \Rightarrow \Hom_{R}(M,E) \twoheadrightarrow \Hom_{R}(M',E)$ \\

i.e., every map $f: M \rightarrow E$ extends 
\[\begin{tikzcd}
	M & {M'} \\
	E
	\arrow[hook, from=1-1, to=1-2]
	\arrow["f"', from=1-1, to=2-1]
	\arrow[dashed, from=1-2, to=2-1]
\end{tikzcd}\]
\end{remark}

\begin{exercise}
\begin{enumerate}
\item Show that $\QQ$ is injective as $\ZZ$-module.
\item Let $M$ be a flat $S$-module and $E$ an injective $R$-module. Then, $\Hom_{R}(M,E)$ is an injective $S$-module.
\end{enumerate}
\end{exercise}


\begin{theorem}
$E$ is injective iff for all ideals $I \hookrightarrow R$ and homomorphism $f: I \rightarrow E$, there exists an extension $R \xlongrightarrow{f'} E$ of $E$, i.e., 
\[\begin{tikzcd}
	I & R \\
	E
	\arrow[hook, from=1-1, to=1-2]
	\arrow["f"', from=1-1, to=2-1]
	\arrow["{f'}", dashed, from=1-2, to=2-1]
\end{tikzcd}\]
\end{theorem}

\begin{definition}
We say that a $R$-module is divisible if $M \xlongrightarrow{\alpha}M$ is surjective for all $a\in R$, i.e., given $m \in R \exists \; n \in M$ such that $an=m$
\end{definition}

\begin{exercise}
\begin{enumerate}
\item Injective implies divisible.
\item If $R$ is PID, then injective iff divisible.
\end{enumerate}
\end{exercise}

\begin{remark}
$\QQ$ is divisible implies $\QQ/\ZZ$ is divisible. Also, $\ZZ$ is a PID therefore $\QQ/\ZZ$ is injective.
\end{remark}

\begin{definition}
Let $M$ be a $R$-module (thus it is also a $\ZZ$-module). Define 
\begin{equation*}
M^{\vee} := \Hom_{\ZZ}(M, \QQ / \ZZ)
\end{equation*}
\end{definition}

\begin{remark}
If $M \neq 0$, then we can always construct a nonzero abelian group homomorphism $M \rightarrow \QQ/\ZZ$. Thus, $M^{\vee} \neq 0$. For $M \in \mathbf{Mod}_{R}$ find $F \twoheadrightarrow M^{\vee}$. As $F$ is free $R$-module therefore $M^{\vee} \hookrightarrow F^{\vee}$ as $\ZZ$-modules. Also, the evaluation map $M \hookrightarrow M^{\vee}$ is injective. Therefore $M \hookrightarrow F^{\vee}$\\

By a previous exercise, since $F$ is flat (in fact free) $R$-module and $\QQ/\ZZ$ is an injective $\ZZ$-module, $\Hom_{\ZZ}(F, \QQ/\ZZ) = F^{\vee}$ is injective $R$-module. Thus, $M \hookrightarrow F^{\vee}$, i.e., $M$ sits inside an injective module.
\end{remark}


\begin{remark}
In the category of $R$-modules, every module sits inside an injective module.
\end{remark}

\section{Applications}

Suppose $M,N$ are $R$-modules. We know that $N$ is a quotient of projective (in fact free) module. Thus, we can construct an exact sequence as follows: 
\[\begin{tikzcd}
	{\cdots } & {P_{2}} & {P_{1}} & {P_{0}} & N & 0 \\
	&& {\ker(f_{1})} & {\ker(f_{0})}
	\arrow["{f_{1}}", from=1-3, to=1-4]
	\arrow[from=1-1, to=1-2]
	\arrow["{f_{2}}", from=1-2, to=1-3]
	\arrow["{f_{0}}", from=1-4, to=1-5]
	\arrow[from=1-5, to=1-6]
	\arrow[two heads, from=1-2, to=2-3]
	\arrow[from=2-3, to=1-3]
	\arrow[two heads, from=1-3, to=2-4]
	\arrow[from=2-4, to=1-4]
\end{tikzcd}\]

This is called a projective resolution of $N$. We write it as $P \rightarrow N \rightarrow 0$. \\

Then, $P \otimes_{R}M \rightarrow N \otimes_{R} M \rightarrow 0$ is a chain complex (not necessarily exact). We define 
\begin{equation*}
\mathrm{Tor}_{R}^i (M,N) := H_{i}(P \otimes_{R} N), i \geq 0
\end{equation*}
Also, it can be proven that 
\begin{eqnarray}
\mathrm{Tor}_{R}^i (M,N) = \mathrm{Tor}_{R}^i (N,M) , \text{ and } \\
\mathrm{Tor}_{R}^0 (M,N) = M \otimes_{R} N
\end{eqnarray}

\begin{theorem}
Let $M$ be a $R$-module. TFAE:
\begin{enumerate}
\item $M$ is flat 
\item $\mathrm{Tor}_{R}^i (M,N) = 0 \; \forall \; i >0, \forall \; N$
\item $\mathrm{Tor}_{R}^1 (M,N) = 0 \; \forall \; N$
\end{enumerate}
\end{theorem}

\begin{exercise}
For $I \subseteq R$ an ideal, $M$ a $R$-module, we have 
\begin{equation*}
\mathrm{Tor}^{1}_{R}(R/I , M) = \{m \in M : am=0 \; \forall \; a\in I\}
\end{equation*}
\end{exercise}

\begin{exercise}
1,3,4,5,7,8,10,11 from Atiyah McDonald chapter 2
\end{exercise}

\chapter{Noetherian and Artinian Rings}

Let $\Sigma$ be a poset with the order given by $\le $

\begin{proposition}
TFAE: 
\begin{enumerate}
\item Every increasing chain $x_{1} \le x_{2} \le \cdots $ is stationary.
\item Every non-empty subset of $\Sigma$ has a maximal element. 
\end{enumerate}
\end{proposition}

\begin{definition}
Suppose $\Sigma$ is a set whose subsets are ordered by inclusions. If $\Sigma$ satisfies any of the conditions of the previous proposition, we say that the subsets of $\Sigma$ satisfy the ascending chain condition (acc). If the subsets of $\Sigma$ are ordered by $\geq$ then we say that the subsets of $\Sigma$ satisfy the descending chain condition (dcc).
\end{definition}

\begin{definition}
\begin{enumerate}
\item We say that the $R$-module $M$ is Noetherian if its submodules satisfy acc.
\item We say that the $R$-module $M$ is Artinian if its submodules satisfy dcc.
\end{enumerate}
\end{definition}

\begin{example}
\begin{enumerate}
\item $M$ a finite abelian group is both Noetherian and Artinian.
\item $\ZZ$ as a $\ZZ$-module is Noetherian ($\ZZ$ is a Noetherian ring) since every element can be factored into finitely many irreducibles. But $\ZZ$ is not Artinian since 
\begin{equation*}
\ZZ/2 \ZZ \supseteq \ZZ / 4\ZZ \supseteq \ZZ / 8 \ZZ \supseteq \cdots 
\end{equation*}
\item Let $k$ be a field. Then a $k$-module $M$ is just a $k$-vector space. Therefore $M$ is Noetherian as $k$-module iff $M$ is finite dimensional iff $M$ is Artinian as a $k$-module. In particular, $k$ itself is just $1$-dimensional $k$-module. Therefore $k$ is Noetherian ring.
\item Let $R =\ZZ, p $ a prime. $M = \{x \in \QQ/\ZZ \mid p^m x =0 \text{ for some } m \geq 0\}$. Then $M$ is a Noetherian $R$-module. This follows from the following claim: \\
\textbf{Claim}: The only subgroups of $M$ are of the form $(\frac{1}{p^m} \ZZ)/\ZZ \simeq \ZZ / p^m \ZZ$ 
\begin{proof}

\end{proof}
In fact, look at 
\[\begin{tikzcd}
	{\ZZ/p\ZZ} & {\ZZ/p^2 \ZZ} & {\ZZ/p^3 \ZZ} & {\cdots } \\
	1 & p & {p^2}
	\arrow[hook, from=1-1, to=1-2]
	\arrow[hook, from=1-2, to=1-3]
	\arrow[hook, from=1-3, to=1-4]
	\arrow[maps to, from=2-1, to=2-2]
	\arrow[maps to, from=2-2, to=2-3]
\end{tikzcd}\]
Then $M = \varinjlim \ZZ/p^n \ZZ$. It can be shown that $M$ is Artinian but not Noetherian.
\item Let $M = \{m/p^n \mid n \geq 0 , m \in \ZZ\} \hookrightarrow \QQ$. Then, 
\[\begin{tikzcd}
	M & \QQ & {\QQ/\ZZ}
	\arrow[hook, from=1-1, to=1-2]
	\arrow[two heads, from=1-2, to=1-3]
	\arrow["\alpha"', curve={height=18pt}, from=1-1, to=1-3]
\end{tikzcd}\]
$N = \ker(\alpha), N' \mathrm{Im}(\alpha)$. Then $$0 \rightarrow N \rightarrow M \rightarrow N' \rightarrow 0$$ is exact. But $M$ is neither Noetherian nor Artininan.
\end{enumerate}
\end{example}

\begin{remark}
View $R$ as a $R$-module over itself. If the module is Noetherian then it has to be Artinian.
\end{remark}

\begin{proposition}
$M$ is Noetherian iff every submodule of $M$ is f.g.
\end{proposition}

\begin{example}
Let $R = k[X_{1}, \hdots , X_{i}, \hdots ]$. We know that $R$ is not Noetherian since $$k \subseteq k [X_{1}] \subseteq k[X_{1},X_{2}] \subseteq \cdots $$
But $R \hookrightarrow \mathrm{Frac}(R)$ and $\mathrm{Frac}(R)$ is Noetherian. \\
Next, $R \subseteq S = \{a/b : X_{1} \nmid b\} \subseteq F$. Is $S$ Noetherian?
\end{example}

\begin{theorem}
$R$ is Noetherain iff every prime ideal of $R$ is f.g.
\end{theorem}

\begin{proposition}
Let $$(*) 0 \rightarrow M' \rightarrow M \rightarrow M'' \rightarrow 0$$ be exact sequence of $R$-modules. Then 
\begin{enumerate}
\item $M$ is Noetherian iff $M'$ and $M''$ are. 
\item $M$ is Artinian iff $M'$ and $M''$ are.
\end{enumerate}
\end{proposition}

\begin{corollary}
If $M = \bigoplus_{i =1}^n M_{i}$, then $M$ is Noetherian (Artinian) iff $M_{i}$'s are Noetherian (Artinian).
\end{corollary}

\begin{definition}
We say that $R$ is Noetherian (resp. Artinian) if it is Noetherian (resp. Artinian) as a $R$-module.
\end{definition}

\begin{example}
Let $X$ be an infinite compact Hausdorff space and $\mathsf{C}(X)$ the set of continuous functions from $X$ to $\RR$. Then, is $\mathsf(X)$ a Noetherian ring?
\end{example}

\begin{exercise}
A compact Hausdorff space is Noetherian iff it is infinite.
\end{exercise}

\begin{proposition}
Suppose $\m_{1}, \hdots ,\m_{n}$ are maximal ideals of $R$ such that $\m_{1} \cdots \m_{n} = 0$. Then, $R$ is Noetherian iff $R$ is Artinian.
\end{proposition}

\begin{corollary}
Suppose $(R,\m)$ is a Noetherian local ring such that $\m^n=0$. Then, $R$ is Artinian.
\end{corollary}

\begin{remark}
In previous proposition, we use the fact that $\m_{i}$ are the only maximal ideals. For, take $\ZZ \times \ZZ/4\ZZ$. Then $2\ZZ$ is a maximal ideal such that $(2\ZZ)^2=0$ but it is not Artinian.
\end{remark}

\begin{exercise}
1,2,4,5,6,7,8 of Atiyah McDonald chapter 6
\end{exercise}

\begin{lemma}
Suppose $R$ is Noetherian and $S$ is a $R$-algebra such that $S$ is f.g. as a $R$-module. Then, $S$ is also a Noetherian ring.
\end{lemma}

\begin{theorem}[Hilbert Basis Theorem]
Let $R$ be a Noetherian Ring. Then $R[X]$ is also Noetherian.
\end{theorem}

\begin{proof}

\end{proof}

\begin{corollary}
$R$ is Noetherian iff $R[X_{1}, \hdots ,X_{n}]$ is Noetherian.
\end{corollary}

\begin{definition}
If $S$ is a $R$-algebra, then we say that $S$ is of finite type (f.g.) over $R$ if $exists \; x_{1}, \hdots , x_{n} \in S$ such that every element of $S$ is a polynomial in $\{x_{1}, \hdots ,x_{n}\}$ with coefficients in $R$. \\

This is equivalent to saying there is a surjection of $R$-algebra $R[x_{1}, \hdots x_{n}] \twoheadrightarrow S$
\end{definition}

\begin{corollary}
$R$ is Noetherian and $S$ a f.g. $R$-algebra implies $S$ is Noetherian.
\end{corollary}


\section{Power Series Ring over $R$}

Denote by $R[[ X ]] = \{\sum_{i \geq 0} a_{i}X^i \mid \; a_{i} \in R\}$ \\

Clearly, $R \hookrightarrow R[X] \hookrightarrow R[[ X ]]$

\begin{theorem}
If $R$ is Noetherian, then $R [[ X ]] $ is Noetherian.
\end{theorem}


\subsection{Interlude to complex analysis}

Let $R$ be the ring of all holomorphic functions on $\CC$ which are holomorphic on some neighbourhood of $0$. \\

Also, $I = \{f \in R : f(0)=0\}$ is the only maximal ideal of $R$. Thus, $R$ is a local ring. Is $R$ Noetherian? \\

Observe that $R = \CC \{ X \} \hookrightarrow \CC [[ X ]]$ where $\CC \{ X \}$ the ring of convergent power series. Thus, $$\CC \hookrightarrow \CC[X] \hookrightarrow \CC \{ X \}=R \hookrightarrow \CC [[ X ]]$$
We know that all the terms in the sequence except $R$ are Noetherian, so we guess that $R$ is also Noetherian. This is, in fact, true. 

\begin{proposition}
Let $S$ be a $R$-algebra which is faithfully flat. Suppose that $S$ is Noetherian. Then, $R$ is also Noetherian.
\end{proposition}

Using this proposition and the fact that $$\CC\{X\} \hookrightarrow \CC [[ X ]]$$ is a ff extension, we deduce that $\CC \{X\}$ is Noetherian.

\begin{corollary}
Let $(R, \m) \hookrightarrow (S, \m')$ be a local ring homomorphism that is also flat. Then, $S$ Noetherian implies $R$ Noetherian.
\end{corollary}

Using this corollary. and the fact that $$\CC \{X\} \hookrightarrow \CC [[ X ]]$$
is a local ring homomorphism, it suffices to check that this extension is flat to deduce that $\CC\{X\}$ is Noetherian.


\begin{definition}
A $R$-module $M$ is said to be faithful if $aM\neq 0 \; \forall \; 0 \neq a \in R$, .i.e., the annihilator of $M$ is $\{0\}$.
\end{definition}

\begin{proposition}
Let $M$ be a $R$-module. If $M$ is Noetherian and faithful, then $R$ is a Noetherian ring.
\end{proposition}

\begin{theorem}[Eakin-Nagata]
Let $R \subseteq S$ be an inclusion of rings such that $S$ is Noetherian ring and is f.g. $R$-module. Then, $R$ is a Noetherian ring.
\end{theorem}

Let $M$ be a $R$-module. Then, 
\begin{equation*}
R \text{ Noetherian} \Leftrightarrow I \subseteq R \text{ is f.g. } \Leftrightarrow \pr \subseteq R \text{ is f.g.}
\end{equation*}

\begin{itemize}
\item Suppose $IM$ is f.g. for all $I \subsetneq R$. Is $M$ Noetherian?
\item Suppose $\pr M$ is f.g. for all $\pr \subsetneq R$. Is $M$ Noetherian?
\end{itemize}
Turns out that both are false! Take $R=k$ a field and $M$ an infinite dimensional vector space. This gives us a counterexample. What happens if $M$ is f.g.

\begin{theorem}
Suppose $M$ is a f.g. $R$-module. Then, 
\begin{equation*}
M \text{ Noetherian } \Leftrightarrow IM \text{ is f.g.} \forall \; I \subsetneq R \Leftrightarrow \pr M \text{ is f.g. } \forall \; \pr \subseteq R
\end{equation*}
\end{theorem}

\begin{proposition}
Let $(R, \m , k)$ be a local Noetherian ring. If $M$ is a f.g. $R$-module. Then TFAE: 
\begin{enumerate}
\item $M$ is free.
\item $M$ is flat.
\item $\m \otimes M \rightarrow M$ is injective.
\item $\mathrm{Tor}_{R}^{1}(k,M) = 0$
\end{enumerate}
\end{proposition}

\chapter{Nullstellansatz}

\section{Interlude to algebraic geometry}

Let $R$ be a commutative ring, $X = \spec(R) =$ the set of all prime ideals of $R$. Declare a subset of $X$ to be closed if it is of the form $$\V(I) = \{\pr \mid I \subseteq \pr \}$$ where $I$ is a fixed ideal of $R$. 

\begin{exercise}
Show that this defines a topology on $X$. This topology is called the Zariski topology.
\end{exercise}

\begin{definition}
An affine scheme is a commutative ring $R$ together with the Zariski topology of $\spec(R)$.
\end{definition}

\begin{exercise}
\begin{enumerate}
\item $\forall \; f\in R$, define $X_{f} = \{\pr \mid f \not \in \pr\} \subseteq X$. Then, show that each $X_{f}$ is open in the Zariski topology. Also, prove that $\{X_{f}\}$ is a basis for the Zariski topology on $\spec(R)$.
\item Show that $\spec(R)$ is quasi-compact.
\end{enumerate}
\end{exercise}

\begin{definition}
A topological space is called irreducible if the intersection of any two non-empty open sets is non-empty. This can be extended to say that that intersection of finitely many non-empty open sets is non-empty. \\
Equivalently, $X$ is not a finite union of closed sets unless one of the closed sets is all of $X$.
\end{definition}

\begin{remark}
$X$ cannot be irreducible by definition if it is Hausdorff.
\end{remark}

\begin{exercise}
Suppose $R$ is an integral domain. Show that $\spec(R)$ is irreducible.
\end{exercise}

\begin{example}
\begin{enumerate}
\item Let $R= \CC[t], X = \spec(R)$. We wish to compare the Zariski topology on $X$ and the natural Euclidean topology on $\CC$. To make this more concrete, observe that the only prime ideals of $\CC[t]$ are of the form $\langle f \rangle$ where $f$ is an irreducible polynomial. But, the only irreducibles are the linear and constant polynomials. Now, $$\Phi : \langle f \rangle \longrightarrow \text{ zeroes of }f$$ is actually surjective from $X\bs \{0\}$ to $\CC$.
\item Let $R = \RR[t], X = \spec(R)$ 
\end{enumerate}
\end{example}

\begin{remark}
\begin{enumerate}
\item If $f \xlongrightarrow{f} S$ is a ring homomorphism, then $f^* : \spec(S) \rightarrow \spec(R)$ given by $\pr \mapsto f^{-1}(\pr)$ is a continuous map.
\item For $R \xlongrightarrow{\pi} R/I$ we have $\pi^* : \spec(R/I) \rightarrow \spec(R)$ is bijective and closed immersion.
\item For $R \xlongrightarrow{\pi} R/\nil(R)$ we have $\pi^* : \spec(R/\nil(R)) \rightarrow \spec(R)$ is a homeomorphism. In other words, Zariski topology does not see the nilradical. More precisely, while studying $\spec(R)$ we can just work with $\spec(R/\nil(R))$ and assume $R$ is reduced ring (has no nonzero nilpotent elements).
\end{enumerate}
\end{remark}

Fix a field $k$, let $R = k[X_{1}, \hdots , X_{n}], X = \spec(R), X_{Zar} = \mathrm{maxsp}(R)= \text{ all maximal ideals }, X_{an} = k^n$ \\
Define 
\begin{align*}
\phi : X_{an} &\longrightarrow X_{Zar} \\
(a_{1}, \hdots , a_{n}) &\mapsto \langle X_{1}-a_{1}, \hdots , X_{n} - a_{n}\rangle
\end{align*}
If $k$ has its own topology, then $X_{an}$ gets the product topology. In this case, is $\phi$ continuous ? \\

Any closed set in $X_{Zar}$ is of the form $$\V(I) = \{\text{maximal ideals containing } I\}$$ for some ideal $I \subseteq R$. \\

Then, $\phi^{-1}(\V(I)) =\{(a_{1}, \hdots ,a_{n}):\V(X_{1}-a_{1}, \hdots , X_{n}-a_{n}) \subseteq \V(I)\}$. But since $R$ is Noetherian we must have $I = \langle f_{1}, \hdots , f_{r} \rangle$. \\

Thus, $$\phi^{-1}(\V(I)) = \{(a_{1}, \hdots ,a_{n}): f_{i}(a_{1}, \hdots , a_{n})= 0 \; \forall \; 1\le i \le r\}$$

In this case $k=\RR$, clearly $\phi^{-1}(\V(I))$ is closed as it is the intersection of set of zeroes of polynomials. Therefore $\phi$ is continuous when $k = \RR$ or $\CC$. Moreover, $\phi$ is bijective as well. \\

However, $\phi$ is not a closed map. Take $k = \RR$ or $\CC$ and $Z = \{z: \sin z=0\}$. Then, $Z$ is infinite and thus it is not closed in the Zariski topology. However, it is closed in the Euclidean topology. Thus, $Z$ is analytically closed but not Zariski closed.

\begin{definition}
A subset $Z \subseteq k^n$ is said to be algebraic if there exists a set $S \subseteq R$ such that $Z = \{(a_{1}, \hdots , a_{n}): f(a_{1}, \hdots , a_{n}) = 0 \; \forall \; f\in S\}$\\

For $Z\subseteq k^n$, $\I(Z) = \{f \in R: f(a_{1}, \hdots , a_{n})=0 \; \forall \; (a_{1}, \hdots ,a_{n}) \in Z\}$ is an ideal in $R$ \\

For an ideal $I \subseteq R$, we define $$\V = \{(a_{1}, \hdots , a_{n}): f(a_{1}, \hdots , a_{n}) = 0 \; \forall \; f\in I\}$$
\end{definition} 

Observe that $\V(I) = \V(\sqrt{I})$\\

Let $\Sigma_{Zar} = \text{ set of all radical ideals of }R$ and $\Sigma_{an} = \text{the set of all algebraic subsets of }k^n$\\

We define $\Phi: \Sigma_{Zar} \longrightarrow \Sigma_{an}$ by $I \mapsto \V(I)$\\

Clearly, $\Phi$ is a surjection. We would like to show that this is also an injection. But this need not be true in general.
\begin{example}
$k=\RR, I_{1}= \langle X^2 + 1 \rangle , I_{2} = \langle X^2 + 2 \rangle$. Then, $\Phi(I_{1}) = \Phi(I_{2}) = \emptyset$ therefore $\Phi$ need not be injective here. Thus, we need an additional assumption. Hence, we assume that $k$ is algebraically closed.
\end{example}

\section{Hilbert Nullstellansatz}

\begin{theorem}[Hilbert Nullstellansatz]
Let $k$ be an algebraically closed field, then $\Phi$ is bijective. In fact, for every ideal $I \subseteq R$, $\I(\V(I)) = \sqrt{I}$. Thus, for radical ideals $I_{1},I_{2}$, $\V(I_{1}) = \V(I_{2}) \Rightarrow \I(\V(I_{1})) = \I(\V(I_{2})) \Rightarrow \sqrt{I_{1}} = \sqrt{I_{2}} \Rightarrow I_{1}=I_{2}$
\end{theorem}

\begin{lemma}
Let $A$ be a Noetherian ring. Let $B \hookrightarrow C$ be an inclusion of $A$-algebras such that $C$ is f.g. as $A$-algebra over $A$ and f.g. as a $B$-module over $B$. Then, $B$ is f.g. over $A$. In particular, $B$ is Noetherian.
\end{lemma}

\begin{lemma}
Let $k$ be a field and $E$ a f.g. $k$-algebra. Suppose that $E$ is a field. Then $E$ is a finite field extension of $k$.
\end{lemma}

\begin{lemma}
Let $k$ be a field, $E = k[X]/\langle X^n + a_{n-1}X^{n-1} + \cdots + a_{1}X + a_{0} \rangle$. Then, $E$ is f.g. $k$-module.
\end{lemma}

\begin{corollary}
Let $k$ be a field, $A$ a f.g. $k$-algebra, $\m \hookrightarrow A$ maximal ideal. Then, $A/\m$ is a finite algebraic extension of $k$. In particular, $k \cong A/\m$ if $k$ is algebraically closed.
\end{corollary}

\begin{proof}[Proof of Nullstellansatz]

\end{proof}

If $k = \bar{k}$, then $\phi : X_{an} \rightarrow X_{Zar}$ defined before is a bijection. \\

Consider the maximal ideal $\m \hookrightarrow R = k[X_{1}, \hdots , X_{n}]$. Then, $$k \hookrightarrow R \xtwoheadrightarrow{\phi} k = R/\m$$ And, $I = \langle X_{1}-a_{1}, \hdots , X_{n} - a_{n} \rangle$ is maximal. $\phi(I) = 0 \Rightarrow I \subseteq \ker \phi = \m \Rightarrow \langle X_{1}, \hdots ,X_{n} \rangle = \m$. 

\begin{corollary}
Suppose $k = \bar{k}$. Suppose we have a system of polynomial equations 
\begin{align*}
f_{1}(x_{1}, \hdots ,x_{n}) &=0 \\
f_{2}(x_{1}, \hdots ,x_{n}) &=0 \\
\vdots \\
f_{n}(x_{1}, \hdots ,x_{n}) &=0
\end{align*}
Then this system has no solution in $k^n$ iff there exists polynomials $g_{1}, \hdots ,g_{r}$ such that $\sum_{i=1}^m f_{i}g_{i} = 1$
\end{corollary}


\begin{exercise}
1,6,10,13 from Atiyah McDonald chapter 7
\end{exercise}


\chapter{Localisation}
Let $R$ be a commutative ring, $S$ a multiplicatively closed subset containing $1$.
\begin{theorem}
There exists a ring homomorphism $\theta : R \rightarrow S^{-1}R$ which has the following properties: 
\begin{enumerate}
\item $\theta(s) \in (S^{-1}R)^{\times} \; \forall \; s\in S$
\item Given any ring homomorphism $g: R \rightarrow A$ such that $g(s) \in A^{\times} \; \forall \; s \in S \; \exists \; ! \;$ ring homomorphism $\tilde{g}: S^{-1}R \rightarrow A$ such that $g = \tilde{g} \circ \theta$.
\[\begin{tikzcd}
	R &&&& {} \\
	{S^{-1}R} & A
	\arrow["\theta"', from=1-1, to=2-1]
	\arrow["g", from=1-1, to=2-2]
	\arrow["{\exists \;!\; \tilde{g}}"', dashed, from=2-1, to=2-2]
\end{tikzcd}\]
\end{enumerate}
\end{theorem}

\begin{proof}
\underline{Existence}: Define the relation on $R \times S$ given by $(a,s) \sim (a',s') $ if $t(as' - a's) =0$ for some $t \in S$. It can be checked that this is an equivalence relation. Define  
\end{proof}

\begin{remark}
$\ker \theta = \{a \in R : as=0 \text{ for some }s \in S\}$. This is because $\theta(a)=0 \Leftrightarrow \frac{a}{1}= \frac{0}{1}$ in $S^{-1}R \Leftrightarrow (a,1) \sim (0,1) \Leftrightarrow (a\cdot 1 - 0 \cdot 1)s=0$ for some $s\in S$\\

In particular, if $R$ is an integral domain, then $\theta : R \rightarrow S^{-1}R$ is injective.
\end{remark}

Let $M$ be a $R$-module. We can define $$S^{-1}M = (M \times S)/ \sim$$ where $(m,s) \sim (m',s')$ if $(ms'-m's)t=0$ for some $t \in S$. We denote by $[(m,s)]$ by $\frac{m}{s}$. By definition, it is clear that 
\begin{enumerate}
\item $S^{-1}M$ is a $S^{-1}R$ module by the action $$\frac{a}{s} \cdot \frac{m}{s'} = \frac{am}{ss'}$$
\item $\theta_{M}: M \rightarrow S^{-1}M$ given by $m \mapsto m/1$ is a $R$-module homomorphism
\end{enumerate}

\begin{proposition}
The functor $\mathbf{Mod}_{R} \rightarrow \mathbf{Mod}_{S^{-1}R}$ given by $M \mapsto S^{-1}M$ is an exact functor.
\end{proposition}

Consider \begin{align*}
\alpha : S^{-1}M \times M &\rightarrow S^{-1}M \\
(\frac{a}{s},m) &\mapsto \frac{am}{s}
\end{align*}
We can check that $\alpha$ is a bilinear $R$-module homomorphism. Hence, $\alpha$ induces a $R$-linear map 
\begin{align*}
\alpha : S^{-1}M \otimes_{R} M &\rightarrow S^{-1}M \\
\frac{a}{s}\otimes m &\mapsto \frac{am}{s}
\end{align*}
It is easy to see that $\alpha$ is in fact $S^{-1}R$ linear since 
\begin{align*}
\alpha \left( \frac{a'}{s'} \left( \frac{a}{s} \otimes m \right) \right) &= \alpha \left(\frac{aa'}{ss'} \otimes m \right) \\
&= \frac{aa'm}{ss'}\\
&= \frac{a'}{s'} \alpha \left( \frac{a}{s} \otimes m \right)
\end{align*}

\begin{theorem}
$\alpha$ is an isomorphism.
\end{theorem}

\begin{corollary}
$S^{-1}R$ is flat over $R$.
\end{corollary}

\begin{remark}
However, if $S\bs R^{\times} \neq \emptyset$ then $S^{-1}R$ is not ff. Take $a\in S$ such that $a \not \in R^{\times}$. Then, $\langle a \rangle \neq R$ so if $I = \langle a \rangle$ then $R/IR$ is a nonzero $R$-module. Any element of $R/IR$ is of the form $r + I$. \\

In $R/IR \otimes_{R} S^{-1}R$
\begin{align*}
(r+ I) \otimes \frac{b}{t} &= (ar+I) \otimes \frac{b}{at} \\
&= I \otimes \frac{b}{at} \\
&= 0
\end{align*}
Hence, $R/IR \otimes_{R} S^{-1}R = \{0\}$. Thus, $$0 \longrightarrow R/IR \longrightarrow 0$$ is not exact but $$0 \longrightarrow R/IR\otimes_{R} S^{-1}R \longrightarrow 0$$ is exact. Hence, $S^{-1}R$ is almost never ff.
\end{remark}

\begin{example}
\begin{enumerate}
\item Let $R$ be a ring and $\pr$ be a prime ideal. Let $S = R\bs \pr$. Then, $S$ is multiplicatively closed. We write $R_{\pr} := S^{-1}R$\\

\textbf{Claim}: $R_{\pr}$ is a local ring with maximal ideal $\pr R_{\pr}$

\item Let $f \in R\bs\{0\}$. Set $S = \{f^n \mid n\geq 0\}$. Then, $S$ is a multiplicatively closed and so we can write $R_{f}:= S^{-1}R$.  \\

Observe that  $R_{\pr} = \ds{\varinjlim_{f \in \pr} R_{f}}$.
\item If $R=\ZZ, S= \ZZ\bs \langle p \rangle$ for some prime $p$, then $$S^{-1}R = \{\frac{a}{b}: a,b \in \ZZ , p \nmid b\}$$
\item Let $k$ be a field, $R = k[X_{1}, \hdots , X_{n}]$. Suppose $k = \bar{k}$. Then by Nullstellansatz, every maximal ideal of $R$ is of the form $\m = \langle X_{1} - a_{1}, \hdots , X_{n}-a_{n} \rangle$ for some $(a_{1}, \hdots , a_{n} ) \in k^n$. \\
We also had the continuous map $\theta : \CC^n \rightarrow \CC^n_{Zar}$. Then, for $\m \langle X_{1}-a_{1},\hdots , X_{n}-a_{n} \rangle$, we have $R_{\m} = \mathcal{O}_{X,x}$ the ring of rational functions which are regular (holomorphic) in a Zariski neighbourhood of $x=(a_{1}, a_{2}, \hdots , a_{n})$. This is because any element of $R_{\m}$ is a rational function $f/g$ with $g \not \in \m$, i.e., $g$ does not vanish at $x$. Hence, if $A = \{y \in X\mid  g(y)=0\}$, then $A$ is Zariski closed. Thus, $A^c$ is Zariski neighbourhood of $x$. Hence, $g$ is nonvanishing on a Zariski neighbourhood of $x$. Thus, $f/g$ is regular in this Zariski neighbourhood of $x$. \\

Since the topology is finer than the Zariski topology, we get the inclusion of local rings 
\begin{align*}
\mathcal{O}_{X_{Zar},x} &\hookrightarrow \mathcal{O}_{X_{an},x} \\
\text{ring of rational functions which are regular in a Zariski nbd of }x & \text{ring of rational functions which are regular in an analytic nbd of }x 
\end{align*} 
\end{enumerate}
\end{example}

\begin{proposition}
Localisation commutes with tensor product, i.e., if $M,N$ are $R$-modules, $S \subseteq R$ multiplicatively closed set, then 
\begin{align*}
S^{-1}M \otimes_{S^{-1}R} S^{-1}N &\xlongrightarrow{\sim} S^{-1} (M \otimes_{R} N )\\
\frac{m}{s} \otimes \frac{n}{s'} &\mapsto \frac{m \otimes n}{ss'}
\end{align*}
\end{proposition}

\begin{remark}
Also, $S^{-1}(M \oplus N) \simeq S^{-1}M \oplus S^{-1}N$ as $S^{-1}R$ modules by the mapping $\frac{m+n}{s} \mapsto \frac{m}{s} + \frac{n}{s}$. Notice that 
\begin{equation}
S^{-1}M \oplus S^{-1}N \simeq (S^{-1}R \otimes_{R} M) \oplus (S^{-1}R \otimes_{R} N) \simeq S^{-1}R \otimes_{R} (M \oplus N) \simeq S^{-1}(M \oplus N )
\end{equation}
\end{remark}

What are the ideals of $S^{-1}R$ and how do they relate to ideals of $R$?\\
We have \begin{equation}
\theta : R \rightarrow S^{-1}R
\end{equation}
For ideal $I \subseteq R$, $T^e = \theta(I)S^{-1}R$ is an ideal of $S^{-1}R$. \\
For ideal $J \subseteq S^{-1}R$, $J^c = \theta^{-1}(J)$ is an ideal of $R$.

\begin{lemma}
For ideal $I \subseteq R$, $I^e = S^{-1}I$
\end{lemma}

\begin{lemma}
Every ideal of $S^{-1}R$ is an extended ideal, i.e., of the form $I^e$ for some ideal $I \subseteq R$
\end{lemma}

\begin{lemma}
Let $I \subseteq R$ be an ideal. Then we know that $I^e \hookrightarrow S^{-1}R$ is an ideal, and $I \hookrightarrow I^{ec}$. In fact, $$I^{ec} = \bigcup_{s\in S} (I:s)$$ where $(I:s) = \{a \in R \mid as \in I\} \supseteq I$
\end{lemma}

\begin{corollary}
If $\pr \subseteq R$ is a prime ideal, then $\pr = \pr^{ec}$ (if $\pr \cap S = \emptyset$)
\end{corollary}

\begin{corollary}
Every proper ideal of $S^{-1}R$ is of the form $S^{-1}I$ for some ideal $I \subseteq R$ such that $I \cap S = \emptyset$
\end{corollary}

\begin{exercise}
An ideal of $R$ is a contracted ideal iff no element of $S$ is a zero divisor of $R/I$
\end{exercise}

\section{Prime ideals in $S^{-1}R$}

\begin{proposition}
\begin{align*}
\{\text{prime ideals of }R \text{ disjoint from }S\} &\rightarrow \{\text{all prime ideals of }S^{-1}R\} \\
\pr &\longmapsto S^{-1}\pr
\end{align*}
This correspondence is well-defined and a bijection. Thus, $\spec(S^{-1}R) \hookrightarrow \spec(R)$ is an inclusion of spaces.
\end{proposition}

\begin{exercise}
Let $I \subseteq R$ be an ideal. Then, $$S^{-1}\sqrt{I} = \sqrt{S^{-1}I}$$ 
\end{exercise}


\begin{corollary}
\begin{equation}
\nil(S^{-1}R) = S^{-1} \nil(R)
\end{equation}
\end{corollary}

\begin{proposition}
Let $\pr \subseteq R$ be a prime ideal. Then, \begin{equation}
\spec(R_{\pr}) = \bigcap_{f \not \pr} \spec(R_{f})
\end{equation}
\end{proposition}

Let $P$ be a property of modules (or rings) over commutative rings. Then, we say that $P$ is local if for every $R$-module $M$ 
\begin{equation}
M \text{ has } P \text{ as a } R \text{ module } \Leftrightarrow M_{\pr} \text{ has } P \text{as a } R_{\pr} \text{ module } \; \forall \; \pr \subseteq R
\end{equation} 

\begin{proposition}
Let $M$ be a $R$-module. TFAE: 
\begin{enumerate}
\item $M=0$
\item $M_{\pr} = 0$ for all $\pr \in \spec(R)$
\item $M_{\m}=0$ for all $\m \in \mathrm{maxSpec}(R)$
\end{enumerate}
\end{proposition}

\begin{proposition}
Let $\phi: M \rightarrow N$ be a $R$-linear map. TFAE: 
\begin{enumerate}
\item $\phi$ is injective
\item $\phi_{\pr}$ is injective for all $\pr \in \spec(R)$
\item $\phi_{\m}$ is injective for all $\m \in \mathrm{maxSpec}(R)$
\end{enumerate}
\end{proposition}

\begin{proposition}
Let $M$ be a $R$-module. TFAE: 
\begin{enumerate}
\item $M$ is $R$-flat
\item $M_{\pr}$ is $R_{\pr}$-flat for all $\pr \in \spec(R)$
\item $M_{\m}$ is $R_{\m}$-flat for all $\m \in \mathrm{maxSpec}(R)$
\end{enumerate}
\end{proposition}

\begin{exercise}
2,3,4,5,6,8,13,17 from Atiyah McDonald chapter 3
\end{exercise}

\chapter{Integral extensions}

\begin{proposition}
Let $A\subseteq B \subseteq C$ be inclusions of commutative rings. If $B$ is integral over $A$ and $C$ is integral over $B$, then $C$ is integral over $A$
\end{proposition}

\begin{corollary}
Let $A \subseteq B$ be an inclusion of rings. Let $C$ be the integral closure of $A$ in $B$. Then, $C$ is integrally closed in $B$.
\end{corollary}

\begin{proposition}
Suppose $A \subseteq B$ is an integral extension. 
\begin{enumerate}
\item Let $J \subseteq B$ be an ideal and let $I = J \cap A = J^c$. Then, $A\I \subseteq B/J$ is an integral extension. 
\item If $S\subseteq A$ is a multiplicatively closed set, then $S^{-1}A \subseteq S^{-1}B$ is an integral extension.
\end{enumerate}
\end{proposition}

\begin{theorem}
Let $f: B \rightarrow B'$ be an integral extension of $A$-algebras and let $C$ be an $A$-algebra. Then, $f \otimes_{A} 1 : B \otimes_{A} C \rightarrow B' \otimes_{A} C$ is an integral extension.
\end{theorem}

\begin{proposition}
Let $A \subseteq B$ be an inclusion of integral domains. Assume that $B$ is integral over $A$. Then, $A$ is a field iff $B$ is a field.
\end{proposition}

\begin{corollary}
$\CC \otimes_{\RR} \CC$ is not an integral domain
\end{corollary}

\begin{corollary}
Let $A \subseteq B$ be an integral extension of rings. Let $\mathfrak{q} \subseteq B$ be a prime ideal. We already know that $\pr = \mathfrak{q}^c = \mathfrak{q}\cap A$ is a prime ideal of $A$. Then, $\mathfrak{q}$ is maximal in $B$ iff $\pr$ is maximal in $A$.
\end{corollary}

\begin{remark}
In general, neither implication is true. Consider $\ZZ \hookrightarrow \QQ$ and $\QQ \rightarrow \QQ[X]$ for counterexamples.
\end{remark}

\begin{corollary}
Let $A \subseteq B$ be an integral extension of rings. Let $\mathfrak{q}_{1} \subseteq \mathfrak{q}_{2} \subseteq B$ be two prime ideals. Then, $\mathfrak{q}_{1} = \mathfrak{q}_{2}$ iff $\mathfrak{q}_{1}^c = \mathfrak{q}_{2}^c$
\end{corollary}

\begin{theorem}
Let $A \subseteq B$ be an integral extension. Let $f: A \rightarrow B$ be this inclusion. Then, $f_{\#} : \spec(B) \rightarrow \spec(A)$ is surjective.
\end{theorem}

\section{Going up and going down}

If $A \subseteq B$ is an integral extension, then $\spec(B) \rightarrow \spec(A)$ is surjective. 

\begin{theorem}
$A \subseteq B$ an integral extension and $\pr_{1} \subseteq \pr_{2} \subseteq \cdots \subseteq \pr_{n}$ is a chain of prime ideals in $A$. Suppose there exists a chain of prime ideals $\mathfrak{q}_{1} \subseteq \cdots \subseteq \mathfrak{q}_{m} (m < n)$ in $B$ such that $\mathfrak{q}_{i} \cap A = \pr_{i}$. Then, $\mathfrak{q}$-chain can be extended to a chain $\mathfrak{q}_{1} \subseteq \cdots \subseteq \mathfrak{q}_{m} \subseteq \mathfrak{q}_{m+1} \subseteq \cdots \subseteq \mathfrak{q}_{n}$ of prime ideals in $B$ such that $\mathfrak{q}_{i} \cap A = \pr_{i} \; \forall \; i$
\end{theorem}

\begin{corollary}
If $A \subseteq B$ is an integral extension and prime ideals of $A$ satisfy acc, then prime ideals of $B$ also satisfy acc.
\end{corollary}

\begin{definition}
An integral domain $A$ is called normal (or integrally closed ) if $A$ is integrally closed in its fraction field.
\end{definition}


\begin{example}
$A=\ZZ$. Any UFD is normal. Normal does not imply UFD
\end{example}


\begin{proposition}
Let $A \subseteq B$ be an inclusion of rings and let $C$ be the integral closure of $A$ in $B$. Let $S \subseteq A$ be a multiplicatively closed set. Then, $S^{-1}C$ is the integral closure of $S^{-1}A$ in $S^{-1}B$.
\end{proposition}


\begin{corollary}
Every localisation of a normal domain is a normal domain.
\end{corollary}

\begin{example}
$A$ normal implies $A_{\pr}$ normal for all $\pr$
\end{example}

Normalisation of $A=$ integral closure of $A$ in $\mathrm{Frac}(A)$
\begin{exercise}
Is normalisation of $A$ always finite over $A$?
\end{exercise}

\begin{example}

\end{example}

\begin{proposition}
Let $A$ be an integral domain. TFAE: 
\begin{enumerate}
\item $A$ is normal
\item $A_{\pr}$ is normal for all $\pr \in \spec(A)$
\item $A_{\m}$ is normal for all $\m \in \mathrm{maxSpec}(A)$
\end{enumerate}
\end{proposition}


Let $A \subseteq B$ be an inclusion of rings. $I \subseteq A$ an ideal. $x\in B$ is integral over $I$ if there exists a monic polynomial $f(x)\in I[x]$ such that $f(x)=0$.

Suppose $A \subseteq B$ is an extension and $C$ the integral closure of $A$ in $B$. 

\begin{proposition}
The integral closure of $I$ in $B$ is $\sqrt{IC}$
\end{proposition}

\begin{corollary}
Integral closure of an ideal is a radical ideal.
\end{corollary}

\begin{proposition}
$A \subseteq B$ be an inclusion of integral domains with $A$ normal. Suppose $x\in B$ is integral over $I \subseteq A$. Then, 
\begin{enumerate}
\item $x$ is algebraic over $k= \mathrm{Frac}(A)$
\item If $t^n + a_{n-1}t^{n-1} + \cdots + a_{0} \in k[t]$ is the minimal polynomial of $x$ over $k$, then $a_{i \in \sqrt{I}}$
\end{enumerate}
\end{proposition}

\begin{theorem}
$A\subseteq B$ an inclusion of integral domains with $A$ normal. Let $\pr_{1} \supseteq \cdots \supseteq \pr_{n}$ be a descending chain of ideals in $A$. Let $\mathfrak{q}_{1} \supseteq \cdots \supseteq \mathfrak{q}_{m} (m<n)$ be a descending chain of prime ideals in $B$ such that $\mathfrak{q}_{i} \cap A \pr_{i} \forall \; 1 \le i \le m$. Then, the $\mathfrak{q}$-chain can be extended to a chain of prime ideals $\mathfrak{q}_{1} \supseteq \cdots \supseteq \mathfrak{q}_{m} \supseteq \mathfrak{q}_{m+1} \supseteq \cdots \supseteq \mathfrak{q}_{n}$ such that $q_{i}\cap A = \pr_{i} \; \forall \; 1\le i \le n$. 
\end{theorem}

\begin{theorem}
Let $A$ be a normal domain with $\mathrm{Frac}(A)=k$ and $L/k$ be a finite separable extension. Let $B \subseteq L$ be an integral closure of $A$ in $L$. Then there exists a basis $v_{1}, \hdots , v_{n}$ of $L$ such that $B \subseteq Av_{1} \oplus \cdots \oplus Av_{n}$.
\end{theorem}

\begin{corollary}
If $A$ is Noetherian, then $B$ is Noetherian. In fact, $B$ is finite over $A$.
\end{corollary}

\begin{proposition}
Suppose $A$ is an integral domain integrally closed in its fraction field $K$, $L$ is some finite separable extension of $K$. If $B$ is integral closure of $A$ in $L$, then $B$ is finite $A$-module
\end{proposition}

\begin{corollary}
In the prev. setup $L = \mathrm{Frac}(B)$
\end{corollary}

\begin{remark}
In fact, the corollary is true if $A$ is any domain (not necessarily integrally closed in $K$) and if $L$ is any finite extension of $K$ (not necessarily separable)
\end{remark}

\begin{proposition}
Let $f: A\rightarrow B$ be an integral extension of commutative rings. Then we get $f_{\#}: \spec(B) \rightarrow \spec(A)$ us a closed mapping.
\end{proposition}

\begin{proposition}
Let $f: A \rightarrow B$ be a map of integral domains and an integral extension such that $B$ is Noetherian. Then, $$f_{\#} : \spec(B) \rightarrow \spec(A)$$ is quasi-finite (all fibers are finite)
\end{proposition}


\section{Noether Normalisation Theorem}

\begin{theorem}[Noether Normalisation theorem]
Let $k$ be an infinite field and let $A = k[y_{1}, \hdots , y_{n}]$ be a $k$-algebra (not necessarily a polynomial ring, the $y_{i}$'s may have relations). Then, there exists $x_{1}, \hdots , x_{r} \in A$ which have the following properties:
\begin{enumerate}
\item Each $x_{i}$ is a linear combination of $y_{j}$'s.
\item $x_{i}$'s are algebraically independent over $k$.
\item $A$ is integral over $k[x_{1}, \hdots , x_{r}]$ (a polynomial ring)
\end{enumerate}
\end{theorem}

\begin{remark}
\begin{enumerate}
\item We do not need $k$ to be infinite
\item Since $A$ is integral and finite type over $k[x_{1}, \hdots ,x_{r}]$, then $A$ is actually finite over the polynomial algebra $k[x_{1},\hdots , x_{r}]$
\end{enumerate}
\end{remark}

\begin{exercise}
2,4,5,6,9,11,12,13,14,15 from Atiyah McDonald chapter 5
\end{exercise}

\chapter{Discrete Valuation Rings}

\section{Valuation Rings}

\begin{definition}
A Noetherian local ring $(A, \m)$ is called regular if $\m = \langle f_{1} , \hdots , f_{r} \rangle$ where $r = \dim(A)$
\end{definition}

\begin{definition}
Let $B$ be an integral domain contained in a field $k$. Then $B$ is called a valuation ring of $k$ if for all $x\in k^{\times}$ we have $B \cap \{x,x^{-1}\} \neq \emptyset$
\end{definition}

\begin{proposition}
Let $B$ be a valuation ring of $k$. Then, 
\begin{enumerate}
\item $B$ is local.
\item If $B \hookrightarrow B' \hookrightarrow k$, then $B'$ is also a valuation ring
\item $B$ is integrally closed
\end{enumerate}
\end{proposition}

Let $k$ be a field and $\Omega$ be an algebraically closed field. Then, $\Sigma$ the set of all pairs $(A,f)$ where $A$ is a subring of $k$ and $f: A \rightarrow \Omega$ is a ring homomorphism. \\
We give a partial order to $\Sigma$ by $(A,f) \le (A',f')$ if $A \subseteq A'$ and $f|'_{A} = f$.\\
By Zorn's lemma, $\Sigma$ has a maximal element. Let $(B,f)$ be the maximal element. 

\begin{lemma}
$B$ is local
\end{lemma}

\begin{lemma}
Let $(B,f)$ be as above and let $x \in k^{\times}$. Let $B' = B[x]$ and let $\m[x] = \m B'$ where $\m = \m_{B}$. Then, either $\m[x] \neq B[x]$ or $\m[x^{-1}] \neq B[x^{-1}]$
\end{lemma}

\begin{theorem}
$B$ is a valuation ring.
\end{theorem}

\begin{corollary}
Let $A \hookrightarrow k$ be a subring. Define $\bar{A}$ to the integral closure of $A$ in $k$. Then, $$\bar{A} = \bigcap_{B \subseteq A, \\ B \text{valuation ring of }k}B = \tilde{A}$$
\end{corollary}

\begin{definition}
An integral domain whose localisation at all prime ideals is a valuation ring is called a Pr\"{u}fer domain.
\end{definition}

\begin{definition}
Let $A \hookrightarrow B$ be an inclusion of local rings. We say that $B$ dominates $A (B \geq A)$ if $\m_{B} \cap A = \m_{A} (\m_{A}B \subseteq \m_{B})$ 
\end{definition}

Let $k$ be a field and let $\Sigma'$ be the set of all local subrings of $k$ ordered by domination. Then, Zorn's lemma guarantees the existence of a maximal element. 
\begin{theorem}
Any maximal element of $\Sigma '$ is a valuation ring of $k$.
\end{theorem}

\begin{corollary}
Let $k$ be a field and $A \hookrightarrow k$ be a local ring. Then, $A$ is dominated by a valuation ring of $k$.
\end{corollary}

\begin{remark}
A valuation ring of $k$ is also called a place of $k$. $\spec^*(k) =$ all places of $k$
\end{remark}

\begin{proposition}
Let $A$ be an integral domain with $\mathrm{Frac}(A)=k$. TFAE: 
\begin{enumerate}
\item $A$ is a valuation ring of $k$
\item for all pairs of ideals $I,J \subseteq A$, one has either $I \subseteq J$ or $J \subseteq I$
\end{enumerate}
\end{proposition}

\begin{remark}
$A$ is a valuation ring iff $\mathsf{I}(A)$ the set of all ideals of $A$ is a totally ordered set
\end{remark}

\begin{corollary}
Let $A$ be a valuation ring and $\pr \in \spec(A)$. Then, $A/\pr$ and $A_{\pr}$ are valuation rings.
\end{corollary}

\begin{corollary}
Let $A$ be a valuation ring of $k$. Then, any subring of $k$ containing $A$ must be local. 
\end{corollary}

\section{Totally ordered abelian group}

Let $A$ be valuation ring of a field $k$. Let $U = A^{\times}$ then $ U\hookrightarrow K^{\times}$. Let $\Gamma = K^{\times}/U$ and $\nu: K^{\times} \rightarrow \Gamma$ be quotient map. We define an order on $\Gamma$ as follows: \\
Let $zeta, \eta \in \Gamma$, we say that $\eta \le \zeta$ if there exists $a,b \in K^{\times}$ such that 
\begin{enumerate}
\item $\nu(a)=\zeta, \nu(b)=\eta$
\item $ab^{-1} \in A$
\end{enumerate}

\begin{proposition}
$(\Gamma, \le)$ is a totally ordered abelian group.
\end{proposition}

We have $\nu : K^{\times} \rightarrow \Gamma$ such that $\ker (\nu) = A^{\times}$, and \\
$A = \{a \in K^{\times}\mid \nu(a) \geq 0\}\cup \{0\}$

\begin{exercise}
$\nu(x+y)\geq \min (\nu(x),\nu(y))$ for $x+y \in k^{\times}$
\end{exercise}

\begin{definition}
Let $(\Gamma, \le)$ be a totally ordered abelian group and $k$ be any field. Then a valuation ring is a group homomoprhism $\nu: k^{\times} \rightarrow \Gamma$ such that $\nu(x+y)\geq \min (\nu(x),\nu(y))$ for $x+y \in k^{\times}$ \\

In this case, $\nu(k^{\times})$ is called the value group of $\nu$ \\
$A= \{a \in k^{\times}: \nu(a) \geq 0\} \cup \{0\}$ is called the ring of integers of $\nu$.
\end{definition}

\begin{proposition}
The ring of integers of a valuation on $k$ is a valuation ring of $k$.
\end{proposition}

An integral domain $A$ with $K = \mathrm{Frac}(A)$ is a valuation ring iff there exists a valuation on $K$ whose ring of integers is $A$. \\

We know that valuations on $K \leftrightarrow$ places of the field \\

Set of all valuations on $K = RZ (K) =$ Riemann-Zariski space of $K$

\begin{proposition}
Let $A$ be a valuation ring with residue field $F$ and let $A' \subseteq F$ be a valuation ring of $F$. Let $$R = \{a\in A \mid a \pmod{\m_{A}} \in A'\}$$
Then, $R$ is a valuation ring of $K= \mathrm{Frac}(A)$
\[\begin{tikzcd}
	{R=\phi^{-1}(A)} & A & K \\
	{A'} & F
	\arrow[hook, from=1-1, to=1-2]
	\arrow[hook, from=1-2, to=1-3]
	\arrow[hook, from=2-1, to=2-2]
	\arrow[two heads, from=1-2, to=2-2]
	\arrow[from=1-1, to=2-1]
\end{tikzcd}\]
\end{proposition}

\begin{exercise}
$A'=\mathbb{F}_{q}[[t]], F= \mathbb{F}_{q}((t)), A=A'[[u]], K = \mathbb{F}_{q}((t,u))$
\end{exercise}

\begin{proposition}
Let $A$ be a local integral domain. TFAE: 
\begin{enumerate}
\item $A$ is a valuation ring
\item Every f.g. ideal of $A$ is principal
\end{enumerate}
\end{proposition}
 
\begin{theorem}
\begin{enumerate}
\item Every totally ordered abelian group is the value group of a valuation of field
\item Totally ordered abelian group is torsion free
\item Every torsion free abelian group is a totally ordered abelian group
\item Suppose $A$ is a valuation ring with valuation $\nu : K^{\times} \rightarrow \Gamma$. The rank of $\nu$ is the dimension of $A$.
\end{enumerate}
\end{theorem} 

\begin{theorem}
$\nu$ has rank $1$ iff the corresponding value group is a subgroup of $(\RR, \le)$
\end{theorem}
 
\section{Discrete Valuation Rings}

\begin{definition}
Let $\nu : K^{\times} \twoheadrightarrow \ZZ$ be a valuation with usual order on $\ZZ$. Then, $\nu^{-1}(\ZZ_{\geq 0})$ is called a DVR.
\end{definition}

\begin{theorem}
Let $A$ be an integral domain with function field $K \neq A$. TFAE: 
\begin{enumerate}
\item $A$ is DVR
\item $A$ is a Noetherian valuation ring
\item $A$ is a Noetherian local domain of dimension $1$, which is normal
\item $A$ is Noetherian local domain whose maximal ideal is principal
\item $A$ is Noetherian local domain with maximal ideal $\m$ and residue field $\mathcal{k}$ such that $\dim_{\mathcal{k}}\m / \m^2 = 1$
\item $A$ is Noetherian local domain with maximal ideal $\m$ such that every non-zero ideal is a power of $\m$
\item $A$ is a Noetherian local domain and there exists $x \in A$ such that every non-zero ideal of $A$ is of the form $\langle x^n \rangle$
\end{enumerate}
\end{theorem}


\chapter{Primary decomposition}

Let $A$ be a commutative ring, and $\mathfrak{q} \subsetneq A$ be an ideal 

\begin{definition}
$\mathfrak{q}$ is called a primary ideal if for all $a,b\in A$ such that $ab \in \mathfrak{q}$, we must have either $a \in \mathfrak{q}$ or $b \in \sqrt{\mathfrak{q}}$. \\
Equivalently in $A/\mathfrak{q}$, every non-zero divisor is nilpotent. 
\end{definition}

$A \xrightarrow{f} B$, $\mathfrak{q} \subseteq B$ primary implies $f^{-1}(\mathfrak{q})=\mathfrak{q}^c$ is also primary. \\

Suppose $I$ has a primary decomposition. Then, $I$ has a finite set of associated primes. Furthermore, if $X  \spec(A), X = \V(I)$ is closed, then $Z$ has finitely many irreducible closed subsets $\{V_{1}, \hdots , V_{n}\} = \{\V(\pr_{1}), \hdots , \V(\pr_{r})\}$ where $\{\pr_{1}, \hdots , \pr_{r}\}= \mathrm{Min}(I)$ \\
Suppose $I$ has a primary decomposition, $B= \mathrm{Min}(I) = $ minimal elements in the set $\mathrm{Ass}(I)$

\begin{proposition}
$A=$ minimum elements of $\V(I) = \mathrm{Min}(I)$
\end{proposition}

\begin{proposition}
Assume $I$ has a primary decomposition, $\mathrm{Ass}(I)= \{\pr_{1}, \hdots , \pr_{r}\}$. Then, 
\begin{equation*}
\bigcup_{i=1}^r \pr_{i} = \{x\in A \mid (I:x) \neq I\} = \{x \in A \mid x \pmod \Gamma \text{ is a zero divisor in }A/I\}
\end{equation*}
\end{proposition}

\section{Primary ideals under localisation}

\begin{proposition}
$S\subseteq A$ multiplicatively closed set, $\mathfrak{q}$ a primary ideal with $\sqrt{\mathfrak{q}} =\pr$ 
\begin{enumerate}
\item If $S \cap \pr \neq \emptyset$ then $S^{-1}\mathfrak{q} = S^{-1}A$
\item If $S \cap \pr = \emptyset$ then $S^{-1}\mathfrak{q}$ is a $S^{-1}\pr$ primary ideal and $S^{-1}\mathfrak{q} \cap A = \mathfrak{q}$
\end{enumerate}
\end{proposition}

\begin{proposition}
Let $I = \bigcap \mathfrak{q}_{i}$ be a primary decomposition with $\sqrt{\mathfrak{q}_{i}} = \pr_{i}$. Let $\pr_{1}, \hdots , \pr_{r}$ be numbered so that $\pr_{m+1}, \hdots , \pr_{r}$ meet $S$ and $\pr_{1}, \hdots , \pr_{m}$ do not meet $S$. Then, 
\begin{equation}
S^{-1}I = \bigcap_{i=1}^m S^{-1}\mathfrak{q}_{i} \text{ and } S(I)= \bigcap_{i=1}^m \mathfrak{q}_{i}
\end{equation}
are primary decomposition.
\end{proposition}

\begin{proposition}
Let $\pr_{1}, \hdots , \pr_{s}$ be the minimal primes of $I$. Let $S = A \bs \left( \bigcup_{i=1}^s  \right)$. Then, $S(I) = \mathfrak{q}_{1} \cap \hdots \cap \mathfrak{q}_{s}$
\end{proposition}

Suppose $I$ has a primary decomposition, say 
\begin{align*}
I &= \mathfrak{q}_{1} \cap \cdots \cap \mathfrak{q}_{r} \\
&= (\mathfrak{q}_{1} \cap \cdots \cap \mathfrak{q}_{s}) \cap (q_{s+1} \cap \cdots \cap \mathfrak{q}_{r})
\end{align*}
where the first $s$ ones are minimal primes and $s+1$ to $r$th ones are embedded primes.

\begin{definition}
We say that $I$ is irreducible if $I = J \cap K$, then $I = J$ or $I =K$.
\end{definition}

\begin{theorem}
Suppose $A$ is Noetherian. Then every ideal is finite intersection of irreducible ideals.
\end{theorem}

\begin{proposition}
Every irreducible ideal is primary.
\end{proposition}

\begin{corollary}
In a Noetherian ring, every ideal has a primary decomposition.
\end{corollary}

\chapter{Dedekind domains}

\begin{proposition}
$A$ is Noetherian integral domain of dimension $1$. TFAE: 
\begin{enumerate}
\item $A$ is normal
\item Every primary ideal is a prime power
\item $A_{\pr}$ is a DVR for all non-zero prime ideal $\pr$.
\end{enumerate}
\end{proposition}

\begin{definition}
Let $A$ be a Noetherian local domain of dimension $1$. We say that $A$ is a Dedekind domain if any of the equivalent conditions of the proposition holds.
\end{definition}

\begin{corollary}
In a Dedekind domain, every non-zero ideal has an unique factorisation as a finite product of prime ideals.
\end{corollary}

\begin{remark}
A Dedekind domain is a f.g. algebra over a field is like an affine open subset of a Riemann surface.
\end{remark}

\begin{example}
$A= k [X], \spec(A) = \Aa_{k}^{1} \xlongrightarrow{compactification} \PP^1_{k}$
\end{example}

\begin{proposition}
Let $K$ be a number field and let $A = \mathcal{O}_{K}$. Then, $A$ is a Dedekind domain.
\end{proposition}

\begin{corollary}
Let $A$ be a Dedekind domain and $M$ a torsion free $A$-module. Then, $M$ is flat.
\end{corollary}

\section{Fractional Ideals}

Let $A$ be an integral domain with $\mathrm{Frac}(A)=K$, let $M$ be an $A$-submodule 
\begin{definition}
We say that $M$ is a fractional ideal if there exists $x \in A \bs \{0\}$ such that $xM \subseteq A (M \subseteq Ax^{-1} \subseteq K)$
\end{definition}

If $M$ is a fractional ideal, we write $$(A:M) = \{y \in K\mid yM \subseteq A\}$$
We say that $M \subseteq K$ is an invertible ideal if there exists a submodule $N \subseteq K$ such that $MN=NM=A$. We say that $M$ is principal if $M = Ax^{-1}$ for some $x \neq 0 \in K$.

\begin{proposition}
Suppose $M \subseteq K$ is a f.g. $A$-submodule. Then, $M$ is a fractional ideal.
\end{proposition}

\begin{proposition}
Suppose $A$ is Noetherian. Let $M$ be a fractional ideal of $A$. Then, $M$ is a f.g. $A$-module.
\end{proposition}


In conclusion, if $A$ is Noetherian, then 
\begin{align*}
\{ \text{Fractional ideals } \} &\leftrightarrow \{\text{ f.g. $A$-submodules of $K$}\}
\end{align*}
Suppose $M$ is an invertible ideal, then there exists $N \subseteq K$ submodule such that $MN=A$

\begin{proposition}
$N=(A:M)$
\end{proposition}

\begin{proposition}
An invertible ideal is f.g. and hence fractional.
\end{proposition}

\begin{remark}
The set of invertible ideals form an abelian group under multiplication. This is called the Picard group of $A$, denoted by $\mathrm{Pic}(A)$.
\end{remark}

\begin{proposition}
Let $M$ be a fractional ideal of $A$. TFAE: 
\begin{enumerate}
\item $M$ is invertible
\item $M$ is f.g. and $M_{\pr}$ is invertible for all prime ideals $\pr$
\item $M$ is f.g. and $M_{\m}$ is invertible for all maximal ideals $\m$
\end{enumerate}
\end{proposition}

\begin{proposition}
Assume that $A$ is a local domain. TFAE: 
\begin{enumerate}
\item $A$ is a DVR
\item Every fractional ideal of $A$ is invertible.
\end{enumerate}
\end{proposition}

\begin{theorem}
Let $A$ be an integral domain. Then, $A$ is a Dedekind domain iff every fractional ideal of $A$ is invertible.
\end{theorem}

\begin{corollary}
In a Dedekind domain, the fractional ideals form an abelian group under multiplication whose identity is $A$
\end{corollary}

\begin{corollary}
The group  of fractional ideals is a free abelian group
\end{corollary}

\begin{proposition}
Let $A$ be a Dedekind domain. Then, $A$ is a PID iff the class group of $A$ (Picard group of $A$) is trivial iff $A$ is a UFD.
\end{proposition}

\begin{theorem}
Let $A$ be a number ring. Then, $\mathcal{Cl}(A)$ is finite.
\end{theorem}

\begin{remark}
It is known that every finite abelian group is the class group of a Dedekind domain, but whether it is the class group of a number ring is not yet answered.
\end{remark}

\begin{proposition}
Let $A$ be a Noetherian integral domain. Let $M$ be a f.g. $A$-module. Then, $M$ is projective of rank $1$ iff $M$ is an invertible ideal.
\end{proposition}

\begin{theorem}
Let $A$ be a Dedekind domain and $I \subseteq A$ a non-zero ideal. Then, $I$ is generated by atmost $2$ elements.
\end{theorem}

\begin{exercise}
Prove that a semilocal Dedekind domain is a PID (Show that class group is trivial)
\end{exercise}


\chapter{Completions}

\section{Topological groups}

\begin{definition}
A topological group is a topological space $G$ such that the two maps 
\begin{align*}
G \times G &\rightarrow G && G &\rightarrow G \\
(x,y) &\mapsto xy && x &\mapsto x^{-1}
\end{align*}
are continuous.
\end{definition}

\begin{proposition}
$G$ is Hausdorff iff $\{1\}$ is closed in $G$
\end{proposition}

Let $H$ be the intersection of all open neighbourhoods of $1$
\begin{proposition}
\begin{enumerate}
\item $H$ is a subgroup
\item $H = \overline{\{1\}}$
\item $G/H$ is Hausdorff
\item $G$ is Hausdorff iff $H = \{1\}$
\end{enumerate}
\end{proposition}

\begin{definition}
Let $(a_{n})$ be a sequence in $G$ (written additively). We say that this sequence is Cauchy if for all neighbourhoods $U$ of $0$ there exists $s = s(U) \in \ZZ$ such that $a_{m}-a_{n} \in U \; \forall \; m,n\geq s$. \\

We say that $(a_{n})\sim (b_{n})$ if $a_{n} -b_{n} \rightarrow 0$ as $n \rightarrow \infty$
\end{definition}
Let $\hat{G}$ be the set of all equivalence classes of Cauchy sequences in $G$. Then, $\hat{G}$ is an abelian group and there exists a homomorphism $\phi : G \rightarrow \hat{G}$ such that $a \mapsto (a)$. The Kernel of this map is clearly $H$. We say that $G$ is complete if $G\cong \hat{G}$. \\

Next, assume that $G$ has a filtration by subgroups 
\begin{equation*}
(*) G = G_{0} \supseteq G_{1} \supseteq \cdots 
\end{equation*}
Then $(*)$ defined a unique topology on $G$ such that $(G_{n})$ becomes a system of neighbourhoods of zero (the subspace topology)\\

Next, consider the system $(G/G_{n})$ with $G_{i} \supseteq G_{i+1}$ and a map $$G/G_{n+1} \xlongrightarrow{\gamma_{n+1}} G/G_{n}$$

This clearly gives us an inverse system and thus we have $$\hat{G} = \varprojlim_{n} G/G_{n}$$
\[\begin{tikzcd}
	G & {\widehat{G}} & {\displaystyle{\prod_{n} G/G_{n}}} & {G/G_{n}}
	\arrow["\phi", from=1-1, to=1-2]
	\arrow[hook, from=1-2, to=1-3]
	\arrow[from=1-3, to=1-4]
	\arrow["{\lambda_{n}}"', curve={height=24pt}, from=1-2, to=1-4]
\end{tikzcd}\]

In a topological group, if we quotient by an open subgroup, we get a discrete topology and thus, 
\[\begin{tikzcd}
	G & {\widehat{G}} & {\displaystyle{\prod_{n} G/G_{n}}} & {G/G_{n}}
	\arrow["\phi", from=1-1, to=1-2]
	\arrow[hook, from=1-2, to=1-3]
	\arrow[from=1-3, to=1-4]
	\arrow["\psi"', curve={height=24pt}, from=1-1, to=1-4]
\end{tikzcd}\]
with $\psi$ continuous, which implies $\phi$ is continuous and this concept of $\widehat{G}$ is same as the previous one.

In general, if $I \subseteq R$ is an ideal, we can take the filtration $$R = I_{0} \supseteq I \supseteq I^2 \supseteq \cdots $$

\begin{example}
$R=A[t], I = \langle t \rangle , G_{n} = I^n$ then $\widehat{A[t]} = A[[t]]$
\end{example}

\begin{lemma}
Suppose \[\begin{tikzcd}
	0 & {\{A_{n}\}} & {\{B_{n}\}} & {\{C_{n}\}} & 0
	\arrow[from=1-1, to=1-2]
	\arrow["{\alpha_{n}}"', from=1-2, to=1-3]
	\arrow["{\beta_{n}}"', from=1-3, to=1-4]
	\arrow[from=1-4, to=1-5]
\end{tikzcd}\]
 is an exact sequence of inverse systems. Then, 
 \[\begin{tikzcd}
	0 & {\ds{\varprojlim_{n}A_{n}}} & {\ds{\varprojlim_{n}B_{n}}} & {\ds{\varprojlim_{n}C_{n}}} & 0
	\arrow[tail reversed, no head, from=1-1, to=1-2]
	\arrow["\alpha"', tail reversed, no head, from=1-2, to=1-3]
	\arrow["\beta"', tail reversed, no head, from=1-3, to=1-4]
	\arrow[tail reversed, no head, from=1-4, to=1-5]
\end{tikzcd}\]
 is exact. Furthermore, $\beta$ is injective if $\{A_{n}\}$ is a surjective inverse system.
\end{lemma}

\begin{lemma}
If 
 \[\begin{tikzcd}
	0 & {G'} & {G} & {G''} & 0
	\arrow[from=1-1, to=1-2]
	\arrow["{\alpha}"', from=1-2, to=1-3]
	\arrow["{\beta}"', from=1-3, to=1-4]
	\arrow[from=1-4, to=1-5]
\end{tikzcd}\]
is an exact sequence where $G$ is a topological group and $G$' has subspace topology and $G''$ has quotient topology. Then, 
\[\begin{tikzcd}
	0 & {\widehat{G'}} & {\widehat{G}} & {\widehat{G''}} & 0
	\arrow[from=1-1, to=1-2]
	\arrow["\alpha^*"', from=1-2, to=1-3]
	\arrow["\beta^*"',  from=1-3, to=1-4]
	\arrow[ from=1-4, to=1-5]
\end{tikzcd}\]
is exact.
\end{lemma}

\begin{corollary}
\[\begin{tikzcd}
	0 & {\{G'/G' \cap G_{n}\}} & {\{G/G_{n}\}} & {\{G/G'\}} & 0
	\arrow[from=1-1, to=1-2]
	\arrow[from=1-2, to=1-3]
	\arrow[from=1-3, to=1-4]
	\arrow[from=1-4, to=1-5]
\end{tikzcd}\] is exact.
\end{corollary}

\begin{corollary}
For all $n$, $\widehat{G_{n}}\hookrightarrow \widehat{G}$ and $G/G_{n} \xlongrightarrow{\sim} \widehat{G}/\widehat{G_{n}}$
\end{corollary}

\begin{corollary}
$\widehat{\widehat{G}} \xlongrightarrow{\sim} G$
\end{corollary}

\section{I-adic completion}

Let $A$ be a commutative ring, $I \subseteq A$ an ideal. $\{I^n A\}$ defines the $I$-adic topology on $A$. $\phi : A \rightarrow \widehat{A}$ a map. \\

If $M$ is a $A$-module, then $\{I^n M\}$ defines a topology on $M$ called the $I$-adic topology on $M$ and $$M \longrightarrow \widehat{M}$$ is the $I$-adic completion. 
\begin{itemize}
\item $\widehat{M}$ is a $\widehat{A}$-module.
\item $A/I^n$ acts on $M/I^nM$
\item $\widehat{A} = \ds{\varprojlim_{n} A/I^n}$ acts on $\ds{\varprojlim_{n} M/I^nM}$
\end{itemize}

\begin{definition}
\begin{enumerate}
\item We say that $M$ is a $I$-filtration if $$IM_{n} \subseteq M_{n+1} \; \forall \; n$$
\item We say that $M$ is $I$-stable if $$IM_{n} = M_{n+1} \; \forall \; n >>0$$
\end{enumerate}
\end{definition}

\begin{remark}
$(1) \Rightarrow I^n M \subseteq M_{n} \; \forall \; n \Rightarrow$ $I$-adic topology is finer than the $M$ topology.
\end{remark}

\section{Graded Rings}
\begin{definition}
A graded ring is a commutative ring $A$ with a decomposition of $(A,+)$ $$A = \bigoplus_{m \geq 0}A_{m}$$ such that $A_{m}A_{n} \subseteq A_{m+n}$
\end{definition}
This means $A_{0}$ is a subring of $A$ and $A$ is an $A_{0}$-algebra.\\


\begin{definition}
Suppose $A$ is a graded ring. An $A$-module $M$ is called graded if $$M=\bigoplus_{n \geq 0} M_{n}$$ as abelian group such that $A_{n}M_{m} \subseteq M_{n+m} \; \forall \; n,m$ 
\end{definition}

\begin{definition}
$\phi : M \rightarrow N$ map of graded $A$-modules is an $A$-linear map such that $\phi(M_{m}) \subseteq N_{m} \ ; \forall \; m$
\end{definition}

\section{Projective scheme}

\begin{theorem}
Let $A = \bigoplus_{n} A_{n}$ be a graded ring. TFAE: 
\begin{enumerate}
\item $A$ is Noetherian
\item $A_{0}$ is Noetherian and $A$ is a f.g. $A_{0}$-algebra
\end{enumerate}
\end{theorem}

\section{Rees Algebra}

\begin{definition}
Let $I \subseteq A$ not necessarily graded, then the Rees algebra of I , denoted by $A(I)$ is defined as $$A(I) = \bigoplus_{n \geq 0} I^n$$ such that $I^nI^m \subseteq I^{n+m}$
\end{definition}
Suppose $M$ is an $A$-module with a filtration $M$ which is a $I$-filtration. Then, $$M^{*}  =\bigoplus_{n\geq 0} M_{n}$$ implies $M^*$ becomes an $A(I)$-module.

\begin{lemma}
If $A$ is Noetherian. TFAE:
\begin{enumerate}
\item $\{M\}$ is $I$-stable
\item $M^*$ is a f.g. $A(I)$-stable
\end{enumerate}
\end{lemma}

\begin{theorem}
If $M$ is $I$-stable on $M$, then $\{M \cap M'\}$ is also $I$-stable on $M'$ if $M$ is f.g. $A$-module.
\end{theorem}

\begin{theorem}
If \[\begin{tikzcd}
	0 & {M'} & M & {M''} & 0
	\arrow[from=1-3, to=1-4]
	\arrow[from=1-4, to=1-5]
	\arrow[from=1-2, to=1-3]
	\arrow[from=1-1, to=1-2]
\end{tikzcd}\]
 is an exact sequence of f.g. $A$-modules. Then, 
 \[\begin{tikzcd}
	0 & {\widehat{M'}} & {\widehat{M}} & {\widehat{M''}} & 0
	\arrow[from=1-3, to=1-4]
	\arrow[from=1-4, to=1-5]
	\arrow[from=1-2, to=1-3]
	\arrow[from=1-1, to=1-2]
\end{tikzcd}\]
is exact.
\end{theorem}

\begin{theorem}
\begin{equation}
\widehat{A} \otimes_{A} M \xlongrightarrow{\sim} \widehat{M}
\end{equation}
\end{theorem}

\begin{corollary}
$\widehat{A}$ is a flat $A$-algebra.
\end{corollary}

\begin{theorem}
\begin{enumerate}
\item $\widehat{I} \simeq I\widehat{A} \simeq I \otimes_{A} \widehat{A}$
\item $\widehat{I^n} \simeq (\widehat{I})^n$
\item $\ds{\frac{I^n}{I^{n+1}} \simeq \frac{\widehat{I}^n}{\widehat{I}^{n+1}}}$
\item $\widehat{I} \subseteq J(\widehat{A})$
\end{enumerate}
\end{theorem}

\begin{proposition}
Assume that $(A,\m)$ is local. Then, $(\widehat{A}, \widehat{\m})$ is local and $A/\m \simeq \widehat{A} / \widehat{\m}$
\end{proposition}

\begin{theorem}
Suppose $A$ is Noetherian and $I \subseteq A$ ideal and $M$ a f.g. $A$-module. Then, 
\begin{align*}
\ker(M \rightarrow \widehat{M}) &= \bigcap_{n \geq 1} I^n M \\
&= \{x\in M \mid x \text{ is annihilated by some element of }1 + I\}
\end{align*}
\end{theorem}

\begin{corollary}
Suppose $A$ is an integral domain. Then, $$\bigcap_{n \geq 0} I^n = \{0\}$$
\end{corollary}

\begin{theorem}
If $(A, \m)$ is Noetherian and local. Then, for any f.g. $A$-module $M$, the $\m$-adic topology on $M$ is Hausdorff.
\end{theorem}

\chapter{Dimension Theory}




\part{Algebraic Number Theory}

\chapter{Dedekind Domains}

\chapter{Splitting of primes}

\chapter{Finiteness of class number}

\chapter{Unit theorem}

\chapter{Cyclotomic Fields and Fermat's last theorem}

\chapter{Local Fields}

\chapter{Global Fields}

\chapter{Kronecker Weber}


\chapter{Ad\'{e}les and Id\'{e}les}




\part{Galois Theory}

\chapter{Fundamental theorem of Galois Theory}

\chapter{Infinite Galois Theory}

\chapter{Finite Fields}


\chapter{Cyclotomic Fields}



\part{Representation theory}

\chapter{Introduction}

\chapter{Character theory}

\chapter{Wedderburn theorem}

\chapter{Induced characters}

\chapter{Brauer Induction theorem}



\part{Miscellaneous}

\chapter{Galois representations}

\chapter{Artin $L$-functions}

\chapter{Riemann hypothesis for curves over finite fields}

\chapter{N\`{e}ron models}

\chapter{Nagata-Zariski-Lipman}

The objective is to prove the following theorem.

\begin{theorem}[Nagata-Zariski-Lipman]
Let $(A, \m)$ be a complete Noetherian local ring with $\QQ \subseteq A$. Suppose that $x_{1}, \hdots , x_{r} \in \m$ and $D_{1}, \hdots , D_{r} \in \mathrm{Der}(A)$ are elements satisfying $\det (D_{i}x_{j}) \not \in \m$. Then, 
\begin{enumerate}
\item There exists a subring $C \subseteq A$ such that 
\begin{equation}
A = C[[x_{1}, \hdots , x_{r}]] \simeq C[[X_{1}, \hdots , X_{r}]]
\end{equation}
Therefore $x_{i}$ are analytically independent over $C$ and $A$ is $I$-smooth over $C$ where $I = \sum_{i=1}^{r} Ax_{i}$ and therefore also $\m$-smooth over $C$.
\item If $\mathfrak{g}= \sum_{i=1}^{r} A D_{i}$ is a Lie algebra ($[D_{i},D_{j}] \in \mathfrak{g} \; \forall \; i,j$), then we can take $C$ to be 
\begin{equation}
C=\{a\in A : D_{1}a = D_{2}a  = \cdots = D_{r}a = 0\}
\end{equation}
\end{enumerate}
\end{theorem}

\section{Preliminaries}



\section{Proof}


\end{document}
