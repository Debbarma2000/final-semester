\documentclass[oneside, 12pt, ]{scrbook}

%packages to import
\usepackage[utf8]{inputenc}
\usepackage{amsmath}
\usepackage{amssymb}
\usepackage{amsthm}
\usepackage{float}
\usepackage{minitoc}
\usepackage[width=15cm, height=24cm]{geometry}
\usepackage{hyperref}
\usepackage[usenames,svgnames,dvipsnames]{xcolor}
\usepackage{mathrsfs}
\usepackage{mathtools}
\usepackage{thmtools}
%\usepackage{kpfonts}
\usepackage[lf]{venturis} %% lf option gives lining figures as default; 
			  %% remove option to get oldstyle figures as default
\renewcommand*\familydefault{\sfdefault} %% Only if the base font of the document is to be sans serif
\usepackage{fontenc}
\usepackage{setspace}
\usepackage{cleveref}
\usepackage{backref}
\usepackage{graphicx}
\usepackage{tikz-cd}
\usepackage{quiver}

%section symbol


%no indentation paragraph
\setlength{\parindent}{0pt}


% color those links
\hypersetup{
colorlinks=true,
urlcolor= BrickRed,
linkcolor= BrickRed,
citecolor= Cerulean
}
%colour page black and text white
%\pagecolor{gray}
%\color{white}

%\usepackage{multicol}
\usepackage[utf8]{inputenc}

%custom commands
\newtheorem{statement}{Statement}
\newcommand{\CC}{\mathbb C}
\newcommand{\FF}{\mathbb F}
\newcommand{\NN}{\mathbb N}
\newcommand{\QQ}{\mathbb Q}
\newcommand{\RR}{\mathbb R}
\newcommand{\ZZ}{\mathbb Z}
\newcommand{\p}{\mathfrak{p}}
\newcommand{\V}{\mathcal{V}}
\newcommand{\I}{\mathcal{I}}
\newcommand{\spec}{\mathrm{Spec}}
\newcommand{\Aa}{\mathbb{A}}
\newcommand{\PP}{\mathbb{P}}
\newcommand{\im}{\mathrm{im}}
\newcommand{\pr}{\mathfrak{p}}
\newcommand{\m}{\mathfrak{m}}
\newcommand{\nil}{\mathrm{Nil}}

%theoremstyles 
\usepackage[framemethod=TikZ]{mdframed}

%theorem
\mdfdefinestyle{mdthmbox}{
	linewidth= 3pt,
	rightline=false,
	leftline=true,
	topline=false,
	bottomline=false,
	linecolor= WildStrawberry,
	backgroundcolor= CarnationPink!25 
}

\declaretheoremstyle[
	spaceabove=6pt, spacebelow=6pt,
	headfont=\normalfont \bfseries,
	notefont=\mdseries, notebraces={(}{)},
	bodyfont=\normalfont,
	postheadspace = \newline,
	mdframed = {style= mdthmbox}
]{theobox}

\declaretheorem[style=theobox,name=Theorem,numberwithin=section]{theorem}
\declaretheorem[style=theobox,name=Conjecture,numberwithin=section]{conjecture}

%lemma, proposition, corollary
\mdfdefinestyle{mdlembox}{
	linewidth= 3pt,
	rightline=false,
	leftline=true,
	topline=false,
	bottomline=false,
	linecolor= RoyalBlue,
	backgroundcolor= SkyBlue!30
}

\declaretheoremstyle[
	postheadspace= \newline,
	spaceabove=6pt, spacebelow=6pt,
	headfont=\normalfont \bfseries,
	notefont=\mdseries, notebraces={(}{)},
	bodyfont=\normalfont,
	postheadspace = \newline,
	mdframed = {style= mdlembox}
]{lembox}
\declaretheorem[style=lembox,name=Lemma,sibling=theorem]{lemma}
\declaretheorem[style=lembox,name=Proposition,sibling=theorem]{proposition}
\declaretheorem[style=lembox,name=Corollary,sibling=theorem]{corollary}

% definition
\mdfdefinestyle{mddefbox}{
	linewidth= 3pt,
	rightline=false,
	leftline=true,
	topline=false,
	bottomline=false,
	linecolor= LimeGreen,
	backgroundcolor= GreenYellow!25 
}

\declaretheoremstyle[
	postheadspace= \newline,
	spaceabove=6pt, spacebelow=6pt,
	headfont=\normalfont \bfseries,
	notefont=\mdseries, notebraces={(}{)},
	bodyfont=\normalfont,
	postheadspace = \newline,
	mdframed = {style= mddefbox}
]{defbox}


\declaretheorem[style=defbox, name= Definition, sibling=theorem]{definition}

\mdfdefinestyle{mdrembox}{
	linewidth= 2pt,
	rightline=false,
	leftline=true,
	topline=false,
	bottomline=false,
	linecolor= black,
}

\declaretheoremstyle[
	postheadspace= \newline,
	spaceabove=6pt, spacebelow=6pt,
	headfont=\normalfont \bfseries,
	notefont=\mdseries, notebraces={(}{)},
	bodyfont=\normalfont,
	postheadspace = \newline,
	mdframed = {style= mdrembox}
]{rembox}

\declaretheorem[name=Remark,sibling=theorem,style=rembox]{remark}
\declaretheorem[name=Note,sibling=theorem,style=rembox]{note}


% example
\mdfdefinestyle{mdredbox}{
	skipabove=8pt,
	linewidth=3pt,
	rightline=false,
	leftline=true,
	topline=false,
	bottomline=false,
	linecolor=red,
	backgroundcolor=Salmon!5
}

\declaretheoremstyle[
	postheadspace= \newline,
	spaceabove=6pt, spacebelow=6pt,
	headfont=\normalfont \bfseries,
	notefont=\mdseries, notebraces={(}{)},
	bodyfont=\normalfont,
	postheadspace = \newline,
	mdframed = {style= mdredbox}
]{thmredbox}

\mdfdefinestyle{mdgreybox}{
	skipabove=8pt,
	linewidth=3pt,
	rightline=false,
	leftline=true,
	topline=false,
	bottomline=false,
	linecolor=Goldenrod,
	backgroundcolor=Grey!5
}

\declaretheoremstyle[
	postheadspace= \newline,
	spaceabove=6pt, spacebelow=6pt,
	headfont=\normalfont \bfseries,
	notefont=\mdseries, notebraces={(}{)},
	bodyfont=\normalfont,
	postheadspace = \newline,
	mdframed = {style= mdredbox}
]{thmredbox}

\theoremstyle{theorem}
\declaretheorem[name= Exercise, style= thmgreybox, sibling = theorem]{exercise}
\declaretheorem[name= Example, style= thmredbox, sibling = theorem]{example}

%citations
%\usepackage[
%backend=biber,
%style=alphabetic,
%sorting=nyt
%]{biblatex}
%\addbibresource{citations.bib}

\begin{document}

%titlepage

\begin{center}
\begin{minipage}{0.75\linewidth}
    \centering
%University logo
    \includegraphics[width=1\linewidth]{IISc_logo.png}
    \rule{0.4\linewidth}{0.15\linewidth}\par
    \vspace{2cm}
%Thesis title
    {\uppercase{\Large Semester notes\par  }}
    \vspace{2cm}
%Author's name
    {\Large Irish Debbarma\par}
    \vspace{2cm}
%Degree
    {\Large Department of Mathematics \\ Indian Institute of Science, Bangalore\par}
    \vspace{1cm}
%Date
    {\Large December 2022}
\pagenumbering{gobble}   
\end{minipage}
\end{center}
\clearpage

\frontmatter
\tableofcontents

\mainmatter

\part{Modular Forms}

\chapter{Lecture-1: }

\chapter{Lecture-2: }

\chapter{Lecture-3 (10 January, 2023): Valence formula and Eisenstein series}

\section{Valence formula}

Recall that $M_{k}(\Gamma_{1})$ is the space of modular forms of weight $k$ and level $1$. It is also a vector space over $\CC$. 

\begin{theorem}
$\dim M_{k}(\Gamma_{1}) = \begin{cases}[k/12]+1 & k\not \equiv \pmod{12} \\ [k/12] & k \equiv \pmod{12} \end{cases}$
\end{theorem}

\begin{proposition}
Let $f \in M_{k}(\Gamma_{1})$. Then, $$\sum_{p \in \Gamma_{1}\backslash \mathbb{H}}\frac{1}{n_{p}} \mathrm{ord}_{p}(f) + \mathrm{ord}_{\infty}(f) = \frac{k}{12}$$
\end{proposition}

\begin{proof}
Let $\epsilon >0$ be "small enough". Remove $\epsilon$-balls around $\infty , i , \omega , \omega + 1$ in $\mathcal{F}_{1}$. $\epsilon$ is small enough so that the removed balls are disjoint. Truncate $\mathcal{F}_{1}$ at the line $y = \epsilon^{-1}$ and call the enclosed region $D$. \\
By Cauchy's theorem $$\int_{\partial D} d(\log f(z)) = 0$$ This integral on the two vertical strips (just the straight lines not the semicircle part) is $0$ since the contribution of left is same as right but orientation is different. On the segment joining $-1/2 + iY , 1/2 + iY$, the integral is $2 \pi i \,\mathrm{ord}_{\infty}(f)$. Again,integral around each removed point in $\mathcal{F}_{1}$ is $\frac{1}{n_{p}} \mathrm{ord}_{p}(f)$. Next, divide the bottom arc into left and right parts and observe that $$d(\log f(S\cdot z)) = d(\log f(z)) + k \frac{dz}{z}$$ $$\int_{C} d(\log f(z)) = \frac{k \pi i }{6}$$ 
\end{proof}

\begin{corollary}
$\dim M_{k}(\Gamma_{1}) = \begin{cases} 0 & k < 0 \\ 0 & k \text{ is odd } \\ 1 & k=0 \\ \begin{cases}[k/12]+1 & k\not \equiv \pmod{12} \\ [k/12] & k \equiv \pmod{12} \end{cases} \end{cases}$
\end{corollary}

\begin{proof}
\begin{itemize}
\item If $k <0$, then $f$ has poles but is holomorphic.
\item  If $k=0$, then $f$ is the constant function. 
\item We have seen
\item For $m = [k/12] + 1$ let $f_{1}, \hdots , f_{m+1} \in M_{k}(\Gamma_{1})$. Let $P_{1}, \hdots , P_{m}$ be any points on $\mathcal{F}_{1}$ not equal to $i , \omega , \omega + 1$ and consider $(f_{i}(P_{j}))_{i \in [m+1], j \in [m]}$. \\
There exists a linear combination $f = \sum_{i=1}^{m+1} c_{i}f_{i}$ not all $c_{i}$ being zero, such that $f(P_{j})=0$ for $1 \le j \le m$. \\
From the previous theorem we get $f \equiv 0$ and this implies $\{f_{i}\}$ is linearly independent and thus $\dim_{\CC} M_{k}(\Gamma_{1}) \le m$. \\
For $k \equiv 2 \pmod{12}$, the relation in previous theorem holds only if there is atleast a simple zero at $p=i$ and atleast a double zero at $p = \omega$. This gives $$\frac{k}{12} - \frac{7}{6} = m-1$$ Repeat the argument above.  
\end{itemize}
\end{proof}

A slight notation. For $\gamma  = \begin{pmatrix}
a & b \\ c & d
\end{pmatrix} \in \mathrm{SL}_{2} (\ZZ)$ we set $f|_{\gamma} (z) = (cz+d)^{-k}f(\gamma \cdot z)$. \\
Thus, $1|_{\gamma}(z) = (cz+d)^{-k}$. If $1|_{\gamma}(z) =1 \Rightarrow c =0$. Conversely, if $c=0$, then $d^{-k}=1$. So, $1|_{\gamma}(z)=1 \Leftrightarrow c=0$. \\
$\Gamma_{\infty} = \left\{ \begin{pmatrix}
a & b \\ 0 & d 
\end{pmatrix} \in \mathrm{SL}_{2}(\ZZ) \right\} = \mathrm{stab}(\infty)$

\section{Eisenstein series}

\begin{definition}
The Eisenstein series $E_{k}(z)$ is defined to be $$E_{k}(z) = \sum_{\gamma \in \Gamma_{\infty} \backslash \Gamma_{1}} 1|_{\gamma}(z)$$
\end{definition}

\begin{proposition}
$$E_{k}(z) = \frac{1}{2} \sum_{(c,d) \in \ZZ^{2} \backslash \{(0,0)\}, \gcd(c,d)=1} \frac{1}{(cz+d)^k}$$
\end{proposition}

\begin{proof}

\end{proof}

\begin{proposition}
$$\sum_{(c,d) \in \ZZ^{2} \backslash \{(0,0)\}, \gcd(c,d)=1} \frac{1}{(cz+d)^k}$$ converges absolutely for $k>2$
\end{proposition}

\begin{proof}

\end{proof}

\begin{theorem}
$E_{k}(z) \in M_{k}(\Gamma_{1})$ for $k >2$.
\end{theorem}

\begin{proof}

\end{proof}

\begin{proposition}
$E_{k}(z) \not \equiv 0$ for $k>2$, even.
\end{proposition}

\begin{proof}
Observe that $$\frac{1}{(cz+d)^k} \rightarrow 0 , \mathfrak{I}(z) \rightarrow \infty , c\neq 0$$ and if $c = 0$, then $c=\pm 1$. Hence, $E_{k}(z) = 1 +$ bounded term as $\mathfrak{I}(z) \rightarrow \infty$. This implies $E_{k}(z) \not \equiv 0$ and $$E_{k}(z) = 1 + \sum_{n=1}^{\infty} a_{n}e^{2 \pi i z}$$
\end{proof}

Another way of looking at Eisenstein series is a function on a lattice. \\
Consider $G_{k}(z) = G_{k}(\ZZ z + \ZZ) = \frac{1}{2} \sum_{(c,d) \in \ZZ^2}^{'} \frac{1}{(cz+d)^k}$

\begin{proposition}
$G_{k}(z)$ converges absolutely for $k>2$.
\end{proposition} 

\begin{proposition}
$G_{k}(z) = \zeta(k) E_{k}(z)$
\end{proposition}

\begin{proposition}
$\mathbb{G}_{k}(z) = \frac{(k-1)!}{(2 \pi i)^k} G_{k}(z) = -\frac{B_{k}}{2k} + \sum_{n=1}^{\infty} \sigma_{k-1}(n)q^n$ for $k>2$, even.
\end{proposition}

\chapter{Lecture-4 (12th January, 2023): Eisenstein series}

\section{Eisenstein series contd..}

Recall that $$M_{*}(\Gamma_{1}) = \bigoplus_{k \in \ZZ}M_{k}(\Gamma_{1})$$  is a graded ring.

\begin{proposition}
The graded ring $M_{*}(\Gamma_{1})$ is freely generated by $E_{4},E_{6}$. This means that the map 
\begin{align*}
f: C[X,Y] &\rightarrow M_{*}(\Gamma_{1})\\
X &\mapsto E_{4} \\
Y &\mapsto E_{6}
\end{align*}
is an isomorphism of graded rings. Here, $\deg X = 4, \deg Y=6$. 
\end{proposition}

\begin{proof}
We want to show that $E_{4}$ and $E_{6}$ are algebraically independent. We start by showing that $E_{4}^3$ and $E_{6}^2$ are linearly independent over $\CC$. Suppose $E_{6}(z)^2 = \lambda E_{4}(z)^{3}$. Consider $f(z) = E_{6}(z)/ E_{4}(z)$. Now observe that $f(z)^2 = \lambda E_{4}(z)$. This means that $f^2$ is holomorphic and thus $f$ is also holomorphic. But $f$ is weakly modular of weight $2$ which is a contradiction. So, our claim is proven. \\

\textbf{Claim}: Let $f_{1},f_{2}$ be two nonzero modular forms of same weight. If $f_{1},f_{2}$ are linearly independent, then they are algebraically independent as well. \\

Let $P(t_{1},t_{2}) \in \CC[t_{1},t_{2}] \backslash \{0\}$ be such that $P(f_{1},f_{2})=0$. Let $P_{d}(t_{1},t_{2})$ be the $d$ degree parts of $P$. Using the fact that modular forms of different weights are linearly independent, we get that $P_{d}(f_{1},f_{2})=0 \; \forall \; d \geq 0$. If $p_{d}(t_{1}/t_{2}) = P_{d}(t_{1},t_{2})/t_{2}^d$, then $p_{d}(f_{1}/f_{2})=0$. But this means that $f_{1}/f_{2}$ is a constant. But, $f_{1},f_{2}$ are linearly independent which implies that they are algebraically independent as well. \\

All of this implies that $E_{4},E_{6}$ are algebraically independent. Using
\end{proof}

\begin{corollary}
$\dim_{\CC} M_{k} (\Gamma_{1}) = \begin{cases}[k/12]+1 & k\not \equiv \pmod{12} \\ [k/12] & k \equiv \pmod{12} \end{cases} $
\end{corollary}

\subsection{Fourier expansions of $E_{k}(z)$}

\begin{proposition}
$$\mathbb{G}_{k}(z) = \frac{(k-1)!}{(2 \pi i)^k} G_{k}(z) = -\frac{B_{k}}{2k} + \sum_{n=1}^{\infty} \sigma_{k-1}(n)q^n$$ for $k>2$, even and $B_{k}$ are Bernoulli numbers.
\end{proposition}

\begin{proof}
Use $$\frac{\pi}{\tan \pi z} = \sum_{n \in \ZZ} \frac{1}{z+n} = \lim_{M,N \rightarrow \infty , N-M < \infty} \sum_{-M}^{N} \frac{1}{z+n}$$ and 
$$\frac{\pi}{\tan \pi z} = \frac{\pi \cos \pi z}{\sin \pi z} = \pi i \frac{e^{\pi i z} + e^{-\pi i z}}{e^{\pi i z} - e^{-\pi i z}} = - \pi i \frac{1+q}{1-q} = - 2 \pi i \left( \frac{1}{2} + \sum_{r=1}^{\infty} q^r \right)$$ This leads to the equality 
$$\sum_{n\in \ZZ}\frac{1}{z+n} = -  2 \pi i \left( \frac{1}{2} + \sum_{r=1}^{\infty} q^r \right)$$
Differentiate both sides of equality $k-1$ times and divide by $(k-1)!$ to get $$\sum_{n \in \ZZ}\frac{1}{(z+n)^k} = \frac{(-2 \pi i)^k}{(k-1)!} \sum_{r=1}^{\infty} r^{k-1}q^r$$
Next, if we look at 
\begin{align*}
G_{k}(z) &= \frac{1}{2} \sum' \frac{1}{(mz+n)^k} \\
&= \frac{1}{2} \sum_{n \in \ZZ , n \neq 0} \frac{1}{n^k} + \frac{1}{2} \sum_{(m,n) \in \ZZ^2 , m \neq 0} \frac{1}{(mz + n)^k} \\
&= \zeta(k) + \sum_{m=1}^{\infty} \sum_{n=-\infty}^{\infty} \frac{1}{(mz+n)^k} \\
&= \zeta(k) + \frac{(2 \pi i)^k}{(k-1)!} \sum_{m=1}^{\infty} \sum_{r=1}^{\infty} r^{k-1} q^{mr} \\
&= \zeta(k) + \frac{(2 \pi i)^k}{(k-1)!} \sum_{m=1}^{\infty} \sum_{r=1}^{\infty} \sigma_{k-1}(n) q^{n}
\end{align*}
The expression of $\mathbb{G}_{k}(z)$ is trivial after noting $$\frac{(k-1)!}{(2 \pi i)^k} \zeta(k) = B_{k}$$
\end{proof} 

\begin{remark}
\begin{enumerate}
\item $\mathbb{G}_{4}(z) = \frac{1}{240} + q + 9q^2 + 28q^3 + 73q^4 + \cdots $
\item $\mathbb{G}_{6}(z) = -\frac{1}{504} + q + 33q^2 + 244q^3 + \cdots $
\item $\mathbb{G}_{8}(z) = \frac{1}{480} + q + 129q^2 + 2188q^3 + \cdots $
\end{enumerate}
\end{remark}

\begin{proposition}
$$\sum_{m=1}^{n-1} \sigma_{3}(m) \sigma_{3}(n-m) = \frac{\sigma_{7}(n) - \sigma_{3}(n)}{120}$$
\end{proposition}

\begin{proof}

\end{proof}

\subsection{Weight $2$ Eisenstein series}

\begin{definition}
\begin{align*}
\mathbb{G}_{2}(z) &= -\frac{1}{24} + \sum_{n=1}^{\infty} \sigma_{1}(n)q^n \\
&= -\frac{1}{24} + q + 3q^2 + 4q^3 + 7q^4 + \cdots 
\end{align*} 
\end{definition}

This converges rapidly on $\mathbb{H}$ and defines a holomorphic function. 

\begin{proposition}
$$G_{2}(z) = - 4 \pi^2 \mathbb{G}_{2}(z)$$
\end{proposition}

\begin{proof}
Since we know that $$G_{2}(z) = \sum_{(m,n) \in \ZZ^2 \backslash \{(0,0)\}}\frac{1}{(mz+n)^2}$$ does not converge absolutely, we define $$G_{2}(z) = \frac{1}{2} \sum_{n \in \ZZ , n \neq 0} \frac{1}{n^2} + \frac{1}{2} \sum_{m \neq 0} \sum_{n \in \ZZ} \frac{1}{(mz+n)^2}$$
This sum converges absolutely and we can show that this satisfies the functional equation as required.
\end{proof}

\begin{proposition}
For $\gamma = \begin{pmatrix}
a & b \\ c & d 
\end{pmatrix} \in \mathrm{SL}_{2}(\ZZ)$ we have $$G_{2}\left( \frac{az+b}{cz+d} \right) = (cz+d)^2 G_{2}(z) - \pi i c(cz+d)$$
\end{proposition}

$G_{2}$ is called a quasi modular form. \\

Introduce (due to Hecke): $$G_{2,s}(z) = \frac{1}{2} \sum_{(m,n)\in \ZZ^2 \backslash \{(0,0)\}} \frac{1}{(mz+n)^2 |mz+n|^{2s}} , \mathfrak{R}(s) >0$$

\section{Modular forms of higher level}

Let $N \in \ZZ_{\geq 1}$ 

$$\mathrm{SL}_{2}(\ZZ / N \ZZ) =\left\{ \begin{pmatrix}
a & b \\ c & d
\end{pmatrix} \in M_{2}(\ZZ / N \ZZ) \mid ad-bc \equiv 1 \pmod{N} \right\}$$

\begin{lemma}
The map 
\begin{align*}
\mathrm{SL}_{2} &\rightarrow \mathrm{SL}_{2}(\ZZ / N \ZZ) \\
\begin{pmatrix}
a & b \\ c & d
\end{pmatrix}  &\mapsto \begin{pmatrix}
\bar{a} & \bar{b} \\ \bar{c} & \bar{d}
\end{pmatrix}
\end{align*}
is a group homomorphism.
\end{lemma}

\begin{definition}
$$\Gamma (N) = \ker (\mathrm{SL}_{2} \rightarrow \mathrm{SL}_{2}(\ZZ / N \ZZ))$$ is called the principal congruence subgroup.
\end{definition}

\begin{definition}
A subgroup $\Gamma$ of $\mathrm{SL}_{2}(\ZZ)$ is called a congruence subgroup if there exists $N$ such that $\Gamma (N) \subseteq \Gamma$.
\end{definition}

$$\Gamma_{0}(N) =\left\{ \begin{pmatrix}
a & b \\ c & d
\end{pmatrix} \in M_{2}(\ZZ / N \ZZ) \mid c\equiv 0 \pmod{N} \right\}$$
$$\Gamma_{1}(N) =\left\{ \begin{pmatrix}
a & b \\ c & d
\end{pmatrix} \in M_{2}(\ZZ / N \ZZ) \mid c\equiv d \equiv 1 \pmod{N} \right\}$$








\part{Elliptic Curves}

\chapter{Lecture-1:}

\chapter{Lecture-2:}

\chapter{Lecture-3 (10 January, 2023): Projective varieties}

\section{Projective varieties}

\begin{definition}
A Projective $n$-space over $k$ denoted by $\mathbb{P}^n$ or $\mathbb{P}^{n}(\bar{K})$ is the set $\mathbb{A}^{n+1} \backslash \{(0,\hdots , 0)\} / \sim$ with  
$$(x_{0}, \hdots ,x_{n}) \sim (y_{0},\hdots ,y_{n})$$ iff $\exists \lambda \in \bar{k}^{\times}$ such that $(y_{0},\hdots ,y_{n}) = (\lambda x_{0},\hdots ,\lambda x_{n})$\\
The equivalence class $(x_{0},\hdots ,x_{n+1})$ is denoted by $[x_{0},\hdots ,x_{n}]$\\
The set of $k$-rational points of $\mathbb{P}^n$ is $$\mathbb{P}^n = \{[x_{0}, \hdots , x_{n}]\; \mid \; x_{i} \in k\}$$
\end{definition}

\textcolor{red}{Caution}: If $p = [x_{0},\hdots ,x_{n}] \in \mathbb{P}^n(k)$ and $x_{i} \neq 0$ for some $i$, then $x_{j}/x_{i} \in k \forall j$

\begin{definition}
Let $p = [x_{0},\hdots ,x_{n}] \in \mathbb{P}^n(\bar{k})$. The minimal field of definition for $p$ is the field $$k(p) = k(x_{0}/x_{i},\hdots ,x_{n}/x_{i}) \text{ for any } i \text{ such that } x_{i} \neq 0$$
$k(p) \frac{x_{i}}{x_{j}} = k(x_{0}/x_{j},\hdots ,x_{n}/x_{j})$ is the same as $k(p)$ as $x_{i}/x_{j} \in k(p)$
\end{definition}  

For $\sigma \in G(\bar{k}/k)$ and $p = [x_{0},\hdots ,x_{n}] \in \mathbb{P}^n$, we have the following action $$\sigma(p) = [\sigma(x_{0}),\hdots ,\sigma(x_{n})]$$
This action is well defined as $$\sigma(\lambda p) = [\sigma(\lambda)\sigma(x_{0}),\hdots ,\sigma(\lambda) \sigma(x_{n})] \sim [\sigma(x_{0}),\hdots ,\sigma(x_{n})]$$

\begin{definition}
A polynomial $f \in \bar{k}[X_{0}, \hdots , X_{n}]$ is homogenous of degree $d$ if $$f(\lambda x_{0} , \hdots , \lambda x_{n})  = \lambda^d f(x_{0} , \hdots ,x_{n}) \forall \lambda \in \bar{k}$$
\end{definition}

\begin{definition}
An ideal $I \subseteq \bar{k}[X_{0} , \hdots , X_{n}]$ is called a homogenous ideal if it is generated by homogenous polynomial.  
\end{definition}

\begin{definition}
Let $I \subseteq \bar{k}[X_{0} , \hdots , X_{n}]$ be a homogenous ideal. Then, $$V(I) = \{p \in \mathbb{P}^n (\bar{k}) \mid f(p)=0 \; \forall \; f\in I\}$$
\end{definition}

\begin{definition}
\begin{itemize}
\item A projective algebraic set is any set of the form $V(I)$ for some homogenous ideal $I$. 
\item If $V$ is a projective algebraic set, the homogenous ideal of $V$, denoted by $I(V)$ is the ideal of $\bar{k}[X_{0} \hdots , X_{n}]$ generated by $\{f \in \bar{k}[X_{0} \hdots , X_{n}]\mid f \text{ is homogenous and }f(p)= 0 \; \forall \; p\in V\}$
\item Such a $V$ is defined over $k$, denoted by $V/k$ if its ideal $I(V)$ can be generated by homogenous polynomials $k[X_{0} \hdots , X_{n}]$. 
\item If $V$ is defined over $k$, then the set of $k$-rational points of $V$ is $$V(k) = V \cap \mathbb{P}^{n}(k) = \{p \in V \mid \sigma(p)=p \; \forall \; \sigma \in G(\bar{k}/k)\}$$
\end{itemize}
\end{definition}

\begin{example}
A line in $\mathbb{P}^2$ is given by the equation $aX + bY + cZ=0$ with $a,b,c \in \bar{k}$ and not all $0$ simultaneously. \\
If $c\neq 0$, then such a line is defined over a field containing $a/c, b/c$. \\
More generally, a hyperplane in $\mathbb{P}^n$ is given by an equation $a_{0}X_{0} + \cdots + a_{n}X_{n} = 0$ with all $a_{i} \neq 0$ simultaneously.
\end{example}

\begin{example}
Let $V$ be the projective algebraic set in $\mathbb{P}^2$ given by $X^2 + Y^2 = Z^2$. 
\begin{align*}
\mathbb{P}^1 & \overbrace{\rightarrow}^{\sim} V \\
[s,t] & \mapsto [s^2 - t^2 : 2st : s^2+t^2]
\end{align*}
\end{example}

\begin{remark}
For $p \in \mathbb{P}^n(\QQ)$ you can clear the denominators and then divide by common factor so that $x_{i} \in \ZZ$ and $\gcd(x_{0} , \hdots ,x_{n})=1$. So, $I =(f_{1}, \hdots , f_{m})$ and finding a rational point of $V_{I}$ is same as finding coprime integer solutions to $f_{i}'s$. 
\end{remark}

\begin{example}
$V \subseteq \mathbb{P}^2$ such that $X^2 + Y^2 = 3Z^2$ over $\QQ$. To find $V(\QQ)$, we just need to find integers $a,b,c$ such that $a^2 +b^2 = 3c^2$
\end{example}

\begin{example}
$V : 3X^3 + 4Y^3 + 5Z^3 = 0$. $V(\QQ) = \emptyset$ but for all prime $p$ we have $V(\QQ_{p}) \neq \emptyset$
\end{example}

\begin{definition}
A projective algebraic set is called a projective variety if its homogenous ideal $I(V)$ is prime $\bar{k}[X_{0}, \hdots , X_{n}]$
\end{definition}

Relation between affine and projective varieties: \\

For $ 0 \le i \le n$
\begin{center}
\begin{align*}
\phi_{i}: \mathbb{A}^n &\rightarrow \mathbb{P}^n \\
(Y_{1}, \hdots , Y_{n}) &\mapsto [Y_{1}, \hdots , Y_{i-1}, 1, Y_{i+1} , \hdots , Y_{n}]
\end{align*}
\end{center}
$\mathrm{Im}(\phi) = U_{i} = \{p \in \mathbb{P}^n \mid \; p =[x_{0}: \hdots  : x_{n}] \text{ with } x_{i} \neq 0\} = \mathbb{P}^n \backslash H_{i}$. \\
This process can also be reversed by the following map : 
\begin{center}
\begin{align*}
\phi_{i}^{-1}: U_{i} &\rightarrow \mathbb{A}^n \\
[x_{0}: \hdots : x_{n}) &\mapsto [x_{0}/x_{i}, \hdots , x_{i-1}/x_{i}, x_{i+1}/x_{i} , \hdots , x_{n}/x_{i}]
\end{align*}
\end{center}

Let $V$ be a projective algebraic set with homogenous ideal $I(V) \subseteq \bar{k}[X_{0} ,\hdots , X_{n}]$. Then, $$V \cap \mathbb{A}^n = \phi_{i}^{-1}(V \cap U_{i}) \text{ for fixed } i $$ is an affine algebraic set with $I(V \cap \mathbb{A}^n ) \subset \bar{k}[X_{0} , \hdots , X_{i-1},X_{i+1}, \hdots , X_{n}]$

\begin{definition}
Let $V \subseteq \mathbb{A}^n$ be an affine algebraic set with ideal $I(V)$ and consider $V \subseteq \mathbb{P}^n$ and $\phi_{i}$ defined as before. \\
The projective closure of $V$ is $\bar{V}$ is the projective algebraic set whose homogenous ideal $I(V)$ is generated by $\{f^* \mid f \in I(V)\}$. \\
Here, for $f \in k[X_{0}, \hdots , X_{i-1}, X_{i+1}, \hdots , X_{n}]$ we define $$f^* (X_{0} , \hdots , X_{n}) = X_{i}^d (f(X_{0}/X_{i}, \hdots , X_{i-1}/X_{i}, X_{i+1}/X_{i}, \hdots , X_{n}/X_{i}))$$ with $d = \deg (f)$.
\end{definition}

\begin{definition}
Dehomogenization of $f(X_{0}, \hdots , X_{n})$ with respect to $i$ is $f(X_{0}, \hdots , X_{i-1}, 1 , X_{i+1}, \hdots ,X_{n})$
\end{definition}

\begin{proposition}
\begin{enumerate}
\item Let $V$ be an affine variety. Then $\bar{V}$ is a projective variety and $V = \bar{V} \cap \mathbb{A}^n$. 
\item Let $V$ be a projective variety. Then, $V \cap \mathbb{A}^n$ is an affine variety and either $V \cap \mathbb{A}^n = \emptyset$ or $V = \overline{V \cap \mathbb{A}^n}$. 
\item If an affine (resp. projective) variety $V$ is defined over $k$, then $\bar{V}$ (resp. $V \cap \mathbb{A}^n$) is also defined over $k$.
\end{enumerate}
\end{proposition}

\begin{proof}
\begin{enumerate}
\item 

\item

\item
\end{enumerate}
\end{proof}

\begin{example}
$V : Y^2 = X^3 + 17 \subseteq \mathbb{A}^2 \rightarrow \mathbb{P}^2$ with $(X,Y) \mapsto [X: Y: 1]$. Here, $\overline{V} : Y^2Z =X^3 +17Z^3$ and $\overline{V} \backslash V = \{[0:1:0]\}$
\end{example}

\chapter{Lecture-4 (12th January, 2023): }




\part{Basic Algebraic Geometry}

\chapter{Lecture-1:}

\chapter{Lecture-2 (10 January, 2023):}

\section{Ideals}
For $I,J$ ideals $$I+J = \{x+y \mid x\in I, y\in J\}$$ $$IJ = \{\sum x_{i}y_{i} \mid x_{i} \in I, y_{i} \in J\}$$

\begin{itemize}
\item $IJ \subseteq I \cap J$.
\item If $I+J = R$, then $I^2 + J^2 = R$. This is because, say $I^2 + J^2 \neq R$, then there is a maximal ideal $\mathrm{m}$ such that $I^2 + J^2 \subseteq \mathfrak{m}$. This means $I^2, J^2 \subseteq \mathfrak{m}$. But $\mathfrak{m}$ is prime ideal, therefore $I,J \subseteq \mathfrak{m} \Rightarrow I+J \subseteq \mathfrak{m}$ which is a contradiction. Thus, we are done. 
\item If $\mathfrak{p}$ is a prime ideal and $IJ \subseteq \mathfrak{p}$. Then, $I \subseteq \mathfrak{p}$ or $J \subseteq \mathfrak{p}$. Suppose not, then there exists $x \in I \backslash \mathfrak{p} , y \in I \backslash \mathfrak{p}$. But then $xy \in IJ \subseteq \mathfrak{p}$.
\item $\mathfrak{p} \supseteq I \cap J \Leftrightarrow IJ \subseteq \mathfrak{p}$.
\end{itemize}

\section{Zariski topology}

\begin{definition}
\begin{itemize}
\item For an ideal $I$, let $$V(I) = \{\mathfrak{p} \text{ prime ideal }\mid I \subseteq \mathfrak{p}\}$$
\item $\mathrm{Spec}(R)=\{ \text{ collection of all prime ideals of }R\}$
\end{itemize}
\end{definition}

\begin{definition}[Zariski Topology]
It is the topology defined on $\mathrm{Spec}(R)$ such that the closed sets are $V(I)$.
\end{definition}

Verification that this indeed is a topology. 
\begin{enumerate}
\item $V(0) = \mathrm{Spec}(R), V(R) = \emptyset$.
\item $V(I) \cup V(J) = V(I\cap J) = V(IJ)$.
\item $\bigcap_{k \in k} V_{k} = V(\sum_{k \in K} I_{k})$. This is because $\mathfrak{p} \supseteq I_{k} \Leftrightarrow \mathfrak{p} \supseteq \sum_{k \in K}I_{k}$
\end{enumerate}

Let us now look at the open sets of this topology. The basis for the open sets is given by $$D(f \in R) = \{ \text{ all prime ideals not containing } f\}$$
Clearly, $$(V(I))^c = \bigcup_{f \in I} D(f)$$ and moreover, each $D(f)$ is open since $D(f) = (V(\langle f \rangle))^c$

\begin{theorem}
$\mathrm{Spec}(R)$ is quasi-compact.
\end{theorem}

\begin{proof}
We wish to prove that every open cover has a finite subcover. This is equivalent to saying every cover by $D(f_{i})$ has a finite subcover. Say $$\spec(R) = \bigcup_{i \in I} D(f_{i})$$ Take $J$ to be the ideal generated by $f_{i}'s$. Either $J = R$ or $J \subseteq \mathfrak{m}$. Suppose $J \subseteq \m$, then $f_{i} \in \m \in \spec(R) \Rightarrow \m \not \in D(f_{i}) \; \forall \; i \Rightarrow D(f_{i})$ does not cover $\m$. A contradiction. Therefore, $J=R$ and this implies $1 =$ some linear combination of $f_{i}$ and notice that this sum is finite. So, just consider these finitely many $f_{i}'s$ (say the indexing set is $K$). These cover $J$. Suppose that $\{ D(f_{k}), k \in K\}$ do not cover $\spec(R)$. Then, there is a prime ideal $\pr \not \in \bigcup_{k\in K}D(f_{k}) \Rightarrow \pr \ni f_{k} \; \forall\; k \in K \Rightarrow R \subseteq \pr \Rightarrow \Leftarrow$. Hence, it covers all of $\spec(R)$ as required. \\

\textbf{Another proof:}\\
Suppose $\spec(R) = \bigcup_{j \in J} U_{j} = \bigcup_{j \in J} \spec(R) \backslash \V(I_{j}) = \spec(R) \backslash \bigcap_{j \in J} \V(I_{j}) =\spec(R) \backslash \V(\sum_{j \in J} I_{j})$. This is equivalent to saying that $\V(\sum_{j \in J} I_{j}) = \emptyset$. So, we conclude that $\sum_{j \in J}I_{j} = R \Rightarrow \sum_{k \in K}a_{k} = 1$ for some finite set $K$. We claim that $\{U_{k}: k \in K\}$ covers $\spec(R)$. This is because 
\begin{align*}
\V(\sum_{k\in K} I_{k}) &= 0 \\
\Rightarrow \spec(R) &= \spec(R) \backslash \V(\sum_{k\in K} I_{k})  \\
&= \bigcup_{k \in K} \spec(R) \backslash \V(I_{k}) \\
&= \bigcup_{k \in K} U_{k}
\end{align*}
This completes the proof. 
\end{proof}

\begin{proposition}
Each $D(f)$ is quasi-compact.
\end{proposition}

\begin{proof}
Suppose $$D(f) = \bigcup D(g_{i})$$ and let $J$ be the ideal generated by $g_{i}'s$. Take $\mathfrak{p} \supseteq J$. Then, each $g_{i} \in J \subseteq \mathfrak{p} \Rightarrow \mathfrak{p} \not \in D(g_{i}) \Rightarrow \mathfrak{p} \not \in D(f) \Rightarrow f \in \mathfrak{p} \Rightarrow f \in \bigcap_{\mathfrak{p} \supseteq J} \mathfrak{p}$. \textcolor{BrickRed}{Before completing this proof, we need to understand this intersection much better. Refer to following content on nilpotent elements and come back.}\\

$f$ is nilpotent from the result proven below. 
\end{proof}

\begin{definition}
$x\in R$ is nilpotent if $x^n =0$ for some $n \in \NN$.
\end{definition}

\begin{remark}
Any nilpotent element ($x^n = 0$ for some $n$ ) is clearly in every prime ideal ($0 \in \mathfrak{p}$) and thus in the intersection of all prime ideals. This can be recorded as $$\bigcap_{\pr \supseteq J} \pr \supseteq \nil(R)$$
\end{remark}

\begin{proposition}
$$\bigcap_{\pr \supseteq J} \pr \subseteq \nil(R)$$
\end{proposition}

\begin{proof}
Take an element $x \in R \backslash \nil(R)$ (not nilpotent) and consider the set $$\Sigma = \{ I \unlhd R \mid x^n \not \in I \; \forall \; n >0\}$$
Notice that $\Sigma$ is a poset with respect to inclusion. And every chain $I_{1} \subseteq I_{2} \subseteq I_{3} \subseteq \cdots $ has an upper bound (union of all the ideals). Thus, we can apply Zorn's lemma to get a maximal element $\pr$ which we claim is prime. Indeed, if $ab \in \pr$ but $a\not \in \pr, b \not \in \pr$ then $\pr + \langle a \rangle, \pr + \langle b \rangle$ are ideals strictly containing $\pr$ contradicting maximality of $\pr$. Therefore, we can conclude that $x \not \in \pr \Rightarrow x \not \in \bigcap_{\pr \supseteq J} \pr$ or rather not nilpotent implies not in intersection and hence we have proved the required inclusion. 
\end{proof}


\chapter{Lecture-3 (12th January): }






\part{Algebraic Geometry I}

\chapter{Lecture-1 (9th January, 2023): Topological properties and Zariski Topology}
\section{Topological properties}

Consider a topological space $X$. 
\begin{definition}
\begin{enumerate}
\item We say $X$ is quasi-compact if every open cover of $X$ admits a finite subcover. 
\item If $f: X \rightarrow Y$ is continuous, we call $f$ quasi-compact if $f^{-1}(V)$ is quasi-compact for all quasi-compact open $V \subseteq Y$.
\end{enumerate}
\end{definition} 

\begin{exercise}
Composition of quasi-compact maps is quasi-compact.
\end{exercise} 

\begin{lemma}
$X$ quasi-compact and $Y \subseteq X$ is closed implies $Y$ is quasi-compact.
\end{lemma}

\begin{proof}

\end{proof}

\begin{proposition}
If $X$ is quasi-compact and Hausdorff, then $E \subseteq X$ is quasi-compact iff $E$ is closed.
\end{proposition}

\begin{proof}

\end{proof}

\begin{lemma}
Any finite union of quasi-compact spaces is quasi-compact.
\end{lemma}

\begin{proof}

\end{proof}

Suppose $\Sigma$ is a poset. $\Sigma$ satisfies acc if every ascending chain $$x_{1} \le x_{2} \le \cdots $$ is stationary.

\begin{lemma}
The following are equivalent: 
\begin{enumerate}
\item $\Sigma$ satisfies acc. 
\item Every non-empty subset of $\Sigma$ has maximal element.
\end{enumerate}
\end{lemma}

\begin{definition}
A topological space is called Noetherian if set of all closed subsets of $X$ satisfies dcc.
\end{definition}

\begin{lemma}
$X$ Noetherian implies $X$ is quasi-compact.
\end{lemma}

\begin{lemma}
If $X_{1}, \hdots , X_{n}$ are Noetherian subspaces of $X$, then so is $X_{1} \cup X_{2} \cup \hdots \cup X_{n}$
\end{lemma}

\begin{lemma}
Quasi-compact and locally Noetherian implies Noetherian.
\end{lemma}

\begin{exercise}
Give an example of a ring $R$ such that $\spec(R)$ is Noetherian but $R$ is not. 
\end{exercise}

\begin{definition}
A topological space $X$ is called irreducible if it cannot be written as finite union of proper closed subsets. \\

A closed subset $Y \subseteq X$ is called irreducible component of $X$ if it is a maximal irreducible closed subset of $X$.
\end{definition}

\begin{lemma}
If $X$ is Noetherian and $Y \subseteq X$ is a subspace, then $Y$ is Noetherian. 
\end{lemma}

\begin{lemma}
Let $X$ be Noetherian. Then, $X$ has finitely many irreducible components. 
\end{lemma}

\begin{lemma}
$X$ is Noetherian implies there exists an unique expression $X = X_{1} \cup \cdots \cup X_{n}$ where $X_{i}'s$ are irreducible components of $X$.
\end{lemma}

\begin{lemma}
Suppose $X$ is Noetherian and $X_{1} \subseteq X$ an irreducible component. Then, $X_{1}$ contains a non-empty open set in $X$.
\end{lemma}

\begin{definition}
Let $X$ be a topological space. We say that $X$ is a spectral space if the following holds: 
\begin{enumerate}
\item $X$ is quasi-compact.
\item $X$ is $T_{0}$.
\item $X$ has a basis of quasi-compact open sets.
\item Every irreducible closed subset of $X$ has a generic point ($\exists x \in Y$ such that $\overline{\{x\}}  = X$)
\end{enumerate}
\end{definition}

\section{Zariski Topology}
Let $A$ be a commutative ring with identity and $X = \spec(A)$. \\

Zariski topology is the unique topology such that a subset $Y\subseteq X$ is closed iff $Y = \V (I)$ for some ideal $I \unlhd A$. Here, $$\V(I) = \{\pr \in X \mid \pr \supseteq I\}$$

\begin{theorem}
$\spec(A)$ is always spectral.
\end{theorem}

\begin{proof}
\begin{enumerate}
\item $X$ is $T_{0}$
\item $X$ is quasi-compact. \\
Let $\{U_{i}\}$ be an open cover of $X$. WLOG, we can assume that $U_{i} = \spec(A_{f_{i}}), f \neq 0$. Let $I$ be the ideal generated by these $f_{i}s$. \\
\textbf{Case-1}: Suppose that $I \neq A$. Then there exists a maximal ideal $\m \supseteq I \Rightarrow \V(\m) \subseteq \V(I) \Rightarrow X \backslash \V(\m) \supseteq X\backslash \V(I) = X \backslash \bigcap_{i \in I}\V(f_{i}) = \bigcup U_{i} = X$ which is absurd. Hence, we conclude that $I=A$. Next, 
\begin{align*}
1 &= \sum_{i=1}^n a_{i}f_{i} &\text{ for some } a_{i} \in A \\
\Rightarrow \bigcup_{i=1}^n U_{i} &= \bigcup_{i=1}^n X \backslash \V(f_{i})
\end{align*}
And, we get the required refinement. 
\item $X$ has a basis of quasi-compact open sets follows from the above.
\item Let $Y \subseteq X$ be an irreducible closed subset. Then, $Y = \spec(A/I)$. WLOG, we can assume $X$ is irreducible. Next, observe that $\spec(A) = \spec(A_{\mathrm{red}}) = \spec(A/ \nil(A))$.
\end{enumerate}
\end{proof}

\chapter{Lecture-2 (11th January, 2023): Zariski topology and affine schemes}

\section{Zariski topology contd..}

\begin{theorem}[Hochster]
Every spectral space is homeomorphic to $\spec(A)$ for some commutative ring $A$.
\end{theorem}

\textbf{Notation:} $\mathrm{\textbf{Ring}}$ be the category of commutative rings, $\mathrm{\textbf{Top}}$ be the category of topological spaces.

\begin{theorem}
There is a contravariant functor 
\begin{align*}
sp: \mathrm{\textbf{Ring}} &\rightarrow \mathrm{\textbf{Top}} \\
\spec(B) &\mapsto \spec(A)
\end{align*}
\end{theorem}

\begin{proof}
Consider $f: A \rightarrow B$. This induces a map $$f_{\#} : \spec(B) \rightarrow \spec(A)$$ such that $f_{\#}(\pr) = f^{-1}(\pr)$. We claim that $f_{\#}$ is continuous. This can be seen as follows: \\
Take a basic open set $U = \spec(A_{a})$ and $b = f(a)$. Then, it is easy to see that $f_{\#}^{-1}(U_{b}) = U_{a}$.
\end{proof}

\section{Affine schemes}

\begin{definition}
$\spec(A)$ will be called an affine "scheme" (we will see this properly later on).
\end{definition}

\begin{definition}
Let $X= \spec(A), Y = \spec(B)$. Let $f: Y \rightarrow X$ be a continuous map. We call such a map $f$ regular (holomorphic) if there is a ring homomorphism $g: A \rightarrow B$ such that $f = g_{\#}$
\end{definition}

\begin{example}
Take $\spec(\ZZ)$ and consider the constant map. This cannot be regular because any ring homomorphism must take $1$ to $1$ and as a consequence fixes every element. 
\end{example}

\begin{proposition}
If $X = \spec(A)$. A regular function on $X$ is a regular map from $X$ to $\spec(\ZZ[t])$.
\end{proposition}

\begin{remark}
On an affine scheme, the set of all regular maps is the ring $A$ itself since, the map $\ZZ[t] \rightarrow A$ is determined by where $t$ is sent to. 
\end{remark}

\begin{lemma}
Every affine scheme has a closed point.
\end{lemma}

\begin{proof}
Every commutative ring has a maximal ideal. 
\end{proof}

\begin{definition}
Open in affine is called quasi-affine.
\end{definition}

\begin{example}
Take $A$ a local integral domain with $\m$ the maximal ideal. Suppose that all prime ideals of $A$ are of the form $$\langle 0 \rangle \subset \pr_{1} \subset \pr_{2} \subset \cdots \subset \{ \m \}$$ Consider $X = \spec(A) \backslash \m$. $X$ is open in affine scheme but has no closed point. \\

An example of such a ring is $$\Gamma = \ZZ x_{1} \oplus \ZZ x_{2} \oplus \cdots $$ Give an ordering: $\sum a_{i}x_{i} \geq 0$ if the first nonzero term is $>0$ or all $a_{i}=0$
\end{example}

\begin{exercise}
Let $A = k[X_{1}, X_{2}, \hdots ], B= A_{\m}, X = \spec(B) \backslash \m , \m = \langle X_{1}, X_{2}, \hdots , \rangle$. Claim is that $X$ has no closed point.
\end{exercise}

\subsection{Fiber products of affine schemes}

Suppose $A$ is a commutative ring, $B,C$ are $A$-algebras. Let $X= \spec(A), Y = \spec(B), Z = \spec(C)$. Next, suppose we have 
\[\begin{tikzcd}
	A & B \\
	C
	\arrow["f", from=1-1, to=1-2]
	\arrow["g"', from=1-1, to=2-1]
\end{tikzcd}\]

\textbf{Universal property of fiber products}:
\[\begin{tikzcd}
	{W'} \\
	& Y \times_{X} Z & Z \\
	& Y & X
	\arrow["{f_{\#}}"', from=3-2, to=3-3]
	\arrow["{g_{\#}}", from=2-3, to=3-3]
	\arrow[from=2-2, to=2-3]
	\arrow[from=2-2, to=3-2]
	\arrow[curve={height=12pt}, from=1-1, to=3-2]
	\arrow[curve={height=-12pt}, from=1-1, to=2-3]
	\arrow["{\exists \; ! }", dashed, from=1-1, to=2-2]
\end{tikzcd}\]

\begin{definition}
If a $W$ exists such that the universal property is satisfied, then $W$ is called the fiber product of $Y,Z$ over $X$ and we write $W = Y \times_{X} Z$
\end{definition}

\begin{theorem}
$\mathrm{\textbf{Aff}_{\ZZ}}=$ category of affine schemes admits fiber products.
\end{theorem}

\begin{proof}
Consider the following data: 
\[\begin{tikzcd}
	A & B \\
	C
	\arrow["f", from=1-1, to=1-2]
	\arrow["g"', from=1-1, to=2-1]
\end{tikzcd}\]

Let $D = B \otimes_{A} C$. We have the natural maps $f_{1}: B \rightarrow B \otimes B \otimes C$ sending $b \mapsto b \otimes 1$ and $f_{2}: C \rightarrow B \otimes C$ sending $c \mapsto 1 \otimes c$. Both are ring homomorphisms and fit into the following diagram due to the nature of tensor product 
 \[\begin{tikzcd}
	A & B \\
	C & {B\otimes C}
	\arrow["f", from=1-1, to=1-2]
	\arrow["g"', from=1-1, to=2-1]
	\arrow["{f_{1}}", from=1-2, to=2-2]
	\arrow["{g_{1}}"', from=2-1, to=2-2]
\end{tikzcd}\]

Now, let $W = \spec(B \otimes_{A} C)$ and we claim that this satisfies the universal property of fibre product. Apply $\spec(-)$ functor to the diagram to get 
\[\begin{tikzcd}
	A & B \\
	C & {\mathrm{Spec}(B\otimes_{A} C)}
	\arrow["{g_{1\#}}"', tail reversed, no head, from=2-1, to=2-2]
	\arrow["{f_{1\#}}"', from=2-2, to=1-2]
	\arrow["{f_{\#}}"', from=1-2, to=1-1]
	\arrow["{g_{\#}}", from=2-1, to=1-1]
\end{tikzcd}\]

From the universal property of tensor product we have the following diagram

\[\begin{tikzcd}
	A & B \\
	C & {B \otimes_{A}C} \\
	&& U
	\arrow["f", from=1-1, to=1-2]
	\arrow["g"', from=1-1, to=2-1]
	\arrow["{f_{1}}", from=1-2, to=2-2]
	\arrow["{g_{1}}"', from=2-1, to=2-2]
	\arrow[curve={height=18pt}, from=2-1, to=3-3]
	\arrow[curve={height=-18pt}, from=1-2, to=3-3]
	\arrow["{\exists \; ! }",dashed, from=2-2, to=3-3]
\end{tikzcd}\]

Again, apply the $\spec(-)$ functor. 

\[\begin{tikzcd}
	X & Y \\
	Z & {\mathrm{Spec}(B\otimes_{A} C)} \\
	&& {\mathrm{Spec}(U)}
	\arrow["{g_{1\#}}"', tail reversed, no head, from=2-1, to=2-2]
	\arrow["{f_{1\#}}"', from=2-2, to=1-2]
	\arrow["{f_{\#}}"', from=1-2, to=1-1]
	\arrow["{g_{\#}}", from=2-1, to=1-1]
	\arrow[curve={height=-24pt}, from=3-3, to=2-1]
	\arrow[curve={height=18pt}, from=3-3, to=1-2]
	\arrow["{\exists \; ! }",dashed, from=3-3, to=2-2]
\end{tikzcd}\]

This completes the proof.
\end{proof}




\chapter{Lecture-3:}

\part{Topics in Analytic Number Theory}

\chapter{Lecture-1: Hardy-Littlewood proof of infinitely many zeros on the line $\mathfrak{R}(s) = 1/2$}

\chapter{Lecture-2: }

\chapter{Lecture-3 (10th January, 2023): Siegel's theorem }

\begin{theorem}[Siegel]
Let $\chi(q)$ be a real Dirichlet character modulo $q\geq 3$. Given any $\epsilon >0$, we have $$L(1, \chi) \geq \frac{C_{\epsilon}}{q^{\epsilon}}$$
\end{theorem}

A trivial lower bound: $L(1, \chi) \gg q^{-1/2}$

\begin{proof}[Goldfeld's proof]
Consider $$f(s) = \zeta(s)L(s,\chi_{1})L(s,\chi_{2})L(s,\chi_{1}\chi_{2})$$ with $\chi_{i}, i=1,2$ primitive quadratic characters. Notice that $f(s) = \sum_{n} b_{n}n^{-s}$ with $b_{1} =1 , b_{n} \geq 0$. Let $\lambda = \mathrm{Res}_{s=1}f(s) = L(1,\chi_{1})L(1,\chi_{2})L(1,\chi_{1}\chi_{2})$

\begin{lemma}
Given any $\epsilon >0$, one can find $\chi_{1}(q_{1})$ and $\beta$ with $1-\epsilon < \beta < 1$ such that $f(\beta) \le 0$, independent of what $\chi_{2}(q_{2})$ is. 
\end{lemma}

\begin{proof}
\textbf{Case-1:} If there are no real zeros of $L(s, \psi)$ for any primitive quadratic character in $(1-\epsilon,1)$, then $f(\beta) < 0$ for any $\beta \in (1 - \epsilon,1)$. This is because $$f(\beta) = \underbrace{\zeta(\beta)}_{<0} \underbrace{L(s,\chi_{1})L(s,\chi_{2})L(s,\chi_{1}\chi_{2})}_{>0}$$ as $L(1,\chi)>0$ and $L$ is continuous so any change of sign will lead to a zero which is a contradiction. \\
\textbf{Case-2:} If we cannot find such a $\psi$, then just set $\chi_{1}=\chi$ and let $\beta$ be the real zero. Then, $f(\beta)=0$. We are done. 
\end{proof}

Next, consider the integral 

\end{proof}

\begin{corollary}
\begin{align*}
h(-d) &= \frac{L(1,\chi_{d}) \sqrt{|d|} \;\omega}{2 \pi} \\
&= \frac{L(1,\chi_{d})}{\log \epsilon_{d}}
\end{align*}
\end{corollary}

\begin{theorem}[Y. Zhang]
$$L(1, \chi) \geq \frac{c}{(\log q)^{2022}}$$
\end{theorem}

\begin{theorem}
If $\chi(q)$ does not have a Siegel zero, then $L(1, \chi) \gg \frac{1}{\log q}$
\end{theorem}


\chapter{Lecture-4 (12th January, 2023): }

\end{document}