\documentclass[oneside, 12pt, ]{scrbook}

%packages to import
\usepackage[utf8]{inputenc}
\usepackage{amsmath}
\usepackage{amssymb}
\usepackage{amsthm}
\usepackage{float}
\usepackage{minitoc}
\usepackage[width=15cm, height=24cm]{geometry}
\usepackage{hyperref}
\usepackage[usenames,svgnames,dvipsnames]{xcolor}
\usepackage{mathrsfs}
\usepackage{mathtools}
\usepackage{thmtools}
%\usepackage{kpfonts}
\usepackage[lf]{venturis} %% lf option gives lining figures as default; 
			  %% remove option to get oldstyle figures as default
\renewcommand*\familydefault{\sfdefault} %% Only if the base font of the document is to be sans serif
\usepackage{fontenc}
\usepackage{setspace}
\usepackage{cleveref}
\usepackage{backref}
\usepackage{graphicx}
\usepackage{tikz-cd}
\usepackage{quiver}

%section symbol


%no indentation paragraph
\setlength{\parindent}{0pt}


% color those links
\hypersetup{
colorlinks=true,
urlcolor= BrickRed,
linkcolor= BrickRed,
citecolor= Cerulean
}
%colour page black and text white
%\pagecolor{gray}
%\color{white}

%\usepackage{multicol}
\usepackage[utf8]{inputenc}

%custom commands
\newtheorem{statement}{Statement}
\newcommand{\CC}{\mathbb C}
\newcommand{\FF}{\mathbb F}
\newcommand{\NN}{\mathbb N}
\newcommand{\QQ}{\mathbb Q}
\newcommand{\RR}{\mathbb R}
\newcommand{\ZZ}{\mathbb Z}
\newcommand{\p}{\mathfrak{p}}
\newcommand{\V}{\mathcal{V}}
\newcommand{\I}{\mathcal{I}}
\newcommand{\spec}{\mathrm{Spec}}
\newcommand{\Aa}{\mathbb{A}}
\newcommand{\PP}{\mathbb{P}}
\newcommand{\im}{\mathrm{im}}
\newcommand{\pr}{\mathfrak{p}}
\newcommand{\m}{\mathfrak{m}}
\newcommand{\nil}{\mathrm{Nil}}
\newcommand{\ord}{\mathrm{ord}}
\newcommand{\SL}{\mathrm{SL}}
\newcommand{\GL}{\mathrm{GL}}

%theoremstyles 
\usepackage[framemethod=TikZ]{mdframed}

%theorem
\mdfdefinestyle{mdthmbox}{
	linewidth= 3pt,
	rightline=false,
	leftline=true,
	topline=false,
	bottomline=false,
	linecolor= WildStrawberry,
	backgroundcolor= CarnationPink!25 
}

\declaretheoremstyle[
	spaceabove=6pt, spacebelow=6pt,
	headfont=\normalfont \bfseries,
	notefont=\mdseries, notebraces={(}{)},
	bodyfont=\normalfont,
	postheadspace = \newline,
	mdframed = {style= mdthmbox}
]{theobox}

\declaretheorem[style=theobox,name=Theorem,numberwithin=section]{theorem}
\declaretheorem[style=theobox,name=Conjecture,numberwithin=section]{conjecture}

%lemma, proposition, corollary
\mdfdefinestyle{mdlembox}{
	linewidth= 3pt,
	rightline=false,
	leftline=true,
	topline=false,
	bottomline=false,
	linecolor= RoyalBlue,
	backgroundcolor= SkyBlue!30
}

\declaretheoremstyle[
	postheadspace= \newline,
	spaceabove=6pt, spacebelow=6pt,
	headfont=\normalfont \bfseries,
	notefont=\mdseries, notebraces={(}{)},
	bodyfont=\normalfont,
	postheadspace = \newline,
	mdframed = {style= mdlembox}
]{lembox}
\declaretheorem[style=lembox,name=Lemma,sibling=theorem]{lemma}
\declaretheorem[style=lembox,name=Proposition,sibling=theorem]{proposition}
\declaretheorem[style=lembox,name=Corollary,sibling=theorem]{corollary}

% definition
\mdfdefinestyle{mddefbox}{
	linewidth= 3pt,
	rightline=false,
	leftline=true,
	topline=false,
	bottomline=false,
	linecolor= LimeGreen,
	backgroundcolor= GreenYellow!25 
}

\declaretheoremstyle[
	postheadspace= \newline,
	spaceabove=6pt, spacebelow=6pt,
	headfont=\normalfont \bfseries,
	notefont=\mdseries, notebraces={(}{)},
	bodyfont=\normalfont,
	postheadspace = \newline,
	mdframed = {style= mddefbox}
]{defbox}


\declaretheorem[style=defbox, name= Definition, sibling=theorem]{definition}

\mdfdefinestyle{mdrembox}{
	linewidth= 2pt,
	rightline=false,
	leftline=true,
	topline=false,
	bottomline=false,
	linecolor= black,
}

\declaretheoremstyle[
	postheadspace= \newline,
	spaceabove=6pt, spacebelow=6pt,
	headfont=\normalfont \bfseries,
	notefont=\mdseries, notebraces={(}{)},
	bodyfont=\normalfont,
	postheadspace = \newline,
	mdframed = {style= mdrembox}
]{rembox}

\declaretheorem[name=Remark,sibling=theorem,style=rembox]{remark}
\declaretheorem[name=Note,sibling=theorem,style=rembox]{note}


% example
\mdfdefinestyle{mdredbox}{
	skipabove=8pt,
	linewidth=3pt,
	rightline=false,
	leftline=true,
	topline=false,
	bottomline=false,
	linecolor=red,
	backgroundcolor=Salmon!5
}

\declaretheoremstyle[
	postheadspace= \newline,
	spaceabove=6pt, spacebelow=6pt,
	headfont=\normalfont \bfseries,
	notefont=\mdseries, notebraces={(}{)},
	bodyfont=\normalfont,
	postheadspace = \newline,
	mdframed = {style= mdredbox}
]{thmredbox}

\mdfdefinestyle{mdgreybox}{
	skipabove=8pt,
	linewidth=3pt,
	rightline=false,
	leftline=true,
	topline=false,
	bottomline=false,
	linecolor=Goldenrod,
	backgroundcolor=Grey!5
}

\declaretheoremstyle[
	postheadspace= \newline,
	spaceabove=6pt, spacebelow=6pt,
	headfont=\normalfont \bfseries,
	notefont=\mdseries, notebraces={(}{)},
	bodyfont=\normalfont,
	postheadspace = \newline,
	mdframed = {style= mdredbox}
]{thmredbox}

\theoremstyle{theorem}
\declaretheorem[name= Exercise, style= thmgreybox, sibling = theorem]{exercise}
\declaretheorem[name= Example, style= thmredbox, sibling = theorem]{example}

%citations
%\usepackage[
%backend=biber,
%style=alphabetic,
%sorting=nyt
%]{biblatex}
%\addbibresource{citations.bib}

\begin{document}

%titlepage

\begin{center}
\begin{minipage}{0.75\linewidth}
    \centering
%University logo
    \includegraphics[width=1\linewidth]{IISc_logo.png}
    \rule{0.4\linewidth}{0.15\linewidth}\par
    \vspace{2cm}
%Thesis title
    {\uppercase{\Large Semester notes\par  }}
    \vspace{2cm}
%Author's name
    {\Large Irish Debbarma\par}
    \vspace{2cm}
%Degree
    {\Large Department of Mathematics \\ Indian Institute of Science, Bangalore\par}
    \vspace{1cm}
%Date
    {\Large December 2022}
\pagenumbering{gobble}   
\end{minipage}
\end{center}
\clearpage

\frontmatter
\tableofcontents

\mainmatter

\part{Modular Forms}

\chapter{Lecture-1 (3rd January): Introduction }

\chapter{Lecture-2 (5th January, 2023): }

\chapter{Lecture-3 (10th January, 2023): Valence formula and Eisenstein series}

\section{Valence formula}

Recall that $M_{k}(\Gamma_{1})$ is the space of modular forms of weight $k$ and level $1$. It is also a vector space over $\CC$. 

\begin{theorem}
$\dim M_{k}(\Gamma_{1}) = \begin{cases}[k/12]+1 & k\not \equiv \pmod{12} \\ [k/12] & k \equiv \pmod{12} \end{cases}$
\end{theorem}

\begin{proposition}
Let $f \in M_{k}(\Gamma_{1})$. Then, $$\sum_{p \in \Gamma_{1}\backslash \mathbb{H}}\frac{1}{n_{p}} \mathrm{ord}_{p}(f) + \mathrm{ord}_{\infty}(f) = \frac{k}{12}$$
\end{proposition}

\begin{proof}
Let $\epsilon >0$ be "small enough". Remove $\epsilon$-balls around $\infty , i , \omega , \omega + 1$ in $\mathcal{F}_{1}$. $\epsilon$ is small enough so that the removed balls are disjoint. Truncate $\mathcal{F}_{1}$ at the line $y = \epsilon^{-1}$ and call the enclosed region $D$. \\
By Cauchy's theorem $$\int_{\partial D} d(\log f(z)) = 0$$ This integral on the two vertical strips (just the straight lines not the semicircle part) is $0$ since the contribution of left is same as right but orientation is different. On the segment joining $-1/2 + iY , 1/2 + iY$, the integral is $2 \pi i \,\mathrm{ord}_{\infty}(f)$. Again,integral around each removed point in $\mathcal{F}_{1}$ is $\frac{1}{n_{p}} \mathrm{ord}_{p}(f)$. Next, divide the bottom arc into left and right parts and observe that $$d(\log f(S\cdot z)) = d(\log f(z)) + k \frac{dz}{z}$$ $$\int_{C} d(\log f(z)) = \frac{k \pi i }{6}$$ 
\end{proof}

\begin{corollary}
$\dim M_{k}(\Gamma_{1}) = \begin{cases} 0 & k < 0 \\ 0 & k \text{ is odd } \\ 1 & k=0 \\ \begin{cases}[k/12]+1 & k\not \equiv \pmod{12} \\ [k/12] & k \equiv \pmod{12} \end{cases} \end{cases}$
\end{corollary}

\begin{proof}
\begin{itemize}
\item If $k <0$, then $f$ has poles but is holomorphic.
\item  If $k=0$, then $f$ is the constant function. 
\item We have seen
\item For $m = [k/12] + 1$ let $f_{1}, \hdots , f_{m+1} \in M_{k}(\Gamma_{1})$. Let $P_{1}, \hdots , P_{m}$ be any points on $\mathcal{F}_{1}$ not equal to $i , \omega , \omega + 1$ and consider $(f_{i}(P_{j}))_{i \in [m+1], j \in [m]}$. \\
There exists a linear combination $f = \sum_{i=1}^{m+1} c_{i}f_{i}$ not all $c_{i}$ being zero, such that $f(P_{j})=0$ for $1 \le j \le m$. \\
From the previous theorem we get $f \equiv 0$ and this implies $\{f_{i}\}$ is linearly independent and thus $\dim_{\CC} M_{k}(\Gamma_{1}) \le m$. \\
For $k \equiv 2 \pmod{12}$, the relation in previous theorem holds only if there is atleast a simple zero at $p=i$ and atleast a double zero at $p = \omega$. This gives $$\frac{k}{12} - \frac{7}{6} = m-1$$ Repeat the argument above.  
\end{itemize}
\end{proof}

A slight notation. For $\gamma  = \begin{pmatrix}
a & b \\ c & d
\end{pmatrix} \in \mathrm{SL}_{2} (\ZZ)$ we set $f|_{\gamma} (z) = (cz+d)^{-k}f(\gamma \cdot z)$. \\
Thus, $1|_{\gamma}(z) = (cz+d)^{-k}$. If $1|_{\gamma}(z) =1 \Rightarrow c =0$. Conversely, if $c=0$, then $d^{-k}=1$. So, $1|_{\gamma}(z)=1 \Leftrightarrow c=0$. \\
$\Gamma_{\infty} = \left\{ \begin{pmatrix}
a & b \\ 0 & d 
\end{pmatrix} \in \mathrm{SL}_{2}(\ZZ) \right\} = \mathrm{stab}(\infty)$

\section{Eisenstein series}

\begin{definition}
The Eisenstein series $E_{k}(z)$ is defined to be $$E_{k}(z) = \sum_{\gamma \in \Gamma_{\infty} \backslash \Gamma_{1}} 1|_{\gamma}(z)$$
\end{definition}

\begin{proposition}
$$E_{k}(z) = \frac{1}{2} \sum_{(c,d) \in \ZZ^{2} \backslash \{(0,0)\}, \gcd(c,d)=1} \frac{1}{(cz+d)^k}$$
\end{proposition}

\begin{proof}

\end{proof}

\begin{proposition}
$$\sum_{(c,d) \in \ZZ^{2} \backslash \{(0,0)\}, \gcd(c,d)=1} \frac{1}{(cz+d)^k}$$ converges absolutely for $k>2$
\end{proposition}

\begin{proof}

\end{proof}

\begin{theorem}
$E_{k}(z) \in M_{k}(\Gamma_{1})$ for $k >2$.
\end{theorem}

\begin{proof}

\end{proof}

\begin{proposition}
$E_{k}(z) \not \equiv 0$ for $k>2$, even.
\end{proposition}

\begin{proof}
Observe that $$\frac{1}{(cz+d)^k} \rightarrow 0 , \mathfrak{I}(z) \rightarrow \infty , c\neq 0$$ and if $c = 0$, then $c=\pm 1$. Hence, $E_{k}(z) = 1 +$ bounded term as $\mathfrak{I}(z) \rightarrow \infty$. This implies $E_{k}(z) \not \equiv 0$ and $$E_{k}(z) = 1 + \sum_{n=1}^{\infty} a_{n}e^{2 \pi i z}$$
\end{proof}

Another way of looking at Eisenstein series is a function on a lattice. \\
Consider $G_{k}(z) = G_{k}(\ZZ z + \ZZ) = \frac{1}{2} \sum_{(c,d) \in \ZZ^2}^{'} \frac{1}{(cz+d)^k}$

\begin{proposition}
$G_{k}(z)$ converges absolutely for $k>2$.
\end{proposition} 

\begin{proposition}
$G_{k}(z) = \zeta(k) E_{k}(z)$
\end{proposition}

\begin{proposition}
$\mathbb{G}_{k}(z) = \frac{(k-1)!}{(2 \pi i)^k} G_{k}(z) = -\frac{B_{k}}{2k} + \sum_{n=1}^{\infty} \sigma_{k-1}(n)q^n$ for $k>2$, even.
\end{proposition}

\chapter{Lecture-4 (12th January, 2023): Eisenstein series}

\section{Eisenstein series contd..}

Recall that $$M_{*}(\Gamma_{1}) = \bigoplus_{k \in \ZZ}M_{k}(\Gamma_{1})$$  is a graded ring.

\begin{proposition}
The graded ring $M_{*}(\Gamma_{1})$ is freely generated by $E_{4},E_{6}$. This means that the map 
\begin{align*}
f: C[X,Y] &\rightarrow M_{*}(\Gamma_{1})\\
X &\mapsto E_{4} \\
Y &\mapsto E_{6}
\end{align*}
is an isomorphism of graded rings. Here, $\deg X = 4, \deg Y=6$. 
\end{proposition}

\begin{proof}
We want to show that $E_{4}$ and $E_{6}$ are algebraically independent. We start by showing that $E_{4}^3$ and $E_{6}^2$ are linearly independent over $\CC$. Suppose $E_{6}(z)^2 = \lambda E_{4}(z)^{3}$. Consider $f(z) = E_{6}(z)/ E_{4}(z)$. Now observe that $f(z)^2 = \lambda E_{4}(z)$. This means that $f^2$ is holomorphic and thus $f$ is also holomorphic. But $f$ is weakly modular of weight $2$ which is a contradiction. So, our claim is proven. \\

\textbf{Claim}: Let $f_{1},f_{2}$ be two nonzero modular forms of same weight. If $f_{1},f_{2}$ are linearly independent, then they are algebraically independent as well. \\

Let $P(t_{1},t_{2}) \in \CC[t_{1},t_{2}] \backslash \{0\}$ be such that $P(f_{1},f_{2})=0$. Let $P_{d}(t_{1},t_{2})$ be the $d$ degree parts of $P$. Using the fact that modular forms of different weights are linearly independent, we get that $P_{d}(f_{1},f_{2})=0 \; \forall \; d \geq 0$. If $p_{d}(t_{1}/t_{2}) = P_{d}(t_{1},t_{2})/t_{2}^d$, then $p_{d}(f_{1}/f_{2})=0$. But this means that $f_{1}/f_{2}$ is a constant. But, $f_{1},f_{2}$ are linearly independent which implies that they are algebraically independent as well. \\

All of this implies that $E_{4},E_{6}$ are algebraically independent. Using
\end{proof}

\begin{corollary}
$\dim_{\CC} M_{k} (\Gamma_{1}) = \begin{cases}[k/12]+1 & k\not \equiv \pmod{12} \\ [k/12] & k \equiv \pmod{12} \end{cases} $
\end{corollary}

\subsection{Fourier expansions of $E_{k}(z)$}

\begin{proposition}
$$\mathbb{G}_{k}(z) = \frac{(k-1)!}{(2 \pi i)^k} G_{k}(z) = -\frac{B_{k}}{2k} + \sum_{n=1}^{\infty} \sigma_{k-1}(n)q^n$$ for $k>2$, even and $B_{k}$ are Bernoulli numbers.
\end{proposition}

\begin{proof}
Use $$\frac{\pi}{\tan \pi z} = \sum_{n \in \ZZ} \frac{1}{z+n} = \lim_{M,N \rightarrow \infty , N-M < \infty} \sum_{-M}^{N} \frac{1}{z+n}$$ and 
$$\frac{\pi}{\tan \pi z} = \frac{\pi \cos \pi z}{\sin \pi z} = \pi i \frac{e^{\pi i z} + e^{-\pi i z}}{e^{\pi i z} - e^{-\pi i z}} = - \pi i \frac{1+q}{1-q} = - 2 \pi i \left( \frac{1}{2} + \sum_{r=1}^{\infty} q^r \right)$$ This leads to the equality 
$$\sum_{n\in \ZZ}\frac{1}{z+n} = -  2 \pi i \left( \frac{1}{2} + \sum_{r=1}^{\infty} q^r \right)$$
Differentiate both sides of equality $k-1$ times and divide by $(k-1)!$ to get $$\sum_{n \in \ZZ}\frac{1}{(z+n)^k} = \frac{(-2 \pi i)^k}{(k-1)!} \sum_{r=1}^{\infty} r^{k-1}q^r$$
Next, if we look at 
\begin{align*}
G_{k}(z) &= \frac{1}{2} \sum' \frac{1}{(mz+n)^k} \\
&= \frac{1}{2} \sum_{n \in \ZZ , n \neq 0} \frac{1}{n^k} + \frac{1}{2} \sum_{(m,n) \in \ZZ^2 , m \neq 0} \frac{1}{(mz + n)^k} \\
&= \zeta(k) + \sum_{m=1}^{\infty} \sum_{n=-\infty}^{\infty} \frac{1}{(mz+n)^k} \\
&= \zeta(k) + \frac{(2 \pi i)^k}{(k-1)!} \sum_{m=1}^{\infty} \sum_{r=1}^{\infty} r^{k-1} q^{mr} \\
&= \zeta(k) + \frac{(2 \pi i)^k}{(k-1)!} \sum_{m=1}^{\infty} \sum_{r=1}^{\infty} \sigma_{k-1}(n) q^{n}
\end{align*}
The expression of $\mathbb{G}_{k}(z)$ is trivial after noting $$\frac{(k-1)!}{(2 \pi i)^k} \zeta(k) = B_{k}$$
\end{proof} 

\begin{remark}
\begin{enumerate}
\item $\mathbb{G}_{4}(z) = \frac{1}{240} + q + 9q^2 + 28q^3 + 73q^4 + \cdots $
\item $\mathbb{G}_{6}(z) = -\frac{1}{504} + q + 33q^2 + 244q^3 + \cdots $
\item $\mathbb{G}_{8}(z) = \frac{1}{480} + q + 129q^2 + 2188q^3 + \cdots $
\end{enumerate}
\end{remark}

\begin{proposition}
$$\sum_{m=1}^{n-1} \sigma_{3}(m) \sigma_{3}(n-m) = \frac{\sigma_{7}(n) - \sigma_{3}(n)}{120}$$
\end{proposition}

\begin{proof}

\end{proof}

\subsection{Weight $2$ Eisenstein series}

\begin{definition}
\begin{align*}
\mathbb{G}_{2}(z) &= -\frac{1}{24} + \sum_{n=1}^{\infty} \sigma_{1}(n)q^n \\
&= -\frac{1}{24} + q + 3q^2 + 4q^3 + 7q^4 + \cdots 
\end{align*} 
\end{definition}

This converges rapidly on $\mathbb{H}$ and defines a holomorphic function. 

\begin{proposition}
$$G_{2}(z) = - 4 \pi^2 \mathbb{G}_{2}(z)$$
\end{proposition}

\begin{proof}
Since we know that $$G_{2}(z) = \sum_{(m,n) \in \ZZ^2 \backslash \{(0,0)\}}\frac{1}{(mz+n)^2}$$ does not converge absolutely, we define $$G_{2}(z) = \frac{1}{2} \sum_{n \in \ZZ , n \neq 0} \frac{1}{n^2} + \frac{1}{2} \sum_{m \neq 0} \sum_{n \in \ZZ} \frac{1}{(mz+n)^2}$$
This sum converges absolutely and we can show that this satisfies the functional equation as required.
\end{proof}

\begin{proposition}
For $\gamma = \begin{pmatrix}
a & b \\ c & d 
\end{pmatrix} \in \mathrm{SL}_{2}(\ZZ)$ we have $$G_{2}\left( \frac{az+b}{cz+d} \right) = (cz+d)^2 G_{2}(z) - \pi i c(cz+d)$$
\end{proposition}

$G_{2}$ is called a quasi modular form. \\

Introduce (due to Hecke): $$G_{2,s}(z) = \frac{1}{2} \sum_{(m,n)\in \ZZ^2 \backslash \{(0,0)\}} \frac{1}{(mz+n)^2 |mz+n|^{2s}} , \mathfrak{R}(s) >0$$

\section{Modular forms of higher level}

Let $N \in \ZZ_{\geq 1}$ 

$$\mathrm{SL}_{2}(\ZZ / N \ZZ) =\left\{ \begin{pmatrix}
a & b \\ c & d
\end{pmatrix} \in M_{2}(\ZZ / N \ZZ) \mid ad-bc \equiv 1 \pmod{N} \right\}$$

\begin{lemma}
The map 
\begin{align*}
\mathrm{SL}_{2}(\ZZ) &\rightarrow \mathrm{SL}_{2}(\ZZ / N \ZZ) \\
\begin{pmatrix}
a & b \\ c & d
\end{pmatrix}  &\mapsto \begin{pmatrix}
\bar{a} & \bar{b} \\ \bar{c} & \bar{d}
\end{pmatrix}
\end{align*}
is a surjective group homomorphism.
\end{lemma}

\begin{proof}

\end{proof}

\begin{definition}
$$\Gamma (N) = \ker (\mathrm{SL}_{2}(\ZZ) \rightarrow \mathrm{SL}_{2}(\ZZ / N \ZZ))$$ is called the principal congruence subgroup.
\end{definition}

\begin{definition}
A subgroup $\Gamma$ of $\mathrm{SL}_{2}(\ZZ)$ is called a congruence subgroup if there exists $N$ such that $\Gamma (N) \subseteq \Gamma$.
\end{definition}

$$\Gamma_{0}(N) =\left\{ \begin{pmatrix}
a & b \\ c & d
\end{pmatrix} \in M_{2}(\ZZ / N \ZZ) \mid c\equiv 0 \pmod{N} \right\}$$
$$\Gamma_{1}(N) =\left\{ \begin{pmatrix}
a & b \\ c & d
\end{pmatrix} \in M_{2}(\ZZ / N \ZZ) \mid c\equiv d \equiv 1 \pmod{N} \right\}$$

\chapter{Lecture-5 (17th January, 2023): Congruence subgroups and $\Delta$ function}

\section{$\Delta$ function}

Consider $$\Delta (z) = \frac{1}{1729} (E_{4}^3(z) - E_{6}^2(z)) = q + q^2() + \cdots $$
Clearly, $\Delta(z)$ is a normalised cusp form of weight $12$ and level $1$. 

\begin{theorem}
$$\Delta (z) = q \prod_{n \geq 1} (1-q^{n})^{24}, q=e^{2\pi i z}$$
\end{theorem}

\begin{proposition}
$\Delta(z)$ has no zero in $\mathbb{H}$.
\end{proposition}

\begin{proof}
From the valence formula we have $$\sum_{p \in \mathbb{H}} \frac{1}{n_{p}} \ord_{p}(\Delta(z)) + \ord_{\infty} (\Delta(z)) = k/12 =1$$ Moreover, $\ord_{\infty} (\Delta(z))=1$. Hence, we can conclude that $\ord_{p}(\Delta(z)) =0 \; \forall \; p \in \mathbb{H}$.
\end{proof}

\textbf{Application}: We use $\Delta(z)$ to write any modular form as a polynomial in $E_{4},E_{6}$. \\

Take $f(z) \in M_{k} (\Gamma_{1})$ with $4a+6b , k \geq 4, a,b\geq 0$. The Fourier expansion of $f(z)$ can we written as $$f(z) = a_{0} + a_{1}q + \cdots $$
Clearly, $f(z) - a_{0}E_{4}^a(z)E_{6}^b(z) \in M_{k}(\Gamma_{1}) \subseteq S_{k}(\Gamma_{1})$.

Next, $$h(z) = \frac{f(z)-a_{0}E_{4}^a(z)E_{6}^b(z)}{\Delta(z)} \in M_{k-12}(\Gamma_{1})$$ Recursively, we can now find expression for $f(z)$. 

\begin{proposition}
$$j(z) = \frac{E_{4}^3}{\Delta(z)} = q^{-1} + \cdots $$
\begin{align*}
j: \bar{\mathbb{H}}/\Gamma_{1} &\rightarrow \PP^1(\CC) \\
z &\mapsto j(z)
\end{align*}

is a bijection.
\end{proposition}

\begin{proof}
$E_{4}^3(z)$ and $\Delta(z)$ are linearly independent. For any $\lambda_{1},\lambda_{2} \in \CC$ both not zero, $\lambda_{1}E_{4}^3(z) + \lambda_{2}\Delta(z)$ has an unique zero in $\bar{\mathbb{H}}/\Gamma_{1}$. 
\end{proof}

\begin{remark}
This $j$ is called the $j$-invariant modular function. It attaches an elliptic curve in $\PP^1 (\CC)$ to any lattice in $\Lambda_{z} = \ZZ z + \ZZ$ and vice versa.
\end{remark}

Next, the Fourier series of $\Delta(z)$ is of the form $\Delta(z) = \sum_{n \geq 1} \tau(n)q^n$ where $\tau(n)$ satisfies the following properties: 
\begin{enumerate}
\item $\tau(pq) =\tau(p)\tau(q)$ if $p,q$ are dinstinct primes. 
\item $\tau(p^2) = \tau(p)^2 - p^{12-1}$.
\item $|\tau(p)| \le 2 p^{\frac{12-1}{2}}$.
\item 
\begin{align*}
\mathbb{G}_{12}(z) &= \Delta(z) + \frac{691}{156} \left( \frac{E_{4}^3(z)}{720} + \frac{E_{6}^2}{1008} \right) \\
\mathbb{G}_{12} &= -\frac{B_{12}}{24} + \sum_{n \geq 1} \sigma_{11}(n) q^n \\
&= \frac{691}{65520} + \sum_{n \geq 1} \sigma_{11}(n) q^n \\ 
\mathbb{G}_{12}(z) &\equiv \Delta(z) \pmod{691}
\end{align*}
To conclude $$\tau(n) = \sigma_{11}(n) \pmod{691}$$
(Related to the fact that $691 \mid \# \mathcal{Cl}(\QQ(\gamma_{691}))$)
\end{enumerate}

\section{Congruence subgroup}

\begin{proposition}
Let $N=p_{1}^{a_{1}}\cdots p_{r}^{a_{r}}$ be the prime factorisation. Then, $$\mathrm{SL}_{2} (\ZZ / N \ZZ) = \prod_{i=1}^{r} \mathrm{SL}_{2}(\ZZ / p^{a_{i}} \ZZ) $$ 
\end{proposition}

\begin{lemma}
$$\# \mathrm{SL}_{2}(\ZZ / N \ZZ) = N^3 \prod_{p \mid N} \left( 1 - \frac{1}{p^2} \right)$$
\end{lemma}

\begin{definition}
A subgroup $\Gamma \subseteq \mathrm{SL}_{2}(\ZZ)$ is called congruence subgroups if $\Gamma(N) \subseteq \Gamma$ for some $N \geq 1$.
\end{definition}

\begin{lemma}
A congruence subgroup has finite index in $\mathrm{SL}_{2}(\ZZ)$.
\end{lemma}

\begin{remark}
There are non-congruence subgroups of finite index in $\mathrm{SL}_{2}(\ZZ)$.
\end{remark}

\textbf{Properties}:
\begin{enumerate}
\item $\mathrm{PSL}_{2}(\ZZ)$ is generated freely by an element of order $2$ and an element of order $3$.
\item $S_{7}$ is generated by an element of order $2$ and an element of order $3$. There is a surjection 
\begin{align*}
\mathrm{PSL}_{2}(\ZZ) &\xrightarrow{\pi} S_{7} \\
\pi^{-1}(\mathrm{Stab}_{1}) \subseteq \mathrm{PSL}_{2}(\ZZ) \\
\end{align*}
\item $\SL_{2}(\ZZ/p\ZZ)$ is a simple group for $p \geq 5$.
\end{enumerate}

\begin{remark}
$\Gamma$ is the smallest index subgroup that is non-congruence.
\end{remark}

\begin{definition}
A holomorphic function $f: \mathbb{H} \rightarrow \CC$ is a modular form of weight $k$ and level $\Gamma$ if 
\begin{enumerate}
\item $f\left(\frac{az+b}{cz+d}\right) = (cz+d)^k f(z)$ for all $\begin{pmatrix}
a & b \\ c & d
\end{pmatrix} \in \Gamma$
\item $f$ is holomorphic at all cusps.
\end{enumerate}
Cusps of $X(\Gamma)$ are just elements of $\Gamma \backslash \PP^1 (\QQ)$.
\end{definition}

\begin{proposition}
$$\mathrm{orbit}(\infty) = \{ \frac{a}{c}: p\mid c , p \nmid a\}, \mathrm{orbit}(1) = \{ \frac{a}{c}:p \nmid c\}$$
\end{proposition}




\part{Elliptic Curves}

\chapter{Lecture-1 (3rd January): Introduction }

\chapter{Lecture-2 (5th January, 2023): Affine varieties }

\section{Affine Varieties}

Suppose $k$ is a perfect field (every extension is separable). \\
Let $G(\bar{k}/k)$ be the Galois group of the extension. It can also be viewed as $\displaystyle{\varinjlim_{L/K \text{Galois, }L \text{ finite}} \mathrm{Gal}(L/K)}$. \\



\chapter{Lecture-3 (10 January, 2023): Projective varieties}

\section{Projective varieties}

\begin{definition}
A Projective $n$-space over $k$ denoted by $\mathbb{P}^n$ or $\mathbb{P}^{n}(\bar{K})$ is the set $\mathbb{A}^{n+1} \backslash \{(0,\hdots , 0)\} / \sim$ with  
$$(x_{0}, \hdots ,x_{n}) \sim (y_{0},\hdots ,y_{n})$$ iff $\exists \lambda \in \bar{k}^{\times}$ such that $(y_{0},\hdots ,y_{n}) = (\lambda x_{0},\hdots ,\lambda x_{n})$\\
The equivalence class $(x_{0},\hdots ,x_{n+1})$ is denoted by $[x_{0},\hdots ,x_{n}]$\\
The set of $k$-rational points of $\mathbb{P}^n$ is $$\mathbb{P}^n = \{[x_{0}, \hdots , x_{n}]\; \mid \; x_{i} \in k\}$$
\end{definition}

\textcolor{red}{Caution}: If $p = [x_{0},\hdots ,x_{n}] \in \mathbb{P}^n(k)$ and $x_{i} \neq 0$ for some $i$, then $x_{j}/x_{i} \in k \forall j$

\begin{definition}
Let $p = [x_{0},\hdots ,x_{n}] \in \mathbb{P}^n(\bar{k})$. The minimal field of definition for $p$ is the field $$k(p) = k(x_{0}/x_{i},\hdots ,x_{n}/x_{i}) \text{ for any } i \text{ such that } x_{i} \neq 0$$
$k(p) \frac{x_{i}}{x_{j}} = k(x_{0}/x_{j},\hdots ,x_{n}/x_{j})$ is the same as $k(p)$ as $x_{i}/x_{j} \in k(p)$
\end{definition}  

For $\sigma \in G(\bar{k}/k)$ and $p = [x_{0},\hdots ,x_{n}] \in \mathbb{P}^n$, we have the following action $$\sigma(p) = [\sigma(x_{0}),\hdots ,\sigma(x_{n})]$$
This action is well defined as $$\sigma(\lambda p) = [\sigma(\lambda)\sigma(x_{0}),\hdots ,\sigma(\lambda) \sigma(x_{n})] \sim [\sigma(x_{0}),\hdots ,\sigma(x_{n})]$$

\begin{definition}
A polynomial $f \in \bar{k}[X_{0}, \hdots , X_{n}]$ is homogenous of degree $d$ if $$f(\lambda x_{0} , \hdots , \lambda x_{n})  = \lambda^d f(x_{0} , \hdots ,x_{n}) \forall \lambda \in \bar{k}$$
\end{definition}

\begin{definition}
An ideal $I \subseteq \bar{k}[X_{0} , \hdots , X_{n}]$ is called a homogenous ideal if it is generated by homogenous polynomial.  
\end{definition}

\begin{definition}
Let $I \subseteq \bar{k}[X_{0} , \hdots , X_{n}]$ be a homogenous ideal. Then, $$V(I) = \{p \in \mathbb{P}^n (\bar{k}) \mid f(p)=0 \; \forall \; f\in I\}$$
\end{definition}

\begin{definition}
\begin{itemize}
\item A projective algebraic set is any set of the form $V(I)$ for some homogenous ideal $I$. 
\item If $V$ is a projective algebraic set, the homogenous ideal of $V$, denoted by $I(V)$ is the ideal of $\bar{k}[X_{0} \hdots , X_{n}]$ generated by $\{f \in \bar{k}[X_{0} \hdots , X_{n}]\mid f \text{ is homogenous and }f(p)= 0 \; \forall \; p\in V\}$
\item Such a $V$ is defined over $k$, denoted by $V/k$ if its ideal $I(V)$ can be generated by homogenous polynomials $k[X_{0} \hdots , X_{n}]$. 
\item If $V$ is defined over $k$, then the set of $k$-rational points of $V$ is $$V(k) = V \cap \mathbb{P}^{n}(k) = \{p \in V \mid \sigma(p)=p \; \forall \; \sigma \in G(\bar{k}/k)\}$$
\end{itemize}
\end{definition}

\begin{example}
A line in $\mathbb{P}^2$ is given by the equation $aX + bY + cZ=0$ with $a,b,c \in \bar{k}$ and not all $0$ simultaneously. \\
If $c\neq 0$, then such a line is defined over a field containing $a/c, b/c$. \\
More generally, a hyperplane in $\mathbb{P}^n$ is given by an equation $a_{0}X_{0} + \cdots + a_{n}X_{n} = 0$ with all $a_{i} \neq 0$ simultaneously.
\end{example}

\begin{example}
Let $V$ be the projective algebraic set in $\mathbb{P}^2$ given by $X^2 + Y^2 = Z^2$. 
\begin{align*}
\mathbb{P}^1 & \overbrace{\rightarrow}^{\sim} V \\
[s,t] & \mapsto [s^2 - t^2 : 2st : s^2+t^2]
\end{align*}
\end{example}

\begin{remark}
For $p \in \mathbb{P}^n(\QQ)$ you can clear the denominators and then divide by common factor so that $x_{i} \in \ZZ$ and $\gcd(x_{0} , \hdots ,x_{n})=1$. So, $I =(f_{1}, \hdots , f_{m})$ and finding a rational point of $V_{I}$ is same as finding coprime integer solutions to $f_{i}'s$. 
\end{remark}

\begin{example}
$V \subseteq \mathbb{P}^2$ such that $X^2 + Y^2 = 3Z^2$ over $\QQ$. To find $V(\QQ)$, we just need to find integers $a,b,c$ such that $a^2 +b^2 = 3c^2$
\end{example}

\begin{example}
$V : 3X^3 + 4Y^3 + 5Z^3 = 0$. $V(\QQ) = \emptyset$ but for all prime $p$ we have $V(\QQ_{p}) \neq \emptyset$
\end{example}

\begin{definition}
A projective algebraic set is called a projective variety if its homogenous ideal $I(V)$ is prime $\bar{k}[X_{0}, \hdots , X_{n}]$
\end{definition}

Relation between affine and projective varieties: \\

For $ 0 \le i \le n$
\begin{center}
\begin{align*}
\phi_{i}: \mathbb{A}^n &\rightarrow \mathbb{P}^n \\
(Y_{1}, \hdots , Y_{n}) &\mapsto [Y_{1}, \hdots , Y_{i-1}, 1, Y_{i+1} , \hdots , Y_{n}]
\end{align*}
\end{center}
$\mathrm{Im}(\phi) = U_{i} = \{p \in \mathbb{P}^n \mid \; p =[x_{0}: \hdots  : x_{n}] \text{ with } x_{i} \neq 0\} = \mathbb{P}^n \backslash H_{i}$. \\
This process can also be reversed by the following map : 
\begin{center}
\begin{align*}
\phi_{i}^{-1}: U_{i} &\rightarrow \mathbb{A}^n \\
[x_{0}: \hdots : x_{n}) &\mapsto [x_{0}/x_{i}, \hdots , x_{i-1}/x_{i}, x_{i+1}/x_{i} , \hdots , x_{n}/x_{i}]
\end{align*}
\end{center}

Let $V$ be a projective algebraic set with homogenous ideal $I(V) \subseteq \bar{k}[X_{0} ,\hdots , X_{n}]$. Then, $$V \cap \mathbb{A}^n = \phi_{i}^{-1}(V \cap U_{i}) \text{ for fixed } i $$ is an affine algebraic set with $I(V \cap \mathbb{A}^n ) \subset \bar{k}[X_{0} , \hdots , X_{i-1},X_{i+1}, \hdots , X_{n}]$

\begin{definition}
Let $V \subseteq \mathbb{A}^n$ be an affine algebraic set with ideal $I(V)$ and consider $V \subseteq \mathbb{P}^n$ and $\phi_{i}$ defined as before. \\
The projective closure of $V$ is $\bar{V}$ is the projective algebraic set whose homogenous ideal $I(V)$ is generated by $\{f^* \mid f \in I(V)\}$. \\
Here, for $f \in k[X_{0}, \hdots , X_{i-1}, X_{i+1}, \hdots , X_{n}]$ we define $$f^* (X_{0} , \hdots , X_{n}) = X_{i}^d (f(X_{0}/X_{i}, \hdots , X_{i-1}/X_{i}, X_{i+1}/X_{i}, \hdots , X_{n}/X_{i}))$$ with $d = \deg (f)$.
\end{definition}

\begin{definition}
Dehomogenization of $f(X_{0}, \hdots , X_{n})$ with respect to $i$ is $f(X_{0}, \hdots , X_{i-1}, 1 , X_{i+1}, \hdots ,X_{n})$
\end{definition}

\begin{proposition}
\begin{enumerate}
\item Let $V$ be an affine variety. Then $\bar{V}$ is a projective variety and $V = \bar{V} \cap \mathbb{A}^n$. 
\item Let $V$ be a projective variety. Then, $V \cap \mathbb{A}^n$ is an affine variety and either $V \cap \mathbb{A}^n = \emptyset$ or $V = \overline{V \cap \mathbb{A}^n}$. 
\item If an affine (resp. projective) variety $V$ is defined over $k$, then $\bar{V}$ (resp. $V \cap \mathbb{A}^n$) is also defined over $k$.
\end{enumerate}
\end{proposition}

\begin{proof}
\begin{enumerate}
\item 

\item

\item
\end{enumerate}
\end{proof}

\begin{example}
$V : Y^2 = X^3 + 17 \subseteq \mathbb{A}^2 \rightarrow \mathbb{P}^2$ with $(X,Y) \mapsto [X: Y: 1]$. Here, $\overline{V} : Y^2Z =X^3 +17Z^3$ and $\overline{V} \backslash V = \{[0:1:0]\}$
\end{example}

\chapter{Lecture-4 (12th January, 2023): Projective varieties and maps between varieties}

\section{Projective varieties contd..}

\begin{definition}
\begin{itemize}
\item Let $Y / k$ be a projective variety and choose $\Aa^n \subseteq \PP^n$ such that $V \cap \Aa^n \neq \emptyset$. The dimension of $V$ is just dimension of $V \cap \Aa^n$.
\item The function field of $V$, $\bar{k}(V) = \bar{k}(V \cap \Aa^n)$ is the function field for $V \cap \Aa^n$ over $\bar{k}$.
\item Similarly, $k(V) = k(V \cap \Aa^n)$
\begin{align*}
\phi_{i}: \Aa^n \rightarrow \PP^n &\I(V \cap \Aa_{i}^n) \\
\phi_{j}: \Aa^n \rightarrow \PP^n &\I(V \cap \Aa_{j}^n) \\ 
\end{align*}
For different $\phi_{i}$ we obtain $k(V)$s but they are canonically isomorphic to each other. This is because we can just switch $x_{i},x_{j}$ are dehomogenise accordingly. 
\end{itemize}
\end{definition}

\begin{definition}
Let $V$ be a projective variety and $p \in V$. Choose $\Aa^n \subseteq \PP^n$ with $p \in \Aa^n$. Then, $V$ is non-singular (or smooth) at $p$ if $V\cap \Aa^n$ is non-singular at $p$. \\

The local ring of $V$ at $p$, $\bar{k}[V]_{p}$ is just the local ring of $\bar{k}[V \cap \Aa^n]_{p}$
\end{definition}

\begin{remark}
Function field of a projective variety $V$ is field of rational functions $f(X)/g(X)$ such that 
\begin{enumerate}
\item $f,g$ are homogenous of same degree. 
\item $g \in \I(V)$.
\item $f_{1}/g_{1} = f_{2}/g_{2}$ iff $f_{1}g_{2} - f_{2}g_{1} \in \I(V)$ 
\end{enumerate}
Equivalently, take $f,g \in \bar{k}[X]/I(V)$ satisfying $1,2$.\\

\textcolor{BrickRed}{Here, $X$ is just a short form for $(X_{0}, \hdots , X_{n})$}
\end{remark}

\section{Maps between varieties}

\begin{definition}
Let $V_{1},V_{2}\in \PP^n$ be projective varieties. A rational map $$\phi : V_{1} \rightarrow V_{2}$$ $\phi = [f_{0} : \cdots : f_{n}]$ where $f_{i} \in \bar{k}(V_{1})$ such that $\forall p \in V_{1}$ at which $f_{i}$ are defined, we have $$\phi(p)= [f_{0}(p): \cdots : f_{n}(p)]$$
\end{definition}

If $V_{1},V_{2}$ are defined over $k$, we have a Galois action. For $\sigma \in G(\bar{k}/k)$ we have $$\sigma(\phi)(p) = [\sigma(f_{0}): \cdots : \sigma(f_{n})(p)]$$
We can check that $\sigma(\phi(p)) = \sigma(\phi)(\sigma(p))$.\\

\begin{definition}
If $\exists \lambda \in \bar{k}^{\times}$ such that $\lambda f_{i} \in k(V_{1})$, then $\phi$ is said to be defined over $k$.
\end{definition}

\begin{proposition}
$\phi$ is defined over $k$ iff $\phi =\sigma(\phi) \; \forall \; \sigma \in G(\bar{k}/k)$.
\end{proposition}

\begin{definition}
A rational map $\phi: V_{1} \rightarrow V_{2}$ is said to be regular if there exists a function $g \in \bar{k}(V_{1})$ such that 
\begin{enumerate}
\item Each $gf_{i}$ is regular at $p$.
\item There exists some $i$ such that $(gf_{i})(p) \neq 0$
\end{enumerate}
If such a $g$ exists, then we set $$\phi(p) = [(gf_{0})(p) : \cdots : (gf_{n})(p)]$$
\end{definition}

\begin{definition}
A rational map is called a morphism if it is regular everywhere.
\end{definition}

\begin{remark}
Let $V_{1},V_{2} \in \PP^n$ be projective varieties. \\
$\bar{k}(V_{1}) = $ quotient of homogenous polynomials in $\bar{k}[X]$ of same degree.\\ 
A rational map $\phi = [f_{0}, \hdots , f_{n}]$ can be multiplied by a homogenous polynomial to clear denominators and get $\phi = [\phi_{0} , \hdots ,  \phi_{n}]$ such that 
\begin{enumerate}
\item $\phi_{i} \in \bar{k}[X]$ homogenous polynomials not all in $\I(V_{1})$ and have same degree.
\item For all $f \in \I(V_{2})$ we have $f(\phi_{0}(X), \hdots , \phi_{n}(X)) \in \I(V_{1})$. 
\end{enumerate}
\end{remark}

\begin{definition}
A rational map $\phi = [\phi_{0} , \hdots , \phi_{n}]: V_{1} \rightarrow V_{2}$ as  above is regular at $p \in V_{1}$ if there exists homogenous polynomials $\psi_{0},\hdots , \psi_{n} \in \bar{k}[X]$ such that 
\begin{enumerate}
\item $\psi_{i}$s have the same degree 
\item $\phi_{i}\psi_{j} \equiv \phi_{j}\psi_{j} \pmod{\I(V_{1})}$ for all $0  \le i,j \le n$
\item $\psi_{i}(p) \neq 0$ for some $i$.
\end{enumerate}
If this happens, we set $$\phi(p) = [\psi_{0}(p), \hdots , \psi_{n}(p)]$$
\end{definition}

\begin{remark}
Let $\phi= [\phi_{0}, \hdots , \phi_{n}]: \PP^m \rightarrow \PP^n$ be a rational map. $\phi_{i}$s are homogenous polynomials having same degree. We can cancel common factors to assume $\gcd(\phi_{0} , \hdots , \phi_{n})=1$. \\

And, $\phi$ is regular at a point $p \in \PP^n$ iff $\phi_{i}(p) \neq 0$ for some $i$.  \\
So, $\phi$ is a morphism if $\phi_{i}$s have no common zeros in $\PP^n$. 
\end{remark}

\begin{definition}
Let $V_{1},V_{2}$ be two projective varieties. We say that $V_{1},V_{2}$ are isomorphic if there are morphisms $$\phi: V_{1}\rightarrow V_{2}, \psi : V_{2} \rightarrow V_{1}$$ such that $\phi \circ \psi = \mathrm{id}_{V_{2}}, \psi \circ \phi = \mathrm{id}_{V_{1}}$. \\
$V_{1}/k$ and $V_{2}/k$ are isomorphic over $k$ if both maps are defined over $k$.
\end{definition}

\begin{example}
$\mathrm{char}(k) \neq 2$, $V: X^2 + Y^2 = Z^2$. 
\begin{align*}
\phi: V &\rightarrow \PP^2 \\
[X:Y:Z] &\mapsto [X+Z:Y]
\end{align*}
$\phi$ is regular everywhere except $[1:0:1]$\\
Since $(X+Z)(X-Z) \equiv -Y^2 \equiv \pmod{\I(V)}$, we have $[X+Z:Y] = [X^2-Z^2 : Y(X-Z)] = [-Y^2 : Y(X-Z)] = [-Y: X-Z] = \psi$ \\
\begin{align*}
\psi: \PP^1 &\rightarrow V \\
[s:t] &\rightarrow [s^2-t^2 : 2st : s^2 +t^2]
\end{align*}
$\psi\circ \phi$ and $\phi \circ \psi$ are both identity maps.
\end{example}

\begin{example}
\begin{align*}
\phi: \PP^2 &\rightarrow \PP^2 \\
[X:Y:Z] &\mapsto [X^2 : YZ : Z^2]
\end{align*}
 is regular everywhere but $[0:1:0]$ and this cannot be salvaged.
\end{example}

\begin{example}
$V: Y^2Z = X^3 + X^2 Z$
\begin{align*}
\psi: \PP^1 &\rightarrow V \\
[s:t] &\mapsto [(s^2 -t^2)t : (s^2 -t^2)s : t^3]
\phi: V &\rightarrow \PP^1 \\
[X:Y:Z] &\mapsto [X:Y]
\end{align*}
$\phi$ is not regular at $[0:0:1]$. $[0:0:1]$ is a singular point of $V$ which implies $\phi$ cannot be extended. So $\phi \circ \psi$ and $\psi \circ  \phi$ are identities  when they are defined.
\end{example}

\begin{example}
$V_{1}: X^2 + Y^2 = Z^2 , V_{2}: X^2 + Y^2 = 3Z^2$. $V_{1} \not \cong V_{2}$ over $\QQ$ but $V_{1}\cong V_{2}$ over $\QQ(\sqrt{3})$.
\end{example}

\chapter{Lecture-5 (17th January, 2023):}

\section{Curves}

\begin{definition}
A curve is a projective variety of dimension $1$.
\end{definition}

\begin{example}
Vanishing set of an irreducible polynomial in $\PP^2$.
\end{example}

\begin{proposition}
Let $C$ be a curve and $p \in C$ be a smooth (non-singular) point. Then, $\bar{k}[C]_{p}$ is a discrete valuation ring.
\end{proposition}

\begin{proof}
$p \in C$ smooth implies $M_{p}/M_{p}^2$ is one dimensional over $\bar{k}[C]_{p}/M_{p} = \bar{k}$. Now, Nakayama will give us $M_{p}$ is a principal ideal. \\

\textbf{Claim}:
$\bigcap_{n} M_{p}^n = 0$.

\begin{proof}
If $\alpha \in \bigcap_{n} M_{p}^n$, then $\alpha = a_{1}t = a_{2} t^2 = a_{3} t^3 = \cdots $. This implies $a_{1} = a_{2}t = a_{3}t^2 = \cdots $. But this gives us a chain $$\langle a_{1} \rangle \subseteq \langle a_{2} \rangle \subseteq \langle a_{3} \rangle \subseteq \cdots $$ that must terminate at some point. This implies $t$ is an unit which is a contradiction. Hence, we are done.
\end{proof}
\end{proof}

\begin{definition}
Let $C$ be a curve and $p \in C$ a smooth point. The normalised valuation on $\bar{k}[C]_{p}$ is 
\begin{align*}
\mathrm{ord}_{p}: \bar{k}[C]_{p} &\rightarrow \NN \cup \{0, \infty\} \\
f &\mapsto \sup \{d \in \ZZ \mid f \in M_{p}^d \} \\
\mathrm{ord}_{p}(\frac{f}{g}) &= \mathrm{ord}_{p}(f) - \mathrm{ord}_{p}(g)
\end{align*}
Thus we can define $$\mathrm{ord}_{p} : \bar{k}[C] \rightarrow \ZZ \cup \{\infty\}$$
\end{definition}

\begin{definition}
An uniformiser for $C$ at $p$ is any function $t \in \bar{k}(C)$ with $\mathrm{ord}_{p}(t) = 1$ that is the generator of $M_{p}$
\end{definition}

\begin{remark}
If $C$ is defined over $k$, we can find a unit $t \in k(C)$.
\end{remark}

\begin{definition}
Let $C$ be a curve and $p \in C$ a smooth point, $f \in \bar{k}(C)$, $\ord_{p}(f) =$ order of $f$ at $p$.

\begin{enumerate}
\item If $\ord_{p}(f)> 0$, then $f$ has a zero at $p$.
\item If $\ord_{p}(f)< 0$, then $f$ has a pole at $p$.
\item If $\ord_{p}(f)\geq 0$, then $f$ is regular at $p$.
\end{enumerate}
\end{definition}

\begin{proposition}
Let $C$ be a smooth curve and $0 \neq f \in \bar{k}(C)$. Then, there are only finitely many points in $C$ at which $f$ has a pole or $0$. If $f$ has no poles, then $f\in \bar{k}$.
\end{proposition}

\begin{proof}
A standard exercise in Riemann surfaces.
\end{proof}

\begin{example}
Suppose $C_{1}: Y^2 = X^3 + X , C_{2}: Y^2 = X^3 + X^2$. $C_{1}$ is smooth everywhere but $C_{2}$ is smooth everywhere except $p = [0:0:1]$.\\

In $\bar{k}[C_{1}]_{p}$, $M_{p} = \langle X,Y \rangle$ and $X \in M_{p}^2$.
\end{example}

\begin{proposition}
Let $C/k$ be a curve and $p \in C$ be a smooth point, and $t \in k(C)$ an uniformiser at $p$. Then, $k(C)$ is a finite separable extension of $k(t)$.
\end{proposition}

\begin{proof}
$k(C)$ is a finite algebraic extension as it is finitely generated over $k$ and has transcendence degree $0$ over $k(t)$ as $t$ is not algebraic over $k$ (it is a local coordinate of $C$ at $p$). \\

Now, take $x\in k(C)$ and let $\Phi(T,X) = \sum a_{ij}T^{i}X^{j}$ be the minimal polynomial at $x$ over $k(t)$. Say $q = \mathrm{char}(k)$. If $\Phi (T,X)$ is not separable, then $\frac{\partial \Phi(T,X)}{\partial X} = 0$ as $\Phi(T,X)$ is irreducible. 

\begin{align*}
\Phi(T,X) &= \Psi (T,X^p) \\
&= \sum_{k=0}^{q-1} \left( \sum_{i,j} b_{ijk} T^{iq} X^{iq} \right) T^k \\
&= \sum_{k=0}^{q-1} \left( \Phi_{k}(T,X) \right)^q T^k \text{     since $k$ is perfect, every element is a $q$-th power} \\
\sum_{k=0}^{q-1} \left( \Phi_{k}(t,x) \right)^q t^k &=0\\
\ord_{p} (\Phi_{k}(t,x)^q t^k) &\equiv k \pmod{q}
\end{align*}
This implies that every term in the final sum has distinct order at $p$. And, hence $$\Phi_{0}(t,x) = \Phi_{1}(t,x) = \cdots = \Phi_{q-1}(t,x) = 0$$ Atleast one of the $\Phi_{i}$s should have a nonzero power of $X$ and $X - \deg \Phi_{i} < X - \deg \Phi$ and hence $\Phi_{k}(t,x) =0$ which contradicts minimality of $\Phi$. Hence, we are done.
\end{proof}

\section{Morphism between curves}

\begin{proposition}
Let $C$ be a curve, $V \subseteq \PP^n$ be a variety, $p \in C$ a smooth point and $$\phi : C \rightarrow V$$ a rational map. Then, $\phi$ is regular at $p$. In particular, if $C$ is smooth, then $\phi$ is a morphism.
\end{proposition}

\begin{proof}
Suppose $\phi = [f_{0} : \cdots : f_{n}]$ with $f_{i} \bar{k}(C)$ and $t \in \bar{k}(C)$ an uniformiser for $C$ at $p$. Let $$n = \min \ord_{p} f_{i}$$ Then, $\ord_{p}(t^{-n}f_{i}) \geq 0 \; \forall \; i$ and $\ord_{p}(t^{-n}f_{j})=0$ for some $j$. But then this means $t^{-n}f_{i}$ are regular at $p$, $t^{-n}f_{j}(p) \neq 0$ and thus $\phi$ is regular at $p$.
\end{proof}

\begin{remark}
This proposition is not true if either $\dim(C) > 1$ or $p$ is singular 
\begin{enumerate}
\item $\phi: \PP^n \rightarrow \PP^n$ be $[X:Y:Z] \mapsto [X^2 : YZ : Z^2]$ is not regular at $p = [0:1:0]$.
\item Suppose $V : Y^2 Z = X^3 + X^2 Z$ and $V \rightarrow \PP^1$ be given by $[X:Y:Z] \mapsto [Y:X]$ is not regular at $[0:0:1]$.
\end{enumerate}
\end{remark}

\begin{example}
\begin{enumerate}
\item $V: X^2 + Y^2 = Z^2$
\end{enumerate}
\end{example}





















\part{Basic Algebraic Geometry}

\chapter{Lecture-1 (5th January): Introduction}

\chapter{Lecture-2 (10 January, 2023): Ideals and Zariski topology}

\section{Ideals}
For $I,J$ ideals $$I+J = \{x+y \mid x\in I, y\in J\}$$ $$IJ = \{\sum x_{i}y_{i} \mid x_{i} \in I, y_{i} \in J\}$$

\begin{itemize}
\item $IJ \subseteq I \cap J$.
\item If $I+J = R$, then $I^2 + J^2 = R$. This is because, say $I^2 + J^2 \neq R$, then there is a maximal ideal $\mathrm{m}$ such that $I^2 + J^2 \subseteq \mathfrak{m}$. This means $I^2, J^2 \subseteq \mathfrak{m}$. But $\mathfrak{m}$ is prime ideal, therefore $I,J \subseteq \mathfrak{m} \Rightarrow I+J \subseteq \mathfrak{m}$ which is a contradiction. Thus, we are done. 
\item If $\mathfrak{p}$ is a prime ideal and $IJ \subseteq \mathfrak{p}$. Then, $I \subseteq \mathfrak{p}$ or $J \subseteq \mathfrak{p}$. Suppose not, then there exists $x \in I \backslash \mathfrak{p} , y \in I \backslash \mathfrak{p}$. But then $xy \in IJ \subseteq \mathfrak{p}$.
\item $\mathfrak{p} \supseteq I \cap J \Leftrightarrow IJ \subseteq \mathfrak{p}$.
\end{itemize}

\section{Zariski topology}

\begin{definition}
\begin{itemize}
\item For an ideal $I$, let $$V(I) = \{\mathfrak{p} \text{ prime ideal }\mid I \subseteq \mathfrak{p}\}$$
\item $\mathrm{Spec}(R)=\{ \text{ collection of all prime ideals of }R\}$
\end{itemize}
\end{definition}

\begin{definition}[Zariski Topology]
It is the topology defined on $\mathrm{Spec}(R)$ such that the closed sets are $V(I)$.
\end{definition}

Verification that this indeed is a topology. 
\begin{enumerate}
\item $V(0) = \mathrm{Spec}(R), V(R) = \emptyset$.
\item $V(I) \cup V(J) = V(I\cap J) = V(IJ)$.
\item $\bigcap_{k \in k} V_{k} = V(\sum_{k \in K} I_{k})$. This is because $\mathfrak{p} \supseteq I_{k} \Leftrightarrow \mathfrak{p} \supseteq \sum_{k \in K}I_{k}$
\end{enumerate}

Let us now look at the open sets of this topology. The basis for the open sets is given by $$D(f \in R) = \{ \text{ all prime ideals not containing } f\}$$
Clearly, $$(V(I))^c = \bigcup_{f \in I} D(f)$$ and moreover, each $D(f)$ is open since $D(f) = (V(\langle f \rangle))^c$

\begin{theorem}
$\mathrm{Spec}(R)$ is quasi-compact.
\end{theorem}

\begin{proof}
We wish to prove that every open cover has a finite subcover. This is equivalent to saying every cover by $D(f_{i})$ has a finite subcover. Say $$\spec(R) = \bigcup_{i \in I} D(f_{i})$$ Take $J$ to be the ideal generated by $f_{i}'s$. Either $J = R$ or $J \subseteq \mathfrak{m}$. Suppose $J \subseteq \m$, then $f_{i} \in \m \in \spec(R) \Rightarrow \m \not \in D(f_{i}) \; \forall \; i \Rightarrow D(f_{i})$ does not cover $\m$. A contradiction. Therefore, $J=R$ and this implies $1 =$ some linear combination of $f_{i}$ and notice that this sum is finite. So, just consider these finitely many $f_{i}'s$ (say the indexing set is $K$). These cover $J$. Suppose that $\{ D(f_{k}), k \in K\}$ do not cover $\spec(R)$. Then, there is a prime ideal $\pr \not \in \bigcup_{k\in K}D(f_{k}) \Rightarrow \pr \ni f_{k} \; \forall\; k \in K \Rightarrow R \subseteq \pr \Rightarrow \Leftarrow$. Hence, it covers all of $\spec(R)$ as required. \\

\textbf{Another proof:}\\
Suppose $\spec(R) = \bigcup_{j \in J} U_{j} = \bigcup_{j \in J} \spec(R) \backslash \V(I_{j}) = \spec(R) \backslash \bigcap_{j \in J} \V(I_{j}) =\spec(R) \backslash \V(\sum_{j \in J} I_{j})$. This is equivalent to saying that $\V(\sum_{j \in J} I_{j}) = \emptyset$. So, we conclude that $\sum_{j \in J}I_{j} = R \Rightarrow \sum_{k \in K}a_{k} = 1$ for some finite set $K$. We claim that $\{U_{k}: k \in K\}$ covers $\spec(R)$. This is because 
\begin{align*}
\V(\sum_{k\in K} I_{k}) &= 0 \\
\Rightarrow \spec(R) &= \spec(R) \backslash \V(\sum_{k\in K} I_{k})  \\
&= \bigcup_{k \in K} \spec(R) \backslash \V(I_{k}) \\
&= \bigcup_{k \in K} U_{k}
\end{align*}
This completes the proof. 
\end{proof}

\begin{proposition}
Each $D(f)$ is quasi-compact.
\end{proposition}

\begin{proof}
Suppose $$D(f) = \bigcup D(g_{i})$$ and let $J$ be the ideal generated by $g_{i}'s$. Take $\mathfrak{p} \supseteq J$. Then, each $g_{i} \in J \subseteq \mathfrak{p} \Rightarrow \mathfrak{p} \not \in D(g_{i}) \Rightarrow \mathfrak{p} \not \in D(f) \Rightarrow f \in \mathfrak{p} \Rightarrow f \in \bigcap_{\mathfrak{p} \supseteq J} \mathfrak{p}$. \textcolor{BrickRed}{Before completing this proof, we need to understand this intersection much better. Refer to following content on nilpotent elements and come back.}\\
Now, we know that $f \in \mathrm{rad}(J)$ which implies $\exists n$ such that $f^n \in J$. We get $$f^n = \sum_{\text{finite}} r_{i}g_{i}$$ Finally, we claim that these $D(g_{i})$s cover $D(f)$. 
\end{proof}

\begin{definition}
$x\in R$ is nilpotent if $x^n =0$ for some $n \in \NN$.
\end{definition}

\begin{remark}
Any nilpotent element ($x^n = 0$ for some $n$ ) is clearly in every prime ideal ($0 \in \mathfrak{p}$) and thus in the intersection of all prime ideals. This can be recorded as $$\bigcap_{\pr \in \spec(R)} \pr \supseteq \nil(R)$$
\end{remark}

\begin{proposition}
$$\bigcap_{\pr \in \spec(R)} \pr \subseteq \nil(R)$$
\end{proposition}

\begin{proof}
Take an element $x \in R \backslash \nil(R)$ (not nilpotent) and consider the set $$\Sigma = \{ I \unlhd R \mid x^n \not \in I \; \forall \; n >0\}$$
Notice that $\Sigma$ is a poset with respect to inclusion. And every chain $I_{1} \subseteq I_{2} \subseteq I_{3} \subseteq \cdots $ has an upper bound (union of all the ideals). Thus, we can apply Zorn's lemma to get a maximal element $\pr$ which we claim is prime. Indeed, if $ab \in \pr$ but $a\not \in \pr, b \not \in \pr$ then $\pr + \langle a \rangle, \pr + \langle b \rangle$ are ideals strictly containing $\pr$ contradicting maximality of $\pr$. Therefore, we can conclude that $x \not \in \pr \Rightarrow x \not \in \bigcap_{\pr \supseteq J} \pr$ or rather not nilpotent implies not in intersection and hence we have proved the required inclusion. 
\end{proof}

$$\nil (R) = \bigcap_{\pr \in \spec (R)} \pr = \bigcap_{\pr \subseteq \{0\}} \pr$$
$$\{x \mid x^n \in J\} = \mathrm{rad}(J) = \bigcap_{\pr \supseteq J} \pr $$

\chapter{Lecture-3 (12th January): Zariski topology }

\section{Zariski topology contd..}

\begin{definition}
If $J = \mathrm{rad}(J)$, then $J$ is called radical ideal.
\end{definition}

\textbf{Properties:}
\begin{enumerate}
\item Every radical ideal is an intersection of prime ideals.
\item $\V(J) = \V(\mathrm{rad}(J))$ 
\item $\V(J) = \V(J')$ implies $\mathrm{rad}(J) = \mathrm{rad}(J')$
\end{enumerate}

Suppose $S\subseteq R$ such that 
\begin{itemize}
\item $1\in S, 0 \not \in S$
\item If $x,y \in S \Rightarrow xy \in S$
\end{itemize}

\begin{proposition}
Take an ideal maximal wrt not intersecting $S$. Then, it is prime.
\end{proposition}

\begin{proof}
Suppose $\m$ is the ideal in question. Next, suppose $\m$ is not prime which implies  $\exists a,b \in R \text{ such that } ab \in \m$ but $a,b \not \in \m$. Then, $\m + \langle a \rangle \supsetneq \m , \m + \langle b \rangle \supsetneq \m$. But, this means $(\m + \langle a \rangle) \cap S \neq \emptyset \Rightarrow m + ra \in S$ for some $m\in \m , r  \in R$. Similarly, $n + sb \in S$ for some $n \in \m , s \in R$. But, $S$ is multiplicative therefore $(m+ra)(n + sb) \in S \Rightarrow mn + ran + msb + rsab \in S \Rightarrow ((\langle ab \rangle + \m)= \m) \cap S \neq \emptyset$. This is a contradiction. Hence, we are done.  
\end{proof}

\begin{proposition}
Say $J$ is maximal wrt not being principal. Then, $J$ is prime.
\end{proposition}

\begin{proof}
Suppose $\m$ is the ideal in question. Next, suppose $\m$ is not prime which implies  $\exists a,b \in R \text{ such that } ab \in \m$ but $a,b \not \in \m$. Next, we can consider the ideal $I = \m + \langle a \rangle$. By maximality of $\m$, we have $I = \langle c \rangle$ for some $c\in R$. Now, consider $J = \{x \in R \mid xc \in \m\}$. Clearly, $I \subseteq J$. Notice that $c = m + ar$ for some $m \in \m , r \in R$. 
\begin{align*}
bc &= b(m+ar) \\
&= bm + (ba)r \\
\Rightarrow bc &\in \m \\
\Rightarrow b &\in J 
\end{align*} 
This means $b \in J \backslash \m$. Therefore $V$ is also principal and hence $V = \langle d \rangle$. Since $\m \in I$, therefore $m = cr$ for some $r\in R$. But this means that $r\in V \Rightarrow r = r'd$ for some $r' \in R$. Hence, $m = cd r' \in \langle cd \rangle \Rightarrow \m \subseteq \langle cd \rangle$. For the other direction, since $d\in V \Rightarrow cd \in U$. All of these tells us that $\m = \langle cd \rangle$ a contradiction to our assumption. Therefore, $\m$ must be prime. 
\end{proof}


\begin{proposition}
Say $J$ is maximal wrt not being finitely generated. Then, $J$ is prime.
\end{proposition}

\begin{proof}
Suppose $\m$ is the ideal in question. Next, suppose $\m$ is not prime which implies  $\exists a,b \in R \text{ such that } ab \in \m$ but $a,b \not \in \m$. \\

If we now look at $\m + \langle a \rangle$, by our assumption, this ideal is finitely generated by say $u_{1}, \hdots , u_{m}$. 
\end{proof}

\begin{exercise}
Suppose $J$ is maximal wrt not being generated by a cardinal number of generators. Then, $J$ is prime.
\end{exercise}

\begin{definition}
A topological space $X$ is said to be irreducible if it cannot be written as the union of proper closed subsets of $X$
\end{definition}

\section{Identify closed irreducible subsets of $\spec(R)$}

\begin{proposition}
The sets $\V(\pr)$ are exactly the irreducible components of $\spec(R)$.
\end{proposition}

\begin{lemma}
Let $I \subseteq R$ be a radical ideal. If $\V(I)$ is irreducible, then $I$ is prime.  
\end{lemma}

\begin{proof}
Suppose $I$ is not prime. Then there exists $a,b$ such that $ab \in I$ but $a\not \in I$ and $b \not \in I$. Consider a prime ideal $\pr$ that contains $I$, it will also contain $ab$ and thus $\pr$ contains either $a$ or $b$. This is summarised as $$\V(I) = (\V(I)\cap \V(a)) \cup (\V(I) \cap \V(b))$$ Thus $\V(I)$ is union of closed sets. It remains to be shown that the sets are proper in order to conclude that $\V(I)$ is not irreducible. Since $\V(I) \cap \V(a) = \V(I + \langle a \rangle)$ and $a \not \in I$ therefore $\V(I +\langle a \rangle)$ is a proper closed subset of $I$ and same for $b$. This is a contradiction to our hypothesis. So, we are done. 
\end{proof}

\begin{lemma}
$\V(\pr)$ is an irreducible closed subset for $\pr$ prime.
\end{lemma}

\begin{proof}
Suppose $\V(\pr) = V_{1} \cup V_{2}$ with $V_{1},V_{2}$ proper closed subsets of $V(\pr)$. Then there exists ideals $I,J$ such that $\V(\pr) = \V(I) \cup \V(J)$. Since $\pr \in \V(\pr)$ this implies $\pr \in \V(I)$ or $\pr \in \V(J)$. Suppose $\pr \in \V(I)$, then $I \subseteq \pr \Rightarrow \V(\pr) \subseteq \V(I) \Rightarrow \V(\pr) = \V(I)$. This is a contradiction to our assumption and hence we are done. $\V(\pr)$ is irreducible.
\end{proof}

\begin{proposition}
Every irreducible closed subset of $\spec(R)$ has an unique generic point.
\end{proposition}

\begin{proof}
Notice that any irreducible closed subset is of the form $\V(\pr)$. Now, $\V(\pr)$ is the closure of $\pr$. This is because $\mathrm{cl}(\pr)$ is a closed set and hence of the form $\V(I)$ for some ideal $I$. Moreover $\pr \supseteq I$. The biggest ideal $I$ such that $I \subseteq \pr$ is $\pr$ and this gives us what we want because $\V$ reverses inclusions. Therefore, $\mathrm{cl}(\pr) = \V(\pr)$. And, such a generic point is unique for suppose $\V(\pr) = \V(\mathfrak{q})$ then clearly $\pr \subseteq \frak{q}$ and $\frak{q} \subseteq \pr$. So, we are done. 
\end{proof}

To summarise, Zariski topology has the following properties: 
\begin{enumerate}
\item $\spec(R)$ is quasi-compact
\item $\spec(R)$ has a basis of quasi-compact opens which is closed under intersection.
\item Every irreducible closed subset has a generic point.
\end{enumerate}

\begin{theorem}[Hochster]
Any topological space with the $3$ properties is the spectrum of some commutative ring.
\end{theorem}



Suppose $X$ is spectral. Define a new space $X^{*}$ with open sets as finite union of quasi-compact open sets in $X$. This new space is called the Hochster dual.

\begin{theorem}
$X^{*}$ is also spectral.
\end{theorem} 

\begin{proof}

\end{proof}


\chapter{Lecture-5 (17th January, 2023): Noetherian spaces}

\section{Noetherian spaces}

First, let us try to remember all the equivalent definitions of a ring being Noetherian. 

\begin{proposition}
The following are equivalent:
\begin{enumerate}
\item Every ideal is finitely generated. 
\item Every ascending chain of ideals $$I_{1} \subseteq I_{2} \subseteq I_{3} \subseteq \cdots $$ stabilises. 
\item Every non-empty family of ideals has a maximal element. 
\end{enumerate}
\end{proposition}

Nowhere do we use Zorn's lemma, so in some sense, these properties are essentially about some "finite-ness" property. Thus, Noetherian means strong finiteness in some sense. 

\begin{definition}

\end{definition}

\begin{definition}

\end{definition}

\begin{theorem}
A module $M$ over $R$ is Noetherian iff the module is finitely generated and finitely presented. 
\end{theorem}

\begin{proof}

\end{proof}

\begin{proposition}
The direct sum of projective modules is projective. 
\end{proposition}

\begin{proposition}
The direct product of injective modules is injective.
\end{proposition}

A question we can ask is when is the direct sum of injective modules injective. 

\begin{proposition}
Direct sum of injective modules is injective iff the module is Noetherian.
\end{proposition}



\part{Algebraic Geometry I}

\chapter{Lecture-1 (9th January, 2023): Topological properties and Zariski Topology}
\section{Topological properties}

Consider a topological space $X$. 
\begin{definition}
\begin{enumerate}
\item We say $X$ is quasi-compact if every open cover of $X$ admits a finite subcover. 
\item If $f: X \rightarrow Y$ is continuous, we call $f$ quasi-compact if $f^{-1}(V)$ is quasi-compact for all quasi-compact open $V \subseteq Y$.
\end{enumerate}
\end{definition} 

\begin{exercise}
Composition of quasi-compact maps is quasi-compact.\\

Consider the two maps $f: X \rightarrow Y$ and $g: Y \rightarrow Z$. Next, look at the composition $g \circ f : X \rightarrow Z$. For all quasi-compact open $V \subseteq Z$, $(g \circ f)^{-1} (V) = f^{-1} \circ g^{-1} (V)$. Since $g$ is quasi-compact and continuous, $g^{-1}(V)$ is also quasi-compact and open. Similarly, $f$ is also quasi-compact and continuous, therefore $f^{-1}(g^{-1}(V))$ is also quasi-compact and we are done.
\end{exercise} 

\begin{lemma}
$X$ quasi-compact and $Y \subseteq X$ is closed implies $Y$ is quasi-compact.
\end{lemma}

\begin{proof}
Let $\{U_{i}\}_{i \in I}$ be an open cover of $Y$. Set $U = X - Y$. Since $U_{i}$ is open in $Y$, we have $U_{i} = Y \cap V_{i}$ where $V_{i}$ is open in $X$. Now we note that $\{V_{i}\}_{i \in I} \cup U$ covers $X$ but $X$ is quasi-compact and we obtain a finite subcover $\{V_{i}\}_{i \in J} \cup U$ where $J$ is finite. The corresponding $U_{i}, i \in J$ must therefore cover $Y$ and we are done.
\end{proof}

\begin{proposition}
If $X$ is quasi-compact and Hausdorff, then $E \subseteq X$ is quasi-compact iff $E$ is closed.
\end{proposition}

\begin{proof}
$\Leftarrow$ direction is done. \\
$\Rightarrow$ direction is what we need to prove. \\
Take $x\in X \backslash E$. For each $y \in E$, due to Hausdorff-ness we have two disjoint open sets $U_{y}$ and $U_{y}$ containing $x$ and $y$ respectively. Do this for all $y \in E$. The collection $\{U_{y}\}_{y \in E}$ covers $E$ but it is quasi-compact thus we get a finite subcover $\{U_{y_{i}}\}_{i \in I}$ with $I$ finite. Now, let $$U = \bigcap_{i \in I} U_{y_{i}}$$ $U$ is clearly open, contains $x$ and is disjoint from $E$. Since $x$ was chosen arbitrarily, $X\backslash E$ must be open. 
\end{proof}

\begin{lemma}
Any finite union of quasi-compact spaces is quasi-compact.
\end{lemma}

\begin{proof}
Suppose $X_{i}, i =1,2,\hdots ,n$ are the spaces in question. We want to show that $$X = \bigcup_{i=1}^n X_{i}$$ is also quasi-compact. Take any cover $\{U_{i}\}_{i \in I}$  be an open cover of $X$. Then for each $i=1,2, \hdots , n$ it is clear that $\{U_{i}\}_{i \in I}$ also covers $X_{i}$. Using quasi-compactness of $X_{i}$ we can get a finite subcollection $\{U_{i_{j}}: j =1 , \hdots , n_{i}\}$. This can be done for all $i$. Now, consider $\bigcup_{i=1}^{n} \bigcup_{j=1}^{n_{i}} U_{i_{j}}$. This union covers $X$ and is finite. So, we are done. 
\end{proof}

\begin{lemma}
Suppose $f: X \rightarrow Y$ is continuous, if $X$ is quasi-compact then so is $f(X)$.
\end{lemma}

\begin{proof}
Let $\{U_{i}\}_{i \in I}$ be an open cover of $f(X)$. Now, $\{f^{-1}(U_{i})\}_{i \in I}$ covers $X$ and by continuity, each of them are open. Use quasi-compactness of $X$ to get a finite subcover that covers $X$.
\begin{align*}
X &= \bigcup_{i=1}^n f^{-1}(U_{i}) \\
\because f(f^{-1}(U_{i})) &\subseteq U_{i} \\
\therefore f(X) &\subseteq \bigcup_{i=1}^n U_{i} 
\end{align*}
\end{proof}

Suppose $\Sigma$ is a poset. $\Sigma$ satisfies acc if every ascending chain $$x_{1} \le x_{2} \le \cdots $$ is stationary.

\begin{lemma}
The following are equivalent: 
\begin{enumerate}
\item $\Sigma$ satisfies acc. 
\item Every non-empty subset of $\Sigma$ has maximal element.
\end{enumerate}
\end{lemma}

\begin{proof}
$1 \Rightarrow 2$. Suppose $S \subseteq \Sigma$ has no maximal element. \\
Then choose $x_{0} \in S$ non-maximal, then we can find a $x_{1}$ such that $x_{0} \lneq x_{1}$. By induction we can construct an infinite chain $x_{0} \lneq x_{1} \lneq \cdots \neq x_{i} \lneq \cdots$ which does not terminate which is a contradiction to our hypothesis. Thus, $S$ must have a maximal element.\\
$2 \Rightarrow 1$. Suppose $x_{1} \le x_{2} \le \cdots \le x_{i} \le $ is an infinite ascending chain, then $S = \{x_{i}\mid i \geq 1\}$ has no maximal element. 
\end{proof}

\begin{definition}
A topological space is called Noetherian if set of all closed subsets of $X$ satisfies dcc.
\end{definition}

\begin{lemma}
$X$ Noetherian implies $X$ is quasi-compact.
\end{lemma}

\begin{proof}
Let $\mathcal{U}=\{U_{i}\}_{i \in I}$ be an open cover of $X$ that does not have a finite subcover. Consider the collection $\mathcal{F}$ of union of finite number of elements of $\mathcal{U}$. Since being Noetherian is equivalent to saying any finite subset of open subsets has a maximal element, we know that $\mathcal{F}$ has a maximal element. Suppose that maximal element is $U_{i_{1}} \cup \hdots \cup U_{i_{n}}$. If this does not cover $X$, take an element $x$ in the complement of the maximal element. Since $\mathcal{U}$ covers $X$, there is an $i \in I$ such that $x \in U_{i}$. Notice that now $U_{i_{1}} \cup \hdots \cup U_{i_{n}} \subseteq U_{i_{1}} \cup \hdots \cup U_{i_{n}} \cup U_{i}$ which contradicts the maximality. Thus, we are done.  
\end{proof}

\begin{remark}
The converse need not be true. Consider $[0,1]$ covered by $[1/2^n , 1]$.
\end{remark}

\begin{lemma}
If $X_{1}, \hdots , X_{n}$ are Noetherian subspaces of $X$, then so is $X=X_{1} \cup X_{2} \cup \hdots \cup X_{n}$
\end{lemma}

\begin{proof}
Let $Y_{i}$s be closed in $X$ that forms the chain $$X \supseteq Y_{1} \supseteq Y_{2} \supseteq Y_{3} \supseteq \cdots $$ For each $i$, we get a chain of closed sets in $X_{i}$ by intersecting with $X_{i}$. This gives us $$X_{i} \supseteq Y_{1}\cap X_{i} \supseteq Y_{2}\cap X_{i} \supseteq Y_{3}\cap X_{i} \supseteq \cdots $$ Since $X_{i}$ is Noetherian, this chain terminates at say $r_{i}$. Now, take $r = \max_{i} r_{i}$. The original chain will terminate after this point. Suppose $y \in Y_{i}$ with $i \le r$, there is an $j$ such that $y \in X_{j}$. This means $y \in X_{j} \cap Y_{i} = X_{j} \cap Y_{r}$. Hence, $y \in Y_{r}$ and we are done.
\end{proof}

\begin{definition}
Locally Noetherian means every point $x\in X$ has a neighbourhood $U$ which is Noetherian wrt subspace topology.
\end{definition}

\begin{lemma}
Quasi-compact and locally Noetherian implies Noetherian.
\end{lemma}

\begin{proof}
Since $X$ is locally Noetherian, for each $x\in X$ we have a nbd. $U_{x}$ that is Noetherian. $\{U_{x}\}_{x \in X}$ is an open cover of $X$. Quasi-compactness gives us a finite subcover $\{U_{x_{i}}\}_{i=1}^n$, i.e., $$X = \bigcup_{i=1}^n U_{x_{i}}$$ $X$ is Noetherian from previous lemma.
\end{proof}

\begin{exercise}
Give an example of a ring $R$ such that $\spec(R)$ is Noetherian but $R$ is not. \\

Consider the ring $R=k[X_{1}, X_{2}, \hdots , ]$ and the ideal $I = \langle X_{1}^2 , X_{2}^2 , \hdots , \rangle$. Now, look at $R'=R/I$. $\spec(R')$ is a singleton. 
\end{exercise}

\begin{definition}
A topological space $X$ is called irreducible if it cannot be written as finite union of proper closed subsets. \\

A closed subset $Y \subseteq X$ is called irreducible component of $X$ if it is a maximal irreducible closed subset of $X$.
\end{definition}

\begin{lemma}
If $X$ is Noetherian and $Y \subseteq X$ is a subspace, then $Y$ is Noetherian. 
\end{lemma}

\begin{proof}
Let $Y_{i}$s be closed in $Y$ that forms the chain $$Y \supseteq Y_{1} \supseteq Y_{2} \supseteq Y_{3} \supseteq \cdots $$ For each $i$, we have a closed set in $X$ such that $Y_{i} = Y \cap X_{i}$. This gives us $$Y \supseteq X_{1}\cap Y \supseteq X_{2}\cap Y \supseteq X_{3}\cap Y \supseteq \cdots $$
\end{proof}

\begin{lemma}
Let $X$ be Noetherian. Then, $X$ has finitely many irreducible components. 
\end{lemma}

\begin{proof}
More generally, we will show that every closed subset for $X$ has finitely many irreducible components. \\

Suppose that this is false. Let $\Sigma$ be the collection of closed subsets of $X$ that does not satisfy our condition. Order this as follows: $A \le B$ if $A \supseteq B$. If $\{C_{i}\}$ is a chain in $\Sigma$, then it must eventually stabilise since $X$ is Noetherian. This $C_{\alpha}$ is an upper bound for this chain. Therefore, by Zorn's lemma, there is a maximal element $Y$. Since $Y \in \Sigma$, therefore it is not irreducible. Suppose $Y = Y_{1} \cup Y_{2}$ with $Y_{1},Y_{2}$ proper closed subsets of $Y$. $Y \le Y_{1}, Y \le Y_{2}$. Since $Y \in \Sigma$, $Y$ is not a finite union of irreducible components. Hence, either $Y_{1}$ or $Y_{2}$ is not irreducible. If $Y_{1}$ is not irreducible but $Y_{1} \in \Sigma$, since $Y$ is maximal in $\Sigma$ and $Y \le Y_{1}$, therefore $Y =Y_{1}$ a contradiction that $Y_{1}$ is a proper subset of $Y$. Thus, $\Sigma$ must be empty and the claim is proven. 
\end{proof}

\begin{lemma}
$X$ is Noetherian implies there exists an unique expression $X = X_{1} \cup \cdots \cup X_{n}$ where $X_{i}'s$ are irreducible components of $X$.
\end{lemma}

\begin{proof}
Suppose $$X = X_{1} \cup \cdots \cup X_{n} = X_{1}' \cup \cdots \cup X_{m}'$$ Clearly $X_{1}' \subseteq X$, this means $X_{1}' = \bigcup_{i=1}^n X_{1}' \cap X_{i}$. Since $X_{1}'$ is irreducible, there must be a $i_{1}$ such that $X_{1}' = X_{i_{1}} \cap X_{1}'$. Thus, $X_{1}' \subseteq X_{i_{1}}$. We can choose $i_{1}$ to be $1$ to get $X_{1}' \subseteq X_{1}$. Similarly, $X_{1} \subseteq X_{j_{1}}'$. Since $X_{1}' \subseteq X_{j_{1}}'$ and our assumption that $X_{i} \not \in X_{j}$ for $i \neq j$ we conclude that $j_{1}=1$. Finally, we conclude that $X_{1}=X_{1}'$. Let $Z$ be the closure of $X - X_{1}$, then $Z = X_{2} \cup \cdots \cup X_{n} = X_{2}' \cup \cdots \cup X_{m}' $. We can argue inductively and conclude that $X_{i} = X_{i}'$ and $n=m$. 
\end{proof}

\begin{lemma}
Suppose $X$ is Noetherian and $X_{1} \subseteq X$ an irreducible component. Then, $X_{1}$ contains a non-empty open set in $X$.
\end{lemma}

\begin{proof}
Consider $U = X \backslash X_{2} \cup \cdots \cup X_{n}$. Clearly, $U$ is non-empty and open. Moreover, $U \subseteq X_{1}$ and we are done. 
\end{proof}

\begin{definition}
Let $X$ be a topological space. We say that $X$ is a spectral space if the following holds: 
\begin{enumerate}
\item $X$ is quasi-compact.
\item $X$ is $T_{0}$.
\item $X$ has a basis of quasi-compact open sets.
\item Every irreducible closed subset of $X$ has a generic point ($\exists x \in Y$ such that $\overline{\{x\}}  = X$)
\end{enumerate}
\end{definition}

\section{Zariski Topology}
Let $A$ be a commutative ring with identity and $X = \spec(A)$. \\

Zariski topology is the unique topology such that a subset $Y\subseteq X$ is closed iff $Y = \V (I)$ for some ideal $I \unlhd A$. Here, $$\V(I) = \{\pr \in X \mid \pr \supseteq I\}$$

\begin{theorem}
$\spec(A)$ is always spectral.
\end{theorem}

\begin{proof}
\begin{enumerate}
\item $X$ is $T_{0}$ \\
For all $f \neq 0$ in $A$,  let $A_{f} = S^{-1}A$ be the localisation of $A$ at $f$ where $A_{f} = \{f^n \mid n \geq 0\}$. Next, let $V_{f} = X \backslash V(f) = \spec (A_{f})$. This forms a basis for the Zariski topology. \\
Now, let $\pr, \mathfrak{P}$ be two distinct primes. 
\begin{itemize}
\item Suppose $\pr \not \subseteq \mathfrak{P}$. \\
$Y = V(\pr)$ is closed set and $\mathfrak{P} \not \in V(\pr)$. Take $Y^c$. Then $\mathfrak{P} \in Y^c$ and $\pr \not \in Y^c$. 
\item If $\pr \subseteq \mathfrak{P}$ \\
Then consider $\V(\mathfrak{P})$. Clearly, $\pr \not \in \V(\mathfrak{P})$. Take $U = \V(\mathfrak{P})^c$, then $\pr \in U$ but $\mathfrak{P} \not\in U$. 
\end{itemize}
\item $X$ is quasi-compact. \\
Let $\{U_{i}\}$ be an open cover of $X$. WLOG, we can assume that $U_{i} = \spec(A_{f_{i}}), f \neq 0$. Let $I$ be the ideal generated by these $f_{i}s$. \\
\textbf{Case-1}: Suppose that $I \neq A$. Then there exists a maximal ideal $\m \supseteq I \Rightarrow \V(\m) \subseteq \V(I) \Rightarrow X \backslash \V(\m) \supseteq X\backslash \V(I) = X \backslash \bigcap_{i \in I}\V(f_{i}) = \bigcup U_{i} = X$ which is absurd. Hence, we conclude that $I=A$. Next, 
\begin{align*}
1 &= \sum_{i=1}^n a_{i}f_{i} &\text{ for some } a_{i} \in A \\
\Rightarrow \bigcup_{i=1}^n U_{i} &= \bigcup_{i=1}^n X \backslash \V(f_{i})
\end{align*}
And, we get the required refinement. 
\item $X$ has a basis of quasi-compact open sets follows from the above.
\item Let $Y \subseteq X$ be an irreducible closed subset. Then, $Y = \spec(A/I)$. WLOG, we can assume $X$ is irreducible. Next, observe that $\spec(A) = \spec(A_{\mathrm{red}}) = \spec(A/ \nil(A))$. Since $A$ is irreducible and reduced, we conclude that $A$ is an integral domain. We are now done since $0$ is a generic point in that case.
\end{enumerate}
\end{proof}

\chapter{Lecture-2 (11th January, 2023): Zariski topology and affine schemes}

\section{Zariski topology contd..}

\begin{theorem}[Hochster]
Every spectral space is homeomorphic to $\spec(A)$ for some commutative ring $A$.
\end{theorem}

\textbf{Notation:} $\mathrm{\textbf{Ring}}$ be the category of commutative rings, $\mathrm{\textbf{Top}}$ be the category of topological spaces.

\begin{theorem}
There is a contravariant functor 
\begin{align*}
sp: \mathrm{\textbf{Ring}} &\rightarrow \mathrm{\textbf{Top}} \\
\spec(B) &\mapsto \spec(A)
\end{align*}
\end{theorem}

\begin{proof}
Consider $f: A \rightarrow B$. This induces a map $$f_{\#} : \spec(B) \rightarrow \spec(A)$$ such that $f_{\#}(\pr) = f^{-1}(\pr)$. \\
\textbf{Well-defined:} Suppose $xy \in f^{-1}(\pr) \Rightarrow f(xy) = f(x)f(y) \in \pr \Rightarrow$ either $x$ or $y$ lies in $f^{-1}(\pr)$ which completes our check. \\
We claim that $f_{\#}$ is continuous. This can be seen as follows: \\
Take a basic open set $D(a), a\in A$. Enough to show for these sets since $D(a)$ forms a basis for the topology on $\spec(A)$. Now, $$\pr \in f_{\#}^{-1}(D(a))\Leftrightarrow f_{\#}(\pr) \in D(a) \Leftrightarrow a \not \in f^{-1}(\pr)$$ But this means $$a \not \in f^{-1}(\pr) \Leftrightarrow f(a) \not \in \pr \Leftrightarrow \pr \in D(f(a))$$
\end{proof}

\section{Affine schemes}

\begin{definition}
$\spec(A)$ will be called an affine "scheme" (we will see this properly later on).
\end{definition}

\begin{definition}
Let $X= \spec(A), Y = \spec(B)$. Let $f: Y \rightarrow X$ be a continuous map. We call such a map $f$ regular (holomorphic) if there is a ring homomorphism $g: A \rightarrow B$ such that $f = g_{\#}$
\end{definition}

\begin{example}
Take $\spec(\ZZ)$ and consider the constant map. This cannot be regular because any ring homomorphism must take $1$ to $1$ and as a consequence fixes every element. 
\end{example}

\begin{proposition}
If $X = \spec(A)$. A regular function on $X$ is a regular map from $X$ to $\spec(\ZZ[t])$.
\end{proposition}

\begin{proof}

\end{proof}

\begin{remark}
On an affine scheme, the set of all regular maps is the ring $A$ itself since, the map $\ZZ[t] \rightarrow A$ is determined by where $t$ is sent to. 
\end{remark}

\begin{lemma}
Every affine scheme has a closed point.
\end{lemma}

\begin{proof}
Every commutative ring has a maximal ideal. 
\end{proof}

\begin{definition}
Open in affine is called quasi-affine.
\end{definition}

\begin{example}
Take $A$ a local integral domain with $\m$ the maximal ideal. Suppose that all prime ideals of $A$ are of the form $$\langle 0 \rangle \subset \pr_{1} \subset \pr_{2} \subset \cdots \subset \{ \m \}$$ Consider $X = \spec(A) \backslash \m$. $X$ is open in affine scheme but has no closed point. \\

An example of such a ring is $$\Gamma = \ZZ x_{1} \oplus \ZZ x_{2} \oplus \cdots $$ Give an ordering: $\sum a_{i}x_{i} \geq 0$ if the first nonzero term is $>0$ or all $a_{i}=0$\\
$\Gamma$ is a totally ordered abelian group and hence there exists a valuation ring $A$ with value group $\Gamma$ and the prime ideals of $\Gamma$ are in $1-1$ correspondence with prime ideals of $A$.
\end{example}

\begin{exercise}
Let $A = k[X_{1}, X_{2}, \hdots ], B= A_{\m}, X = \spec(B) \backslash \m , \m = \langle X_{1}, X_{2}, \hdots , \rangle$. Claim is that $X$ has no closed point.
\end{exercise}

\subsection{Fiber products of affine schemes}

Suppose $A$ is a commutative ring, $B,C$ are $A$-algebras. Let $X= \spec(A), Y = \spec(B), Z = \spec(C)$. Next, suppose we have 
\[\begin{tikzcd}
	A & B \\
	C
	\arrow["f", from=1-1, to=1-2]
	\arrow["g"', from=1-1, to=2-1]
\end{tikzcd}\]

\textbf{Universal property of fiber products}:
\[\begin{tikzcd}
	{W'} \\
	& Y \times_{X} Z & Z \\
	& Y & X
	\arrow["{f_{\#}}"', from=3-2, to=3-3]
	\arrow["{g_{\#}}", from=2-3, to=3-3]
	\arrow[from=2-2, to=2-3]
	\arrow[from=2-2, to=3-2]
	\arrow[curve={height=12pt}, from=1-1, to=3-2]
	\arrow[curve={height=-12pt}, from=1-1, to=2-3]
	\arrow["{\exists \; ! }", dashed, from=1-1, to=2-2]
\end{tikzcd}\]

\begin{definition}
If a $W$ exists such that the universal property is satisfied, then $W$ is called the fiber product of $Y,Z$ over $X$ and we write $W = Y \times_{X} Z$
\end{definition}

\begin{theorem}
$\mathrm{\textbf{Aff}_{\ZZ}}=$ category of affine schemes admits fiber products.
\end{theorem}

\begin{proof}
Consider the following data: 
\[\begin{tikzcd}
	A & B \\
	C
	\arrow["f", from=1-1, to=1-2]
	\arrow["g"', from=1-1, to=2-1]
\end{tikzcd}\]

Let $D = B \otimes_{A} C$. We have the natural maps $f_{1}: B \rightarrow  B \otimes_{A} C$ sending $b \mapsto b \otimes 1$ and $f_{2}: C \rightarrow B \otimes_{A} C$ sending $c \mapsto 1 \otimes c$. Both are ring homomorphisms and fit into the following diagram due to the nature of tensor product 
 \[\begin{tikzcd}
	A & B \\
	C & {B\otimes_{A} C}
	\arrow["f", from=1-1, to=1-2]
	\arrow["g"', from=1-1, to=2-1]
	\arrow["{f_{1}}", from=1-2, to=2-2]
	\arrow["{g_{1}}"', from=2-1, to=2-2]
\end{tikzcd}\]

Now, let $W = \spec(B \otimes_{A} C)$ and we claim that this satisfies the universal property of fibre product. Apply $\spec(-)$ functor to the diagram to get 
\[\begin{tikzcd}
	A & B \\
	C & {\mathrm{Spec}(B\otimes_{A} C)}
	\arrow["{g_{1\#}}"', tail reversed, no head, from=2-1, to=2-2]
	\arrow["{f_{1\#}}"', from=2-2, to=1-2]
	\arrow["{f_{\#}}"', from=1-2, to=1-1]
	\arrow["{g_{\#}}", from=2-1, to=1-1]
\end{tikzcd}\]

From the universal property of tensor product we have the following diagram

\[\begin{tikzcd}
	A & B \\
	C & {B \otimes_{A}C} \\
	&& U
	\arrow["f", from=1-1, to=1-2]
	\arrow["g"', from=1-1, to=2-1]
	\arrow["{f_{1}}", from=1-2, to=2-2]
	\arrow["{g_{1}}"', from=2-1, to=2-2]
	\arrow[curve={height=18pt}, from=2-1, to=3-3]
	\arrow[curve={height=-18pt}, from=1-2, to=3-3]
	\arrow["{\exists \; ! }",dashed, from=2-2, to=3-3]
\end{tikzcd}\]

Again, apply the $\spec(-)$ functor. 

\[\begin{tikzcd}
	X & Y \\
	Z & {\mathrm{Spec}(B\otimes_{A} C)} \\
	&& {\mathrm{Spec}(U)}
	\arrow["{g_{1\#}}"', tail reversed, no head, from=2-1, to=2-2]
	\arrow["{f_{1\#}}"', from=2-2, to=1-2]
	\arrow["{f_{\#}}"', from=1-2, to=1-1]
	\arrow["{g_{\#}}", from=2-1, to=1-1]
	\arrow[curve={height=-24pt}, from=3-3, to=2-1]
	\arrow[curve={height=18pt}, from=3-3, to=1-2]
	\arrow["{\exists \; ! }",dashed, from=3-3, to=2-2]
\end{tikzcd}\]

This completes the proof.
\end{proof}

\chapter{Lecture-3 (16th January, 2023): Category theory brushup}

Suppose we have a ring homomorphism $f : A \rightarrow B$ and $X = \spec(A), Y = \spec(B)$. This induces a map $f_{\#}: Y\rightarrow X$. From, the previous discussion, there is a fiber product $Y \times_{X} Y$ such that the following diagram makes sense

\[\begin{tikzcd}
	X & Y \\
	Y & Y \times_{X} Y \\
	&& Y
	\arrow["{p_{1}}"', tail reversed, no head, from=2-1, to=2-2]
	\arrow["{p_{2}}"', from=2-2, to=1-2]
	\arrow["{f_{\#}}"', from=1-2, to=1-1]
	\arrow["{f_{\#}}", from=2-1, to=1-1]
	\arrow[curve={height=-24pt}, from=3-3, to=2-1]
	\arrow[curve={height=18pt}, from=3-3, to=1-2]
	\arrow["{\exists \; ! \; \Delta_{Y} }",dashed, from=3-3, to=2-2]
\end{tikzcd}\]

Here, $p_{1} \circ \Delta_{Y} = p_{2} \circ \Delta_{Y} = \mathrm{id}$ where $$\Delta_{Y} : Y \rightarrow Y\times_{X} Y$$ is called the relative diagonal of $Y/X$. 

\begin{definition}
Say $X_{1},X_{2}$ are affine schemes. $X_{1} \rightarrow X_{2}$ is a closed immersion iff $A_{1} \rightarrow A_{2}$ is a surjective. Here, $\spec(A_{i}) = X_{i}, i =1,2$. 
\end{definition}

\begin{lemma}
$\Delta_{Y}$ is a closed immersion.
\end{lemma}

\begin{proof}
$B \otimes_{B} B \rightarrow B$ is a surjection. 
\end{proof}

\begin{example}
Take $A = \ZZ, B = \ZZ[t]/ \langle t^n \rangle$ for some $n \geq 2$. There is a canonical inclusion $f: A \rightarrow B$. This induces a map $Y = \spec(B) \rightarrow X = \spec(A)$ which is an identity map in terms of sets. Thus, it is a closed inclusion but not a closed immersion.  
\end{example}

\begin{remark}
We know that diagonal is closed iff the space is Hausdorff. This seems to contradict our assumptions! But we are fine because this claim is true only when the topology is the product topology. Here, the topology we have is not the product topology. 
\end{remark}

\begin{definition}
A regular map $f: X \rightarrow Y$ is called separated morphism if the relative diagonal of $Y$ over $X$ is closed in $Y \times_{X} Y$.
\end{definition}

\begin{lemma}
Let $X = \spec(A)$. Suppose $U_{1},U_{2}$ are two open affine subsets of $X$. Then, $U_{1}\cap U_{2}$ is also affine.
\end{lemma}

\begin{proof}
We have two natural injections $$U_{1} \xhookrightarrow{j_{1}} X , U_{2} \xhookrightarrow{j_{2}} X$$ then we naturally have the following $$U_{1} \times_{Z} U_{2} \xrightarrow{j_{1} \times j_{2}} X \times_{Z} X $$ where $Z = \spec(\ZZ)$ (if it is blank, just assume $Z$ by default). \\
From previous discussion we get 
\[\begin{tikzcd}
	{U_{1}\times_{Z}U_{2}} & {} & {X\times_{Z}X} \\
	{} && {} \\
	&& X
	\arrow["{j_{1}\times j_{2}}", from=1-1, to=1-3]
	\arrow["{\Delta_{X}}"', from=3-3, to=1-3]
\end{tikzcd}\]
Since each term is affine, we can take the fiber product of $U_{1}\times_{Z}U_{2}$ and $X$. Say the fiber product is $W$.
\[\begin{tikzcd}
	& {U_{2}} && X \\
	{U_{1}} & {U_{1}\times_{Z}U_{2}} & {} & {X\times_{Z}X} & X \\
	& {} & \square & {} \\
	& W && X
	\arrow["{\Delta_{X}}"', from=4-4, to=2-4]
	\arrow["{j_{1}\times j_{2}}", from=2-2, to=2-4]
	\arrow["{\Delta'}", from=4-2, to=2-2]
	\arrow["j"', from=4-2, to=4-4]
	\arrow["{p_{1}}"', from=2-4, to=1-4]
	\arrow["{p_{2}}"', from=2-4, to=2-5]
	\arrow["{q_{2}}"', from=2-2, to=2-1]
	\arrow["{q_{1}}"', from=2-2, to=1-2]
\end{tikzcd}\]

Then, we claim that \\

\textbf{Claim}: $W = U_{1} \cap U_{2}$

\begin{proof}
Suppose $x\in W$, then 
\end{proof}

It now remains to show that $W$ is affine but it is clear from the definition of fiber products.

\end{proof}

\begin{remark}
If $\Delta_{X}$ is a closed immersion then so is $\Delta '$. That is, closed immersions are preserved under fiber products. Follows from right exactness of tensor product. 
\end{remark}

Now, that we have discussed intersection, we naturally ask : What happens to $U_{1} \cup U_{2}$. Is it still affine ? \\

The answer turns out to be NO. To see this,

\begin{example}[NON-example]
Consider $k$ be an algebraically closed field. $A  = k [t_{1},t_{2}]$ and $X =\spec(A)$. Let $U_{i}  = \{x \mid t_{i}(x) \neq 0\} = X \backslash \V(t_{i})$. Clearly, $U_{i}$ is open and affine ($ = \spec(A_{t_{i}})$). But $U_{1} \cup U_{2} $ is not affine. 
\[\begin{tikzcd}
	&& {} \\
	&& {} \\
	{} & {} && {} & {t_{2}=0} \\
	\\
	&& {t_{1}=0} \\
	&& {}
	\arrow[tail reversed, from=1-3, to=5-3]
	\arrow[tail reversed, from=3-5, to=3-1]
\end{tikzcd}\]
$U_{1}$ is complement of the horizontal axis and $U_{2}$ of the vertical axis. But $U_{1}\cup U_{2}$ is the complement of origin. The question is asking if the complement of origin is affine or not. A highly NON-TRIVIAL question to answer.
\end{example}

\begin{exercise}[not trivial but do think about it]
Suppose $X = \spec(A)$ and $U \hookrightarrow X$ is affine open. Does this imply $U = \spec(S^{-1}A)$ for some multiplicatively closed set $S \subseteq A$?
\end{exercise}

\begin{definition}
Suppose $S = \spec(A)$ and $x\in X$. Let $K(A) = S^{-1}(A)$ where $S$ is the set of all nonzero divisors in $A$. Here, we have $A \hookrightarrow S^{-1}(A)=$ the ring of all meromorphic functions on $X$. Then, $$\mathcal{O}_{X,x} = \{f \in K(A) \mid f \text{ is regular in a nbd of } x \}$$ is called the germ of regular function. 
\end{definition}

\begin{lemma}
$$\mathcal{O}_{X,x} = A_{\pr}$$ where $\pr =x$.
\end{lemma}

\begin{proof}
Suppose $f$ is regular in a nbd of $\pr$ iff there exists $b \not \in \pr $ such that $f \not \in \V(b)$. But this means $f \not \in A_{b}$ which in turn implies $f \in \bigcup_{b \not \in \pr} A_{b} = A_{\pr}$.
\end{proof}

\begin{definition}
The germs of analytic functions at $x$ is the completion of $\mathcal{O}_{X,x}$, denoted by $\mathcal{O}_{X,x}^{\wedge}$ with respect to its maximal ideal. 
\end{definition}

\begin{remark}
We have the natural map $\mathcal{O}_{X,x} \rightarrow \mathcal{O}_{X,x}^{\wedge}$ but if $\mathcal{O}_{X,x}$ is Noetherian then this map is also injective. 
\end{remark}

\section{Categories and functors}

A category $\mathcal{C}$ consists of a collection $\mathrm{ob}(\mathcal{C})$ and for all $X,Y \in \mathrm{ob}(\mathcal{C})$, there is a set $\mathrm{Hom}_{\mathcal{C}}(X,Y)$ and a map $$\mathrm{Hom}_{\mathcal{C}}(X,Y) \times \mathrm{Hom}_{\mathcal{C}}(Y,Z) \rightarrow \mathrm{Hom}_{\mathcal{C}}(X,Z)$$ satisfying 
\begin{enumerate}
\item $\forall \; X \in \mathrm{ob}(\mathcal{C}) \; \exists \; 1_{X} \in \mathrm{Hom}_{\mathcal{C}}(X,Y)$ such that $f \circ 1_{X} = 1_{X} \circ f = f$
\item $f \circ (g \circ h) = (f \circ g)\circ h$
\end{enumerate}

A functor (contravariant) $\mathcal{F}: \mathcal{C}_{1} \rightarrow \mathcal{C}_{2}$ is a function $\mathcal{F}: \mathrm{ob}(\mathcal{C}_{1}) \rightarrow \mathrm{ob}(\mathcal{C}_{2})$ and a map of sets $\mathcal{F}: \mathrm{Hom}_{\mathcal{C}_{1}}(X,Y)\rightarrow \mathrm{Hom}_{\mathcal{C}_{2}}(\mathcal{F}(X),\mathcal{F}(Y))$ such that 
\begin{enumerate}
\item $f(1_{X}) = 1_{\mathcal{F}(X)}$
\item $\mathcal{F}(f \circ g) = \mathcal{F}(f) \circ \mathcal{F}(g)$
\end{enumerate}

To each category $\mathcal{C}$, we associate a category $\mathcal{C}^{\mathrm{op}}$ such that $$\mathrm{ob}(\mathcal{C}) = \mathrm{ob}(\mathcal{C}^{\mathrm{op}})$$ and $$\mathrm{Hom}_{\mathcal{C}^{\mathrm{op}}}(X,Y) = \mathrm{Hom}_{\mathcal{C}}(Y,X)$$

Suppose $\mathcal{F},\mathcal{F}' : \mathcal{C} \rightarrow \mathcal{C}'$ be two functors. Then, a natural transformation is $T: \mathcal{F} \rightarrow \mathcal{F}'$ consisting of the following data: 
\begin{enumerate}
\item $\forall \; X \in \mathcal{C} , \exists \; T_{X} : \mathcal{F}(X) \rightarrow \mathcal{F}'(X)$ ,i.e., $T_{X} \in \mathrm{Hom}_{\mathcal{C}'}(\mathcal{F}(X),\mathcal{F}' (X))$ such that for all $f: X \rightarrow Y$, the diagram commutes\[\begin{tikzcd}
	{\mathcal{F}(X)} & {} & {\mathcal{F}'(X)} \\
	{} & \circlearrowright & {} \\
	{\mathcal{F}(Y)} && {\mathcal{F}'(Y)}
	\arrow["{T_{X}}", from=1-1, to=1-3]
	\arrow["{\mathcal{F}(f)}"', from=1-1, to=3-1]
	\arrow["{\mathcal{F}'(f)}", from=1-3, to=3-3]
	\arrow["{T_{Y}}"', from=3-1, to=3-3]
\end{tikzcd}\]
\end{enumerate}

Given, $\mathcal{C}, \mathcal{C}'$ then $F(\mathcal{C}, \mathcal{C}')=$ all functors from $\mathcal{C}$ to $\mathcal{C}'$ is a category and $\mathrm{Hom}_{F(\mathcal{C}, \mathcal{C}')}(F_{1},F_{2})=$ all natural transformations from $F_{1}$ to $F_{2}$ 






\part{Topics in Analytic Number Theory}

\chapter{Lecture-1: Hardy-Littlewood proof of infinitely many zeros on the line $\mathfrak{R}(s) = 1/2$}

\chapter{Lecture-2: }

\chapter{Lecture-3 (10th January, 2023): Siegel's theorem }

\begin{theorem}[Siegel]
Let $\chi(q)$ be a real Dirichlet character modulo $q\geq 3$. Given any $\epsilon >0$, we have $$L(1, \chi) \geq \frac{C_{\epsilon}}{q^{\epsilon}}$$
\end{theorem}

A trivial lower bound: $L(1, \chi) \gg q^{-1/2}$

\begin{proof}[Goldfeld's proof]
Consider $$f(s) = \zeta(s)L(s,\chi_{1})L(s,\chi_{2})L(s,\chi_{1}\chi_{2})$$ with $\chi_{i}, i=1,2$ primitive quadratic characters. Notice that $f(s) = \sum_{n} b_{n}n^{-s}$ with $b_{1} =1 , b_{n} \geq 0$. Let $\lambda = \mathrm{Res}_{s=1}f(s) = L(1,\chi_{1})L(1,\chi_{2})L(1,\chi_{1}\chi_{2})$

\begin{lemma}
Given any $\epsilon >0$, one can find $\chi_{1}(q_{1})$ and $\beta$ with $1-\epsilon < \beta < 1$ such that $f(\beta) \le 0$, independent of what $\chi_{2}(q_{2})$ is. 
\end{lemma}

\begin{proof}
\textbf{Case-1:} If there are no real zeros of $L(s, \psi)$ for any primitive quadratic character in $(1-\epsilon,1)$, then $f(\beta) < 0$ for any $\beta \in (1 - \epsilon,1)$. This is because $$f(\beta) = \underbrace{\zeta(\beta)}_{<0} \underbrace{L(s,\chi_{1})L(s,\chi_{2})L(s,\chi_{1}\chi_{2})}_{>0}$$ as $L(1,\chi)>0$ and $L$ is continuous so any change of sign will lead to a zero which is a contradiction. \\
\textbf{Case-2:} If we cannot find such a $\psi$, then just set $\chi_{1}=\chi$ and let $\beta$ be the real zero. Then, $f(\beta)=0$. We are done. 
\end{proof}

Next, consider the integral 

\end{proof}

\begin{corollary}
\begin{align*}
h(-d) &= \frac{L(1,\chi_{d}) \sqrt{|d|} \;\omega}{2 \pi} \\
&= \frac{L(1,\chi_{d})}{\log \epsilon_{d}}
\end{align*}
\end{corollary}

\begin{theorem}[Y. Zhang]
$$L(1, \chi) \geq \frac{c}{(\log q)^{2022}}$$
\end{theorem}

\begin{theorem}
If $\chi(q)$ does not have a Siegel zero, then $L(1, \chi) \gg \frac{1}{\log q}$
\end{theorem}


\chapter{Lecture-4 (12th January, 2023): PNT for Dirichlet characters and APs}

\begin{lemma}
If $\rho = \beta + i \gamma$ runs through nontrivial zeros of $L(s, \chi)$, then $$\sum_{\rho} \frac{1}{1 + (T - \gamma)^2} = \mathcal{O} (\log q(|T| + 2)) \forall T \in \RR $$
\end{lemma}

\begin{lemma}
$$N(T+1, \chi) - N(T , \chi) = \mathcal{O} (\log q (|T| + 2))$$
\end{lemma}

\begin{lemma}
$$\sum_{\rho: |\gamma - t|\le 1} \frac{1}{s - \rho} + \mathcal{O}(\log qt) = \frac{L'}{L}(s,\chi)$$ for $-1 \le \sigma \le 2 , |t|\geq 2, L(s,\chi) \neq 0$
\end{lemma}

\begin{lemma}
Let $\chi(q)$ be primitive, $q \geq 3, T \geq 2$. Then, there exists $T_{1} \in [T,T+1]$ such that $\frac{L'}{L}(\sigma \pm iT_{1}, \chi) \ll (\log qT)^2 , -1 \le \sigma \le 2$.
\end{lemma}

\begin{lemma}
Put $a=1$ if $\chi$ is even and $0$ otherwise. $$\mathcal{A}(a) := \{ s \in \CC \mid \sigma \le -1 , |s + 2n - a| \geq \frac{1}{4} \; \forall \; n \geq 1\}$$ Then, $$\frac{L'}{L} (s, \chi) \ll \log (q(|s|+1))$$ on $\mathcal{A}(a)$
\end{lemma}

These are all the ingredients needed to prove the the explicit formula for $\psi_{0}(x, \chi)$. 

\begin{theorem}
$$\psi(s,\chi) = \sum_{n \le x} \Lambda(n) \chi(n)$$ $$\psi_{0}(x,\chi)  = \frac{1}{2}(\psi(x^{+}, \chi) + \psi(x^{-}, \chi)) = - \sum_{\rho : |\gamma|\le t} \frac{x^{\rho}}{\rho} - \frac{1}{2} \log (x-1) - \frac{\chi(-1)}{2} \log (x+1) + C_{\chi} + R_{\chi}(T)$$ where $C_{\chi} = \frac{L'}{L} (1, \overline{\chi}) + \log \frac{q}{2 \pi} - \gamma$ and $R_{\chi}(T) \ll (\log x)\min(1, x/T <x>) + \frac{x}{T} (\log (qxT))^2$. Letting $T \rightarrow \infty$ we see that $R_{\chi}(T) \rightarrow 0$.
\end{theorem}

\begin{theorem}[Brun-Titsmarsh inequality]
Let $x\geq 0, y \geq 2q$. Then, $$\pi(x+y; q,a) - \pi (x; q,a) \le \frac{2y}{\phi(q) \log (\frac{y}{q})} \left( 1 + \mathcal{O}(\frac{1}{\log (\frac{y}{q})}) \right)$$
\end{theorem}

\textcolor{BrickRed}{Remind him to prove this later; uses Sieve theoretic methods}

\begin{theorem}[PNT for Dirichlet characters]
There exists a $c_{1} \geq 0$ such that for all $q \le \exp (c_{1} \sqrt{\log x})$, we have $$\psi(x, \chi) = \sum_{n \le x}\Lambda (n) \chi (n) = \begin{cases} E_{0}(x) + \mathcal{O} (x\exp(-c_{1}\sqrt{\log x})) & \chi \text{ has no Siegel zero} \\ -\frac{x^{\beta_{1}}}{\beta_{1}} + \mathcal{O} (x\exp(-c_{1}\sqrt{\log x})) & \chi \text{ has Siegel zero} \end{cases}$$ Here, $E_{0}(\chi) = 1$ if $\chi = \chi_{0}$ and $0$ otherwise. 
\end{theorem}

Recall from MA317 that $L(x,\chi) \neq 0$ when $\sigma \geq  1 - \frac{c}{\log q\tau}$ for some constant $c>0$ with the exception of atmost one real zero ($\beta_{1}$ the Siegel zero)

\begin{proposition}
Let $c$ be as above and assume that $\sigma \geq 1 - \frac{c}{2 \log q \tau}$. Then, 
\begin{enumerate}
\item If $L(s,\chi)$ has no Siegel zero or if $\beta_{1}$ is a Siegel zero (thus $\chi$ quadratic) but $|s- \beta_{1}| \geq \frac{1}{\log q}$, then $$\frac{L'}{L}(s, \chi) \ll \log q \tau$$ $$|\log L(s,\chi)| \ll \log \log q\tau + \mathcal{O}(1)$$ $$\frac{1}{L(s,\chi)} \ll \log q\tau$$
\item If $\beta_{1}$ is a Siegel zero and $|s - \beta_{1}| \le \frac{1}{\log q}$, then $$\frac{L'}{L}(s, \chi) = \frac{1}{s- \beta_{1}} + \mathcal{O}(\log q)$$ $$|\mathrm{arg} L(s,\chi)| \le \log \log q + \mathcal{O}(1)$$ $$|s- \beta_{1}| \ll |L(s,\chi)| \ll |s-\beta_{1}|(\log q)^2$$
\end{enumerate}
\end{proposition}


















\part{Commutative Algebra}




\end{document}