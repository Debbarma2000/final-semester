\documentclass[oneside, 12pt, ]{article}

%packages to import
\usepackage[utf8]{inputenc}
\usepackage{amsmath}
\usepackage{amssymb}
\usepackage{amsthm}
\usepackage{float}
\usepackage{minitoc}
\usepackage[width=15cm, height=24cm]{geometry}
\usepackage{hyperref}
\usepackage[usenames,svgnames,dvipsnames]{xcolor}
\usepackage{mathrsfs}
\usepackage{mathtools}
\usepackage{thmtools}
%\usepackage{kpfonts}
\usepackage[lf]{venturis} %% lf option gives lining figures as default; 
			  %% remove option to get oldstyle figures as default
\renewcommand*\familydefault{\sfdefault} %% Only if the base font of the document is to be sans serif
\usepackage{fontenc}
\usepackage{setspace}
\usepackage{cleveref}
\usepackage{backref}
\usepackage{graphicx}
\usepackage{tikz-cd}
%\usepackage{quiver}
\usepackage{hyperref}
\usepackage{extpfeil}

%section symbol


%no indentation paragraph
\setlength{\parindent}{0pt}


% color those links
\hypersetup{
colorlinks=true,
urlcolor= BrickRed,
linkcolor= BrickRed,
citecolor= Cerulean
}
%colour page black and text white
%\pagecolor{gray}
%\color{white}

%\usepackage{multicol}
\usepackage[utf8]{inputenc}

%custom commands
\newtheorem{statement}{Statement}
\newcommand{\CC}{\mathbb C}
\newcommand{\FF}{\mathbb F}
\newcommand{\NN}{\mathbb N}
\newcommand{\QQ}{\mathbb Q}
\newcommand{\RR}{\mathbb R}
\newcommand{\ZZ}{\mathbb Z}
\newcommand{\PP}{\mathbb P}
\newcommand{\Aa}{\mathbb{A}}


\title{MA 353: Elliptic Curves \\ Assignment-1}
\author{Irish Debbarma, 16696}
\date{due: 19th February, 2023}

\begin{document}

\maketitle

\begin{enumerate}
\item $V/\QQ$ is the variety $V: 5X^2 + 6XY + 2Y^2 = 2YZ + Z^2$ \\

\textbf{Claim}: $V(\QQ) = \emptyset$. 
\begin{proof}
Since the equation is homogenous, and we are working over rationals $\QQ$, can assume that the solutions $[x:y:z]$ have $\gcd(a,b,c)=1$ and $a,b,c\in \ZZ$. 
\begin{itemize}
\item Observe $\pmod{2}$. We have $5X^2 \equiv Z^2 \pmod{2} \Rightarrow X \equiv Z \pmod{2}$. If $X,Z$ are both even, then 
\begin{itemize}
\item Observe $\pmod{4}$. We get $X^2 + 2XY + 2Y^2  = 2YZ + Z^2$. If $X,Z$ are both even, then $Y$ is even as well which contradicts our assumption on the gcd. Thus, we can assume $X,Z$ to be odd.  
\end{itemize}
\item  If we consider $\pmod{3}$, we have $2X^2 + 2Y^2 = 2YZ + Z^2 \Rightarrow 2X^2 + 3Y^2 = (Y+Z)^2 \Rightarrow X^2 = (Y+Z)^2$. Therefore, $3 \mid X$ and $3 \mid Y+Z$.
\begin{itemize}
\item Now, consider $\pmod{9}$. $2Y^2 = 2YZ + Z^2 \Rightarrow 3Y^2 = (Y+Z)^2 \Rightarrow Y^2 = 0$. Thus, $3 \mid Y \Rightarrow 3 \mid Z$. This contradicts our assumption $\gcd(X,Y,Z)=1$. 
\end{itemize}
\end{itemize} 
\end{proof}

\item For each prime $p\geq 3$, let $V_{p} \subseteq \PP^2$ be the variety corresponding to the curve $$V_{p}: X^2 + Y^2 = pZ^2$$

\begin{enumerate}
\item \textbf{Claim}: $V_{p} \cong \PP^1 $ over $\QQ$ iff $p \equiv 1 \pmod{4}$
\begin{proof}
$(\Rightarrow)$ \\

Suppose $V_{p}(\QQ) \simeq \PP^1(\QQ)$ but $p \equiv 3 \pmod{4}$. Consider the equation mod $p$ to get $X^2 + Y^2 \equiv 0 \pmod{p} \Rightarrow X^2 \equiv -Y^2 \pmod{p}$. Solving this is equivalent to checking if $-1$ is a quadratic residue of $p$ but from Euler's criterion $-1$ is a quadratic residue iff $(-1)^{(p-1)/2} =1$. Since $p = 4k+3$, the condition is not satisfied and hence $-1$ is not a quadratic residue. Thus, $V_{p}(\QQ) = \emptyset$ but clearly $\PP^1(\QQ)$ is non-empty, a contradiction. Hence, our assumption is wrong. $p$ must be congruent $1 \pmod{4}$. \\

$(\Leftarrow)$\\

Suppose $p\equiv 1 \pmod{4}$. Then there exists integers $a,b$ such that $p = a^2 + b^2$. Consider the map 
\begin{align*}
\phi : V_{p}(\QQ) &\longrightarrow \PP^1(\QQ) \\
[X,Y,Z] &\mapsto [aX + bY + pZ , (aY-bX)]
\end{align*}
This map is regular except maybe at the point $aY-bX=0, aX + bY + Z=0$, i.e., the point $[a:b:-1]$. \\

Note that 
\end{proof}

\item \textbf{Claim}: For $p \equiv 3 \pmod{4}$, no two $V_{p}$s are isomorphic.
\begin{proof}

\end{proof}
\end{enumerate}

\item Let $F(x,y,x) \in k[x,y,z]$ be a homogeneous polynomial polynomial of degree $d\geq 1$ and the curve corresponding to $F$ is non-singular. \\

\textbf{Claim}: 
\begin{equation*}
\mathfrak{g}(C) = \frac{(d-1)(d-2)}{2}
\end{equation*} 

\begin{proof}

\end{proof}

\item 
\begin{enumerate}
\item $L : 2x+5y-1=0$\\

Homogenisation gives us $2x + 5y - Z=0$. The point at infinity is the point where $z=0$. Then, $2x + 5y=0 \Rightarrow [-5:2:0]$ is the point at infinity.

\item $C : x^2 -4xy + 3y^2 -3x + 5y - 10=0$ \\

Homogenisation gives us $x^2 - 4xy + 3y^2 - 3xz + 5yz - 10z^2$. The point at infinity is the point where $z=0$. Thus, 
\begin{align*}
x^2 - 4xy + 3y^2 &=0\\
(x-2y)^2 -y^2 &=0\\
(x-2y + y)(x-2y-y) &= 0\\
(x-y)(x-3y) &=0
\end{align*}
Thus, $x=y$ or $x=3y$. The points at infinity are thus $[1:1:0]$ and $[3:1:0]$.
\end{enumerate}

\item Given $f(x,y) = y^2 - x^3 - ax^2 - bx$ \\



\item Suppose $E$ is an elliptic curve given by the Weierstrass equation $y^2 = x^3 + ax^2 + bx + c$ and $P = (x,y)$ a point on $E$.
\begin{enumerate}
\item The slope at $P$ is $\lambda = (3x^2 + 2ax  + b)/2y$ and the intercept is $\nu = (-x^3 + bx + 2c)/2y$. The line is given by $Y = \lambda X + \nu$. From the formula given in Silverman, the coordinate of $2P$ is given by 
\begin{align*}
x_{2} &= \lambda^2 - a - 2x \\
y_{2} &= -\lambda x_{2} - \nu
\end{align*}
Since we want to solve for $3P=0$, we can just solve for $2P=-P$. Again, using the formula given in Silverman, we want $x_{2}=x, y_{2} = -y$. 

\begin{align*}
- \lambda x_{2} - \nu &= -y \\
\lambda x + \nu &= y\\
\lambda = \frac{y - \nu}{x}
\end{align*}

Using this we can do the following: 
\begin{align*}
\lambda^2 - a - 2x &= x\\
\lambda^2 &= a + 3x \\
(y-\nu)^2 &= ax^2 + 3x^3 \\
y^2 + \nu^2 - 2y\nu &= ax^2 + 3x^2 \\
x^3 + ax^2 + bx + c + \nu^2 + x^3 - bx -2c &= ax^2 + 3x^3 \\
\nu^2 - c &= x^3 \\
(-x^3 + bx + 2c)^2 &= 4(x^3 + c)(x^3 + ax^2 + bx + c) \\
x^6 + b^2x^2 + 4c^2 - 2bx^4 - 4cx^3 + 4bcx &= 4x^6 + 4ax^5 
+  4bx^4 + 4cx^3 +\\& 4cx^3 + 4acx^2 + 4bcx + 4c^2 \\
3x^6 + 4ax^5 + 6bx^4 + 12x^3c + (4ac - b^2)x^2 &=0
\end{align*}
Thus, either $x=0$ or $3x^4 + 4ax^3 + 6bx^2 + 12xc + (4ac - b^2) = 0$. 

\item Now, in the particular case of $Y^2 = X^3 + 1$, we have $a=0=b, c=1$. Thus, we have two cases: 
\begin{itemize}
\item $x=0$, then $y^2=1$. Hence, the points are $[0:1], [0:-1]$.

\item \begin{align*}
3x^4 + 12x &= 0 \\
x^3 &= -4
\end{align*}
Thus, $x=\sqrt{-4}, \sqrt{-4} \omega $ or $\sqrt{-4}\omega^3$ where $\omega$ is primitive $3$rd root of unity. \\
Now, solve for $y^2 = x^3 + 1 = -2$. Therefore, $y = \sqrt{-2} i , -\sqrt{-2} i$
\end{itemize}
\end{enumerate}

\item Given $E: y^2 =x^3 + 17$ is an elliptic curve over $\QQ$

\begin{enumerate}
\item $P = (-1,4), Q = (2,5)$. We wish to find $P + Q$

\item $P =(-2,3), Q = (2,5)$. We wish to find $-P + 2Q$ 
\end{enumerate} 
\end{enumerate}


\end{document}